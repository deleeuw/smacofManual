% Options for packages loaded elsewhere
\PassOptionsToPackage{unicode}{hyperref}
\PassOptionsToPackage{hyphens}{url}
%
\documentclass[
  12pt,
]{article}
\usepackage{amsmath,amssymb}
\usepackage{iftex}
\ifPDFTeX
  \usepackage[T1]{fontenc}
  \usepackage[utf8]{inputenc}
  \usepackage{textcomp} % provide euro and other symbols
\else % if luatex or xetex
  \usepackage{unicode-math} % this also loads fontspec
  \defaultfontfeatures{Scale=MatchLowercase}
  \defaultfontfeatures[\rmfamily]{Ligatures=TeX,Scale=1}
\fi
\usepackage{lmodern}
\ifPDFTeX\else
  % xetex/luatex font selection
    \setmainfont[]{Times New Roman}
\fi
% Use upquote if available, for straight quotes in verbatim environments
\IfFileExists{upquote.sty}{\usepackage{upquote}}{}
\IfFileExists{microtype.sty}{% use microtype if available
  \usepackage[]{microtype}
  \UseMicrotypeSet[protrusion]{basicmath} % disable protrusion for tt fonts
}{}
\makeatletter
\@ifundefined{KOMAClassName}{% if non-KOMA class
  \IfFileExists{parskip.sty}{%
    \usepackage{parskip}
  }{% else
    \setlength{\parindent}{0pt}
    \setlength{\parskip}{6pt plus 2pt minus 1pt}}
}{% if KOMA class
  \KOMAoptions{parskip=half}}
\makeatother
\usepackage{xcolor}
\usepackage[margin=1in]{geometry}
\usepackage{longtable,booktabs,array}
\usepackage{calc} % for calculating minipage widths
% Correct order of tables after \paragraph or \subparagraph
\usepackage{etoolbox}
\makeatletter
\patchcmd\longtable{\par}{\if@noskipsec\mbox{}\fi\par}{}{}
\makeatother
% Allow footnotes in longtable head/foot
\IfFileExists{footnotehyper.sty}{\usepackage{footnotehyper}}{\usepackage{footnote}}
\makesavenoteenv{longtable}
\usepackage{graphicx}
\makeatletter
\def\maxwidth{\ifdim\Gin@nat@width>\linewidth\linewidth\else\Gin@nat@width\fi}
\def\maxheight{\ifdim\Gin@nat@height>\textheight\textheight\else\Gin@nat@height\fi}
\makeatother
% Scale images if necessary, so that they will not overflow the page
% margins by default, and it is still possible to overwrite the defaults
% using explicit options in \includegraphics[width, height, ...]{}
\setkeys{Gin}{width=\maxwidth,height=\maxheight,keepaspectratio}
% Set default figure placement to htbp
\makeatletter
\def\fps@figure{htbp}
\makeatother
\setlength{\emergencystretch}{3em} % prevent overfull lines
\providecommand{\tightlist}{%
  \setlength{\itemsep}{0pt}\setlength{\parskip}{0pt}}
\setcounter{secnumdepth}{5}
% definitions for citeproc citations
\NewDocumentCommand\citeproctext{}{}
\NewDocumentCommand\citeproc{mm}{%
  \begingroup\def\citeproctext{#2}\cite{#1}\endgroup}
\makeatletter
 % allow citations to break across lines
 \let\@cite@ofmt\@firstofone
 % avoid brackets around text for \cite:
 \def\@biblabel#1{}
 \def\@cite#1#2{{#1\if@tempswa , #2\fi}}
\makeatother
\newlength{\cslhangindent}
\setlength{\cslhangindent}{1.5em}
\newlength{\csllabelwidth}
\setlength{\csllabelwidth}{3em}
\newenvironment{CSLReferences}[2] % #1 hanging-indent, #2 entry-spacing
 {\begin{list}{}{%
  \setlength{\itemindent}{0pt}
  \setlength{\leftmargin}{0pt}
  \setlength{\parsep}{0pt}
  % turn on hanging indent if param 1 is 1
  \ifodd #1
   \setlength{\leftmargin}{\cslhangindent}
   \setlength{\itemindent}{-1\cslhangindent}
  \fi
  % set entry spacing
  \setlength{\itemsep}{#2\baselineskip}}}
 {\end{list}}
\usepackage{calc}
\newcommand{\CSLBlock}[1]{\hfill\break\parbox[t]{\linewidth}{\strut\ignorespaces#1\strut}}
\newcommand{\CSLLeftMargin}[1]{\parbox[t]{\csllabelwidth}{\strut#1\strut}}
\newcommand{\CSLRightInline}[1]{\parbox[t]{\linewidth - \csllabelwidth}{\strut#1\strut}}
\newcommand{\CSLIndent}[1]{\hspace{\cslhangindent}#1}
\usepackage{tcolorbox}
\usepackage{amssymb}
\usepackage{yfonts}
\usepackage{bm}
\usepackage{titlesec}
\usepackage{kbordermatrix}


\newtcolorbox{greybox}{
  colback=white,
  colframe=blue,
  coltext=black,
  boxsep=5pt,
  arc=4pt}
  
\newcommand{\sectionbreak}{\clearpage}

 
\newcommand{\ds}[4]{\sum_{{#1}=1}^{#3}\sum_{{#2}=1}^{#4}}
\newcommand{\us}[3]{\mathop{\sum\sum}_{1\leq{#2}<{#1}\leq{#3}}}

\newcommand{\ol}[1]{\overline{#1}}
\newcommand{\ul}[1]{\underline{#1}}

\newcommand{\amin}[1]{\mathop{\text{argmin}}_{#1}}
\newcommand{\amax}[1]{\mathop{\text{argmax}}_{#1}}

\newcommand{\ci}{\perp\!\!\!\perp}

\newcommand{\mc}[1]{\mathcal{#1}}
\newcommand{\mb}[1]{\mathbb{#1}}
\newcommand{\mf}[1]{\mathfrak{#1}}

\newcommand{\eps}{\epsilon}
\newcommand{\lbd}{\lambda}
\newcommand{\alp}{\alpha}
\newcommand{\df}{=:}
\newcommand{\am}[1]{\mathop{\text{argmin}}_{#1}}
\newcommand{\ls}[2]{\mathop{\sum\sum}_{#1}^{#2}}
\newcommand{\ijs}{\mathop{\sum\sum}_{1\leq i<j\leq n}}
\newcommand{\jis}{\mathop{\sum\sum}_{1\leq j<i\leq n}}
\newcommand{\sij}{\sum_{i=1}^n\sum_{j=1}^n}
	
\ifLuaTeX
  \usepackage{selnolig}  % disable illegal ligatures
\fi
\usepackage{bookmark}
\IfFileExists{xurl.sty}{\usepackage{xurl}}{} % add URL line breaks if available
\urlstyle{same}
\hypersetup{
  pdftitle={Multidimensional Voronoi Scaling of Categorical Data},
  pdfauthor={Jan de Leeuw - University of California Los Angeles},
  hidelinks,
  pdfcreator={LaTeX via pandoc}}

\title{Multidimensional Voronoi Scaling of Categorical Data}
\author{Jan de Leeuw - University of California Los Angeles}
\date{Started April 13 2024, Version of April 15, 2024}

\begin{document}
\maketitle
\begin{abstract}
smacofVO
\end{abstract}

{
\setcounter{tocdepth}{3}
\tableofcontents
}
\textbf{Note:} This is a working paper which will be expanded/updated frequently. All suggestions for improvement are welcome.

\section{Introduction}\label{introduction}

\subsection{Simultaneous Nonmetric Unfolding}\label{simultaneous-nonmetric-unfolding}

In non-metric unfolding
\[
\sigma(X,Y)=\sum_{i=1}^n\min_{\delta_i\in\Delta_i}\sum_{j=1}^mw_{ij}(\delta_{ij}-d(x_i,y_j))^2
\]
If we have data where the same individuals give preference
judgments over multiple domains, or over the same domain at multiple occasions, or over the same domain with different experimental conditions,
then we can use the loss function

\[
\sigma(X,Y_1,\cdots,Y_s)=\sum_{r=1}^s\sum_{i=1}^n\min_{\delta_i^r\in\Delta_i^r}\sum_{j=1}^{m_r}w_{ij}^r(\delta_{ij}^r-d(x_i,y_j^r))^2
\]
Note that for each occasion there are different transformations \(\Delta_r\)
and different matrices of column scores \(Y_j\), but there is only a single matrix of row scores \(X\).

z\#\# Homogeneity Analysis

Homogeneity Analysis, also known as Multiple Correspondence Analysis,
is a basic multivariate data analysis technique. The Gifi System (Gifi (1990), Michailidis and De Leeuw (1998), De Leeuw and Mair (2009)) presents the familiar linear multivariate analysis techniques (regression, analysis of variance, canonical analysis, discriminant analysis, principal component analysis) as special cases of homogeneity analysis.

We give a somewhat non-standard introduction to homogeneity analysis here, to highlight the
similarities with unfolding and the voronoi technique we will present further on in this paper.

The data are a number of indicator matrices \(G_1,\cdots,G_s\). Indicator matrices are binary matrices, with rows that add up to one. They are
used to code categorical variables. Rows corresponds with objects
(or individuals), column with the categories (or levels) of a variable.
An element \(g_{ij}\) is one in row if object \(i\) is in category \(j\),
and all other elements in row \(i\) are zero.

Homogeneity analysis makes a joint maps in \(p\) dimensions of individuals
and categories (both represented as points) in such a way that category points are close to the points for the individuals in the category. And, vice versa, individuals are close to the category points that they score in.
If there is only one variable then it is trivial to make such a
homogeneous map. We just make sure the individual points coincide with
their category points. But there are \(s>1\) indicator matrices, corresponding with \(s\) categorical variables, and the solution is a compromise trying to achieve homogeneity as well as possible for all variables simultaneously.

\[
\sigma(X,Y_1,\cdots,Y_s)=\sum_{r=1}^s\sum_{i=1}^nw_i^rd^2(x_i,y_i^r)=
\sum_{r=1}^s\text{tr}\ (X-G_rY_r)'W_r(X-G_rY_r)
\]
\[
w_i^r:=\sum_{j=1}^{m^r}g_{ij}^rw_{ij}^r
\]
\[
y_{i}^r:=\sum_{j=1}^{m^r}g_{ij}^ry_j^r
\]
Constraint is that \(\delta_{ij}^r\) is zero if \(i\) is in category \(j\) of
variable \(r\). There are no constraints on the other delta's in row \(i\).
of variable \(r\). Thus we want an object to coincide with all \(s\) categories
it is in.

star plot

\[Y_r=(G_r'W_rG_r)^{-1}G_r'W_rX\]

\[
\min_Y\sigma(X,Y_1,\cdots,Y_s)=\text{tr}\ X'\left\{\sum_{r=1}^s\left\{W_r-W_rG_r(G_r'W_rG_r)^{-1}G_r'W_r\right\}\right\}X
\]

\(X'W_\star X=I\)

\section{Voronoi Loss Function}\label{voronoi-loss-function}

\subsection{The Unconstrained Case}\label{the-unconstrained-case}

Now consider the closely related problem in which we do not require,
as in homogeneity analysis, that
\[
\delta_{i}^r:=\sum_{j=1}^{m_r}g_{ij}^r\delta_{ij}^r=0
\]
but we merely require that \(\delta_{i}^r\) is less than or equal to all
other \(\delta_{ij}^r\) in row \(i\). Formally
\[
\delta_{i}^r\leq\delta_{ij}^r\qquad\forall j=1,\cdots,m_r
\]
or
\[
g^j_{il}=1\Rightarrow\delta^j_{il}\leq\delta^j_{iv}\qquad\forall v\not=\ell
\]

\[
\sigma(X,Y_1,\cdots,Y_m) = 
\sum_{j=1}^m\left\{\sum_{i=1}^n\sum_{\ell=1}^{k_j}w_{il}^j(\delta_{i\ell}^j-d(x_i,y_\ell^j))^2\right\}
\]
\[
\sum_{j=1}^m\left\{\sum_{i=1}^n\sum_{\ell=1}^{k_j}w_{il}^jd^2(x_i,y_\ell^j)\right\}=1
\]

\[
Z_j=\kbordermatrix{
\mbox{\ }&p\\
n&X\\
k_j&Y_j}
\]

\[
d^2(x_i,y_\ell^j)=\text{tr}\ Z_j'A^j_{i\ell}Z_j^{\ }
\]
\[
\sum_{i=1}^n\sum_{\ell=1}^{k_j}w_{il}^jd^2(x_i,y_\ell^j)=\text{tr}\ Z_j'V_jZ_j
\]

\[
V_j=\sum_{i=1}^n\sum_{\ell=1}^{k_j}w_{il}^jA^j_{i\ell}
\]
\[
V_j=\kbordermatrix{&n&k_j\\
n&\ddots&\Box\\
k_j&\Box&\ddots}
\]

\[
\sum_{i=1}^n\sum_{\ell=1}^{k_j}w_{il}^j\delta_{i\ell}^jd(x_i,y_\ell^j)=\text{tr}\ Z_j'B_j(Z_j)Z_j\geq\text{tr}\ Z_j'B_j(\tilde Z_j)\tilde Z_j=\text{tr}\ Z_j'V_jG(\tilde Z_j)
\]

\[
\sigma(Z_1,\cdots,Z_m)\leq\sum_{j=1}^m\left\{\eta^2(\Delta_j)+\text{tr}\ (Z_j-G(\tilde Z_j))'V_j(Z_j-G(\tilde Z_j))-\text{tr}\ G(\tilde Z_j)'V_jG(\tilde Z_j)\right\}
\]

\[
\sum_{j=1}^m Z_j'V_jZ_j=X'\{\sum_{j=1}^mH_j'V_jH_j\}X.
\]
\[
\text{tr}\ (X-\tilde X_j)'V_{11}^j(X-\tilde X_j)+2\text{tr}\ (X-\tilde X_j)'V_{12}^j(Y_j-\tilde Y_j)+\text{tr}\ (Y_j-\tilde Y_j)'V_{22}^j(Y_j-\tilde Y_j)
\]
\#\# The Unweighted Case

If unweighted then \(V_{11}^j=k_jI\) and \(V_{22}^j=nI\). \(V_{12}^j\) is \(n\times k_j\) with all elements equal to \(-1\). The rank of \(V_j\) is \(n+k_j-1\), and the only vectors in the null-space are proportional to the vector with all elements equal to one ().

\[V_j^+=(V_j+ee)^{-1}-\frac{1}{n+k_j}ee'\]
But
\[V_j+ee'=(V_{11}^j+ee')\oplus(V_{22}^j+ee')=(k_jI+ee')\oplus(nI+ee')\]
\[(V_j+ee')^{-1}=(k_jI+ee')^{-1}\oplus(nI+ee')^{-1}\]
Lemma: If \(A\) is a symmetric matrix of order \(n\) of the form
\(\alpha I+ee'\) then \((\alpha I+ee')^{-1}=\)

Proof:

Heiser and De Leeuw (1979)

Sherman-Morrison

\[
(n I+ee')^{-1}=\frac{1}{n}\left\{I-\frac{1}{k_j+n}ee'\right\}\]
\[
(k_j I+ee')^{-1}=\frac{1}{k_j}\left\{I-\frac{1}{k_j+n}ee'\right\}\]

Schur complement \[V_{11}-V_{12}V_{22}^{-1}V_{21}=kI-e_ne_k'\frac{1}{n}e_ke_n'=k\{I-\frac{1}{n}e_ne_n'\}=kJ_n\]

\subsection{The Constrained Case}\label{the-constrained-case}

\[
Z_j=
\kbordermatrix{
\mbox{\ }&p\\
n&I\\
k_j&D_j^{-1}G_j'}X=H_jX
\]

\section*{References}\label{references}
\addcontentsline{toc}{section}{References}

\phantomsection\label{refs}
\begin{CSLReferences}{1}{0}
\bibitem[\citeproctext]{ref-deleeuw_mair_A_09a}
De Leeuw, J., and P. Mair. 2009. {``{Homogeneity Analysis in {R}: the Package homals}.''} \emph{Journal of Statistical Software} 31 (4): 1--21. \url{https://www.jstatsoft.org/v31/i04/}.

\bibitem[\citeproctext]{ref-gifi_B_90}
Gifi, A. 1990. \emph{Nonlinear Multivariate Analysis}. New York, N.Y.: Wiley.

\bibitem[\citeproctext]{ref-heiser_deleeuw_A_79}
Heiser, W. J., and J. De Leeuw. 1979. {``Metric Multidimensional Unfolding.''} \emph{Methoden En Data Nieuwsbrief SWS/VVS} 4: 26--50.

\bibitem[\citeproctext]{ref-michailidis_deleeuw_A_98}
Michailidis, G., and J. De Leeuw. 1998. {``The Gifi System for Descriptive Multivariate Analysis.''} \emph{Statistical Science} 13: 307--36.

\end{CSLReferences}

\end{document}
