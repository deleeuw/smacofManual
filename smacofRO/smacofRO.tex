% Options for packages loaded elsewhere
\PassOptionsToPackage{unicode}{hyperref}
\PassOptionsToPackage{hyphens}{url}
%
\documentclass[
  12pt,
]{article}
\usepackage{amsmath,amssymb}
\usepackage{iftex}
\ifPDFTeX
  \usepackage[T1]{fontenc}
  \usepackage[utf8]{inputenc}
  \usepackage{textcomp} % provide euro and other symbols
\else % if luatex or xetex
  \usepackage{unicode-math} % this also loads fontspec
  \defaultfontfeatures{Scale=MatchLowercase}
  \defaultfontfeatures[\rmfamily]{Ligatures=TeX,Scale=1}
\fi
\usepackage{lmodern}
\ifPDFTeX\else
  % xetex/luatex font selection
    \setmainfont[]{Times New Roman}
\fi
% Use upquote if available, for straight quotes in verbatim environments
\IfFileExists{upquote.sty}{\usepackage{upquote}}{}
\IfFileExists{microtype.sty}{% use microtype if available
  \usepackage[]{microtype}
  \UseMicrotypeSet[protrusion]{basicmath} % disable protrusion for tt fonts
}{}
\makeatletter
\@ifundefined{KOMAClassName}{% if non-KOMA class
  \IfFileExists{parskip.sty}{%
    \usepackage{parskip}
  }{% else
    \setlength{\parindent}{0pt}
    \setlength{\parskip}{6pt plus 2pt minus 1pt}}
}{% if KOMA class
  \KOMAoptions{parskip=half}}
\makeatother
\usepackage{xcolor}
\usepackage[margin=1in]{geometry}
\usepackage{color}
\usepackage{fancyvrb}
\newcommand{\VerbBar}{|}
\newcommand{\VERB}{\Verb[commandchars=\\\{\}]}
\DefineVerbatimEnvironment{Highlighting}{Verbatim}{commandchars=\\\{\}}
% Add ',fontsize=\small' for more characters per line
\usepackage{framed}
\definecolor{shadecolor}{RGB}{248,248,248}
\newenvironment{Shaded}{\begin{snugshade}}{\end{snugshade}}
\newcommand{\AlertTok}[1]{\textcolor[rgb]{0.94,0.16,0.16}{#1}}
\newcommand{\AnnotationTok}[1]{\textcolor[rgb]{0.56,0.35,0.01}{\textbf{\textit{#1}}}}
\newcommand{\AttributeTok}[1]{\textcolor[rgb]{0.13,0.29,0.53}{#1}}
\newcommand{\BaseNTok}[1]{\textcolor[rgb]{0.00,0.00,0.81}{#1}}
\newcommand{\BuiltInTok}[1]{#1}
\newcommand{\CharTok}[1]{\textcolor[rgb]{0.31,0.60,0.02}{#1}}
\newcommand{\CommentTok}[1]{\textcolor[rgb]{0.56,0.35,0.01}{\textit{#1}}}
\newcommand{\CommentVarTok}[1]{\textcolor[rgb]{0.56,0.35,0.01}{\textbf{\textit{#1}}}}
\newcommand{\ConstantTok}[1]{\textcolor[rgb]{0.56,0.35,0.01}{#1}}
\newcommand{\ControlFlowTok}[1]{\textcolor[rgb]{0.13,0.29,0.53}{\textbf{#1}}}
\newcommand{\DataTypeTok}[1]{\textcolor[rgb]{0.13,0.29,0.53}{#1}}
\newcommand{\DecValTok}[1]{\textcolor[rgb]{0.00,0.00,0.81}{#1}}
\newcommand{\DocumentationTok}[1]{\textcolor[rgb]{0.56,0.35,0.01}{\textbf{\textit{#1}}}}
\newcommand{\ErrorTok}[1]{\textcolor[rgb]{0.64,0.00,0.00}{\textbf{#1}}}
\newcommand{\ExtensionTok}[1]{#1}
\newcommand{\FloatTok}[1]{\textcolor[rgb]{0.00,0.00,0.81}{#1}}
\newcommand{\FunctionTok}[1]{\textcolor[rgb]{0.13,0.29,0.53}{\textbf{#1}}}
\newcommand{\ImportTok}[1]{#1}
\newcommand{\InformationTok}[1]{\textcolor[rgb]{0.56,0.35,0.01}{\textbf{\textit{#1}}}}
\newcommand{\KeywordTok}[1]{\textcolor[rgb]{0.13,0.29,0.53}{\textbf{#1}}}
\newcommand{\NormalTok}[1]{#1}
\newcommand{\OperatorTok}[1]{\textcolor[rgb]{0.81,0.36,0.00}{\textbf{#1}}}
\newcommand{\OtherTok}[1]{\textcolor[rgb]{0.56,0.35,0.01}{#1}}
\newcommand{\PreprocessorTok}[1]{\textcolor[rgb]{0.56,0.35,0.01}{\textit{#1}}}
\newcommand{\RegionMarkerTok}[1]{#1}
\newcommand{\SpecialCharTok}[1]{\textcolor[rgb]{0.81,0.36,0.00}{\textbf{#1}}}
\newcommand{\SpecialStringTok}[1]{\textcolor[rgb]{0.31,0.60,0.02}{#1}}
\newcommand{\StringTok}[1]{\textcolor[rgb]{0.31,0.60,0.02}{#1}}
\newcommand{\VariableTok}[1]{\textcolor[rgb]{0.00,0.00,0.00}{#1}}
\newcommand{\VerbatimStringTok}[1]{\textcolor[rgb]{0.31,0.60,0.02}{#1}}
\newcommand{\WarningTok}[1]{\textcolor[rgb]{0.56,0.35,0.01}{\textbf{\textit{#1}}}}
\usepackage{longtable,booktabs,array}
\usepackage{calc} % for calculating minipage widths
% Correct order of tables after \paragraph or \subparagraph
\usepackage{etoolbox}
\makeatletter
\patchcmd\longtable{\par}{\if@noskipsec\mbox{}\fi\par}{}{}
\makeatother
% Allow footnotes in longtable head/foot
\IfFileExists{footnotehyper.sty}{\usepackage{footnotehyper}}{\usepackage{footnote}}
\makesavenoteenv{longtable}
\usepackage{graphicx}
\makeatletter
\def\maxwidth{\ifdim\Gin@nat@width>\linewidth\linewidth\else\Gin@nat@width\fi}
\def\maxheight{\ifdim\Gin@nat@height>\textheight\textheight\else\Gin@nat@height\fi}
\makeatother
% Scale images if necessary, so that they will not overflow the page
% margins by default, and it is still possible to overwrite the defaults
% using explicit options in \includegraphics[width, height, ...]{}
\setkeys{Gin}{width=\maxwidth,height=\maxheight,keepaspectratio}
% Set default figure placement to htbp
\makeatletter
\def\fps@figure{htbp}
\makeatother
\setlength{\emergencystretch}{3em} % prevent overfull lines
\providecommand{\tightlist}{%
  \setlength{\itemsep}{0pt}\setlength{\parskip}{0pt}}
\setcounter{secnumdepth}{5}
% definitions for citeproc citations
\NewDocumentCommand\citeproctext{}{}
\NewDocumentCommand\citeproc{mm}{%
  \begingroup\def\citeproctext{#2}\cite{#1}\endgroup}
\makeatletter
 % allow citations to break across lines
 \let\@cite@ofmt\@firstofone
 % avoid brackets around text for \cite:
 \def\@biblabel#1{}
 \def\@cite#1#2{{#1\if@tempswa , #2\fi}}
\makeatother
\newlength{\cslhangindent}
\setlength{\cslhangindent}{1.5em}
\newlength{\csllabelwidth}
\setlength{\csllabelwidth}{3em}
\newenvironment{CSLReferences}[2] % #1 hanging-indent, #2 entry-spacing
 {\begin{list}{}{%
  \setlength{\itemindent}{0pt}
  \setlength{\leftmargin}{0pt}
  \setlength{\parsep}{0pt}
  % turn on hanging indent if param 1 is 1
  \ifodd #1
   \setlength{\leftmargin}{\cslhangindent}
   \setlength{\itemindent}{-1\cslhangindent}
  \fi
  % set entry spacing
  \setlength{\itemsep}{#2\baselineskip}}}
 {\end{list}}
\usepackage{calc}
\newcommand{\CSLBlock}[1]{\hfill\break\parbox[t]{\linewidth}{\strut\ignorespaces#1\strut}}
\newcommand{\CSLLeftMargin}[1]{\parbox[t]{\csllabelwidth}{\strut#1\strut}}
\newcommand{\CSLRightInline}[1]{\parbox[t]{\linewidth - \csllabelwidth}{\strut#1\strut}}
\newcommand{\CSLIndent}[1]{\hspace{\cslhangindent}#1}
\usepackage{tcolorbox}
\usepackage{amssymb}
\usepackage{yfonts}
\usepackage{bm}
\usepackage{titlesec}
\usepackage{kbordermatrix}


\newtcolorbox{greybox}{
  colback=white,
  colframe=blue,
  coltext=black,
  boxsep=5pt,
  arc=4pt}
  
\newcommand{\sectionbreak}{\clearpage}

 
\newcommand{\ds}[4]{\sum_{{#1}=1}^{#3}\sum_{{#2}=1}^{#4}}
\newcommand{\us}[3]{\mathop{\sum\sum}_{1\leq{#2}<{#1}\leq{#3}}}

\newcommand{\ol}[1]{\overline{#1}}
\newcommand{\ul}[1]{\underline{#1}}

\newcommand{\amin}[1]{\mathop{\text{argmin}}_{#1}}
\newcommand{\amax}[1]{\mathop{\text{argmax}}_{#1}}

\newcommand{\ci}{\perp\!\!\!\perp}

\newcommand{\mc}[1]{\mathcal{#1}}
\newcommand{\mb}[1]{\mathbb{#1}}
\newcommand{\mf}[1]{\mathfrak{#1}}

\newcommand{\eps}{\epsilon}
\newcommand{\lbd}{\lambda}
\newcommand{\alp}{\alpha}
\newcommand{\df}{=:}
\newcommand{\am}[1]{\mathop{\text{argmin}}_{#1}}
\newcommand{\ls}[2]{\mathop{\sum\sum}_{#1}^{#2}}
\newcommand{\ijs}{\mathop{\sum\sum}_{1\leq i<j\leq n}}
\newcommand{\jis}{\mathop{\sum\sum}_{1\leq j<i\leq n}}
\newcommand{\sij}{\sum_{i=1}^n\sum_{j=1}^n}
	
\ifLuaTeX
  \usepackage{selnolig}  % disable illegal ligatures
\fi
\usepackage{bookmark}
\IfFileExists{xurl.sty}{\usepackage{xurl}}{} % add URL line breaks if available
\urlstyle{same}
\hypersetup{
  pdfauthor={Jan de Leeuw - University of California Los Angeles},
  hidelinks,
  pdfcreator={LaTeX via pandoc}}

\title{Smacof at 50: A Manual\\
Part x: smacofRO - Non-metric Smacof for Rank Order Data}
\author{Jan de Leeuw - University of California Los Angeles}
\date{Started May 12 2024, Version of May 12, 2024}

\begin{document}
\maketitle

{
\setcounter{tocdepth}{3}
\tableofcontents
}
\textbf{Note:} This is a working manuscript which will be expanded/updated
frequently. All suggestions for improvement are welcome. All Rmd, tex,
html, pdf, R, and C files are in the public domain. Attribution will be
appreciated, but is not required. The files can be found at
\url{https://github.com/deleeuw} in the repositories smacofCode, smacofManual, and smacofExamples.

\section{Introduction}\label{introduction}

The smacofRO program is a convential non-metric MDS program designed
to handle a partial order over pairs from a single set of objects.
Typically the partial order is derived from a symmetric matrix of
numerical dissimilarities or from a ranking (with or without ties)
of dissimilarities.

\section{Data}\label{data}

\section{Program}\label{program}

\subsection{Parameters}\label{parameters}

\begin{Shaded}
\begin{Highlighting}[]
\NormalTok{smacofRO }\OtherTok{\textless{}{-}} \ControlFlowTok{function}\NormalTok{(data,}
\NormalTok{                     ndim,}
                     \AttributeTok{xold =} \ConstantTok{NULL}\NormalTok{,}
                     \AttributeTok{labels =} \ConstantTok{NULL}\NormalTok{,}
                     \AttributeTok{width =} \DecValTok{15}\NormalTok{,}
                     \AttributeTok{precision =} \DecValTok{10}\NormalTok{,}
                     \AttributeTok{itmax =} \DecValTok{1000}\NormalTok{,}
                     \AttributeTok{eps =} \FloatTok{1e{-}10}\NormalTok{,}
                     \AttributeTok{verbose =} \ConstantTok{TRUE}\NormalTok{,}
                     \AttributeTok{kitmax =} \DecValTok{5}\NormalTok{,}
                     \AttributeTok{keps =} \FloatTok{1e{-}10}\NormalTok{,}
                     \AttributeTok{kverbose =} \DecValTok{0}\NormalTok{,}
                     \AttributeTok{init =} \DecValTok{1}\NormalTok{,}
                     \AttributeTok{ties =} \DecValTok{1}\NormalTok{)}
\end{Highlighting}
\end{Shaded}

\subsection{Input}\label{input}

The data are a five column table. Here are the first five rows for
the Gruijter example, analyzed below.

\begin{Shaded}
\begin{Highlighting}[]
\NormalTok{     i j delta     weight ties}
\NormalTok{[}\DecValTok{1}\NormalTok{,] }\DecValTok{7} \DecValTok{6}  \FloatTok{3.20} \FloatTok{0.02777778}    \DecValTok{1}
\NormalTok{[}\DecValTok{2}\NormalTok{,] }\DecValTok{2} \DecValTok{1}  \FloatTok{4.08} \FloatTok{0.02777778}    \DecValTok{2}
\NormalTok{[}\DecValTok{3}\NormalTok{,] }\DecValTok{3} \DecValTok{2}  \FloatTok{4.59} \FloatTok{0.02777778}    \DecValTok{3}
\NormalTok{[}\DecValTok{4}\NormalTok{,] }\DecValTok{6} \DecValTok{5}  \FloatTok{4.60} \FloatTok{0.02777778}    \DecValTok{4}
\NormalTok{[}\DecValTok{5}\NormalTok{,] }\DecValTok{8} \DecValTok{4}  \FloatTok{4.67} \FloatTok{0.02777778}    \DecValTok{5}
\end{Highlighting}
\end{Shaded}

The first two columns give the indices of the dissimilarities. Always \(1\leq j<i\leq n\).
Column three gives the value of delta, which can be an observed numerical value or
a rank number. The data are in non-increasing order within their
column. Column four gives the weights, adding up to one over all observations.
And the last column codes tieblocks. It is allowed to have are fewer than \(\frac12n(n-1)\) rows.
The missing rows are computationally equivalent to non-missing rows with weight equal to zero.

Typically we create the data by using the utility function, which is part of the
smacofRO package. Here delta is a symmetric matrix or an object of class dist.
If weights is NULL all weights are assumed to be equal.

\begin{Shaded}
\begin{Highlighting}[]
\NormalTok{smacofMakeRankOrderData }\OtherTok{\textless{}{-}}\ControlFlowTok{function}\NormalTok{(delta, }\AttributeTok{weights =} \ConstantTok{NULL}\NormalTok{, }\AttributeTok{tieblocks =} \ConstantTok{TRUE}\NormalTok{)}
\end{Highlighting}
\end{Shaded}

\subsection{Algorithm}\label{algorithm}

The ALS algorithm alternates a number of Guttman iterations with a single monotone
regression (with either one of the three options for ties, cf De Leeuw (1977)).
Guttman iterations are normalized explicitly, using \(\sum\sum w_{ij}d_{ij}^2(X)=1\).
Monotone regression is done with the pava algorithm from De Leeuw (2017).

\subsection{Output}\label{output}

\section{Examples}\label{examples}

\subsection{De Gruijter (1967)}\label{degruijter_67}

In this example the number of tieblocks is 35, which means
there is a single tieblock of two observations, the other 34 observations are
untied. We expect that the three different ways of handling ties will not make
much of a difference.

\begin{Shaded}
\begin{Highlighting}[]
\NormalTok{h1 }\OtherTok{\textless{}{-}} \FunctionTok{smacofRO}\NormalTok{(gruijterData, }\DecValTok{2}\NormalTok{, }\AttributeTok{ties =} \DecValTok{1}\NormalTok{, }\AttributeTok{labels =}\NormalTok{ labels, }\AttributeTok{verbose =} \ConstantTok{FALSE}\NormalTok{)}
\NormalTok{h2 }\OtherTok{\textless{}{-}} \FunctionTok{smacofRO}\NormalTok{(gruijterData, }\DecValTok{2}\NormalTok{, }\AttributeTok{ties =} \DecValTok{2}\NormalTok{, }\AttributeTok{labels =}\NormalTok{ labels, }\AttributeTok{verbose =} \ConstantTok{FALSE}\NormalTok{)}
\NormalTok{h3 }\OtherTok{\textless{}{-}} \FunctionTok{smacofRO}\NormalTok{(gruijterData, }\DecValTok{2}\NormalTok{, }\AttributeTok{ties =} \DecValTok{3}\NormalTok{, }\AttributeTok{labels =}\NormalTok{ labels, }\AttributeTok{verbose =} \ConstantTok{FALSE}\NormalTok{)}
\end{Highlighting}
\end{Shaded}

\begin{itemize}
\tightlist
\item
  Ties = 1 uses 88 iterations and stops at stress 0.0042180140
\item
  Ties = 2 uses 65 iterations and stops at stress 0.0042573281
\item
  Ties = 3 uses 69 iterations and stops at stress 0.0040850898
\end{itemize}

And indeed the solutions are practically the same, although the number of iterations needed for convergence
differs somewhat between options. We only show one Shepardplot, because the others are virtually the same.

\begin{center}\includegraphics{smacofRO_files/figure-latex/plotgruijter-1} \end{center}

We can also use this example to show the effect of using different initial configurations.

\begin{Shaded}
\begin{Highlighting}[]
\NormalTok{h1 }\OtherTok{\textless{}{-}} \FunctionTok{smacofRO}\NormalTok{(gruijterData, }\DecValTok{2}\NormalTok{, }\AttributeTok{init =} \DecValTok{1}\NormalTok{, }\AttributeTok{labels =}\NormalTok{ labels, }\AttributeTok{verbose =} \ConstantTok{FALSE}\NormalTok{)}
\NormalTok{h2 }\OtherTok{\textless{}{-}} \FunctionTok{smacofRO}\NormalTok{(gruijterData, }\DecValTok{2}\NormalTok{, }\AttributeTok{init =} \DecValTok{2}\NormalTok{, }\AttributeTok{labels =}\NormalTok{ labels, }\AttributeTok{verbose =} \ConstantTok{FALSE}\NormalTok{)}
\NormalTok{h3 }\OtherTok{\textless{}{-}} \FunctionTok{smacofRO}\NormalTok{(gruijterData, }\DecValTok{2}\NormalTok{, }\AttributeTok{init =} \DecValTok{3}\NormalTok{, }\AttributeTok{labels =}\NormalTok{ labels, }\AttributeTok{verbose =} \ConstantTok{FALSE}\NormalTok{)}
\end{Highlighting}
\end{Shaded}

\begin{itemize}
\tightlist
\item
  Init = 1 uses 88 iterations and stops at stress 0.0042180140
\item
  Init = 2 uses 76 iterations and stops at stress 0.0039894695
\item
  Init = 3 uses 69 iterations and stops at stress 0.0060410599
\end{itemize}

\begin{center}\includegraphics{smacofRO_files/figure-latex/gruiterconfs-1} \end{center}

\begin{center}\includegraphics{smacofRO_files/figure-latex/gruiterconfs-2} \end{center}

\begin{center}\includegraphics{smacofRO_files/figure-latex/gruiterconfs-3} \end{center}

Using different initial configurations in this example makes a huge difference. All three plots
show the (CPN,PvdA,PSP) leftist cluster, the liberal (VVD,D'66) cluster, the protest BP outlier which is its own cluster, and the (KVP,ARP,CHU) christian democrat cluster. In the three plots the
clusters are distributed differently over the plane.

\subsection{Ekman (1954)}\label{ekman_54}

In the Ekman data there are 47 tieblocks out of
91 observations, and we expect the
ties option to make some difference.

\begin{Shaded}
\begin{Highlighting}[]
\NormalTok{h1 }\OtherTok{\textless{}{-}} \FunctionTok{smacofRO}\NormalTok{(ekmanData, }\DecValTok{2}\NormalTok{, }\AttributeTok{ties =} \DecValTok{1}\NormalTok{, }\AttributeTok{verbose =} \ConstantTok{FALSE}\NormalTok{, }\AttributeTok{labels =}\NormalTok{ labels)}
\NormalTok{h2 }\OtherTok{\textless{}{-}} \FunctionTok{smacofRO}\NormalTok{(ekmanData, }\DecValTok{2}\NormalTok{, }\AttributeTok{ties =} \DecValTok{2}\NormalTok{, }\AttributeTok{verbose =} \ConstantTok{FALSE}\NormalTok{, }\AttributeTok{labels =}\NormalTok{ labels)}
\NormalTok{h3 }\OtherTok{\textless{}{-}} \FunctionTok{smacofRO}\NormalTok{(ekmanData, }\DecValTok{2}\NormalTok{, }\AttributeTok{ties =} \DecValTok{3}\NormalTok{, }\AttributeTok{verbose =} \ConstantTok{FALSE}\NormalTok{, }\AttributeTok{labels =}\NormalTok{ labels)}
\end{Highlighting}
\end{Shaded}

\begin{itemize}
\tightlist
\item
  Ties = 1 uses 40 iterations and stops at stress 0.0002668633
\item
  Ties = 2 uses 32 iterations and stops at stress 0.0004988332
\item
  Ties = 3 uses 1000 iterations and stops at stress 0.0000001193
\end{itemize}

\begin{center}\includegraphics{smacofRO_files/figure-latex/plotekman-1} \end{center}

\begin{center}\includegraphics{smacofRO_files/figure-latex/plotekman-2} \end{center}

\begin{center}\includegraphics{smacofRO_files/figure-latex/plotekman-3} \end{center}

Although the stress values are indeed different, the solutions are practically the same. When looking at the
Shepard plots we have to remember that for ties = 1 and ties = 3 we can have different disparity values for the
same dissimilarity value. Thus tieblocks are represented as intervals on the vertical axis, and strictly spoken we do not have a functional relationship between delta and dhat. This is clear from the plot for the
tertiary approach, which shows some of the intervals, mostly for the larger dissimilarities. It reinforces the
idea that the tertiary approach is only useful if there are many small tieblocks, in which case it will be quite similar to the primary and secondary approach.

\begin{center}\includegraphics{smacofRO_files/figure-latex/plotekmanconfs-1} \end{center}

\begin{center}\includegraphics{smacofRO_files/figure-latex/plotekmanconfs-2} \end{center}

\begin{center}\includegraphics{smacofRO_files/figure-latex/plotekmanconfs-3} \end{center}

\subsection{Rothkopf (1957)}\label{rothkopf_57}

\begin{Shaded}
\begin{Highlighting}[]
\NormalTok{h1 }\OtherTok{\textless{}{-}} \FunctionTok{smacofRO}\NormalTok{(morseData, }\DecValTok{2}\NormalTok{, }\AttributeTok{ties =} \DecValTok{1}\NormalTok{, }\AttributeTok{verbose =} \ConstantTok{FALSE}\NormalTok{, }\AttributeTok{itmax =} \DecValTok{10000}\NormalTok{)}
\NormalTok{h2 }\OtherTok{\textless{}{-}} \FunctionTok{smacofRO}\NormalTok{(morseData, }\DecValTok{2}\NormalTok{, }\AttributeTok{ties =} \DecValTok{2}\NormalTok{, }\AttributeTok{verbose =} \ConstantTok{FALSE}\NormalTok{, }\AttributeTok{itmax =} \DecValTok{10000}\NormalTok{)}
\NormalTok{h3 }\OtherTok{\textless{}{-}} \FunctionTok{smacofRO}\NormalTok{(morseData, }\DecValTok{2}\NormalTok{, }\AttributeTok{ties =} \DecValTok{3}\NormalTok{, }\AttributeTok{verbose =} \ConstantTok{FALSE}\NormalTok{, }\AttributeTok{itmax =} \DecValTok{10000}\NormalTok{)}
\end{Highlighting}
\end{Shaded}

\begin{itemize}
\tightlist
\item
  Ties = 1 uses 34 iterations and stops at stress 0.0163278414
\item
  Ties = 2 uses 30 iterations and stops at stress 0.0203202336
\item
  Ties = 3 uses 1129 iterations and stops at stress 0.0000000226
\end{itemize}

\begin{center}\includegraphics{smacofRO_files/figure-latex/plotmorse-1} \end{center}

\begin{center}\includegraphics{smacofRO_files/figure-latex/plotmorse-2} \end{center}

\begin{center}\includegraphics{smacofRO_files/figure-latex/plotmorse-3} \end{center}

We also run morse with different values of kitmax, the number of inner Guttman iterations. The results are

\begin{Shaded}
\begin{Highlighting}[]
\NormalTok{h1 }\OtherTok{\textless{}{-}} \FunctionTok{smacofRO}\NormalTok{(morseData, }\DecValTok{2}\NormalTok{, }\AttributeTok{kitmax =} \DecValTok{1}\NormalTok{, }\AttributeTok{verbose =} \ConstantTok{FALSE}\NormalTok{, }\AttributeTok{itmax =} \DecValTok{10000}\NormalTok{)}
\NormalTok{h2 }\OtherTok{\textless{}{-}} \FunctionTok{smacofRO}\NormalTok{(morseData, }\DecValTok{2}\NormalTok{, }\AttributeTok{kitmax =} \DecValTok{5}\NormalTok{, }\AttributeTok{verbose =} \ConstantTok{FALSE}\NormalTok{, }\AttributeTok{itmax =} \DecValTok{10000}\NormalTok{)}
\NormalTok{h3 }\OtherTok{\textless{}{-}} \FunctionTok{smacofRO}\NormalTok{(morseData, }\DecValTok{2}\NormalTok{, }\AttributeTok{kitmax =} \DecValTok{10}\NormalTok{, }\AttributeTok{verbose =} \ConstantTok{FALSE}\NormalTok{, }\AttributeTok{itmax =} \DecValTok{10000}\NormalTok{)}
\end{Highlighting}
\end{Shaded}

\begin{itemize}
\tightlist
\item
  Kitmax = 1 uses 138 major iterations and stops at stress 0.0163278417
\item
  Kitmax = 5 uses 34 major iterations and stops at stress 0.0163278414
\item
  Kitmax = 10 uses 22 major iterations and stops at stress 0.0163278414
\end{itemize}

We see that the resulting stress values are the same, but the number of outer iterations differs. This may be
significant, because it means that using only one Guttman iteration means doing 138 monotone
rgressions, while having 10 Guttman iterations per major iteration only uses 22 monotone
regressions.

\section*{References}\label{references}
\addcontentsline{toc}{section}{References}

\phantomsection\label{refs}
\begin{CSLReferences}{1}{0}
\bibitem[\citeproctext]{ref-degruijter_67}
De Gruijter, D. N. M. 1967. {``{The Cognitive Structure of Dutch Political Parties in 1966}.''} Report E019-67. Psychological Institute, University of Leiden.

\bibitem[\citeproctext]{ref-deleeuw_A_77}
De Leeuw, J. 1977. {``Correctness of Kruskal's Algorithms for Monotone Regression with Ties.''} \emph{Psychometrika} 42: 141--44.

\bibitem[\citeproctext]{ref-deleeuw_E_17h}
---------. 2017. {``{Exceedingly Simple Monotone Regression (with Ties)}.''} 2017.

\bibitem[\citeproctext]{ref-ekman_54}
Ekman, G. 1954. {``{Dimensions of Color Vision}.''} \emph{Journal of Psychology} 38: 467--74.

\bibitem[\citeproctext]{ref-rothkopf_57}
Rothkopf, E. Z. 1957. {``{A Measure of Stimulus Similarity and Errors in some Paired-associate Learning}.''} \emph{Journal of Experimental Psychology} 53: 94--101.

\end{CSLReferences}

\end{document}
