% Options for packages loaded elsewhere
\PassOptionsToPackage{unicode}{hyperref}
\PassOptionsToPackage{hyphens}{url}
%
\documentclass[
  12pt,
]{article}
\usepackage{amsmath,amssymb}
\usepackage{iftex}
\ifPDFTeX
  \usepackage[T1]{fontenc}
  \usepackage[utf8]{inputenc}
  \usepackage{textcomp} % provide euro and other symbols
\else % if luatex or xetex
  \usepackage{unicode-math} % this also loads fontspec
  \defaultfontfeatures{Scale=MatchLowercase}
  \defaultfontfeatures[\rmfamily]{Ligatures=TeX,Scale=1}
\fi
\usepackage{lmodern}
\ifPDFTeX\else
  % xetex/luatex font selection
    \setmainfont[]{Times New Roman}
\fi
% Use upquote if available, for straight quotes in verbatim environments
\IfFileExists{upquote.sty}{\usepackage{upquote}}{}
\IfFileExists{microtype.sty}{% use microtype if available
  \usepackage[]{microtype}
  \UseMicrotypeSet[protrusion]{basicmath} % disable protrusion for tt fonts
}{}
\makeatletter
\@ifundefined{KOMAClassName}{% if non-KOMA class
  \IfFileExists{parskip.sty}{%
    \usepackage{parskip}
  }{% else
    \setlength{\parindent}{0pt}
    \setlength{\parskip}{6pt plus 2pt minus 1pt}}
}{% if KOMA class
  \KOMAoptions{parskip=half}}
\makeatother
\usepackage{xcolor}
\usepackage[margin=1in]{geometry}
\usepackage{longtable,booktabs,array}
\usepackage{calc} % for calculating minipage widths
% Correct order of tables after \paragraph or \subparagraph
\usepackage{etoolbox}
\makeatletter
\patchcmd\longtable{\par}{\if@noskipsec\mbox{}\fi\par}{}{}
\makeatother
% Allow footnotes in longtable head/foot
\IfFileExists{footnotehyper.sty}{\usepackage{footnotehyper}}{\usepackage{footnote}}
\makesavenoteenv{longtable}
\usepackage{graphicx}
\makeatletter
\def\maxwidth{\ifdim\Gin@nat@width>\linewidth\linewidth\else\Gin@nat@width\fi}
\def\maxheight{\ifdim\Gin@nat@height>\textheight\textheight\else\Gin@nat@height\fi}
\makeatother
% Scale images if necessary, so that they will not overflow the page
% margins by default, and it is still possible to overwrite the defaults
% using explicit options in \includegraphics[width, height, ...]{}
\setkeys{Gin}{width=\maxwidth,height=\maxheight,keepaspectratio}
% Set default figure placement to htbp
\makeatletter
\def\fps@figure{htbp}
\makeatother
\setlength{\emergencystretch}{3em} % prevent overfull lines
\providecommand{\tightlist}{%
  \setlength{\itemsep}{0pt}\setlength{\parskip}{0pt}}
\setcounter{secnumdepth}{5}
% definitions for citeproc citations
\NewDocumentCommand\citeproctext{}{}
\NewDocumentCommand\citeproc{mm}{%
  \begingroup\def\citeproctext{#2}\cite{#1}\endgroup}
\makeatletter
 % allow citations to break across lines
 \let\@cite@ofmt\@firstofone
 % avoid brackets around text for \cite:
 \def\@biblabel#1{}
 \def\@cite#1#2{{#1\if@tempswa , #2\fi}}
\makeatother
\newlength{\cslhangindent}
\setlength{\cslhangindent}{1.5em}
\newlength{\csllabelwidth}
\setlength{\csllabelwidth}{3em}
\newenvironment{CSLReferences}[2] % #1 hanging-indent, #2 entry-spacing
 {\begin{list}{}{%
  \setlength{\itemindent}{0pt}
  \setlength{\leftmargin}{0pt}
  \setlength{\parsep}{0pt}
  % turn on hanging indent if param 1 is 1
  \ifodd #1
   \setlength{\leftmargin}{\cslhangindent}
   \setlength{\itemindent}{-1\cslhangindent}
  \fi
  % set entry spacing
  \setlength{\itemsep}{#2\baselineskip}}}
 {\end{list}}
\usepackage{calc}
\newcommand{\CSLBlock}[1]{\hfill\break\parbox[t]{\linewidth}{\strut\ignorespaces#1\strut}}
\newcommand{\CSLLeftMargin}[1]{\parbox[t]{\csllabelwidth}{\strut#1\strut}}
\newcommand{\CSLRightInline}[1]{\parbox[t]{\linewidth - \csllabelwidth}{\strut#1\strut}}
\newcommand{\CSLIndent}[1]{\hspace{\cslhangindent}#1}
\usepackage{tcolorbox}
\usepackage{amssymb}
\usepackage{yfonts}
\usepackage{bm}
\usepackage{titlesec}
\usepackage{kbordermatrix}


\newtcolorbox{greybox}{
  colback=white,
  colframe=blue,
  coltext=black,
  boxsep=5pt,
  arc=4pt}
  
\newcommand{\sectionbreak}{\clearpage}

 
\newcommand{\ds}[4]{\sum_{{#1}=1}^{#3}\sum_{{#2}=1}^{#4}}
\newcommand{\us}[3]{\mathop{\sum\sum}_{1\leq{#2}<{#1}\leq{#3}}}

\newcommand{\ol}[1]{\overline{#1}}
\newcommand{\ul}[1]{\underline{#1}}

\newcommand{\amin}[1]{\mathop{\text{argmin}}_{#1}}
\newcommand{\amax}[1]{\mathop{\text{argmax}}_{#1}}

\newcommand{\ci}{\perp\!\!\!\perp}

\newcommand{\mc}[1]{\mathcal{#1}}
\newcommand{\mb}[1]{\mathbb{#1}}
\newcommand{\mf}[1]{\mathfrak{#1}}

\newcommand{\eps}{\epsilon}
\newcommand{\lbd}{\lambda}
\newcommand{\alp}{\alpha}
\newcommand{\df}{=:}
\newcommand{\am}[1]{\mathop{\text{argmin}}_{#1}}
\newcommand{\ls}[2]{\mathop{\sum\sum}_{#1}^{#2}}
\newcommand{\ijs}{\mathop{\sum\sum}_{1\leq i<j\leq n}}
\newcommand{\jis}{\mathop{\sum\sum}_{1\leq j<i\leq n}}
\newcommand{\sij}{\sum_{i=1}^n\sum_{j=1}^n}
	
\ifLuaTeX
  \usepackage{selnolig}  % disable illegal ligatures
\fi
\usepackage{bookmark}
\IfFileExists{xurl.sty}{\usepackage{xurl}}{} % add URL line breaks if available
\urlstyle{same}
\hypersetup{
  pdfauthor={Jan de Leeuw - University of California Los Angeles},
  hidelinks,
  pdfcreator={LaTeX via pandoc}}

\title{Smacof at 50: A Manual\\
Part 5: Unfolding in Smacof}
\author{Jan de Leeuw - University of California Los Angeles}
\date{Started December 12 2022, Version of April 29, 2024}

\begin{document}
\maketitle
\begin{abstract}
TBD
\end{abstract}

{
\setcounter{tocdepth}{3}
\tableofcontents
}
\textbf{Note:} This is a working manuscript which will be expanded/updated
frequently. All suggestions for improvement are welcome. All Rmd, tex,
html, pdf, R, and C files are in the public domain. Attribution will be
appreciated, but is not required. The files can be found at
\url{https://github.com/deleeuw} in the repositories smacofCode, smacofManual,
and smacofExamples.

\section{Introduction}\label{introduction}

In Multidimensional Unfolding (MDU) the objects of an MDS problem are partitioned into two sets.
There is a set of \(n\) row-objects and a set of
\(m\) column-objects, and a corresponding \(n\times p\) row-configuration \(X\) and \(m\times p\) column-configuration
\(Y\). We minimize stress defined as
\begin{equation}
\sigma(X,Y):=\sum_{i=1}^{n}\sum_{j=1}^{n}w_{ij}(\delta_{ij}-d(x_i,y_j)^2.
\label{eq:ufstress}
\end{equation}
over both \(X\) and \(Y\). Here
\begin{equation}
d(x_i,y_j):=\sqrt{(x_i-y_j)'(x_i-y_j)}
\label{eq:ufdist}
\end{equation}
Thus the within-set
dissimilarities are missing, or ignored even if they are available, and only the between-set dissimilarities are fitted by between-set distances.

If we define \(Z\) as
\begin{equation}
Z:=
\kbordermatrix{
\mbox{\ }&p\\
n&X\\
m&Y
}
\label{eq:ufzdef}
\end{equation}
and \(U\) as
\begin{equation}
U:=
\kbordermatrix{
&n&m\\
n&0&W\\
m&W'&0
}
\label{eq:ufudef}
\end{equation}
then we can also write
\begin{equation}
\sigma(Z)=\sum_{i=1}^{n+m}\sum_{j=1}^{n+m}u_{ij}(\delta_{ij}-d_{ij}(Z))^2.
\label{eq:ufzstress}
\end{equation}

Data Preferences Type A and Type B Conditional

The unfolding model for preference judgments is often attributed to Clyde H. Coombs (1950),
with further developments by Coombs and his co-workers reviewed in C. H. Coombs (1964).
After this path-breaking work the digital computer took over, and minimization of loss function
\eqref{eq:ufstress} and its variations was started by
Roskam (1968) and Kruskal and Carroll (1969).

In this manual we are not interested in MDU as a psychological theory, as a model for
preference judgments. We merely are interested in mapping off-diagonal dissimilarity
relations into low-dimensional Euclidean space, i.e.~in making a picture of the
data. In some cases (distance completion, distances with errors, spatial basis)

Challenges: degeneracy

\section{Loss function}\label{loss-function}

\subsection{Metric}\label{metric}

\subsection{Non-linear}\label{non-linear}

\subsection{Non-metric}\label{non-metric}

\subsection{Constraints}\label{constraints}

\section{smacofUF}\label{smacofuf}

\subsection{Initial Configuration}\label{initial-configuration}

\[
\sigma(C)=\sigma(\tilde C+(C-\tilde C))=\sum_{i=1}^n\sum_{j=1}^m((\delta_{ij}^2-\text{tr}\ A_{ij}\tilde C)-\text{tr}\ A_{ij}(C-\tilde C))^2
\]
\[
\sigma(C)=\sigma(\tilde C)-2\sum_{i=1}^n\sum_{j=1}^m(\delta_{ij}^2-\text{tr}\ A_{ij}\tilde C)\text{tr}\ A_{ij}(C-\tilde C)+\sum_{i=1}^n\sum_{j=1}^m\{\text{tr}\ A_{ij}(C-\tilde C)\}^2
\]

From De Leeuw, Groenen, and Pietersz (2006)
\[
\sum_{i=1}^n\sum_{j=1}^m\{\text{tr}\ A_{ij}(C-\tilde C)\}^2\leq (n+m+2)\text{tr}\ (C-\tilde C)^2
\]
Define
\[
B(\tilde C):=\frac{1}{n+m+2}\sum_{i=1}^n\sum_{j=1}^m(\delta_{ij}^2-\text{tr}\ A_{ij}\tilde C)A_{ij}
\]
So we minimize
\[
-2\ \text{tr}\ B(\tilde C)C+\text{tr}\ C^2-2\text{tr}\ C\tilde C=\\
\text{tr}\ (C-\{\tilde C + B(\tilde C)\})^2
\]

\subsection{Constraints}\label{constraints-1}

\subsubsection{Centroid Restriction}\label{centroid-restriction}

\[
Z=\begin{bmatrix}
X\\Y
\end{bmatrix}=
\begin{bmatrix}
I\\
D^{-1}G'
\end{bmatrix}X=HX
\]
Minimize \(\text{tr}\ (\overline{Z}-HX)'V(\overline{Z}-HX)\). If there are
no further restrictions on \(X\) the minimum is attained at \(\hat X=(H'VH)^+H'V\overline{Z}\).
Otherwise write \(X=\hat X+(X-\hat X)\). Then
\[
\text{tr}\ (\overline{Z}-H\hat X-H(X-\hat X))'V(\overline{Z}-H\hat X-H(X-\hat X))=\\
\text{tr}\ (\overline{Z}-H\hat X)'V(\overline{Z}-H\hat X)+
\text{tr}\ (X-\hat X)'H'VH(X-\hat X)
\]
and we must minimize \(\text{tr}\ (X-\hat X)'H'VH(X-\hat X)\) for example over \(X'H'VHX=I\).
That is maximizing \(H'VH\hat X=H'VHXM\) with \(M\) a symmetric matrix of Lagrange multipliers.
Thus \(M^2=\hat X'H'VH\hat X\) and \(X=\hat X(\hat X'H'VH\hat X)^{-\frac12}\).

Over \(\text{tr}\ X'H'VHX=1\) we get \(H'VH(X-\hat X)=\lambda H'VHX\) or
\(X=(1-\lambda)H'VHX=H'VH\hat X\), wich means \(X\) is proportional to \(\hat X\) and we just have to normalize \(\hat X\) to find \(X\).

The constraint \(X'V_{11}X=I\) is more difficult to deal with
\[
\text{tr}\ (X-\hat X)'H'VH(X-\hat X)=
\]

Oblique Procrustus

\subsubsection{Linearly Restricted Unfolding}\label{linearly-restricted-unfolding}

\paragraph{External Unfolding}\label{external-unfolding}

\[
\begin{bmatrix}
X\\Y
\end{bmatrix}=
\begin{bmatrix}
R&0\\
0&S
\end{bmatrix}
\begin{bmatrix}
P\\
Q
\end{bmatrix}
\]
\#\#\# Normalization Restrictions

\(X'V_{11}X=I\).

\subsubsection{Rank-one Restriction}\label{rank-one-restriction}

\(Y=za'\)

Minimize
\[
2\ \text{tr}\ (za'-\overline{Y})'V_{21}(X-\overline{X})+
\text{tr}\ (za'-\overline{Y})'V_{22}(za'-\overline{Y})
\]
Removing irrelevant terms
\[
2\ z'\{V_{21}(X-\overline{X})-V_{22}\overline{Y}\}a+
a'a.z'V_{22}z
\]
Let \(H=-V_{21}(X-\overline{X})-V_{22}\overline{Y}\). Minimize using \(z'V_{22}z=1\).
\[
a=H'z=\{\overline{Y}'V_{22}-(X-\overline{X})'V_{12})\}z
\]

\section{Indicator Matrices}\label{indicator-matrices}

Suppose \(G_1,\cdots,G_m\) are indicator matrices, all with \(n\) rows, but with
\(k_1,\cdots,k_m\) columns. Their elements are \(g_{il}^j\) with \(i=1,\cdots,n\)
and \(l=1,\cdots,k_j\). Thus \(g_{il}^j\) is either zero or one, and
\[
\sum_{l=1}^{k_j}g_{il}^j=1
\]
for all \(i=1,\cdots,n\) and \(j=1,\cdots,m\).

Now define the smacof loss function
\[
\sigma(X,Y_1,\cdots,Y_m)=\sum_{j=1}^m\sum_{i=1}^n\sum_{l=1}^{k_j}w_{il}^j(\delta_{il}^j-d(x_i,y^j_l))^2
\]
with \(w_{il}^j=g_{il}^j\) and \(\delta_{il}^j=1-g_{il}^j\). This means we are
aiming for \(d(x_i,y^j_l)=0\) whenever \(g_{il}^j=1\) and we do not care what
\(d(x_i,y^j_l)\) is when \(g_{il}^j=0\).

\section{Examples}\label{examples}

\subsection{Roskam}\label{roskam}

\subsection{Breakfast}\label{breakfast}

\subsection{Gold}\label{gold}

\subsection{Indicator matrix / Matrices}\label{indicator-matrix-matrices}

\section*{References}\label{references}
\addcontentsline{toc}{section}{References}

\phantomsection\label{refs}
\begin{CSLReferences}{1}{0}
\bibitem[\citeproctext]{ref-coombs_64}
Coombs, C. H. 1964. \emph{{A Theory of Data}}. Wiley.

\bibitem[\citeproctext]{ref-coombs_50}
Coombs, Clyde H. 1950. {``Psychological Scaling Without a Unit of Measurement.''} \emph{Psychological Review} 57: 148--58.

\bibitem[\citeproctext]{ref-deleeuw_groenen_pietersz_U_06}
De Leeuw, J., P. J. F. Groenen, and R. Pietersz. 2006. {``{Optimizing Functions of Squared Distances}.''} UCLA Department of Statistics. \url{https://jansweb.netlify.app/publication/deleeuw-groenen-pietersz-u-06/deleeuw-groenen-pietersz-u-06.pdf}.

\bibitem[\citeproctext]{ref-kruskal_carroll_69}
Kruskal, J. B., and J. D. Carroll. 1969. {``{Geometrical Models and Badness of Fit Functions}.''} In \emph{Multivariate Analysis, Volume II}, edited by P. R. Krishnaiah, 639--71. North Holland Publishing Company.

\bibitem[\citeproctext]{ref-roskam_68}
Roskam, E. E. 1968. {``{Metric Analysis of Ordinal Data in Psychology}.''} PhD thesis, University of Leiden.

\end{CSLReferences}

\end{document}
