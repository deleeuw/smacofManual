% Options for packages loaded elsewhere
\PassOptionsToPackage{unicode}{hyperref}
\PassOptionsToPackage{hyphens}{url}
%
\documentclass[
  12pt,
]{article}
\usepackage{amsmath,amssymb}
\usepackage{iftex}
\ifPDFTeX
  \usepackage[T1]{fontenc}
  \usepackage[utf8]{inputenc}
  \usepackage{textcomp} % provide euro and other symbols
\else % if luatex or xetex
  \usepackage{unicode-math} % this also loads fontspec
  \defaultfontfeatures{Scale=MatchLowercase}
  \defaultfontfeatures[\rmfamily]{Ligatures=TeX,Scale=1}
\fi
\usepackage{lmodern}
\ifPDFTeX\else
  % xetex/luatex font selection
    \setmainfont[]{Times New Roman}
\fi
% Use upquote if available, for straight quotes in verbatim environments
\IfFileExists{upquote.sty}{\usepackage{upquote}}{}
\IfFileExists{microtype.sty}{% use microtype if available
  \usepackage[]{microtype}
  \UseMicrotypeSet[protrusion]{basicmath} % disable protrusion for tt fonts
}{}
\makeatletter
\@ifundefined{KOMAClassName}{% if non-KOMA class
  \IfFileExists{parskip.sty}{%
    \usepackage{parskip}
  }{% else
    \setlength{\parindent}{0pt}
    \setlength{\parskip}{6pt plus 2pt minus 1pt}}
}{% if KOMA class
  \KOMAoptions{parskip=half}}
\makeatother
\usepackage{xcolor}
\usepackage[margin=1in]{geometry}
\usepackage{longtable,booktabs,array}
\usepackage{calc} % for calculating minipage widths
% Correct order of tables after \paragraph or \subparagraph
\usepackage{etoolbox}
\makeatletter
\patchcmd\longtable{\par}{\if@noskipsec\mbox{}\fi\par}{}{}
\makeatother
% Allow footnotes in longtable head/foot
\IfFileExists{footnotehyper.sty}{\usepackage{footnotehyper}}{\usepackage{footnote}}
\makesavenoteenv{longtable}
\usepackage{graphicx}
\makeatletter
\newsavebox\pandoc@box
\newcommand*\pandocbounded[1]{% scales image to fit in text height/width
  \sbox\pandoc@box{#1}%
  \Gscale@div\@tempa{\textheight}{\dimexpr\ht\pandoc@box+\dp\pandoc@box\relax}%
  \Gscale@div\@tempb{\linewidth}{\wd\pandoc@box}%
  \ifdim\@tempb\p@<\@tempa\p@\let\@tempa\@tempb\fi% select the smaller of both
  \ifdim\@tempa\p@<\p@\scalebox{\@tempa}{\usebox\pandoc@box}%
  \else\usebox{\pandoc@box}%
  \fi%
}
% Set default figure placement to htbp
\def\fps@figure{htbp}
\makeatother
\setlength{\emergencystretch}{3em} % prevent overfull lines
\providecommand{\tightlist}{%
  \setlength{\itemsep}{0pt}\setlength{\parskip}{0pt}}
\setcounter{secnumdepth}{5}
% definitions for citeproc citations
\NewDocumentCommand\citeproctext{}{}
\NewDocumentCommand\citeproc{mm}{%
  \begingroup\def\citeproctext{#2}\cite{#1}\endgroup}
\makeatletter
 % allow citations to break across lines
 \let\@cite@ofmt\@firstofone
 % avoid brackets around text for \cite:
 \def\@biblabel#1{}
 \def\@cite#1#2{{#1\if@tempswa , #2\fi}}
\makeatother
\newlength{\cslhangindent}
\setlength{\cslhangindent}{1.5em}
\newlength{\csllabelwidth}
\setlength{\csllabelwidth}{3em}
\newenvironment{CSLReferences}[2] % #1 hanging-indent, #2 entry-spacing
 {\begin{list}{}{%
  \setlength{\itemindent}{0pt}
  \setlength{\leftmargin}{0pt}
  \setlength{\parsep}{0pt}
  % turn on hanging indent if param 1 is 1
  \ifodd #1
   \setlength{\leftmargin}{\cslhangindent}
   \setlength{\itemindent}{-1\cslhangindent}
  \fi
  % set entry spacing
  \setlength{\itemsep}{#2\baselineskip}}}
 {\end{list}}
\usepackage{calc}
\newcommand{\CSLBlock}[1]{\hfill\break\parbox[t]{\linewidth}{\strut\ignorespaces#1\strut}}
\newcommand{\CSLLeftMargin}[1]{\parbox[t]{\csllabelwidth}{\strut#1\strut}}
\newcommand{\CSLRightInline}[1]{\parbox[t]{\linewidth - \csllabelwidth}{\strut#1\strut}}
\newcommand{\CSLIndent}[1]{\hspace{\cslhangindent}#1}
\usepackage{tcolorbox}
\usepackage{amssymb}
\usepackage{yfonts}
\usepackage{bm}
\usepackage{titlesec}


\newtcolorbox{greybox}{
  colback=white,
  colframe=blue,
  coltext=black,
  boxsep=5pt,
  arc=4pt}
  
 
\newcommand{\ds}[4]{\sum_{{#1}=1}^{#3}\sum_{{#2}=1}^{#4}}
\newcommand{\us}[3]{\mathop{\sum\sum}_{1\leq{#2}<{#1}\leq{#3}}}

\newcommand{\ol}[1]{\overline{#1}}
\newcommand{\ul}[1]{\underline{#1}}

\newcommand{\amin}[1]{\mathop{\text{argmin}}_{#1}}
\newcommand{\amax}[1]{\mathop{\text{argmax}}_{#1}}

\newcommand{\ci}{\perp\!\!\!\perp}

\newcommand{\mc}[1]{\mathcal{#1}}
\newcommand{\mb}[1]{\mathbb{#1}}
\newcommand{\mf}[1]{\mathfrak{#1}}

\newcommand{\eps}{\epsilon}
\newcommand{\lbd}{\lambda}
\newcommand{\alp}{\alpha}
\newcommand{\df}{=:}
\newcommand{\am}[1]{\mathop{\text{argmin}}_{#1}}
\newcommand{\ls}[2]{\mathop{\sum\sum}_{#1}^{#2}}
\newcommand{\ijs}{\mathop{\sum\sum}_{1\leq i<j\leq n}}
\newcommand{\jis}{\mathop{\sum\sum}_{1\leq j<i\leq n}}
\newcommand{\sij}{\sum_{i=1}^n\sum_{j=1}^n}

\newcommand{\sectionbreak}{\pagebreak}
	
\usepackage{bookmark}
\IfFileExists{xurl.sty}{\usepackage{xurl}}{} % add URL line breaks if available
\urlstyle{same}
\hypersetup{
  pdftitle={Accelerated SMACOF Multidimensional Scaling},
  pdfauthor={Jan de Leeuw - University of California Los Angeles},
  hidelinks,
  pdfcreator={LaTeX via pandoc}}

\title{Accelerated SMACOF Multidimensional Scaling}
\author{Jan de Leeuw - University of California Los Angeles}
\date{Started July 23 2024, Version of August 02, 2024}

\begin{document}
\maketitle
\begin{abstract}
TBD
\end{abstract}

{
\setcounter{tocdepth}{3}
\tableofcontents
}
\textbf{Note:} This is a working paper which will be expanded/updated frequently. All suggestions for improvement are welcome.

\section{Introduction}\label{introduction}

In this paper we study minimization of the loss function
\[
\sigma(X):=\frac12\sum_{i=1}^n\sum_{j=1}^n w_{ij}(\delta_{ij}-d_{ij}(X))^2
\]
over all \(n\times p\) \emph{configuration} matrices \(X\). Here \(W=\{w_{ij}\}\) and \(\Delta=\{\delta_{ij}\}\) are known non-negative, symmetric, and hollow matrices of \emph{weights} and \emph{dissimilarities} and \(D(X)=\{d_{ij}(X)\}\) is the matrix of \emph{Euclidean distances} between the rows of \(X\).

Define
\[
\rho(X):=\sum_{i=1}^n\sum_{j=1}^n w_{ij}\delta_{ij}d_{ij}(X)=\text{tr}\ X'B(X)X,
\]
where
\[
B(X):=\sum_{i=1}^n\sum_{j=1}^nw_{ij}\frac{\delta_{ij}}{d_{ij}(X)}A_{ij}.
\]
Aso define
\[
\eta^2(X):=\sum_{i=1}^n\sum_{j=1}^nw_{ij}d_{ij}^2(X)=\text{tr}\ X'VX,
\]
where
\[
V:=\sum_{i=1}^n\sum_{j=1}^n w_{ij}A_{ij}.
\]
Thus
\[
\sigma(X)=1-\rho(X)+\frac12\eta^2(X)
\]
Both \(\rho\) and \(\eta\) are homogeneous convex functions.

\(\rho(X)=0\) iff \(d_{ij}(X)=0\) for all \((i,j)\) for which \(w_{ij}\delta_{ij}>0\).

\section{One-point Methods}\label{one-point-methods}

\subsection{Basic Iteration}\label{basic-iteration}

The \emph{Guttman transform} of a configuration \(X\), named after Guttman (1968), is defined as
\[
\Phi(X)=V^+B(X)X,
\]
with \(V^+\) the Moore-Penrose inverse of \(V\). If \(X=\Phi(X)\), i.e.~if \(X\) is
fixed point of \(\Phi\), then \(VX-B(X)0\), which we can also write as
\(\mathcal{D}\sigma(X)=0\). Thus \(X\) is a fixed point of \(\Phi\) if and only
if the gradient of \(\sigma\) vanishes at \(X\), i.e.~if and only if \(X\) is a
stationary point if \(\sigma\).

We have to be somewhat careful here. Subdifferential
\[
\partial d_{ij}(X)=\frac{1}{d_{ij}(X)}A_{ij}X
\]
\[
\partial d_{ij}(X)=A_{ij}Z
\]

Using the Guttman transform we can derive the basic smacof equality
\[
\sigma(X)=1+\eta^2(X-\Phi(X))-\eta^2(\Phi(X))
\]
for all \(X\) and the inequality
\[
\sigma(X)\leq 1+\eta^2(X-\Phi(Y))-\eta^2(\Phi(Y))
\]
for all \(X\) and \(Y\).

Taken together ,, and ,,, imply the sandwich inequality
\[
\sigma(\Phi(Y))\leq 1-\eta^2(\Phi(Y))\leq 1+\eta^2(Y-\Phi(Y))-\eta^2(\Phi(Y))=\sigma(Y)
\]
If \(Y\) is not a fixed point of \(\Phi\) then the second inequality in the
chain is strict and thus \(\sigma(\Phi(Y))<\sigma(Y)\). It also follows
from \ldots{} that \(\eta^2(\Phi(Y))\leq 1\).

Algorithm

\[
\mathcal{D}\Phi_X(H)=V^+\sum w_{ij}\frac{\delta_{ij}}{d_{ij}(X)}\left\{A_{ij}H-\frac{\text{tr}\ X'A_{ij}H}{ \text{tr}\ X'A_{ij}X}A_{ij}X\right\}
\]
Thus \(\mathcal{D}\Phi_X(X)=0\) for all \(X\). If \(X\) is a fixed point and \(S\) is anti-symmetric \(\mathcal{D}\Phi_X(XS)=V^+B(X)XS=XS\),
which means \(\mathcal{D}\Phi_X\) has \(\frac12p(p-1)\) eigenvalues equal to one.
At a fixed point all eigenvalues are between zero and one.

\[
\mathcal{D}^2\rho_X(G,H)=\sum w_{ij}\frac{\delta_{ij}}{d_{ij}(X)}\left\{\text{tr}\ G'A_{ij}H-\frac{\text{tr}\ H'A_{ij}X\ \text{tr}\ G'A_{ij}X}{ d_{ij}^2(X)}\right\}
\]

\begin{verbatim}
## itel  57 sold 2.1114112739 snew 2.1114112739 chng 0.0000000000 labd 0.7669812392
\end{verbatim}

stress is 2.1114112739076

\begin{verbatim}
##       [,1]           
##  [1,]   +1.0000000000
##  [2,]   +0.7669965027
##  [3,]   +0.7480939418
##  [4,]   +0.7185926293
##  [5,]   +0.7007452300
##  [6,]   +0.6920114811
##  [7,]   +0.6859492532
##  [8,]   +0.6593334523
##  [9,]   +0.6541779410
## [10,]   +0.6477573342
## [11,]   +0.6237683212
## [12,]   +0.6178713315
## [13,]   +0.5735285948
## [14,]   +0.5483330654
## [15,]   +0.5260355535
## [16,]   +0.5112510731
## [17,]   +0.5064703617
## [18,]   +0.5059294793
## [19,]   +0.4919752629
## [20,]   +0.4827646549
## [21,]   +0.4782034983
## [22,]   +0.4757907684
## [23,]   +0.4682965897
## [24,]   +0.4619226490
## [25,]   +0.4559704883
## [26,]   +0.0000000000
## [27,]   +0.0000000000
## [28,]   -0.0000000000
\end{verbatim}

\[
\mathcal{D}\Phi_X(H)=V^+\sum w_{ij}\frac{\delta_{ij}}{d_{ij}(X)}\left\{A_{ij}H-\frac{\text{tr}\ X'A_{ij}H}{ \text{tr}\ X'A_{ij}X}A_{ij}X\right\}
\]
Thus \(\mathcal{D}\Phi_X(X)=0\) for all \(X\). If \(X\) is a fixed point and \(S\) is anti-symmetric \(\mathcal{D}\Phi_X(XS)=V^+B(X)XS=XS\),
which means \(\mathcal{D}\Phi_X\) has \(\frac12p(p-1)\) eigenvalues equal to one.
At a fixed point all eigenvalues are between zero and one.

\[
\mathcal{D}^2\rho_X(G,H)=\sum w_{ij}\frac{\delta_{ij}}{d_{ij}(X)}\left\{\text{tr}\ G'A_{ij}H-\frac{\text{tr}\ H'A_{ij}X\ \text{tr}\ G'A_{ij}X}{ d_{ij}^2(X)}\right\}
\]

\subsection{Rotated Basic Iteration}\label{rotated-basic-iteration}

\begin{verbatim}
## itel  54 sold 2.1114112739 snew 2.1114112739 chng 0.0000000000 labd 0.7669843896
\end{verbatim}

stress is 2.1114112739076

\begin{verbatim}
##       [,1]           
##  [1,]   +0.7669964894
##  [2,]   +0.7480939420
##  [3,]   +0.7185926297
##  [4,]   +0.7007452335
##  [5,]   +0.6920114817
##  [6,]   +0.6859492534
##  [7,]   +0.6593334543
##  [8,]   +0.6541779412
##  [9,]   +0.6477573345
## [10,]   +0.6237683217
## [11,]   +0.6178713316
## [12,]   +0.5735285959
## [13,]   +0.5483330651
## [14,]   +0.5260355534
## [15,]   +0.5112510730
## [16,]   +0.5064703618
## [17,]   +0.5059294793
## [18,]   +0.4919752632
## [19,]   +0.4827646574
## [20,]   +0.4782035029
## [21,]   +0.4757907649
## [22,]   +0.4682965885
## [23,]   +0.4619226493
## [24,]   +0.4559704884
## [25,]   -0.0000000000
## [26,]   -0.0000000000
## [27,]   +0.0000000000
## [28,]   +0.0000000000
\end{verbatim}

\begin{verbatim}
## itel  56 sold 2.1114112739 snew 2.1114112739 chng 0.0000000000 labd 0.7669940008
\end{verbatim}

stress is 2.1114112739076

\begin{verbatim}
##       [,1]           
##  [1,]   +0.7669964993
##  [2,]   +0.7480939419
##  [3,]   +0.7185926294
##  [4,]   +0.7007452309
##  [5,]   +0.6920114813
##  [6,]   +0.6859492533
##  [7,]   +0.6593334529
##  [8,]   +0.6541779410
##  [9,]   +0.6477573343
## [10,]   +0.6237683213
## [11,]   +0.6178713317
## [12,]   +0.5735285948
## [13,]   +0.5483330653
## [14,]   +0.5260355535
## [15,]   +0.5112510731
## [16,]   +0.5064703617
## [17,]   +0.5059294792
## [18,]   +0.4919752629
## [19,]   +0.4827646550
## [20,]   +0.4782034999
## [21,]   +0.4757907672
## [22,]   +0.4682965894
## [23,]   +0.4619226495
## [24,]   +0.4559704888
## [25,]   -0.0000000006
## [26,]   -0.0000000000
## [27,]   -0.0000000000
## [28,]   +0.0000000000
\end{verbatim}

\section{Two Point Iteration}\label{two-point-iteration}

De Leeuw and Heiser (1980) suggested the update
\[
\Psi(X)=2\Phi(X)-X
\]
The reasoning here is two-fold. The smacof inequality says
\[
\sigma(X)\leq 1+\eta^2(X-\Phi(Y))-\eta^2(\Phi(Y))
\]
If \(X=\alpha\Phi(Y)+(1-\alpha)Y\) then this becomes
\[
\sigma(\alpha\Phi(Y)+(1-\alpha)Y)\leq 1+(1-\alpha)^2\eta^2(Y-\Phi(Y))-\eta^2(\Phi(Y))
\]
If \((1-\alpha)^2\leq 1\) then
\[
1+(1-\alpha)^2\eta^2(Y-\Phi(Y))-\eta^2(\Phi(Y))\leq 1+\eta^2(Y-\Phi(Y))-\eta^2(\Phi(Y))=\sigma(Y)
\]
Thus updating with \(X^{(k+1)}=\alpha\Phi(X^{(k)})+(1-\alpha)X^{(k)}\) is a stable
algorithm as long as \(0\leq\alpha\leq 2\).

It turns out that applying the relaxed update
\[
X^{(k+1)}=2V^+B(X^{(k)})X^{(k)}-X^{(k)}
\]
has some unintended consequences.

\begin{verbatim}
## itel  23 sold 3.9946270666 snew 3.9946270666 chng 3.7664315853 labd 1.0000000000
\end{verbatim}

stress is 2.1114112739076

We see that \(\eta^2(X^{(k+1)}-X^{(k)})\) does not converge to zero, and that
\(\sigma_k\) converges to a value which does not even correspond to a local minimum
of \(\sigma\).

\begin{verbatim}
##       [,1]           
##  [1,]   -1.0000000000
##  [2,]   +1.0000000000
##  [3,]   -1.0000000000
##  [4,]   -1.0000000000
##  [5,]   +0.5339929781
##  [6,]   +0.4961878839
##  [7,]   +0.4371852597
##  [8,]   +0.4014904677
##  [9,]   +0.3840229636
## [10,]   +0.3718985068
## [11,]   +0.3186669088
## [12,]   +0.3083558824
## [13,]   +0.2955146690
## [14,]   +0.2475366434
## [15,]   +0.2357426633
## [16,]   +0.1470571918
## [17,]   +0.0966661301
## [18,]   -0.0880590224
## [19,]   -0.0761547015
## [20,]   -0.0634068231
## [21,]   +0.0520711068
## [22,]   -0.0484184707
## [23,]   -0.0435929937
## [24,]   -0.0344706856
## [25,]   +0.0225021460
## [26,]   -0.0160494738
## [27,]   +0.0129407235
## [28,]   +0.0118589584
\end{verbatim}

A more thorough analysis of the results show that the method produces a sequence
\(X^{(k)}\) with two subseqences. If \(\overline{X}\) is a fixed point of
\(\Phi\) then there is a \(\tau>0\) such that
the subsequence with \(k\) even converges to \(\tau\overline{X}\)
while the subsequence with \(k\) odd converges to \((2-\tau)\overline{X}\).

\begin{verbatim}
## itel  18 sold 3.9946270666 snew 3.9946270666 chng 0.0000000000 labd 0.2737973829
\end{verbatim}

stress is 2.1114112739076

\begin{verbatim}
##       [,1]           
##  [1,]   +1.0000000000
##  [2,]   -1.0000000000
##  [3,]   -1.0000000000
##  [4,]   -0.9999999999
##  [5,]   +0.5339930271
##  [6,]   +0.4961878833
##  [7,]   +0.4371852579
##  [8,]   +0.4014904541
##  [9,]   +0.3840229612
## [10,]   +0.3718985063
## [11,]   +0.3186669016
## [12,]   +0.3083558816
## [13,]   +0.2955146680
## [14,]   +0.2475366415
## [15,]   +0.2357426629
## [16,]   +0.1470571877
## [17,]   +0.0966661315
## [18,]   -0.0880590242
## [19,]   -0.0761547024
## [20,]   -0.0634068192
## [21,]   +0.0520711070
## [22,]   -0.0484184575
## [23,]   -0.0435930109
## [24,]   -0.0344706938
## [25,]   +0.0225021463
## [26,]   -0.0160494743
## [27,]   +0.0129407234
## [28,]   +0.0118589585
\end{verbatim}

\begin{verbatim}
## Warning in microbenchmark(smacofAccelerate(delta, ndim = 2, opt = 1, halt = 2,
## : less accurate nanosecond times to avoid potential integer overflows
\end{verbatim}

\begin{verbatim}
## Unit: milliseconds
##                                                                   expr      min
##  smacofAccelerate(delta, ndim = 2, opt = 1, halt = 2, verbose = FALSE) 3.188775
##  smacofAccelerate(delta, ndim = 2, opt = 2, halt = 2, verbose = FALSE) 3.151875
##  smacofAccelerate(delta, ndim = 2, opt = 3, halt = 2, verbose = FALSE) 3.847276
##  smacofAccelerate(delta, ndim = 2, opt = 4, halt = 2, verbose = FALSE) 1.421019
##  smacofAccelerate(delta, ndim = 2, opt = 5, halt = 2, verbose = FALSE) 1.564027
##  smacofAccelerate(delta, ndim = 2, opt = 6, halt = 2, verbose = FALSE) 1.639467
##  smacofAccelerate(delta, ndim = 2, opt = 7, halt = 2, verbose = FALSE) 1.526307
##        lq     mean   median       uq      max neval
##  3.304825 3.602914 3.358987 3.476656 6.070009   100
##  3.241993 3.521317 3.294924 3.374894 5.645618   100
##  3.939567 4.335232 4.029644 4.215682 6.634046   100
##  1.466795 1.601369 1.498222 1.541600 3.919846   100
##  1.618536 1.762409 1.650004 1.712775 3.995286   100
##  1.692829 1.827641 1.720585 1.775444 6.503174   100
##  1.578336 1.718087 1.618106 1.683132 6.439214   100
\end{verbatim}

\begin{verbatim}
## Unit: milliseconds
##                                                                   expr      min
##  smacofAccelerate(delta, ndim = 2, opt = 1, halt = 2, verbose = FALSE) 43.06710
##  smacofAccelerate(delta, ndim = 2, opt = 2, halt = 2, verbose = FALSE) 44.09382
##  smacofAccelerate(delta, ndim = 2, opt = 3, halt = 2, verbose = FALSE) 54.69146
##  smacofAccelerate(delta, ndim = 2, opt = 4, halt = 2, verbose = FALSE) 20.16495
##  smacofAccelerate(delta, ndim = 2, opt = 5, halt = 2, verbose = FALSE) 14.30547
##  smacofAccelerate(delta, ndim = 2, opt = 6, halt = 2, verbose = FALSE) 22.89670
##  smacofAccelerate(delta, ndim = 2, opt = 7, halt = 2, verbose = FALSE) 20.67150
##        lq     mean   median       uq      max neval
##  44.02676 45.24634 44.87503 45.65168 66.89129   100
##  45.18950 46.15677 46.23545 46.68055 50.64677   100
##  56.24936 56.93894 56.64076 57.10158 74.41721   100
##  20.85358 22.01609 22.03133 22.24851 40.76454   100
##  14.63911 15.66496 14.95206 16.28350 34.73516   100
##  24.80289 25.22137 25.04815 25.37377 44.09050   100
##  22.44541 22.52770 22.71785 22.93573 24.76105   100
\end{verbatim}

\section*{References}\label{references}
\addcontentsline{toc}{section}{References}

\phantomsection\label{refs}
\begin{CSLReferences}{1}{0}
\bibitem[\citeproctext]{ref-deleeuw_heiser_C_80}
De Leeuw, J., and W. J. Heiser. 1980. {``Multidimensional Scaling with Restrictions on the Configuration.''} In \emph{Multivariate Analysis, Volume {V}}, edited by P. R. Krishnaiah, 501--22. Amsterdam, The Netherlands: North Holland Publishing Company.

\bibitem[\citeproctext]{ref-guttman_68}
Guttman, L. 1968. {``{A General Nonmetric Technique for Fitting the Smallest Coordinate Space for a Configuration of Points}.''} \emph{Psychometrika} 33: 469--506.

\end{CSLReferences}

\end{document}
