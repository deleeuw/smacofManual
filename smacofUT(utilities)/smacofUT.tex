% Options for packages loaded elsewhere
\PassOptionsToPackage{unicode}{hyperref}
\PassOptionsToPackage{hyphens}{url}
\PassOptionsToPackage{dvipsnames,svgnames,x11names}{xcolor}
%
\documentclass[
  12pt,
  letterpaper,
  DIV=11,
  numbers=noendperiod]{scrartcl}

\usepackage{amsmath,amssymb}
\usepackage{iftex}
\ifPDFTeX
  \usepackage[T1]{fontenc}
  \usepackage[utf8]{inputenc}
  \usepackage{textcomp} % provide euro and other symbols
\else % if luatex or xetex
  \usepackage{unicode-math}
  \defaultfontfeatures{Scale=MatchLowercase}
  \defaultfontfeatures[\rmfamily]{Ligatures=TeX,Scale=1}
\fi
\usepackage{lmodern}
\ifPDFTeX\else  
    % xetex/luatex font selection
    \setmainfont[]{Times New Roman}
\fi
% Use upquote if available, for straight quotes in verbatim environments
\IfFileExists{upquote.sty}{\usepackage{upquote}}{}
\IfFileExists{microtype.sty}{% use microtype if available
  \usepackage[]{microtype}
  \UseMicrotypeSet[protrusion]{basicmath} % disable protrusion for tt fonts
}{}
\makeatletter
\@ifundefined{KOMAClassName}{% if non-KOMA class
  \IfFileExists{parskip.sty}{%
    \usepackage{parskip}
  }{% else
    \setlength{\parindent}{0pt}
    \setlength{\parskip}{6pt plus 2pt minus 1pt}}
}{% if KOMA class
  \KOMAoptions{parskip=half}}
\makeatother
\usepackage{xcolor}
\setlength{\emergencystretch}{3em} % prevent overfull lines
\setcounter{secnumdepth}{5}
% Make \paragraph and \subparagraph free-standing
\makeatletter
\ifx\paragraph\undefined\else
  \let\oldparagraph\paragraph
  \renewcommand{\paragraph}{
    \@ifstar
      \xxxParagraphStar
      \xxxParagraphNoStar
  }
  \newcommand{\xxxParagraphStar}[1]{\oldparagraph*{#1}\mbox{}}
  \newcommand{\xxxParagraphNoStar}[1]{\oldparagraph{#1}\mbox{}}
\fi
\ifx\subparagraph\undefined\else
  \let\oldsubparagraph\subparagraph
  \renewcommand{\subparagraph}{
    \@ifstar
      \xxxSubParagraphStar
      \xxxSubParagraphNoStar
  }
  \newcommand{\xxxSubParagraphStar}[1]{\oldsubparagraph*{#1}\mbox{}}
  \newcommand{\xxxSubParagraphNoStar}[1]{\oldsubparagraph{#1}\mbox{}}
\fi
\makeatother

\usepackage{color}
\usepackage{fancyvrb}
\newcommand{\VerbBar}{|}
\newcommand{\VERB}{\Verb[commandchars=\\\{\}]}
\DefineVerbatimEnvironment{Highlighting}{Verbatim}{commandchars=\\\{\}}
% Add ',fontsize=\small' for more characters per line
\usepackage{framed}
\definecolor{shadecolor}{RGB}{241,243,245}
\newenvironment{Shaded}{\begin{snugshade}}{\end{snugshade}}
\newcommand{\AlertTok}[1]{\textcolor[rgb]{0.68,0.00,0.00}{#1}}
\newcommand{\AnnotationTok}[1]{\textcolor[rgb]{0.37,0.37,0.37}{#1}}
\newcommand{\AttributeTok}[1]{\textcolor[rgb]{0.40,0.45,0.13}{#1}}
\newcommand{\BaseNTok}[1]{\textcolor[rgb]{0.68,0.00,0.00}{#1}}
\newcommand{\BuiltInTok}[1]{\textcolor[rgb]{0.00,0.23,0.31}{#1}}
\newcommand{\CharTok}[1]{\textcolor[rgb]{0.13,0.47,0.30}{#1}}
\newcommand{\CommentTok}[1]{\textcolor[rgb]{0.37,0.37,0.37}{#1}}
\newcommand{\CommentVarTok}[1]{\textcolor[rgb]{0.37,0.37,0.37}{\textit{#1}}}
\newcommand{\ConstantTok}[1]{\textcolor[rgb]{0.56,0.35,0.01}{#1}}
\newcommand{\ControlFlowTok}[1]{\textcolor[rgb]{0.00,0.23,0.31}{\textbf{#1}}}
\newcommand{\DataTypeTok}[1]{\textcolor[rgb]{0.68,0.00,0.00}{#1}}
\newcommand{\DecValTok}[1]{\textcolor[rgb]{0.68,0.00,0.00}{#1}}
\newcommand{\DocumentationTok}[1]{\textcolor[rgb]{0.37,0.37,0.37}{\textit{#1}}}
\newcommand{\ErrorTok}[1]{\textcolor[rgb]{0.68,0.00,0.00}{#1}}
\newcommand{\ExtensionTok}[1]{\textcolor[rgb]{0.00,0.23,0.31}{#1}}
\newcommand{\FloatTok}[1]{\textcolor[rgb]{0.68,0.00,0.00}{#1}}
\newcommand{\FunctionTok}[1]{\textcolor[rgb]{0.28,0.35,0.67}{#1}}
\newcommand{\ImportTok}[1]{\textcolor[rgb]{0.00,0.46,0.62}{#1}}
\newcommand{\InformationTok}[1]{\textcolor[rgb]{0.37,0.37,0.37}{#1}}
\newcommand{\KeywordTok}[1]{\textcolor[rgb]{0.00,0.23,0.31}{\textbf{#1}}}
\newcommand{\NormalTok}[1]{\textcolor[rgb]{0.00,0.23,0.31}{#1}}
\newcommand{\OperatorTok}[1]{\textcolor[rgb]{0.37,0.37,0.37}{#1}}
\newcommand{\OtherTok}[1]{\textcolor[rgb]{0.00,0.23,0.31}{#1}}
\newcommand{\PreprocessorTok}[1]{\textcolor[rgb]{0.68,0.00,0.00}{#1}}
\newcommand{\RegionMarkerTok}[1]{\textcolor[rgb]{0.00,0.23,0.31}{#1}}
\newcommand{\SpecialCharTok}[1]{\textcolor[rgb]{0.37,0.37,0.37}{#1}}
\newcommand{\SpecialStringTok}[1]{\textcolor[rgb]{0.13,0.47,0.30}{#1}}
\newcommand{\StringTok}[1]{\textcolor[rgb]{0.13,0.47,0.30}{#1}}
\newcommand{\VariableTok}[1]{\textcolor[rgb]{0.07,0.07,0.07}{#1}}
\newcommand{\VerbatimStringTok}[1]{\textcolor[rgb]{0.13,0.47,0.30}{#1}}
\newcommand{\WarningTok}[1]{\textcolor[rgb]{0.37,0.37,0.37}{\textit{#1}}}

\providecommand{\tightlist}{%
  \setlength{\itemsep}{0pt}\setlength{\parskip}{0pt}}\usepackage{longtable,booktabs,array}
\usepackage{calc} % for calculating minipage widths
% Correct order of tables after \paragraph or \subparagraph
\usepackage{etoolbox}
\makeatletter
\patchcmd\longtable{\par}{\if@noskipsec\mbox{}\fi\par}{}{}
\makeatother
% Allow footnotes in longtable head/foot
\IfFileExists{footnotehyper.sty}{\usepackage{footnotehyper}}{\usepackage{footnote}}
\makesavenoteenv{longtable}
\usepackage{graphicx}
\makeatletter
\newsavebox\pandoc@box
\newcommand*\pandocbounded[1]{% scales image to fit in text height/width
  \sbox\pandoc@box{#1}%
  \Gscale@div\@tempa{\textheight}{\dimexpr\ht\pandoc@box+\dp\pandoc@box\relax}%
  \Gscale@div\@tempb{\linewidth}{\wd\pandoc@box}%
  \ifdim\@tempb\p@<\@tempa\p@\let\@tempa\@tempb\fi% select the smaller of both
  \ifdim\@tempa\p@<\p@\scalebox{\@tempa}{\usebox\pandoc@box}%
  \else\usebox{\pandoc@box}%
  \fi%
}
% Set default figure placement to htbp
\def\fps@figure{htbp}
\makeatother
% definitions for citeproc citations
\NewDocumentCommand\citeproctext{}{}
\NewDocumentCommand\citeproc{mm}{%
  \begingroup\def\citeproctext{#2}\cite{#1}\endgroup}
\makeatletter
 % allow citations to break across lines
 \let\@cite@ofmt\@firstofone
 % avoid brackets around text for \cite:
 \def\@biblabel#1{}
 \def\@cite#1#2{{#1\if@tempswa , #2\fi}}
\makeatother
\newlength{\cslhangindent}
\setlength{\cslhangindent}{1.5em}
\newlength{\csllabelwidth}
\setlength{\csllabelwidth}{3em}
\newenvironment{CSLReferences}[2] % #1 hanging-indent, #2 entry-spacing
 {\begin{list}{}{%
  \setlength{\itemindent}{0pt}
  \setlength{\leftmargin}{0pt}
  \setlength{\parsep}{0pt}
  % turn on hanging indent if param 1 is 1
  \ifodd #1
   \setlength{\leftmargin}{\cslhangindent}
   \setlength{\itemindent}{-1\cslhangindent}
  \fi
  % set entry spacing
  \setlength{\itemsep}{#2\baselineskip}}}
 {\end{list}}
\usepackage{calc}
\newcommand{\CSLBlock}[1]{\hfill\break\parbox[t]{\linewidth}{\strut\ignorespaces#1\strut}}
\newcommand{\CSLLeftMargin}[1]{\parbox[t]{\csllabelwidth}{\strut#1\strut}}
\newcommand{\CSLRightInline}[1]{\parbox[t]{\linewidth - \csllabelwidth}{\strut#1\strut}}
\newcommand{\CSLIndent}[1]{\hspace{\cslhangindent}#1}

\usepackage{tcolorbox}
\usepackage{amssymb}
\usepackage{yfonts}
\usepackage{bm}


\newtcolorbox{greybox}{
  colback=white,
  colframe=blue,
  coltext=black,
  boxsep=5pt,
  arc=4pt}
  
\newcommand{\sectionbreak}{\clearpage}

 
\newcommand{\ds}[4]{\sum_{{#1}=1}^{#3}\sum_{{#2}=1}^{#4}}
\newcommand{\us}[3]{\mathop{\sum\sum}_{1\leq{#2}<{#1}\leq{#3}}}

\newcommand{\ol}[1]{\overline{#1}}
\newcommand{\ul}[1]{\underline{#1}}

\newcommand{\amin}[1]{\mathop{\text{argmin}}_{#1}}
\newcommand{\amax}[1]{\mathop{\text{argmax}}_{#1}}

\newcommand{\ci}{\perp\!\!\!\perp}

\newcommand{\mc}[1]{\mathcal{#1}}
\newcommand{\mb}[1]{\mathbb{#1}}
\newcommand{\mf}[1]{\mathfrak{#1}}

\newcommand{\eps}{\epsilon}
\newcommand{\lbd}{\lambda}
\newcommand{\alp}{\alpha}
\newcommand{\df}{=:}
\newcommand{\am}[1]{\mathop{\text{argmin}}_{#1}}
\newcommand{\ls}[2]{\mathop{\sum\sum}_{#1}^{#2}}
\newcommand{\ijs}{\mathop{\sum\sum}_{1\leq i<j\leq n}}
\newcommand{\jis}{\mathop{\sum\sum}_{1\leq j<i\leq n}}
\newcommand{\sij}{\sum_{i=1}^n\sum_{j=1}^n}
	
\KOMAoption{captions}{tableheading}
\makeatletter
\@ifpackageloaded{caption}{}{\usepackage{caption}}
\AtBeginDocument{%
\ifdefined\contentsname
  \renewcommand*\contentsname{Table of contents}
\else
  \newcommand\contentsname{Table of contents}
\fi
\ifdefined\listfigurename
  \renewcommand*\listfigurename{List of Figures}
\else
  \newcommand\listfigurename{List of Figures}
\fi
\ifdefined\listtablename
  \renewcommand*\listtablename{List of Tables}
\else
  \newcommand\listtablename{List of Tables}
\fi
\ifdefined\figurename
  \renewcommand*\figurename{Figure}
\else
  \newcommand\figurename{Figure}
\fi
\ifdefined\tablename
  \renewcommand*\tablename{Table}
\else
  \newcommand\tablename{Table}
\fi
}
\@ifpackageloaded{float}{}{\usepackage{float}}
\floatstyle{ruled}
\@ifundefined{c@chapter}{\newfloat{codelisting}{h}{lop}}{\newfloat{codelisting}{h}{lop}[chapter]}
\floatname{codelisting}{Listing}
\newcommand*\listoflistings{\listof{codelisting}{List of Listings}}
\makeatother
\makeatletter
\makeatother
\makeatletter
\@ifpackageloaded{caption}{}{\usepackage{caption}}
\@ifpackageloaded{subcaption}{}{\usepackage{subcaption}}
\makeatother

\usepackage{bookmark}

\IfFileExists{xurl.sty}{\usepackage{xurl}}{} % add URL line breaks if available
\urlstyle{same} % disable monospaced font for URLs
\hypersetup{
  pdfauthor={Jan de Leeuw},
  colorlinks=true,
  linkcolor={blue},
  filecolor={Maroon},
  citecolor={Blue},
  urlcolor={Blue},
  pdfcreator={LaTeX via pandoc}}


\title{Smacof at 50: Part xx\\
Utilities}
\author{Jan de Leeuw}
\date{November 23, 2024}

\begin{document}
\maketitle

\renewcommand*\contentsname{Table of contents}
{
\hypersetup{linkcolor=}
\setcounter{tocdepth}{3}
\tableofcontents
}

\textbf{Note:} This is a working manuscript which will be
expanded/updated frequently. All suggestions for improvement are
welcome. All Rmd, tex, html, pdf, R, and C files are in the public
domain. Attribution will be appreciated, but is not required. The files
can be found at \url{https://github.com/deleeuw} in the repositories
smacofCode, smacofManual, and smacofExamples.

\sectionbreak

\section{Simultaneous Iteration}\label{sec-simit}

The problem in this section is to maximize
\begin{equation}\phantomsection\label{eq-lbddef}{
\lambda(K):=\text{tr}\ K'AK
}\end{equation} over all \(n\times p\) orthonormal \(K\). Suppose for
the time being that \(A\) is positive semi-definite.

Define \(k=\text{vec}(K)\) and
\begin{equation}\phantomsection\label{eq-apdef}{
A_p:=\underbrace{A\oplus\cdots\oplus A}_{p \text{times}}
}\end{equation} Then \(\lambda=k'A_pk\), which shows that \(\lambda\)
i/s convex, and that consequently for all \(k\) and \(\tilde k\) we have
the minorization \begin{equation}\phantomsection\label{eq-kmaj}{
k'A_pk\geq \tilde k'A_p\tilde k+2\tilde k'A_p(k-\tilde k)=2k'A_p\tilde k-\tilde k'A_p\tilde k
}\end{equation} It follows that we increase \(\lambda\) by increasing
\(\text{tr}\ K'A\tilde K\) over \(K'K=I\).
\begin{equation}\phantomsection\label{eq-kupd}{
K^{(\nu+1)}=\mathop{\text{argmax}}_{K'K=I}\text{tr}\ K'AK^{(\nu)}
}\end{equation} Computing K\^{}\{(\nu+1)\} is an orthoginal Procrustus
problem (Gower and Dijksterhuis
(\citeproc{ref-gower_dijksterhuis_04}{2004})). Thus from the singular
value decomposition \(AK^{(\nu)}=P\Phi Q'\) we find \(K^{(\nu+1)}=PQ'\).

There are two remaining elaborations of this result. First, we want to
get rid of the assumption that \(A\) is positive semi-definite. We do
this by adding a constant to the diagonal of \(A\). To compute a
suitable constant \(\mu\), use the fact that a diagonally dominant
symmetric matrix with a non-negative diagonal is positive semidefinite.
Thus if \begin{equation}\phantomsection\label{eq-mubnd}{
\mu\geq\max_i\sum_{j\not= i}|a_{ij}|-a_{ii}
}\end{equation} for all \(i\) then the matrix \(\overline{A}:=A+\mu I\)
is positive semi-definite. We now use Equation~\ref{eq-kupd} with the
adjusted \(\overline{A}\) and after convergence we subtract \(p\mu\)
from \(\lambda\).

Second, instead of computing the singular value decomposition to update
\(K\) in each iteration we can use the \(Q\) from the less expensive QR
decomposition. The argument is the same as in Gifi
(\citeproc{ref-gifi_B_90}{1990}) (page 98-99, page 171). The QR update
of \(K\) is a rotation of the Procrustus update of \(K\), and it
consequently gives the same value of \(\lambda\) from
Equation~\ref{eq-lbddef}.

\sectionbreak

\section{Symmetric Matrix Approximation}\label{sec-symmat}

The problem in this section is to minimize
\begin{equation}\phantomsection\label{eq-eydef}{
\sigma(X)=\text{tr}\ (C-XX')^2
}\end{equation} over all \(n\times p\) matrices \(X\). Note there are no
weights. The solution is well-known (Eckart and Young
(\citeproc{ref-eckart_young_36}{1936}), Keller
(\citeproc{ref-keller_62}{1962})). If \(C=K\Lambda K'\) is the
eigen-decomposition of \(C\), with eigenvalues in non-increasing order
on the diagonal of \(\Lambda\), then define \(\overline{\Lambda}\) with
elements \(\max(0,\lambda_s)\). Then use the \(p\) largest eigenvalues
and corresponding vectors to compute \(X=K_p\overline{\Lambda}_p\). Note
that \(\text{rank}(X)\) is less than \(p\) if \(C\) has fewer than \(p\)
positive eigenvalues.

Our algorithm to minimize, or at least decrease, the loss function in
Equation~\ref{eq-eydef}. Uses a combination of alternating least squares
(De Leeuw (\citeproc{ref-deleeuw_C_94c}{1994})) and majorization (De
Leeuw (\citeproc{ref-deleeuw_C_94c}{1994})) or MM (Lange
(\citeproc{ref-lange_16}{2016})). We first write \(X\) in the form
\(X=K\Lambda\), with \(K'K=I\) and \(\Lambda\) diagonal. Then rewrite
Equation~\ref{eq-eydef} as
\begin{equation}\phantomsection\label{eq-eydefkl}{
\sigma(K,\Lambda)=\text{tr}\ (C-K\Lambda^2 K')^2=\text{tr}\ C^2-2\text{tr}\ K'CK\Lambda^2+\text{tr}\ \Lambda^4.
}\end{equation} Again, for the time being, assume \(C\) is positive
semi-definite. The minimum over \(\Lambda\) for given \(K\) has
\(\lambda_s^2=k_s'Ck_s\). Thus
\begin{equation}\phantomsection\label{eq-eydefklexpmin}{
\min_{K'K=I}\min_\Lambda\sigma(K,\Lambda)=\text{tr}\ C^2-\max_{K'K=I}\sum_{s=1}^p (k_s'Ck_s)^2
}\end{equation} The term depending on \(K\) is convex and can
consequently be minorized by the linear function \[
\sum_{s=1}^p (k_s'Ck_s)^2=\sum_{s=1}^p (\tilde k_s'C\tilde k_s)^2+2\sum_{s=1}^p (\tilde k_s'C\tilde k_s)\tilde k_s'C(k_s-\tilde k_s).
\] In iteration \(\nu + 1\) we must maximize
\(\text{tr}\ K'CK^{(\nu)}\Lambda^{(\nu)}\) over \(K'K=I\), where
\(\Lambda^{(\nu)}=\text{diag}\{K^{(\nu)}\}'CK^{(\nu)}\). Again, this is
an orthogonal Procrustus problem.

\sectionbreak

\section{Various Simple}\label{various-simple}

\begin{Shaded}
\begin{Highlighting}[]
\NormalTok{smacofEi }\OtherTok{\textless{}{-}} \ControlFlowTok{function}\NormalTok{(i, n) \{}
  \FunctionTok{return}\NormalTok{(}\FunctionTok{ifelse}\NormalTok{(i }\SpecialCharTok{==} \DecValTok{1}\SpecialCharTok{:}\NormalTok{n, }\DecValTok{1}\NormalTok{, }\DecValTok{0}\NormalTok{))}
\NormalTok{\}}

\NormalTok{smacofAij }\OtherTok{\textless{}{-}} \ControlFlowTok{function}\NormalTok{(i, j, n) \{}
\NormalTok{  ei }\OtherTok{\textless{}{-}} \FunctionTok{ifelse}\NormalTok{(i }\SpecialCharTok{==} \DecValTok{1}\SpecialCharTok{:}\NormalTok{n, }\DecValTok{1}\NormalTok{, }\DecValTok{0}\NormalTok{)}
\NormalTok{  ej }\OtherTok{\textless{}{-}} \FunctionTok{ifelse}\NormalTok{(j }\SpecialCharTok{==} \DecValTok{1}\SpecialCharTok{:}\NormalTok{n, }\DecValTok{1}\NormalTok{, }\DecValTok{0}\NormalTok{)}
  \FunctionTok{return}\NormalTok{(}\FunctionTok{outer}\NormalTok{(ei }\SpecialCharTok{{-}}\NormalTok{ ej, ei }\SpecialCharTok{{-}}\NormalTok{ ej))}
\NormalTok{\}}

\NormalTok{smacofDoubleCenter }\OtherTok{\textless{}{-}} \ControlFlowTok{function}\NormalTok{(a) \{}
\NormalTok{  r }\OtherTok{\textless{}{-}} \FunctionTok{apply}\NormalTok{(a, }\DecValTok{1}\NormalTok{, mean)}
\NormalTok{  s }\OtherTok{\textless{}{-}} \FunctionTok{mean}\NormalTok{(a)}
  \FunctionTok{return}\NormalTok{(a }\SpecialCharTok{{-}} \FunctionTok{outer}\NormalTok{(r, r, }\StringTok{"+"}\NormalTok{) }\SpecialCharTok{+}\NormalTok{ s)}
\NormalTok{\}}

\NormalTok{smacofCenter }\OtherTok{\textless{}{-}} \ControlFlowTok{function}\NormalTok{(x) \{}
  \FunctionTok{return}\NormalTok{(}\FunctionTok{apply}\NormalTok{(x, }\DecValTok{2}\NormalTok{, }\ControlFlowTok{function}\NormalTok{(x) x }\SpecialCharTok{{-}} \FunctionTok{mean}\NormalTok{(x)))}
\NormalTok{\}}

\NormalTok{smacofTrace }\OtherTok{\textless{}{-}} \ControlFlowTok{function}\NormalTok{(a) \{}
  \FunctionTok{return}\NormalTok{(}\FunctionTok{sum}\NormalTok{(}\FunctionTok{diag}\NormalTok{(a)))}
\NormalTok{\}}

\NormalTok{smacofMakeDoubleCenter }\OtherTok{\textless{}{-}} \ControlFlowTok{function}\NormalTok{(w) \{}
\NormalTok{  v }\OtherTok{\textless{}{-}} \SpecialCharTok{{-}}\NormalTok{w}
  \FunctionTok{diag}\NormalTok{(v) }\OtherTok{\textless{}{-}} \SpecialCharTok{{-}}\FunctionTok{rowSums}\NormalTok{(v)}
  \FunctionTok{return}\NormalTok{(v)}
\NormalTok{\}}

\NormalTok{smacofDoubleCenterGeneralizedInverse }\OtherTok{\textless{}{-}} \ControlFlowTok{function}\NormalTok{(v) \{}
\NormalTok{  n }\OtherTok{\textless{}{-}} \FunctionTok{nrow}\NormalTok{(v)}
  \FunctionTok{return}\NormalTok{(}\FunctionTok{solve}\NormalTok{(v }\SpecialCharTok{+}\NormalTok{ (}\DecValTok{1} \SpecialCharTok{/}\NormalTok{ n)) }\SpecialCharTok{{-}} \DecValTok{1} \SpecialCharTok{/}\NormalTok{ n)}
\NormalTok{\}}
\end{Highlighting}
\end{Shaded}

\section{Data formats}\label{data-formats}

\begin{Shaded}
\begin{Highlighting}[]
\NormalTok{smacofMakeData }\OtherTok{\textless{}{-}}
  \ControlFlowTok{function}\NormalTok{(delta,}
           \AttributeTok{weights =} \FunctionTok{rep}\NormalTok{(}\DecValTok{1}\NormalTok{, }\FunctionTok{length}\NormalTok{(delta)),}
           \AttributeTok{winclude =} \ConstantTok{FALSE}\NormalTok{,}
\NormalTok{           fname) \{}
\NormalTok{    m }\OtherTok{\textless{}{-}} \FunctionTok{length}\NormalTok{(delta)}
\NormalTok{    n }\OtherTok{\textless{}{-}} \FunctionTok{as.integer}\NormalTok{((}\DecValTok{1} \SpecialCharTok{+} \FunctionTok{sqrt}\NormalTok{(}\DecValTok{1} \SpecialCharTok{+} \DecValTok{8} \SpecialCharTok{*}\NormalTok{ m)) }\SpecialCharTok{/} \DecValTok{2}\NormalTok{)}
\NormalTok{    h }\OtherTok{\textless{}{-}} \FunctionTok{fullIndex}\NormalTok{(n)}
\NormalTok{    g }\OtherTok{\textless{}{-}} \FunctionTok{cbind}\NormalTok{(h}\SpecialCharTok{$}\NormalTok{ii, h}\SpecialCharTok{$}\NormalTok{jj, delta, weights)}
    \ControlFlowTok{for}\NormalTok{ (k }\ControlFlowTok{in} \DecValTok{1}\SpecialCharTok{:}\NormalTok{m) \{}
      \ControlFlowTok{if}\NormalTok{ ((g[k, }\DecValTok{4}\NormalTok{] }\SpecialCharTok{==} \DecValTok{0}\NormalTok{) }\SpecialCharTok{||} \FunctionTok{is.na}\NormalTok{(g[k, }\DecValTok{4}\NormalTok{]) }\SpecialCharTok{||} \FunctionTok{is.na}\NormalTok{(g[k, }\DecValTok{3}\NormalTok{])) \{}
\NormalTok{        continue}
\NormalTok{      \} }\ControlFlowTok{else}\NormalTok{ \{}
        \ControlFlowTok{if}\NormalTok{ (winclude) \{}
          \FunctionTok{cat}\NormalTok{(}
            \FunctionTok{formatC}\NormalTok{(g[k, }\DecValTok{1}\NormalTok{], }\AttributeTok{digits =} \DecValTok{3}\NormalTok{, }\AttributeTok{format =} \StringTok{"d"}\NormalTok{),}
            \FunctionTok{formatC}\NormalTok{(g[k, }\DecValTok{2}\NormalTok{], }\AttributeTok{digits =} \DecValTok{3}\NormalTok{, }\AttributeTok{format =} \StringTok{"d"}\NormalTok{),}
            \FunctionTok{formatC}\NormalTok{(g[k, }\DecValTok{3}\NormalTok{], }\AttributeTok{digits =} \DecValTok{6}\NormalTok{, }\AttributeTok{format =} \StringTok{"f"}\NormalTok{),}
            \FunctionTok{formatC}\NormalTok{(g[k, }\DecValTok{4}\NormalTok{], }\AttributeTok{digits =} \DecValTok{6}\NormalTok{, }\AttributeTok{format =} \StringTok{"f"}\NormalTok{),}
            \StringTok{"}\SpecialCharTok{\textbackslash{}n}\StringTok{"}\NormalTok{,}
            \AttributeTok{file =}\NormalTok{ fname, }\AttributeTok{append =} \ConstantTok{TRUE}
\NormalTok{          )}
\NormalTok{        \} }\ControlFlowTok{else}\NormalTok{ \{}
          \FunctionTok{cat}\NormalTok{(}
            \FunctionTok{formatC}\NormalTok{(g[k, }\DecValTok{1}\NormalTok{], }\AttributeTok{digits =} \DecValTok{3}\NormalTok{, }\AttributeTok{format =} \StringTok{"d"}\NormalTok{),}
            \FunctionTok{formatC}\NormalTok{(g[k, }\DecValTok{2}\NormalTok{], }\AttributeTok{digits =} \DecValTok{3}\NormalTok{, }\AttributeTok{format =} \StringTok{"d"}\NormalTok{),}
            \FunctionTok{formatC}\NormalTok{(g[k, }\DecValTok{3}\NormalTok{], }\AttributeTok{digits =} \DecValTok{6}\NormalTok{, }\AttributeTok{format =} \StringTok{"f"}\NormalTok{),}
            \StringTok{"}\SpecialCharTok{\textbackslash{}n}\StringTok{"}\NormalTok{,}
            \AttributeTok{file =}\NormalTok{ fname, }\AttributeTok{append =} \ConstantTok{TRUE}
\NormalTok{          )}
\NormalTok{        \}}
\NormalTok{      \}}
\NormalTok{    \}}
\NormalTok{  \}}


\NormalTok{fullIndex }\OtherTok{\textless{}{-}} \ControlFlowTok{function}\NormalTok{(n) \{}
\NormalTok{  ii }\OtherTok{\textless{}{-}} \FunctionTok{c}\NormalTok{()}
\NormalTok{  jj }\OtherTok{\textless{}{-}} \FunctionTok{c}\NormalTok{()}
  \ControlFlowTok{for}\NormalTok{ (j }\ControlFlowTok{in} \DecValTok{1}\SpecialCharTok{:}\NormalTok{(n }\SpecialCharTok{{-}} \DecValTok{1}\NormalTok{)) \{}
    \ControlFlowTok{for}\NormalTok{ (i }\ControlFlowTok{in}\NormalTok{ (j }\SpecialCharTok{+} \DecValTok{1}\NormalTok{)}\SpecialCharTok{:}\NormalTok{n) \{}
\NormalTok{      ii }\OtherTok{\textless{}{-}} \FunctionTok{c}\NormalTok{(ii, i)}
\NormalTok{      jj }\OtherTok{\textless{}{-}} \FunctionTok{c}\NormalTok{(jj, j)}
\NormalTok{    \}}
\NormalTok{  \}}
  \FunctionTok{return}\NormalTok{(}\FunctionTok{list}\NormalTok{(}\AttributeTok{ii =}\NormalTok{ ii, }\AttributeTok{jj =}\NormalTok{ jj))}
\NormalTok{\}}
\end{Highlighting}
\end{Shaded}

\section{I/O}\label{io}

\begin{Shaded}
\begin{Highlighting}[]
\NormalTok{smacofMatrixPrint }\OtherTok{\textless{}{-}} \ControlFlowTok{function}\NormalTok{(x,}
                   \AttributeTok{digits =} \DecValTok{6}\NormalTok{,}
                   \AttributeTok{width =} \DecValTok{8}\NormalTok{,}
                   \AttributeTok{format =} \StringTok{"f"}\NormalTok{,}
                   \AttributeTok{flag =} \StringTok{"+"}\NormalTok{) \{}
  \FunctionTok{print}\NormalTok{(}\FunctionTok{noquote}\NormalTok{(}
    \FunctionTok{formatC}\NormalTok{(}
\NormalTok{      x,}
      \AttributeTok{digits =}\NormalTok{ digits,}
      \AttributeTok{width =}\NormalTok{ width,}
      \AttributeTok{format =}\NormalTok{ format,}
      \AttributeTok{flag =}\NormalTok{ flag}
\NormalTok{    )}
\NormalTok{  ))}
\NormalTok{\}}
\end{Highlighting}
\end{Shaded}

\section{Indices}\label{indices}

sindex tindex mindex vindex

\sectionbreak

\section*{References}\label{references}
\addcontentsline{toc}{section}{References}

\phantomsection\label{refs}
\begin{CSLReferences}{1}{0}
\bibitem[\citeproctext]{ref-deleeuw_C_94c}
De Leeuw, J. 1994. {``{Block Relaxation Algorithms in Statistics}.''} In
\emph{Information Systems and Data Analysis}, edited by H. H. Bock, W.
Lenski, and M. M. Richter, 308--24. Berlin: Springer Verlag.
\url{https://jansweb.netlify.app/publication/deleeuw-c-94-c/deleeuw-c-94-c.pdf}.

\bibitem[\citeproctext]{ref-eckart_young_36}
Eckart, C., and G. Young. 1936. {``{The Approximation of One Matrix by
Another of Lower Rank}.''} \emph{Psychometrika} 1 (3): 211--18.

\bibitem[\citeproctext]{ref-gifi_B_90}
Gifi, A. 1990. \emph{Nonlinear Multivariate Analysis}. New York, N.Y.:
Wiley.

\bibitem[\citeproctext]{ref-gower_dijksterhuis_04}
Gower, J. C., and G. B. Dijksterhuis. 2004. \emph{Procrustus Problems}.
Oxford University Press.

\bibitem[\citeproctext]{ref-keller_62}
Keller, J. B. 1962. {``{Factorization of Matrices by Least Squares}.''}
\emph{Biometrika} 49: 239--42.

\bibitem[\citeproctext]{ref-lange_16}
Lange, K. 2016. \emph{MM Optimization Algorithms}. SIAM.

\end{CSLReferences}




\end{document}
