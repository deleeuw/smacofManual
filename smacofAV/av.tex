% Options for packages loaded elsewhere
\PassOptionsToPackage{unicode}{hyperref}
\PassOptionsToPackage{hyphens}{url}
\PassOptionsToPackage{dvipsnames,svgnames,x11names}{xcolor}
%
\documentclass[
  12pt,
  letterpaper,
  DIV=11,
  numbers=noendperiod]{scrartcl}

\usepackage{amsmath,amssymb}
\usepackage{iftex}
\ifPDFTeX
  \usepackage[T1]{fontenc}
  \usepackage[utf8]{inputenc}
  \usepackage{textcomp} % provide euro and other symbols
\else % if luatex or xetex
  \usepackage{unicode-math}
  \defaultfontfeatures{Scale=MatchLowercase}
  \defaultfontfeatures[\rmfamily]{Ligatures=TeX,Scale=1}
\fi
\usepackage{lmodern}
\ifPDFTeX\else  
    % xetex/luatex font selection
    \setmainfont[]{Times New Roman}
\fi
% Use upquote if available, for straight quotes in verbatim environments
\IfFileExists{upquote.sty}{\usepackage{upquote}}{}
\IfFileExists{microtype.sty}{% use microtype if available
  \usepackage[]{microtype}
  \UseMicrotypeSet[protrusion]{basicmath} % disable protrusion for tt fonts
}{}
\makeatletter
\@ifundefined{KOMAClassName}{% if non-KOMA class
  \IfFileExists{parskip.sty}{%
    \usepackage{parskip}
  }{% else
    \setlength{\parindent}{0pt}
    \setlength{\parskip}{6pt plus 2pt minus 1pt}}
}{% if KOMA class
  \KOMAoptions{parskip=half}}
\makeatother
\usepackage{xcolor}
\setlength{\emergencystretch}{3em} % prevent overfull lines
\setcounter{secnumdepth}{5}
% Make \paragraph and \subparagraph free-standing
\makeatletter
\ifx\paragraph\undefined\else
  \let\oldparagraph\paragraph
  \renewcommand{\paragraph}{
    \@ifstar
      \xxxParagraphStar
      \xxxParagraphNoStar
  }
  \newcommand{\xxxParagraphStar}[1]{\oldparagraph*{#1}\mbox{}}
  \newcommand{\xxxParagraphNoStar}[1]{\oldparagraph{#1}\mbox{}}
\fi
\ifx\subparagraph\undefined\else
  \let\oldsubparagraph\subparagraph
  \renewcommand{\subparagraph}{
    \@ifstar
      \xxxSubParagraphStar
      \xxxSubParagraphNoStar
  }
  \newcommand{\xxxSubParagraphStar}[1]{\oldsubparagraph*{#1}\mbox{}}
  \newcommand{\xxxSubParagraphNoStar}[1]{\oldsubparagraph{#1}\mbox{}}
\fi
\makeatother

\usepackage{color}
\usepackage{fancyvrb}
\newcommand{\VerbBar}{|}
\newcommand{\VERB}{\Verb[commandchars=\\\{\}]}
\DefineVerbatimEnvironment{Highlighting}{Verbatim}{commandchars=\\\{\}}
% Add ',fontsize=\small' for more characters per line
\usepackage{framed}
\definecolor{shadecolor}{RGB}{241,243,245}
\newenvironment{Shaded}{\begin{snugshade}}{\end{snugshade}}
\newcommand{\AlertTok}[1]{\textcolor[rgb]{0.68,0.00,0.00}{#1}}
\newcommand{\AnnotationTok}[1]{\textcolor[rgb]{0.37,0.37,0.37}{#1}}
\newcommand{\AttributeTok}[1]{\textcolor[rgb]{0.40,0.45,0.13}{#1}}
\newcommand{\BaseNTok}[1]{\textcolor[rgb]{0.68,0.00,0.00}{#1}}
\newcommand{\BuiltInTok}[1]{\textcolor[rgb]{0.00,0.23,0.31}{#1}}
\newcommand{\CharTok}[1]{\textcolor[rgb]{0.13,0.47,0.30}{#1}}
\newcommand{\CommentTok}[1]{\textcolor[rgb]{0.37,0.37,0.37}{#1}}
\newcommand{\CommentVarTok}[1]{\textcolor[rgb]{0.37,0.37,0.37}{\textit{#1}}}
\newcommand{\ConstantTok}[1]{\textcolor[rgb]{0.56,0.35,0.01}{#1}}
\newcommand{\ControlFlowTok}[1]{\textcolor[rgb]{0.00,0.23,0.31}{\textbf{#1}}}
\newcommand{\DataTypeTok}[1]{\textcolor[rgb]{0.68,0.00,0.00}{#1}}
\newcommand{\DecValTok}[1]{\textcolor[rgb]{0.68,0.00,0.00}{#1}}
\newcommand{\DocumentationTok}[1]{\textcolor[rgb]{0.37,0.37,0.37}{\textit{#1}}}
\newcommand{\ErrorTok}[1]{\textcolor[rgb]{0.68,0.00,0.00}{#1}}
\newcommand{\ExtensionTok}[1]{\textcolor[rgb]{0.00,0.23,0.31}{#1}}
\newcommand{\FloatTok}[1]{\textcolor[rgb]{0.68,0.00,0.00}{#1}}
\newcommand{\FunctionTok}[1]{\textcolor[rgb]{0.28,0.35,0.67}{#1}}
\newcommand{\ImportTok}[1]{\textcolor[rgb]{0.00,0.46,0.62}{#1}}
\newcommand{\InformationTok}[1]{\textcolor[rgb]{0.37,0.37,0.37}{#1}}
\newcommand{\KeywordTok}[1]{\textcolor[rgb]{0.00,0.23,0.31}{\textbf{#1}}}
\newcommand{\NormalTok}[1]{\textcolor[rgb]{0.00,0.23,0.31}{#1}}
\newcommand{\OperatorTok}[1]{\textcolor[rgb]{0.37,0.37,0.37}{#1}}
\newcommand{\OtherTok}[1]{\textcolor[rgb]{0.00,0.23,0.31}{#1}}
\newcommand{\PreprocessorTok}[1]{\textcolor[rgb]{0.68,0.00,0.00}{#1}}
\newcommand{\RegionMarkerTok}[1]{\textcolor[rgb]{0.00,0.23,0.31}{#1}}
\newcommand{\SpecialCharTok}[1]{\textcolor[rgb]{0.37,0.37,0.37}{#1}}
\newcommand{\SpecialStringTok}[1]{\textcolor[rgb]{0.13,0.47,0.30}{#1}}
\newcommand{\StringTok}[1]{\textcolor[rgb]{0.13,0.47,0.30}{#1}}
\newcommand{\VariableTok}[1]{\textcolor[rgb]{0.07,0.07,0.07}{#1}}
\newcommand{\VerbatimStringTok}[1]{\textcolor[rgb]{0.13,0.47,0.30}{#1}}
\newcommand{\WarningTok}[1]{\textcolor[rgb]{0.37,0.37,0.37}{\textit{#1}}}

\providecommand{\tightlist}{%
  \setlength{\itemsep}{0pt}\setlength{\parskip}{0pt}}\usepackage{longtable,booktabs,array}
\usepackage{calc} % for calculating minipage widths
% Correct order of tables after \paragraph or \subparagraph
\usepackage{etoolbox}
\makeatletter
\patchcmd\longtable{\par}{\if@noskipsec\mbox{}\fi\par}{}{}
\makeatother
% Allow footnotes in longtable head/foot
\IfFileExists{footnotehyper.sty}{\usepackage{footnotehyper}}{\usepackage{footnote}}
\makesavenoteenv{longtable}
\usepackage{graphicx}
\makeatletter
\def\maxwidth{\ifdim\Gin@nat@width>\linewidth\linewidth\else\Gin@nat@width\fi}
\def\maxheight{\ifdim\Gin@nat@height>\textheight\textheight\else\Gin@nat@height\fi}
\makeatother
% Scale images if necessary, so that they will not overflow the page
% margins by default, and it is still possible to overwrite the defaults
% using explicit options in \includegraphics[width, height, ...]{}
\setkeys{Gin}{width=\maxwidth,height=\maxheight,keepaspectratio}
% Set default figure placement to htbp
\makeatletter
\def\fps@figure{htbp}
\makeatother
% definitions for citeproc citations
\NewDocumentCommand\citeproctext{}{}
\NewDocumentCommand\citeproc{mm}{%
  \begingroup\def\citeproctext{#2}\cite{#1}\endgroup}
\makeatletter
 % allow citations to break across lines
 \let\@cite@ofmt\@firstofone
 % avoid brackets around text for \cite:
 \def\@biblabel#1{}
 \def\@cite#1#2{{#1\if@tempswa , #2\fi}}
\makeatother
\newlength{\cslhangindent}
\setlength{\cslhangindent}{1.5em}
\newlength{\csllabelwidth}
\setlength{\csllabelwidth}{3em}
\newenvironment{CSLReferences}[2] % #1 hanging-indent, #2 entry-spacing
 {\begin{list}{}{%
  \setlength{\itemindent}{0pt}
  \setlength{\leftmargin}{0pt}
  \setlength{\parsep}{0pt}
  % turn on hanging indent if param 1 is 1
  \ifodd #1
   \setlength{\leftmargin}{\cslhangindent}
   \setlength{\itemindent}{-1\cslhangindent}
  \fi
  % set entry spacing
  \setlength{\itemsep}{#2\baselineskip}}}
 {\end{list}}
\usepackage{calc}
\newcommand{\CSLBlock}[1]{\hfill\break\parbox[t]{\linewidth}{\strut\ignorespaces#1\strut}}
\newcommand{\CSLLeftMargin}[1]{\parbox[t]{\csllabelwidth}{\strut#1\strut}}
\newcommand{\CSLRightInline}[1]{\parbox[t]{\linewidth - \csllabelwidth}{\strut#1\strut}}
\newcommand{\CSLIndent}[1]{\hspace{\cslhangindent}#1}

\KOMAoption{captions}{tableheading}
\makeatletter
\@ifpackageloaded{caption}{}{\usepackage{caption}}
\AtBeginDocument{%
\ifdefined\contentsname
  \renewcommand*\contentsname{Table of contents}
\else
  \newcommand\contentsname{Table of contents}
\fi
\ifdefined\listfigurename
  \renewcommand*\listfigurename{List of Figures}
\else
  \newcommand\listfigurename{List of Figures}
\fi
\ifdefined\listtablename
  \renewcommand*\listtablename{List of Tables}
\else
  \newcommand\listtablename{List of Tables}
\fi
\ifdefined\figurename
  \renewcommand*\figurename{Figure}
\else
  \newcommand\figurename{Figure}
\fi
\ifdefined\tablename
  \renewcommand*\tablename{Table}
\else
  \newcommand\tablename{Table}
\fi
}
\@ifpackageloaded{float}{}{\usepackage{float}}
\floatstyle{ruled}
\@ifundefined{c@chapter}{\newfloat{codelisting}{h}{lop}}{\newfloat{codelisting}{h}{lop}[chapter]}
\floatname{codelisting}{Listing}
\newcommand*\listoflistings{\listof{codelisting}{List of Listings}}
\makeatother
\makeatletter
\makeatother
\makeatletter
\@ifpackageloaded{caption}{}{\usepackage{caption}}
\@ifpackageloaded{subcaption}{}{\usepackage{subcaption}}
\makeatother

\ifLuaTeX
  \usepackage{selnolig}  % disable illegal ligatures
\fi
\usepackage{bookmark}

\IfFileExists{xurl.sty}{\usepackage{xurl}}{} % add URL line breaks if available
\urlstyle{same} % disable monospaced font for URLs
\hypersetup{
  pdftitle={Robust Least Squares Multidimensional Scaling},
  pdfauthor={Jan de Leeuw},
  colorlinks=true,
  linkcolor={blue},
  filecolor={Maroon},
  citecolor={Blue},
  urlcolor={Blue},
  pdfcreator={LaTeX via pandoc}}


\title{Robust Least Squares Multidimensional Scaling}
\author{Jan de Leeuw}
\date{September 23, 2024}

\begin{document}
\maketitle
\begin{abstract}
We use an iteratively reweighted version of the smacof algorithm to
minimize various robust multidimensional scaling loss functions. Our
results depend strongly on a general theorem on sharp quadratic
majorization of De Leeuw and Lange
(\citeproc{ref-deleeuw_lange_A_09}{2009}).
\end{abstract}


\newcommand{\sectionbreak}{\pagebreak}
\pagebreak

\section{Introduction}\label{introduction}

The title of this chapter seems something paradoxical. Least squares
estimation is typically not robust, it is sensitive to outliers and pays
a lot of attention to fitting the larger observations. What we mean by
robust least squares MDS, however, is using the smacof machinery
designed to minimize loss of the form \begin{equation}
\sigma_2(X):=\sum w_k(\delta_k-d_k(X))^2\label{eq:stressdef},
\end{equation} to minimize robust loss functions. The prototypical
robust loss function is \begin{equation}
\sigma_1(X):=\sum w_k|\delta_k-d_k(X)|\label{eq:stradddef},
\end{equation} which we will call \emph{strife}, because stress,
sstress, and strain are already taken.

Strife is not differentiable at configurations \(X\) for which there is
at least one \(k\) for which either \(d_k(X)=\delta_k\) or \(d_k(X)=0\)
(or both). This lack of differentiability complicates the minimization
problem. Moreover experience with one-dimensional and city block MDS
suggests that having many points where the loss function is not
differentiable leads to (many) additional local minima.

In this chapter we will discuss (and implement) various variations of
\(\sigma_1\) from \eqref{eq:stradddef}. They can be interpreted in two
different ways. On the one hand we use smoothers of the absolute value
function, and consequently of strife. This is not unlike the distance
smoothing used by Pliner (\citeproc{ref-pliner_96}{1996}) and Groenen,
Heiser, and Meulman (\citeproc{ref-groenen_heiser_meulman_99}{1999}) in
the global minimization of \(\sigma_2\) from \eqref{eq:stressdef}. On
the other hand our modified loss function can be interpreted as more
robust versions of the least squares loss function, and consequently of
stress.

Our robust or smoother loss functions are all of the form
\begin{equation}
\sigma(X):=\sum w_k\ f(\delta_k-d_k(X))\label{eq:strifedef},
\end{equation} for a suitable choice of the real valued function \(f\).
We will define what we mean by ``suitable'' later on. For now, note that
loss \eqref{eq:stressdef} is the special case with \(f(x)=x^2\) and loss
\eqref{eq:stradddiff} is the special case with \(f(x)=|x|\).

\section{Majorizing Strife}\label{majorizing-strife}

The pioneering work in strife minimization using smacof is Heiser
(\citeproc{ref-heiser_88}{1988}), building on earlier work in Heiser
(\citeproc{ref-heiser_87}{1987}). It is based on a creative use of the
Arithmetic Mean-Geometric Mean (AM/GM) inequality to find a majorizer of
the absolute value function. For the general theory of majorization
algorithms (now more commonly known as MM algorithms) we refer to their
original introduction in De Leeuw (\citeproc{ref-deleeuw_C_94c}{1994})
and to the excellent book by Lange (\citeproc{ref-lange_16}{2016}).

The AM/GM inequality says that for all non-negative \(x\) and \(y\) we
have \begin{equation}
|x||y|=\sqrt{x^2y^2}\leq\frac12(x^2+y^2),\label{eq:amgm}
\end{equation} with equality if and only if \(x=y\). If \(y>0\) we can
write \eqref{eq:amgm} as \begin{equation}
|x|\leq\frac12\frac{1}{|y|}(x^2+y^2),\label{eq:amgmmaj}
\end{equation} and this provides a quadratic majorization of \(|x|\) at
\(y\). There is no quadratic majorization of \(|x|\) at \(y=0\), which
is a nuisance we must deal with.

Using the majorization \eqref{eq:amgmmaj}, and assuming
\(\delta_k\not= d_k(Y)\) for all \(k\), we define \begin{equation}
\omega_1(X):=\frac12\sum w_k\frac{1}{|\delta_k-d_k(Y)|}((\delta_k-d_k(Y))^2+(\delta_k-d_k(X))^2).\label{eq:omegadef}
\end{equation} Now \(\sigma_1(X)\leq\omega_1(X)\) for all \(X\) and
\(\sigma_1(Y)=\omega_1(Y)\). Thus \(\omega_1\) majorizes \(\sigma_1\) at
\(Y\).

But what if for some \(k\) we have \(d_k(Y)=\delta_k\) ? Define
\begin{equation}
\begin{subequations}
s_k(Y)=\begin{cases}
w_k\frac{1}{|\delta_k-d_k(Y)|}&\text{ if }d_k(Y)\not =\delta_k,\\
2\ \max_k w_k&\text{ otherwise}.\label{eq:sdef}
\end{cases}
\end{subequations}
\end{equation} Redefine \(\omega_1\) as \begin{equation}
\omega_1(X):=\frac12\sum s_k(Y)\{(\delta_k-d_k(Y))^2+(\delta_k-d_k(X))^2\}.\label{eq:ommaj}
\end{equation} This modified \(\omega_1\) majorizes \(\sigma_1\) at
\(Y\), even in the case that \(\delta_k=d_k(Y)\) for all \(k\).

Reweighted smacof to minimize strife computes \(X^{(k+1)}\) by
decreasing \begin{equation}
\sum s_k(X^{(k)})(\delta_k-d_k(X^{(k)}))^2,\label{eq:sstrf}
\end{equation} using a standard smacof step. It then computes the new
weights \(s_k(X^{(k+1)})\) and uses them in the next smacof step to
update \(X^{(k+1)}\). And so on, until convergence.

To illustrate the problems with differentability we compute the
directional derivatives of strife.

Let \(s_k(X):=w_k|d_k(X)-\delta_k|\).

\begin{enumerate}
\def\labelenumi{\arabic{enumi}.}
\tightlist
\item
  If \(\delta_k=0\) and \(d_k(X)=0\) then \(ds_k(X;Y)=w_kd_k(Y)\).
\item
  If \(\delta_k>0\) and \(d_k(X)=0\) then \(ds(X;Y)=-w_kd_k(Y)\).
\item
  If \(d_k(X)>0\) and \(d_k(X)-\delta_k>0\) then
  \(ds_k(X;Y)=w_k\frac{1}{d_k(X)}\text{tr}\ X'A_kY\).
\item
  If \(d_k(X)>0\) and \(d_k(X)-\delta_k<0\) then
  \(ds_k(X;Y)=-w_k\frac{1}{d_k(X)}\text{tr}\ X'A_kY\).
\item
  If \(d_k(X)>0\) and \(d_k(X)-\delta_k=0\) then
  \(ds_k(X;Y)=w_k\frac{1}{d_k(X)}|\text{tr}\ X'A_kY|\).
\end{enumerate}

The directional derivative of \(\sigma_1\) is consequently the sum of
five terms, corresponding with each of these five cases.

In the case of \(\sigma_2\) the directional derivatives could be used to
prove that if \(w_k\delta_k>0\) for all \(k\) stress then is
differentiable at each local minimum (De Leeuw
(\citeproc{ref-deleeuw_A_84f}{1984})). For strife to be differentiable
we would have to prove that at a local minimum both \(d_k(X)>0\) and
\((d_k(X)-\delta_k)\not= 0\). So far I have no proof and no counter
example, but it's early in the game.

\section{Generalizing Strife}\label{generalizing-strife}

The AM/GM inequality was used in the previous section to construct a
quadratic majorization of strife. To fix the terminology we say that a
function \(g\) \emph{majorizes} a function \(f\) at \(y\) if
\(g(x)\geq f(x)\) for all \(x\) and \(g(y)=f(y)\). Majorization is
\emph{strict} if \(g(x)>f(x)\) for all \(x\not= y\). If \(\mathfrak{H}\)
is a family of functions that all majorize \(f\) at \(y\) then
\(h\in\mathfrak{H}\) is \emph{sharp} if \(h(x)\leq g(x)\) for all
\(g\in\mathfrak{H}\).

We are specifically interested in this chapter in sharp quadratic
majorization, in which \(\mathfrak{H}\) is the set of all convex
quadratics that majorize \(f\) at \(y\). This case has been studied in
detail (in the case of real-valued functions on the line) by De Leeuw
and Lange (\citeproc{ref-deleeuw_lange_A_09}{2009}). Their Theorem 4.5
on page 2478 says

\begin{quote}
Theorem: Suppose \(f(x)\) is an even, differentiable function on
\(\mathbb{R}\) such that the ratio \(f'(x)/x\) is non-increasing on
\((0,\infty)\). Then the even quadratic \begin{equation}
g(x)=\frac{f'(y)}{2y}(x^2-y^2)+f(y)\label{eq:sharp}
\end{equation} is a sharp quadratic majorizer of \(f\) at the point
\(y\).
\end{quote}

We now apply this theorem to functions of the form \begin{equation}
\sigma_f(X):=\sum w_k\ f(\delta_k-d_k(X)),\label{eq:fstressdef}
\end{equation} where \(f\) satisfies the conditions in the theorem. If
\begin{equation}
\omega_f(X):=\sum w_k\frac{f'(\delta_k-d_k(Y))}{2(\delta_k-d_k(Y))}\{(\delta_k-d_k(X))^2-(\delta_k-d_k(Y))^2\}+f(\delta_k-d_k(Y)),\label{eq:fstressmaj}
\end{equation} then \(\omega_f\) is a sharp quadratic majorization at
\(Y\).

Although the absolute value is not differentiable at the origin the
theorem can still be applied. It just does not give a majorizer at
\(y=0\). If \(f(x)=|x|\) then \begin{equation}
g(x)=\frac{1}{2|y|}(x^2-y^2)+|y|=\frac{1}{2|y|}(x^2+y^2),label{eq:abssharp}
\end{equation} which is the same as \eqref{eq:amgmmaj}. Thus the AM/GM
method gives the sharp quadratic majorization.

In iteration \(k\) the robust smacof algorithm does a smacof step
towards minimization of \(\omega_f\) over \(X\). We can ignore the parts
of \eqref{eq:fstressmaj} that only depend on \(Y\), and minimize
\begin{equation}
\sum w_k(X^{(k)})(\delta_k-d_k(X))^2,\label{eq:fstressaux}
\end{equation} with \begin{equation}
w_k(X^{(k)}):=w_k\frac{f'(\delta_k-d_k(X^{(k)}))}{2(\delta_k-d_k(Y))}.\label{eq:wkdef}
\end{equation} It then recomputes the weights \(w_k(X^{(k+1)})\) and
goes to the smacof step again. This can be thought of as iterativey
reweighted least squares, and also as majorization within majorization,
with the smacof majorization within the sharp quadratic majorization of
the loss function.

A straightforward variation of the algorithm does a number of smacof
steps before upgrading the weights. This still leads to a monotone, and
thus convergent, algorithm. How many smacof steps we have to take in the
inner iterations is something that needs further study. It is likely to
depend on the fit of the data, on the shape of the function near the
local minimum, and on how far the iterations are from the local minimum.

\section{Pseudo-Huber Loss, Charbonnier
loss}\label{pseudo-huber-loss-charbonnier-loss}

De Leeuw (\citeproc{ref-deleeuw_E_18f}{2018})

\[
f(x)=\sqrt{x^2 + c^2}
\] \[
f(x)=c\ \sqrt{1+(\frac{x}{c})^2}
\] \[
f'(x)=\frac{1}{\sqrt{x^2+c^2}}x
\]

\[
\frac{f'(x)}{x}=\frac{1}{\sqrt{x^2+c^2}}
\] which is decreasing. \[
\sigma_\epsilon(X):=\sum w_k\sqrt{(\delta_k-d_k(X))^2+\epsilon^2}
\] Now majorization using \[
\sqrt{(\delta_k-d_k(X))^2+\epsilon^2}\leq\frac12\frac{1}{\sqrt{(\delta_k-d_k(Y))^2+\epsilon^2}}\{(\delta_k-d_k(X))^2+(\delta_k-d_k(Y))^2+2\epsilon^2\}
\] Alt:

\[
\sigma_\epsilon(X)=\epsilon^2\left\{\sqrt{1+\frac{x^2}{\epsilon^2}}-1\right\}
\]

\section{Robustifying - Huber Loss}\label{robustifying---huber-loss}

The Huber function (Huber (\citeproc{ref-huber_64}{1964})) is \[
f(x)=\begin{cases}
\frac12x^2&\text{ if }|x|<c,\\
c|x|-\frac12 c^2&\text{ otherwise}.
\end{cases}
\]

The Huber function is differentiable, although not twice diffentiable.
Its derivative is \[
f'(x)=\begin{cases}
c&\text{ if }x\geq c,\\
x&\text{ if }|x|\leq c,\\
-c&\text{ if }x\leq -c.
\end{cases}
\]

The Huber function is even and differentiable. Moreover \(f'(x)/x\)
decreases from. Thus the theorem applies and the sharp quadratic
majorizer at \(y\) is \[
g(x)=\begin{cases}
\end{cases}
\]

\[
\sigma_k(X)=\begin{cases}
\frac12(\delta_k-d_k(X))^2&\text{ if }|\delta_k-d_k(X)|<c,\\
c|\delta_k-d_k(X)|-\frac12 c^2&\text{ if }|\delta_k-d_k(X)|\geq c.
\end{cases}
\]

\[
\omega_k(x,y)=\begin{cases}
\frac12\frac{c}{|y|}(x^2-y^2)-cy-\frac12c^2&\text{ if }y\leq -c,\\
\frac12x^2&\text{ if }|y|<c,\\
\frac12\frac{c}{|y|}(x^2-y^2)+cy-\frac12c^2&\text{ if }y\geq +c.
\end{cases}
\] Now \(x=\delta_k-d_k(X)\) and \(y=\delta_k-d_k(Y)\)

\[
\omega_k(X;Y)=\begin{cases}
\frac12\frac{c}{|\delta_k-d_k(Y)|}\{(\delta_k-d_k(X))^2+(d_k(Y)-\delta_k)^2\}-c(\delta_k-d_k(Y))-\frac12c^2&\text{ if }\delta_k-d_k(Y)\leq -c,\\
\frac12(\delta_k-d_k(X))^2&\text{ if }|\delta_k-d_k(Y)|<c,\\
\frac12\frac{c}{|\delta_k-d_k(Y)|}\{(\delta_k-d_k(X))^2+(d_k(Y)-\delta_k)^2\}+c(\delta_k-d_k(Y))-\frac12c^2&\text{ if }\delta_k-d_k(Y)\geq +c.
\end{cases}
\] Thus the MDS majorization algorithm for the Huber loss is to update
\(Y\) by minimizing (or by performing one smacof step to decrease) \[
\sum w_k(Y)(\delta_k-d_k(X))^2
\] where \[
w_k(Y)=\begin{cases}
w_k&\text{ if }|\delta_k-d_k(Y)|<c,\\
\frac{cw_k}{|\delta_k-d_k(Y)|}&\text{ otherwise}.
\end{cases}
\]

\section{Tukey biweight}\label{tukey-biweight}

\[
f(x)=\begin{cases}
\frac{c^2}{6}\left(1-\left[1-(\frac{x}{c})^2\right]^3\right)&\text{ if }|x|\leq c,\\
\frac{c^2}{6}&\text{ otherwise }.
\end{cases}
\] \[
f'(x)=\begin{cases}
x\left[1-(\frac{x}{c})^2\right]^2&\text{ if }|x|\leq c,\\
0&\text{ otherwise }.
\end{cases}
\] It is easy to see that \(f'(x)/x\) is non-increasing on
\((0,+\infty)\). \[
w_k(Y)=\begin{cases}
\frac12w_k\left[1-(\frac{\delta_k-d_k(Y)}{c})^2\right]^2&\text{ if }|\delta_k-d_k(Y)|<c,\\
0&\text{ otherwise}.
\end{cases}
\]

\section{Convolution}\label{convolution}

In De Leeuw (\citeproc{ref-deleeuw_E_18f}{2018}) we also study the
convolution smoother proposed by Voronin, Ozkaya, and Yoshida
(\citeproc{ref-voronin_ozkaya_yoshida_15}{n.d.}). The idea is to use the
convolution of the absolute value function and a \emph{mollifier} as the
smoothed function.

\begin{quote}
A smooth function \(\psi:\mathbb{R}\rightarrow\mathbb{R}\) is said to be
a pdf if it is non-negative, and has area \(\int\psi(x)dx=1\). For any
pdf \(\psi\) and any \(c>0\), define the parametric function
\(\psi_c:\mathbb{R}\rightarrow\mathbb{R}\) by:
\(\psi_c(x):= \frac{1}{c}\psi (\frac{1}{c})\), for all
\(x\in\mathbb{R}\). Then \(\{\psi_c:c>0\}\) is a family of pdf's, whose
support decreases as \(c\rightarrow 0\), but the volume under the graph
always remains equal to one.
\end{quote}

choose a Gaussian pdf. \[
f(x)=\frac{1}{c\sqrt{2\pi}}\int_{-\infty}^{+\infty}|x-y|\exp\left\{-\frac12(\frac{y}{c})^2\right\}dy
\]

Carrying out the integration gives

\[
f(x)=x\{2\Phi(x/c)-1\}+2c\phi(x/c).
\] The derivative is \[
f'(x)=2\Phi(x/c)-1
\] It may not be immediately obvious in this case that \(f'(x)/x\) is
decreasing. We prove that its derivative is negative on \((0,+\infty)\).
The derivative of \(f'(x)/x\) has the sign of \(xf''(x)-f'(x)\), which
is \(z\phi(z)-\Phi(z)+1/2\), with \(z=x/c\). It remains to show that
\(\Phi(z)-z\phi(z)\geq\frac12\), or equivalently that
\(\int_0^z\phi(x)dx-z\phi(z)\geq 0\). Now if \(0\leq x\leq z\) then
\(\phi(x)\geq\phi(z)\) and thus
\(\int_0^z\phi(x)dx\geq\phi(z)\int_0^zdx=z\phi(z)\), which completes the
proof.

\[
w_k(Y)=
\frac{\Phi((\delta_k-d_k(Y))/c)-\frac12}{\delta_k-d_k(Y)}\\
\]

\section{Barron Loss}\label{barron-loss}

Not surprisingly there are a large number of generalizations of
Huber-like losses in the engineering community, and in their maze of
conference publications. Without having any confidence of selecting a
representative sample from the literature, we mention Barron
(\citeproc{ref-barron_17}{2017}), Barron
(\citeproc{ref-barron_19}{2019}), Gokcesu and Gokcesu
(\citeproc{ref-gokcesu_gokcesu_21}{2021}), Gokcesu and Gokcesu
(\citeproc{ref-gokcesu_gokcesu_22}{2022}). These papers also give a
large number of possibly useful references.

It is also clear that we can use any location-scale family of
probability densities to define convolution smoothers. There is an
infinite number of possible choices, with finite or infinite support,
smooth or nonsmooth, using splines or wavelets, and so on.

\section{Example}\label{example}

\section{Discussion}\label{discussion}

\subsection{Fixed weights}\label{fixed-weights}

\section{Code}\label{code}

We wrote separate programs in R (R Core Team
(\citeproc{ref-r_core_team_24}{2024})) for pseudo-Huber, Huber, Tukey,
and the convulution smoother. Each programs is about 50 lines of code,
and they differ in less tha 10 of the 50 lines. That leads to a lot of
duplicate code, obviously, and it would make sense to merge the four
programs into a single one.

\begin{Shaded}
\begin{Highlighting}[]
\NormalTok{smacofRobustPseudoHuber }\OtherTok{\textless{}{-}} \ControlFlowTok{function}\NormalTok{(delta,}
                                    \AttributeTok{weights =} \DecValTok{1} \SpecialCharTok{{-}} \FunctionTok{diag}\NormalTok{(}\FunctionTok{nrow}\NormalTok{(delta)),}
                                    \AttributeTok{ndim =} \DecValTok{2}\NormalTok{,}
                                    \AttributeTok{cons =} \DecValTok{0}\NormalTok{,}
                                    \AttributeTok{itmax =} \DecValTok{1000}\NormalTok{,}
                                    \AttributeTok{eps =} \FloatTok{1e{-}15}\NormalTok{,}
                                    \AttributeTok{verbose =} \ConstantTok{TRUE}\NormalTok{) \{}
\NormalTok{  nobj }\OtherTok{\textless{}{-}} \FunctionTok{nrow}\NormalTok{(delta)}
\NormalTok{  wmax }\OtherTok{\textless{}{-}} \FunctionTok{max}\NormalTok{(weights)}
\NormalTok{  xold }\OtherTok{\textless{}{-}} \FunctionTok{smacofTorgerson}\NormalTok{(delta, ndim)}
\NormalTok{  dold }\OtherTok{\textless{}{-}} \FunctionTok{as.matrix}\NormalTok{(}\FunctionTok{dist}\NormalTok{(xold))}
\NormalTok{  rold }\OtherTok{\textless{}{-}} \FunctionTok{sqrt}\NormalTok{((delta }\SpecialCharTok{{-}}\NormalTok{ dold) }\SpecialCharTok{\^{}} \DecValTok{2} \SpecialCharTok{+}\NormalTok{ cons)}
\NormalTok{  sold }\OtherTok{\textless{}{-}} \FunctionTok{sum}\NormalTok{(weights }\SpecialCharTok{*}\NormalTok{ rold)}
\NormalTok{  wold }\OtherTok{\textless{}{-}}\NormalTok{ weights }\SpecialCharTok{/}\NormalTok{ (rold }\SpecialCharTok{+} \FunctionTok{diag}\NormalTok{(nobj))}
\NormalTok{  itel }\OtherTok{\textless{}{-}} \DecValTok{1}
  \ControlFlowTok{repeat}\NormalTok{ \{}
\NormalTok{    vmat }\OtherTok{\textless{}{-}} \SpecialCharTok{{-}}\NormalTok{wold}
    \FunctionTok{diag}\NormalTok{(vmat) }\OtherTok{\textless{}{-}} \SpecialCharTok{{-}}\FunctionTok{rowSums}\NormalTok{(vmat)}
\NormalTok{    vinv }\OtherTok{\textless{}{-}} \FunctionTok{solve}\NormalTok{(vmat }\SpecialCharTok{+}\NormalTok{ (}\DecValTok{1} \SpecialCharTok{/}\NormalTok{ nobj)) }\SpecialCharTok{{-}}\NormalTok{ (}\DecValTok{1} \SpecialCharTok{/}\NormalTok{ nobj)}
\NormalTok{    bmat }\OtherTok{\textless{}{-}} \SpecialCharTok{{-}}\NormalTok{wold }\SpecialCharTok{*}\NormalTok{ delta }\SpecialCharTok{/}\NormalTok{ (dold }\SpecialCharTok{+} \FunctionTok{diag}\NormalTok{(nobj))}
    \FunctionTok{diag}\NormalTok{(bmat) }\OtherTok{\textless{}{-}} \SpecialCharTok{{-}}\FunctionTok{rowSums}\NormalTok{(bmat)}
\NormalTok{    xnew }\OtherTok{\textless{}{-}}\NormalTok{ vinv }\SpecialCharTok{\%*\%}\NormalTok{ (bmat }\SpecialCharTok{\%*\%}\NormalTok{ xold)}
\NormalTok{    dnew }\OtherTok{\textless{}{-}} \FunctionTok{as.matrix}\NormalTok{(}\FunctionTok{dist}\NormalTok{(xnew))}
\NormalTok{    rnew }\OtherTok{\textless{}{-}} \FunctionTok{sqrt}\NormalTok{((delta }\SpecialCharTok{{-}}\NormalTok{ dnew) }\SpecialCharTok{\^{}} \DecValTok{2} \SpecialCharTok{+}\NormalTok{ cons)}
\NormalTok{    wnew }\OtherTok{\textless{}{-}}\NormalTok{ weights }\SpecialCharTok{/}\NormalTok{ (rnew }\SpecialCharTok{+} \FunctionTok{diag}\NormalTok{(nobj))}
\NormalTok{    snew }\OtherTok{\textless{}{-}} \FunctionTok{sum}\NormalTok{(weights }\SpecialCharTok{*}\NormalTok{ rnew)}
    \ControlFlowTok{if}\NormalTok{ (verbose) \{}
      \FunctionTok{cat}\NormalTok{(}
        \StringTok{"itel "}\NormalTok{,}
        \FunctionTok{formatC}\NormalTok{(itel, }\AttributeTok{width =} \DecValTok{4}\NormalTok{, }\AttributeTok{format =} \StringTok{"d"}\NormalTok{),}
        \StringTok{"sold "}\NormalTok{,}
        \FunctionTok{formatC}\NormalTok{(sold, }\AttributeTok{digits =} \DecValTok{10}\NormalTok{, }\AttributeTok{format =} \StringTok{"f"}\NormalTok{),}
        \StringTok{"snew "}\NormalTok{,}
        \FunctionTok{formatC}\NormalTok{(snew, }\AttributeTok{digits =} \DecValTok{10}\NormalTok{, }\AttributeTok{format =} \StringTok{"f"}\NormalTok{),}
        \StringTok{"}\SpecialCharTok{\textbackslash{}n}\StringTok{"}
\NormalTok{      )}
\NormalTok{    \}}
    \ControlFlowTok{if}\NormalTok{ ((itel }\SpecialCharTok{==}\NormalTok{ itmax) }\SpecialCharTok{||}\NormalTok{ ((sold }\SpecialCharTok{{-}}\NormalTok{ snew) }\SpecialCharTok{\textless{}}\NormalTok{ eps)) \{}
      \ControlFlowTok{break}
\NormalTok{    \}}
\NormalTok{    xold }\OtherTok{\textless{}{-}}\NormalTok{ xnew}
\NormalTok{    dold }\OtherTok{\textless{}{-}}\NormalTok{ dnew}
\NormalTok{    sold }\OtherTok{\textless{}{-}}\NormalTok{ snew}
\NormalTok{    wold }\OtherTok{\textless{}{-}}\NormalTok{ wnew}
\NormalTok{    rold }\OtherTok{\textless{}{-}}\NormalTok{ rnew}
\NormalTok{    itel }\OtherTok{\textless{}{-}}\NormalTok{ itel }\SpecialCharTok{+} \DecValTok{1}
\NormalTok{  \}}
  \FunctionTok{return}\NormalTok{(}\FunctionTok{list}\NormalTok{(}
    \AttributeTok{x =}\NormalTok{ xnew,}
    \AttributeTok{s =}\NormalTok{ snew,}
    \AttributeTok{d =}\NormalTok{ dnew,}
    \AttributeTok{r =}\NormalTok{ rnew,}
    \AttributeTok{itel =}\NormalTok{ itel}
\NormalTok{  ))}
\NormalTok{\}}

\NormalTok{smacofRobustHuber }\OtherTok{\textless{}{-}} \ControlFlowTok{function}\NormalTok{(delta,}
                              \AttributeTok{weights =} \DecValTok{1} \SpecialCharTok{{-}} \FunctionTok{diag}\NormalTok{(}\FunctionTok{nrow}\NormalTok{(delta)),}
                              \AttributeTok{ndim =} \DecValTok{2}\NormalTok{,}
                              \AttributeTok{cons =} \DecValTok{0}\NormalTok{,}
                              \AttributeTok{itmax =} \DecValTok{1000}\NormalTok{,}
                              \AttributeTok{eps =} \FloatTok{1e{-}10}\NormalTok{,}
                              \AttributeTok{verbose =} \ConstantTok{TRUE}\NormalTok{) \{}
\NormalTok{  nobj }\OtherTok{\textless{}{-}} \FunctionTok{nrow}\NormalTok{(delta)}
\NormalTok{  wmax }\OtherTok{\textless{}{-}} \FunctionTok{max}\NormalTok{(weights)}
\NormalTok{  xold }\OtherTok{\textless{}{-}} \FunctionTok{smacofTorgerson}\NormalTok{(delta, ndim)}
\NormalTok{  dold }\OtherTok{\textless{}{-}} \FunctionTok{as.matrix}\NormalTok{(}\FunctionTok{dist}\NormalTok{(xold))}
\NormalTok{  fold }\OtherTok{\textless{}{-}} \FunctionTok{abs}\NormalTok{(delta }\SpecialCharTok{{-}}\NormalTok{ dold)}
\NormalTok{  rold }\OtherTok{\textless{}{-}} \FunctionTok{ifelse}\NormalTok{(fold }\SpecialCharTok{\textless{}}\NormalTok{ cons, (fold }\SpecialCharTok{\^{}} \DecValTok{2}\NormalTok{) }\SpecialCharTok{/} \DecValTok{2}\NormalTok{, cons }\SpecialCharTok{*}\NormalTok{ fold }\SpecialCharTok{{-}}\NormalTok{ (cons }\SpecialCharTok{\^{}} \DecValTok{2}\NormalTok{) }\SpecialCharTok{/} \DecValTok{2}\NormalTok{)}
\NormalTok{  sold }\OtherTok{\textless{}{-}} \FunctionTok{sum}\NormalTok{(weights }\SpecialCharTok{*}\NormalTok{ rold)}
\NormalTok{  wold }\OtherTok{\textless{}{-}} \FunctionTok{ifelse}\NormalTok{(fold }\SpecialCharTok{\textless{}}\NormalTok{ cons, weights, cons }\SpecialCharTok{*}\NormalTok{ weights }\SpecialCharTok{/}\NormalTok{ (fold }\SpecialCharTok{+} \FunctionTok{diag}\NormalTok{(nobj)))}
\NormalTok{  itel }\OtherTok{\textless{}{-}} \DecValTok{1}
  \ControlFlowTok{repeat}\NormalTok{ \{}
\NormalTok{    vmat }\OtherTok{\textless{}{-}} \SpecialCharTok{{-}}\NormalTok{wold}
    \FunctionTok{diag}\NormalTok{(vmat) }\OtherTok{\textless{}{-}} \SpecialCharTok{{-}}\FunctionTok{rowSums}\NormalTok{(vmat)}
\NormalTok{    vinv }\OtherTok{\textless{}{-}} \FunctionTok{solve}\NormalTok{(vmat }\SpecialCharTok{+}\NormalTok{ (}\DecValTok{1} \SpecialCharTok{/}\NormalTok{ nobj)) }\SpecialCharTok{{-}}\NormalTok{ (}\DecValTok{1} \SpecialCharTok{/}\NormalTok{ nobj)}
\NormalTok{    bmat }\OtherTok{\textless{}{-}} \SpecialCharTok{{-}}\NormalTok{wold }\SpecialCharTok{*}\NormalTok{ delta }\SpecialCharTok{/}\NormalTok{ (dold }\SpecialCharTok{+} \FunctionTok{diag}\NormalTok{(nobj))}
    \FunctionTok{diag}\NormalTok{(bmat) }\OtherTok{\textless{}{-}} \SpecialCharTok{{-}}\FunctionTok{rowSums}\NormalTok{(bmat)}
\NormalTok{    xnew }\OtherTok{\textless{}{-}}\NormalTok{ vinv }\SpecialCharTok{\%*\%}\NormalTok{ (bmat }\SpecialCharTok{\%*\%}\NormalTok{ xold)}
\NormalTok{    dnew }\OtherTok{\textless{}{-}} \FunctionTok{as.matrix}\NormalTok{(}\FunctionTok{dist}\NormalTok{(xnew))}
\NormalTok{    fnew }\OtherTok{\textless{}{-}} \FunctionTok{abs}\NormalTok{(delta }\SpecialCharTok{{-}}\NormalTok{ dnew)}
\NormalTok{    rnew }\OtherTok{\textless{}{-}} \FunctionTok{ifelse}\NormalTok{(fnew }\SpecialCharTok{\textless{}}\NormalTok{ cons, (fnew }\SpecialCharTok{\^{}} \DecValTok{2}\NormalTok{) }\SpecialCharTok{/} \DecValTok{2}\NormalTok{, cons }\SpecialCharTok{*}\NormalTok{ fnew }\SpecialCharTok{{-}}\NormalTok{ (cons }\SpecialCharTok{\^{}} \DecValTok{2}\NormalTok{) }\SpecialCharTok{/} \DecValTok{2}\NormalTok{)}
\NormalTok{    snew }\OtherTok{\textless{}{-}} \FunctionTok{sum}\NormalTok{(weights }\SpecialCharTok{*}\NormalTok{ rnew)}
\NormalTok{    wnew }\OtherTok{\textless{}{-}} \FunctionTok{ifelse}\NormalTok{(fnew }\SpecialCharTok{\textless{}}\NormalTok{ cons, weights, cons }\SpecialCharTok{*}\NormalTok{ weights }\SpecialCharTok{/}\NormalTok{ (fnew }\SpecialCharTok{+} \FunctionTok{diag}\NormalTok{(nobj)))}
    \ControlFlowTok{if}\NormalTok{ (verbose) \{}
      \FunctionTok{cat}\NormalTok{(}
        \StringTok{"itel "}\NormalTok{,}
        \FunctionTok{formatC}\NormalTok{(itel, }\AttributeTok{width =} \DecValTok{4}\NormalTok{, }\AttributeTok{format =} \StringTok{"d"}\NormalTok{),}
        \StringTok{"sold "}\NormalTok{,}
        \FunctionTok{formatC}\NormalTok{(sold, }\AttributeTok{digits =} \DecValTok{10}\NormalTok{, }\AttributeTok{format =} \StringTok{"f"}\NormalTok{),}
        \StringTok{"snew "}\NormalTok{,}
        \FunctionTok{formatC}\NormalTok{(snew, }\AttributeTok{digits =} \DecValTok{10}\NormalTok{, }\AttributeTok{format =} \StringTok{"f"}\NormalTok{),}
        \StringTok{"}\SpecialCharTok{\textbackslash{}n}\StringTok{"}
\NormalTok{      )}
\NormalTok{    \}}
    \ControlFlowTok{if}\NormalTok{ ((itel }\SpecialCharTok{==}\NormalTok{ itmax) }\SpecialCharTok{||}\NormalTok{ ((sold }\SpecialCharTok{{-}}\NormalTok{ snew) }\SpecialCharTok{\textless{}}\NormalTok{ eps)) \{}
      \ControlFlowTok{break}
\NormalTok{    \}}
\NormalTok{    xold }\OtherTok{\textless{}{-}}\NormalTok{ xnew}
\NormalTok{    dold }\OtherTok{\textless{}{-}}\NormalTok{ dnew}
\NormalTok{    sold }\OtherTok{\textless{}{-}}\NormalTok{ snew}
\NormalTok{    wold }\OtherTok{\textless{}{-}}\NormalTok{ wnew}
\NormalTok{    rold }\OtherTok{\textless{}{-}}\NormalTok{ rnew}
\NormalTok{    itel }\OtherTok{\textless{}{-}}\NormalTok{ itel }\SpecialCharTok{+} \DecValTok{1}
\NormalTok{  \}}
  \FunctionTok{return}\NormalTok{(}\FunctionTok{list}\NormalTok{(}
    \AttributeTok{x =}\NormalTok{ xnew,}
    \AttributeTok{s =}\NormalTok{ snew,}
    \AttributeTok{d =}\NormalTok{ dnew,}
    \AttributeTok{r =}\NormalTok{ rnew,}
    \AttributeTok{itel =}\NormalTok{ itel}
\NormalTok{  ))}
\NormalTok{\}}

\NormalTok{smacofRobustTukey }\OtherTok{\textless{}{-}} \ControlFlowTok{function}\NormalTok{(delta,}
                              \AttributeTok{weights =} \DecValTok{1} \SpecialCharTok{{-}} \FunctionTok{diag}\NormalTok{(}\FunctionTok{nrow}\NormalTok{(delta)),}
                              \AttributeTok{ndim =} \DecValTok{2}\NormalTok{,}
                              \AttributeTok{cons =} \DecValTok{0}\NormalTok{,}
                              \AttributeTok{itmax =} \DecValTok{1000}\NormalTok{,}
                              \AttributeTok{eps =} \FloatTok{1e{-}10}\NormalTok{,}
                              \AttributeTok{verbose =} \ConstantTok{TRUE}\NormalTok{) \{}
\NormalTok{  nobj }\OtherTok{\textless{}{-}} \FunctionTok{nrow}\NormalTok{(delta)}
\NormalTok{  wmax }\OtherTok{\textless{}{-}} \FunctionTok{max}\NormalTok{(weights)}
\NormalTok{  xold }\OtherTok{\textless{}{-}} \FunctionTok{smacofTorgerson}\NormalTok{(delta, ndim)}
\NormalTok{  dold }\OtherTok{\textless{}{-}} \FunctionTok{as.matrix}\NormalTok{(}\FunctionTok{dist}\NormalTok{(xold))}
\NormalTok{  fold }\OtherTok{\textless{}{-}}\NormalTok{ delta }\SpecialCharTok{{-}}\NormalTok{ dold}
\NormalTok{  rold }\OtherTok{\textless{}{-}}\NormalTok{ ((cons }\SpecialCharTok{\^{}} \DecValTok{2}\NormalTok{) }\SpecialCharTok{/} \DecValTok{6}\NormalTok{) }\SpecialCharTok{*} \FunctionTok{ifelse}\NormalTok{(}\FunctionTok{abs}\NormalTok{(fold) }\SpecialCharTok{\textless{}}\NormalTok{ cons, (}\DecValTok{1} \SpecialCharTok{{-}}\NormalTok{ (}\DecValTok{1} \SpecialCharTok{{-}}\NormalTok{ (fold }\SpecialCharTok{/}\NormalTok{ cons) }\SpecialCharTok{\^{}} \DecValTok{2}\NormalTok{) }\SpecialCharTok{\^{}} \DecValTok{3}\NormalTok{), }\DecValTok{1}\NormalTok{)}
\NormalTok{  sold }\OtherTok{\textless{}{-}} \FunctionTok{sum}\NormalTok{(weights }\SpecialCharTok{*}\NormalTok{ rold)}
\NormalTok{  wold }\OtherTok{\textless{}{-}} \FunctionTok{ifelse}\NormalTok{(}\FunctionTok{abs}\NormalTok{(fold) }\SpecialCharTok{\textless{}}\NormalTok{ cons, weights }\SpecialCharTok{*}\NormalTok{ (}\DecValTok{1} \SpecialCharTok{{-}}\NormalTok{ (fold }\SpecialCharTok{/}\NormalTok{ cons) }\SpecialCharTok{\^{}} \DecValTok{2}\NormalTok{) }\SpecialCharTok{\^{}} \DecValTok{2}\NormalTok{, }\DecValTok{0}\NormalTok{) }\SpecialCharTok{/} \DecValTok{2}
\NormalTok{  itel }\OtherTok{\textless{}{-}} \DecValTok{1}
  \ControlFlowTok{repeat}\NormalTok{ \{}
\NormalTok{    vmat }\OtherTok{\textless{}{-}} \SpecialCharTok{{-}}\NormalTok{wold}
    \FunctionTok{diag}\NormalTok{(vmat) }\OtherTok{\textless{}{-}} \SpecialCharTok{{-}}\FunctionTok{rowSums}\NormalTok{(vmat)}
\NormalTok{    vinv }\OtherTok{\textless{}{-}} \FunctionTok{solve}\NormalTok{(vmat }\SpecialCharTok{+}\NormalTok{ (}\DecValTok{1} \SpecialCharTok{/}\NormalTok{ nobj)) }\SpecialCharTok{{-}}\NormalTok{ (}\DecValTok{1} \SpecialCharTok{/}\NormalTok{ nobj)}
\NormalTok{    bmat }\OtherTok{\textless{}{-}} \SpecialCharTok{{-}}\NormalTok{wold }\SpecialCharTok{*}\NormalTok{ delta }\SpecialCharTok{/}\NormalTok{ (dold }\SpecialCharTok{+} \FunctionTok{diag}\NormalTok{(nobj))}
    \FunctionTok{diag}\NormalTok{(bmat) }\OtherTok{\textless{}{-}} \SpecialCharTok{{-}}\FunctionTok{rowSums}\NormalTok{(bmat)}
\NormalTok{    xnew }\OtherTok{\textless{}{-}}\NormalTok{ vinv }\SpecialCharTok{\%*\%}\NormalTok{ (bmat }\SpecialCharTok{\%*\%}\NormalTok{ xold)}
\NormalTok{    dnew }\OtherTok{\textless{}{-}} \FunctionTok{as.matrix}\NormalTok{(}\FunctionTok{dist}\NormalTok{(xnew))}
\NormalTok{    fnew }\OtherTok{\textless{}{-}}\NormalTok{ delta }\SpecialCharTok{{-}}\NormalTok{ dnew}
\NormalTok{    rnew }\OtherTok{\textless{}{-}}\NormalTok{ ((cons }\SpecialCharTok{\^{}} \DecValTok{2}\NormalTok{) }\SpecialCharTok{/} \DecValTok{6}\NormalTok{) }\SpecialCharTok{*} \FunctionTok{ifelse}\NormalTok{(}\FunctionTok{abs}\NormalTok{(fnew) }\SpecialCharTok{\textless{}}\NormalTok{ cons, (}\DecValTok{1} \SpecialCharTok{{-}}\NormalTok{ (}\DecValTok{1} \SpecialCharTok{{-}}\NormalTok{ (fnew }\SpecialCharTok{/}\NormalTok{ cons) }\SpecialCharTok{\^{}} \DecValTok{2}\NormalTok{) }\SpecialCharTok{\^{}} \DecValTok{3}\NormalTok{), }\DecValTok{1}\NormalTok{)}
\NormalTok{    snew }\OtherTok{\textless{}{-}} \FunctionTok{sum}\NormalTok{(weights }\SpecialCharTok{*}\NormalTok{ rnew)}
\NormalTok{    wnew }\OtherTok{\textless{}{-}} \FunctionTok{ifelse}\NormalTok{(}\FunctionTok{abs}\NormalTok{(fnew) }\SpecialCharTok{\textless{}}\NormalTok{ cons, weights }\SpecialCharTok{*}\NormalTok{ (}\DecValTok{1} \SpecialCharTok{{-}}\NormalTok{ (fnew }\SpecialCharTok{/}\NormalTok{ cons) }\SpecialCharTok{\^{}} \DecValTok{2}\NormalTok{) }\SpecialCharTok{\^{}} \DecValTok{2}\NormalTok{, }\DecValTok{0}\NormalTok{) }\SpecialCharTok{/} \DecValTok{2}
    \ControlFlowTok{if}\NormalTok{ (verbose) \{}
      \FunctionTok{cat}\NormalTok{(}
        \StringTok{"itel "}\NormalTok{,}
        \FunctionTok{formatC}\NormalTok{(itel, }\AttributeTok{width =} \DecValTok{4}\NormalTok{, }\AttributeTok{format =} \StringTok{"d"}\NormalTok{),}
        \StringTok{"sold "}\NormalTok{,}
        \FunctionTok{formatC}\NormalTok{(sold, }\AttributeTok{digits =} \DecValTok{10}\NormalTok{, }\AttributeTok{format =} \StringTok{"f"}\NormalTok{),}
        \StringTok{"snew "}\NormalTok{,}
        \FunctionTok{formatC}\NormalTok{(snew, }\AttributeTok{digits =} \DecValTok{10}\NormalTok{, }\AttributeTok{format =} \StringTok{"f"}\NormalTok{),}
        \StringTok{"}\SpecialCharTok{\textbackslash{}n}\StringTok{"}
\NormalTok{      )}
\NormalTok{    \}}
    \ControlFlowTok{if}\NormalTok{ ((itel }\SpecialCharTok{==}\NormalTok{ itmax) }\SpecialCharTok{||}\NormalTok{ ((sold }\SpecialCharTok{{-}}\NormalTok{ snew) }\SpecialCharTok{\textless{}}\NormalTok{ eps)) \{}
      \ControlFlowTok{break}
\NormalTok{    \}}
\NormalTok{    xold }\OtherTok{\textless{}{-}}\NormalTok{ xnew}
\NormalTok{    dold }\OtherTok{\textless{}{-}}\NormalTok{ dnew}
\NormalTok{    sold }\OtherTok{\textless{}{-}}\NormalTok{ snew}
\NormalTok{    wold }\OtherTok{\textless{}{-}}\NormalTok{ wnew}
\NormalTok{    rold }\OtherTok{\textless{}{-}}\NormalTok{ rnew}
\NormalTok{    itel }\OtherTok{\textless{}{-}}\NormalTok{ itel }\SpecialCharTok{+} \DecValTok{1}
\NormalTok{  \}}
  \FunctionTok{return}\NormalTok{(}\FunctionTok{list}\NormalTok{(}
    \AttributeTok{x =}\NormalTok{ xnew,}
    \AttributeTok{s =}\NormalTok{ snew,}
    \AttributeTok{d =}\NormalTok{ dnew,}
    \AttributeTok{r =}\NormalTok{ rnew,}
    \AttributeTok{itel =}\NormalTok{ itel}
\NormalTok{  ))}
\NormalTok{\}}

\NormalTok{smacofRobustConvolution }\OtherTok{\textless{}{-}} \ControlFlowTok{function}\NormalTok{(delta,}
                              \AttributeTok{weights =} \DecValTok{1} \SpecialCharTok{{-}} \FunctionTok{diag}\NormalTok{(}\FunctionTok{nrow}\NormalTok{(delta)),}
                              \AttributeTok{ndim =} \DecValTok{2}\NormalTok{,}
                              \AttributeTok{cons =} \DecValTok{0}\NormalTok{,}
                              \AttributeTok{itmax =} \DecValTok{1000}\NormalTok{,}
                              \AttributeTok{eps =} \FloatTok{1e{-}10}\NormalTok{,}
                              \AttributeTok{verbose =} \ConstantTok{TRUE}\NormalTok{) \{}
\NormalTok{  nobj }\OtherTok{\textless{}{-}} \FunctionTok{nrow}\NormalTok{(delta)}
\NormalTok{  wmax }\OtherTok{\textless{}{-}} \FunctionTok{max}\NormalTok{(weights)}
\NormalTok{  xold }\OtherTok{\textless{}{-}} \FunctionTok{smacofTorgerson}\NormalTok{(delta, ndim)}
\NormalTok{  dold }\OtherTok{\textless{}{-}} \FunctionTok{as.matrix}\NormalTok{(}\FunctionTok{dist}\NormalTok{(xold))}
\NormalTok{  fold }\OtherTok{\textless{}{-}}\NormalTok{ delta }\SpecialCharTok{{-}}\NormalTok{ dold}
\NormalTok{  rold }\OtherTok{\textless{}{-}}\NormalTok{ fold }\SpecialCharTok{*}\NormalTok{ (}\DecValTok{2} \SpecialCharTok{*} \FunctionTok{pnorm}\NormalTok{(fold }\SpecialCharTok{/}\NormalTok{ cons) }\SpecialCharTok{{-}} \DecValTok{1}\NormalTok{) }\SpecialCharTok{+} \DecValTok{2} \SpecialCharTok{*}\NormalTok{ cons }\SpecialCharTok{*} \FunctionTok{dnorm}\NormalTok{(fold }\SpecialCharTok{/}\NormalTok{ cons)}
\NormalTok{  sold }\OtherTok{\textless{}{-}} \FunctionTok{sum}\NormalTok{(weights }\SpecialCharTok{*}\NormalTok{ rold)}
\NormalTok{  wold }\OtherTok{\textless{}{-}}\NormalTok{ (}\FunctionTok{pnorm}\NormalTok{(fold }\SpecialCharTok{/}\NormalTok{ cons) }\SpecialCharTok{{-}} \FloatTok{0.5}\NormalTok{) }\SpecialCharTok{/}\NormalTok{ (fold }\SpecialCharTok{+} \FunctionTok{diag}\NormalTok{(nobj))}
\NormalTok{  itel }\OtherTok{\textless{}{-}} \DecValTok{1}
  \ControlFlowTok{repeat}\NormalTok{ \{}
\NormalTok{    vmat }\OtherTok{\textless{}{-}} \SpecialCharTok{{-}}\NormalTok{wold}
    \FunctionTok{diag}\NormalTok{(vmat) }\OtherTok{\textless{}{-}} \SpecialCharTok{{-}}\FunctionTok{rowSums}\NormalTok{(vmat)}
\NormalTok{    vinv }\OtherTok{\textless{}{-}} \FunctionTok{solve}\NormalTok{(vmat }\SpecialCharTok{+}\NormalTok{ (}\DecValTok{1} \SpecialCharTok{/}\NormalTok{ nobj)) }\SpecialCharTok{{-}}\NormalTok{ (}\DecValTok{1} \SpecialCharTok{/}\NormalTok{ nobj)}
\NormalTok{    bmat }\OtherTok{\textless{}{-}} \SpecialCharTok{{-}}\NormalTok{wold }\SpecialCharTok{*}\NormalTok{ delta }\SpecialCharTok{/}\NormalTok{ (dold }\SpecialCharTok{+} \FunctionTok{diag}\NormalTok{(nobj))}
    \FunctionTok{diag}\NormalTok{(bmat) }\OtherTok{\textless{}{-}} \SpecialCharTok{{-}}\FunctionTok{rowSums}\NormalTok{(bmat)}
\NormalTok{    xnew }\OtherTok{\textless{}{-}}\NormalTok{ vinv }\SpecialCharTok{\%*\%}\NormalTok{ (bmat }\SpecialCharTok{\%*\%}\NormalTok{ xold)}
\NormalTok{    dnew }\OtherTok{\textless{}{-}} \FunctionTok{as.matrix}\NormalTok{(}\FunctionTok{dist}\NormalTok{(xnew))}
\NormalTok{    fnew }\OtherTok{\textless{}{-}}\NormalTok{ delta }\SpecialCharTok{{-}}\NormalTok{ dnew}
\NormalTok{    rnew }\OtherTok{\textless{}{-}}\NormalTok{ fnew }\SpecialCharTok{*}\NormalTok{ (}\DecValTok{2} \SpecialCharTok{*} \FunctionTok{pnorm}\NormalTok{(fnew }\SpecialCharTok{/}\NormalTok{ cons) }\SpecialCharTok{{-}} \DecValTok{1}\NormalTok{) }\SpecialCharTok{+} \DecValTok{2} \SpecialCharTok{*}\NormalTok{ cons }\SpecialCharTok{*} \FunctionTok{dnorm}\NormalTok{(fnew }\SpecialCharTok{/}\NormalTok{ cons)}
\NormalTok{    snew }\OtherTok{\textless{}{-}} \FunctionTok{sum}\NormalTok{(weights }\SpecialCharTok{*}\NormalTok{ rnew)}
\NormalTok{    wnew }\OtherTok{\textless{}{-}}\NormalTok{ (}\FunctionTok{pnorm}\NormalTok{(fnew }\SpecialCharTok{/}\NormalTok{ cons) }\SpecialCharTok{{-}} \FloatTok{0.5}\NormalTok{) }\SpecialCharTok{/}\NormalTok{ (fnew }\SpecialCharTok{+} \FunctionTok{diag}\NormalTok{(nobj))}
    \ControlFlowTok{if}\NormalTok{ (verbose) \{}
      \FunctionTok{cat}\NormalTok{(}
        \StringTok{"itel "}\NormalTok{,}
        \FunctionTok{formatC}\NormalTok{(itel, }\AttributeTok{width =} \DecValTok{4}\NormalTok{, }\AttributeTok{format =} \StringTok{"d"}\NormalTok{),}
        \StringTok{"sold "}\NormalTok{,}
        \FunctionTok{formatC}\NormalTok{(sold, }\AttributeTok{digits =} \DecValTok{10}\NormalTok{, }\AttributeTok{format =} \StringTok{"f"}\NormalTok{),}
        \StringTok{"snew "}\NormalTok{,}
        \FunctionTok{formatC}\NormalTok{(snew, }\AttributeTok{digits =} \DecValTok{10}\NormalTok{, }\AttributeTok{format =} \StringTok{"f"}\NormalTok{),}
        \StringTok{"}\SpecialCharTok{\textbackslash{}n}\StringTok{"}
\NormalTok{      )}
\NormalTok{    \}}
    \ControlFlowTok{if}\NormalTok{ ((itel }\SpecialCharTok{==}\NormalTok{ itmax) }\SpecialCharTok{||}\NormalTok{ ((sold }\SpecialCharTok{{-}}\NormalTok{ snew) }\SpecialCharTok{\textless{}}\NormalTok{ eps)) \{}
      \ControlFlowTok{break}
\NormalTok{    \}}
\NormalTok{    xold }\OtherTok{\textless{}{-}}\NormalTok{ xnew}
\NormalTok{    dold }\OtherTok{\textless{}{-}}\NormalTok{ dnew}
\NormalTok{    sold }\OtherTok{\textless{}{-}}\NormalTok{ snew}
\NormalTok{    wold }\OtherTok{\textless{}{-}}\NormalTok{ wnew}
\NormalTok{    rold }\OtherTok{\textless{}{-}}\NormalTok{ rnew}
\NormalTok{    itel }\OtherTok{\textless{}{-}}\NormalTok{ itel }\SpecialCharTok{+} \DecValTok{1}
\NormalTok{  \}}
  \FunctionTok{return}\NormalTok{(}\FunctionTok{list}\NormalTok{(}
    \AttributeTok{x =}\NormalTok{ xnew,}
    \AttributeTok{s =}\NormalTok{ snew,}
    \AttributeTok{d =}\NormalTok{ dnew,}
    \AttributeTok{r =}\NormalTok{ rnew,}
    \AttributeTok{itel =}\NormalTok{ itel}
\NormalTok{  ))}
\NormalTok{\}}

\NormalTok{smacofTorgerson }\OtherTok{\textless{}{-}} \ControlFlowTok{function}\NormalTok{(delta, ndim) \{}
\NormalTok{  dd }\OtherTok{\textless{}{-}}\NormalTok{ delta }\SpecialCharTok{\^{}} \DecValTok{2}
\NormalTok{  rd }\OtherTok{\textless{}{-}} \FunctionTok{apply}\NormalTok{(dd, }\DecValTok{1}\NormalTok{, mean)}
\NormalTok{  md }\OtherTok{\textless{}{-}} \FunctionTok{mean}\NormalTok{(dd)}
\NormalTok{  sd }\OtherTok{\textless{}{-}} \SpecialCharTok{{-}}\NormalTok{.}\DecValTok{5} \SpecialCharTok{*}\NormalTok{ (dd }\SpecialCharTok{{-}} \FunctionTok{outer}\NormalTok{(rd, rd, }\StringTok{"+"}\NormalTok{) }\SpecialCharTok{+}\NormalTok{ md)}
\NormalTok{  ed }\OtherTok{\textless{}{-}} \FunctionTok{eigen}\NormalTok{(sd)}
  \FunctionTok{return}\NormalTok{(ed}\SpecialCharTok{$}\NormalTok{vectors[, }\DecValTok{1}\SpecialCharTok{:}\NormalTok{ndim] }\SpecialCharTok{\%*\%} \FunctionTok{diag}\NormalTok{(}\FunctionTok{sqrt}\NormalTok{(ed}\SpecialCharTok{$}\NormalTok{values[}\DecValTok{1}\SpecialCharTok{:}\NormalTok{ndim])))}
\NormalTok{\}}
\end{Highlighting}
\end{Shaded}

\section*{References}\label{references}
\addcontentsline{toc}{section}{References}

\phantomsection\label{refs}
\begin{CSLReferences}{1}{0}
\bibitem[\citeproctext]{ref-barron_17}
Barron, J. T. 2017. {``A More General Robust Loss Function.''}
\url{https://arxiv.org/pdf/1701.03077v1}.

\bibitem[\citeproctext]{ref-barron_19}
---------. 2019. {``A General and Adaptive Robust Loss Function.''} In
\emph{Proceedings 2019 IEEE/CVF Conferehce on Computer Vision and
Pattern Recognition}, 4331--39.
\url{https://openaccess.thecvf.com/content_CVPR_2019/papers/Barron_A_General_and_Adaptive_Robust_Loss_Function_CVPR_2019_paper.pdf}.

\bibitem[\citeproctext]{ref-deleeuw_A_84f}
De Leeuw, J. 1984. {``{Differentiability of Kruskal's Stress at a Local
Minimum}.''} \emph{Psychometrika} 49: 111--13.

\bibitem[\citeproctext]{ref-deleeuw_C_94c}
---------. 1994. {``{Block Relaxation Algorithms in Statistics}.''} In
\emph{Information Systems and Data Analysis}, edited by H. H. Bock, W.
Lenski, and M. M. Richter, 308--24. Berlin: Springer Verlag.
\url{https://jansweb.netlify.app/publication/deleeuw-c-94-c/deleeuw-c-94-c.pdf}.

\bibitem[\citeproctext]{ref-deleeuw_E_18f}
---------. 2018. {``{MM Algorithms for Smoothed Absolute Values}.''}
2018.

\bibitem[\citeproctext]{ref-deleeuw_lange_A_09}
De Leeuw, J., and K. Lange. 2009. {``Sharp Quadratic Majorization in One
Dimension.''} \emph{Computational Statistics and Data Analysis} 53:
2471--84.

\bibitem[\citeproctext]{ref-gokcesu_gokcesu_21}
Gokcesu, K., and H. Gokcesu. 2021. {``Generalized Huber Loss for Robust
Learning and Its Efficient Minimization for a Robust Statistics.''}
\url{https://arxiv.org/abs/2108.12627}.

\bibitem[\citeproctext]{ref-gokcesu_gokcesu_22}
---------. 2022. {``Nonconvex Extension of Generalized Huber Loss for
Robust Learning and Pseudo-Mode Statistics.''} arXiv:2202.11141v1
{[}stat.ML{]}. \url{https://arxiv.org/abs/2202.11141}.

\bibitem[\citeproctext]{ref-groenen_heiser_meulman_99}
Groenen, P. J. F., W. J. Heiser, and J. J. Meulman. 1999. {``{Global
Optimization in Least-Squares Multidimensional Scaling by Distance
Smoothing}.''} \emph{Journal of Classification} 16: 225--54.

\bibitem[\citeproctext]{ref-heiser_87}
Heiser, W. J. 1987. {``{Correspondence Analysis with Least Absolute
Residuals}.''} \emph{Computational Statistica and Data Analysis} 5:
337--56.

\bibitem[\citeproctext]{ref-heiser_88}
---------. 1988. {``{Multidimensional Scaling with Least Absolute
Residuals}.''} In \emph{Classification and Related Methods of Data
Analysis}, edited by H. H. Bock, 455--62. North-Holland Publishing Co.

\bibitem[\citeproctext]{ref-huber_64}
Huber, P. J. 1964. {``Robust Estimation of a Location Parameter.''}
\emph{Annals of Mathematical Statistics} 35 (1): 73--101.

\bibitem[\citeproctext]{ref-lange_16}
Lange, K. 2016. \emph{MM Optimization Algorithms}. SIAM.

\bibitem[\citeproctext]{ref-pliner_96}
Pliner, V. 1996. {``{Metric Unidimensional Scaling and Global
Optimization}.''} \emph{Journal of Classification} 13: 3--18.

\bibitem[\citeproctext]{ref-r_core_team_24}
R Core Team. 2024. \emph{R: A Language and Environment for Statistical
Computing}. {Vienna, Austria}: R Foundation for Statistical Computing.
\url{https://www.R-project.org/}.

\bibitem[\citeproctext]{ref-voronin_ozkaya_yoshida_15}
Voronin, S., G. Ozkaya, and Y. Yoshida. n.d. {``{Convolution Based
Smooth Approximations to the Absolute Value Function with Application to
Non-smooth Regularization}.''}

\end{CSLReferences}




\end{document}
