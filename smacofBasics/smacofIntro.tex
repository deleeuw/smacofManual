% Options for packages loaded elsewhere
\PassOptionsToPackage{unicode}{hyperref}
\PassOptionsToPackage{hyphens}{url}
%
\documentclass[
  12pt,
]{article}
\usepackage{amsmath,amssymb}
\usepackage{iftex}
\ifPDFTeX
  \usepackage[T1]{fontenc}
  \usepackage[utf8]{inputenc}
  \usepackage{textcomp} % provide euro and other symbols
\else % if luatex or xetex
  \usepackage{unicode-math} % this also loads fontspec
  \defaultfontfeatures{Scale=MatchLowercase}
  \defaultfontfeatures[\rmfamily]{Ligatures=TeX,Scale=1}
\fi
\usepackage{lmodern}
\ifPDFTeX\else
  % xetex/luatex font selection
    \setmainfont[]{Times New Roman}
\fi
% Use upquote if available, for straight quotes in verbatim environments
\IfFileExists{upquote.sty}{\usepackage{upquote}}{}
\IfFileExists{microtype.sty}{% use microtype if available
  \usepackage[]{microtype}
  \UseMicrotypeSet[protrusion]{basicmath} % disable protrusion for tt fonts
}{}
\makeatletter
\@ifundefined{KOMAClassName}{% if non-KOMA class
  \IfFileExists{parskip.sty}{%
    \usepackage{parskip}
  }{% else
    \setlength{\parindent}{0pt}
    \setlength{\parskip}{6pt plus 2pt minus 1pt}}
}{% if KOMA class
  \KOMAoptions{parskip=half}}
\makeatother
\usepackage{xcolor}
\usepackage[margin=1in]{geometry}
\usepackage{longtable,booktabs,array}
\usepackage{calc} % for calculating minipage widths
% Correct order of tables after \paragraph or \subparagraph
\usepackage{etoolbox}
\makeatletter
\patchcmd\longtable{\par}{\if@noskipsec\mbox{}\fi\par}{}{}
\makeatother
% Allow footnotes in longtable head/foot
\IfFileExists{footnotehyper.sty}{\usepackage{footnotehyper}}{\usepackage{footnote}}
\makesavenoteenv{longtable}
\usepackage{graphicx}
\makeatletter
\def\maxwidth{\ifdim\Gin@nat@width>\linewidth\linewidth\else\Gin@nat@width\fi}
\def\maxheight{\ifdim\Gin@nat@height>\textheight\textheight\else\Gin@nat@height\fi}
\makeatother
% Scale images if necessary, so that they will not overflow the page
% margins by default, and it is still possible to overwrite the defaults
% using explicit options in \includegraphics[width, height, ...]{}
\setkeys{Gin}{width=\maxwidth,height=\maxheight,keepaspectratio}
% Set default figure placement to htbp
\makeatletter
\def\fps@figure{htbp}
\makeatother
\setlength{\emergencystretch}{3em} % prevent overfull lines
\providecommand{\tightlist}{%
  \setlength{\itemsep}{0pt}\setlength{\parskip}{0pt}}
\setcounter{secnumdepth}{5}
% definitions for citeproc citations
\NewDocumentCommand\citeproctext{}{}
\NewDocumentCommand\citeproc{mm}{%
  \begingroup\def\citeproctext{#2}\cite{#1}\endgroup}
\makeatletter
 % allow citations to break across lines
 \let\@cite@ofmt\@firstofone
 % avoid brackets around text for \cite:
 \def\@biblabel#1{}
 \def\@cite#1#2{{#1\if@tempswa , #2\fi}}
\makeatother
\newlength{\cslhangindent}
\setlength{\cslhangindent}{1.5em}
\newlength{\csllabelwidth}
\setlength{\csllabelwidth}{3em}
\newenvironment{CSLReferences}[2] % #1 hanging-indent, #2 entry-spacing
 {\begin{list}{}{%
  \setlength{\itemindent}{0pt}
  \setlength{\leftmargin}{0pt}
  \setlength{\parsep}{0pt}
  % turn on hanging indent if param 1 is 1
  \ifodd #1
   \setlength{\leftmargin}{\cslhangindent}
   \setlength{\itemindent}{-1\cslhangindent}
  \fi
  % set entry spacing
  \setlength{\itemsep}{#2\baselineskip}}}
 {\end{list}}
\usepackage{calc}
\newcommand{\CSLBlock}[1]{\hfill\break\parbox[t]{\linewidth}{\strut\ignorespaces#1\strut}}
\newcommand{\CSLLeftMargin}[1]{\parbox[t]{\csllabelwidth}{\strut#1\strut}}
\newcommand{\CSLRightInline}[1]{\parbox[t]{\linewidth - \csllabelwidth}{\strut#1\strut}}
\newcommand{\CSLIndent}[1]{\hspace{\cslhangindent}#1}
\usepackage{tcolorbox}
\usepackage{amssymb}
\usepackage{yfonts}
\usepackage{bm}
\usepackage{titlesec}
\usepackage{kbordermatrix}


\newtcolorbox{greybox}{
  colback=white,
  colframe=blue,
  coltext=black,
  boxsep=5pt,
  arc=4pt}
  
\newcommand{\sectionbreak}{\clearpage}

 
\newcommand{\ds}[4]{\sum_{{#1}=1}^{#3}\sum_{{#2}=1}^{#4}}
\newcommand{\us}[3]{\mathop{\sum\sum}_{1\leq{#2}<{#1}\leq{#3}}}

\newcommand{\ol}[1]{\overline{#1}}
\newcommand{\ul}[1]{\underline{#1}}

\newcommand{\amin}[1]{\mathop{\text{argmin}}_{#1}}
\newcommand{\amax}[1]{\mathop{\text{argmax}}_{#1}}

\newcommand{\ci}{\perp\!\!\!\perp}

\newcommand{\mc}[1]{\mathcal{#1}}
\newcommand{\mb}[1]{\mathbb{#1}}
\newcommand{\mf}[1]{\mathfrak{#1}}

\newcommand{\eps}{\epsilon}
\newcommand{\lbd}{\lambda}
\newcommand{\alp}{\alpha}
\newcommand{\df}{=:}
\newcommand{\am}[1]{\mathop{\text{argmin}}_{#1}}
\newcommand{\ls}[2]{\mathop{\sum\sum}_{#1}^{#2}}
\newcommand{\ijs}{\mathop{\sum\sum}_{1\leq i<j\leq n}}
\newcommand{\jis}{\mathop{\sum\sum}_{1\leq j<i\leq n}}
\newcommand{\sij}{\sum_{i=1}^n\sum_{j=1}^n}
	
\ifLuaTeX
  \usepackage{selnolig}  % disable illegal ligatures
\fi
\usepackage{bookmark}
\IfFileExists{xurl.sty}{\usepackage{xurl}}{} % add URL line breaks if available
\urlstyle{same}
\hypersetup{
  pdfauthor={Jan de Leeuw - University of California Los Angeles},
  hidelinks,
  pdfcreator={LaTeX via pandoc}}

\title{Smacof at 50: A Manual\\
Part 1: Smacof Notation and Theory}
\author{Jan de Leeuw - University of California Los Angeles}
\date{Started February 21 2024, Version of April 16, 2024}

\begin{document}
\maketitle
\begin{abstract}
TBD
\end{abstract}

{
\setcounter{tocdepth}{4}
\tableofcontents
}
\textbf{Note:} This is a working manuscript which will be expanded/updated
frequently. All suggestions for improvement are welcome. All Rmd, tex,
html, pdf, R, and C files are in the public domain. Attribution will be
appreciated, but is not required. The files can be found at
\url{https://github.com/deleeuw/smacofCode}.

\sectionbreak

\section{Introduction}\label{introduction}

In \emph{Multidimensional Scaling (MDS)} the data consists of information
about the similarity or dissimilarity between pairs of objects selected
from a finite set \(\mathcal{O}=\{o_1,\cdots,o_n\}\).

In \emph{metric MDS} we have numerical dissimilarity measures and we want to
map the objects \(o_i\) into \(n\) points \(x_i\) of some metric space in such
a way that the distances between the points approximate the
dissimilarities between the objects. In \emph{smacof}, our framework for MDS
theory, algorithms, and computer programs, the metric space is
\(\mathbb{R}^p\), the space of all \(p\)-tuples of real numbers, and in the
code documented in this manual we assume the distance is the usual
Euclidean distance.

In \emph{non-metric MDS} the information about the dissimilarities is
incomplete. It is usually \emph{ordinal}, i.e.~it tells us in some way or
another that some dissimilarties are larger or smaller than others.
Somewhere between metric and non-metric MDS is MDS with \emph{missing data},
in which some dissimilarities are known numbers while others are
unknown. MDS with missing data is a form of \emph{distance matrix completion}
(Fang and O'Leary (2012)).

\section{Krukal's Stress}\label{krukals-stress}

In the pioneering papers Kruskal (1964a) and Kruskal (1964b) the MDS problem
was formulated for the first time as minimization of an explicit loss
function, which measures the quality of the approximation of the
dissimilarities by the distances.

\subsection{Metric MDS}\label{metric-mds}

The loss function in least squares
metric Euclidean MDS is called \emph{raw stress} or \emph{Kruskal's raw stress}
and is defined as
\begin{equation}
\sigma_R(X):=\frac12\sum w_{ij}(\delta_{ij}-d_{ij}(X))^2.
\label{eq:stressdef}
\end{equation}

The symbol \(:=\) is used for definitions. In definition
\eqref{eq:stressdef} the \(w_{ij}\) are known non-negative \emph{weights}, the
\(\delta_{ij}\) are the known non-negative \emph{dissimilarities} between
objects \(o_i\) and \(o_j\), and the \(d_{ij}(X)\) are the \emph{distances} between
the corresponding points \(x_i\) and \(x_j\). The summation is over all
\(\binom{n}{2}\) pairs \((i,j)\) with \(j>i\), i.e.~over elements below the
diagonal of the matrices \(W\) and \(\Delta\). The subscript \(R\) in \(\sigma_R\)
stands for ``raw''. From now on we use ``metric
MDS'' to mean Least Squares Metric Euclidean MDS.

The \(n\times p\) matrix \(X\), which has the coordinates \(x_i\) of the \(n\)
points as its rows, is called the \emph{configuration}, where \(p\) is the
\emph{dimension} of the Euclidean space in which we make the map. Thus
\begin{equation}
d_{ij}(X)=\sqrt{\sum_{s=1}^p(x_{is}-x_{js})^2}.
\label{eq:ddef}
\end{equation} The metric MDS problem (of dimension \(p\), for given \(W\)
and \(\Delta\)) is the minimization of \eqref{eq:stressdef} over the
\(n\times p\) configurations \(X\).

The weights \(w_{ij}\) can be used to quantify information about the
precision or importance of the corresponding dissimilarities. Some of
the weights may be zero, which can be used to code \emph{missing data}. If
all weights are positive we have \emph{complete data}. If we have complete
data, and all weights are equal to one, we have \emph{unweighted} metric MDS.
The pioneering papers by Shepard, Kruskal, and Guttman only consider the unweighted case. Weights were only introduced in MDS in De Leeuw (1977).

We assume throughout that the weights are \emph{irreducible} (De Leeuw (1977)).
This means there is no partitioning of the index set
\(I_n:=\{1,2,\cdots,n\}\) into subsets for which all between-subset
weights are zero. A reducible metric MDS problems decomposes into a
number of smaller independent metric MDS problems, so the irreducibility
assumption causes no real loss of generality.

The fact that the summation in \eqref{eq:stressdef} is over all \(j<i\)
indicates that the diagonal elements of \(\Delta\) are not used (they are
assumed to be zero) and the elements above the diagonal are not used as
well (they are assumed to be equal to the corresponding elements below
the diagonal). The somewhat mysterious factor \(\frac12\) in definition
\eqref{eq:stressdef} is there because it simplifies some of the formulas
in later sections of this paper.

\subsection{Non-metric MDS}\label{non-metric-mds}

Kruskal was not really interested in metric MDS and the ``raw'' loss
function \eqref{eq:stressdef}. His papers are really about non-metric
MDS, by which we mean least squares non-metric Euclidean MDS. Non-metric
MDS differs from metric MDS because we have incomplete information about
the dissimilarities. As we have seen, that if some dissimilarities are
missing metric MDS can handle this by using zero weights. In some
situations, however, we only know the rank order of the non-missing
dissimilarities. We do not know, or we refuse to use, their actual
numeric values. Or, to put it differently, even if we have numerical
dissimilarities we are looking for a \emph{transformation} of the non-missing
dissimilarities, where the transformation is chosen from a set of
admissible transformations (for instance from all linear or monotone
transformations). If the dissimilarities are non-numerical, for example
rank orders or partitionings, we choose from the set of admissible
\emph{quantifications}.

In non-metric MDS raw stress becomes
\begin{equation}
\sigma_R(X,\Delta):=\frac12\sum w_{ij}(\delta_{ij}-d_{ij}(X))^2,
\label{eq:rawstressdef}
\end{equation}
where \(\Delta\) varies over the quantified or transformed
dissimilarities. In MDS parlance they are also called \emph{pseudo-distances}
or \emph{disparities}. Loss function \eqref{eq:rawstressdef} must be minimized
over both configurations and disparities, with the condition that the
disparities \(\Delta\) are an admissible transformation or quantification
of the data. In Kruskal's non-metric MDS this means
requiring monotonicity. In this paper we will consider various other
choices for the set of admissible transformations. We will use the
symbol \(\mathfrak{D}\) for the set of admissible transformations

The most familiar examples of \(\mathfrak{D}\) (linear,
polynomial, splines, monotone) define convex cones with apex at the origin. This
means that if \(\Delta\in\mathfrak{D}\) then so is \(\lambda\Delta\) for all
\(\lambda\geq 0\). But consequently minimizing \eqref{eq:rawstressdef} over
all \(\Delta\in\mathfrak{D}\) and over all configurations has the trivial
solution \(\Delta=0\) and \(X=0\), corresponding with the global minimum
\(\sigma(X,\Delta)=0\). We need additional constraints to rule out this
trivial solution, and in non-metric MDS this is done by choosing a
\emph{normalization} that keeps the solution away from zero.

Kruskal's original solution is to define \emph{normalized stress} as
\begin{equation}
\sigma(X,\Delta):=\frac{\sum w_{ij}(\delta_{ij}-d_{ij}(X))^2}{\sum w_{ij}d_{ij}^2(X)}.
\label{eq:nstressdef}
\end{equation}
To be precise, in Kruskal's formulation there are no weights,
and he actually takes the square root of \eqref{eq:nstressdef} to define
\emph{Kruskal's stress}. The non-metric Euclidean MDS problem now is to minimize
loss function \eqref{eq:nstressdef} over all \(n\times p\) configurations
\(X\) and all admissible disparities \(\Delta\).

\subsection{Normalization}\label{normalization}

Equation \eqref{eq:nstressdef} is only one way to normalize raw stress. Some obvious alternatives are discussed in detail in
Kruskal and Carroll (1969) and De Leeuw (1975). In the terminology of
De Leeuw (1975) there are \emph{explicit} and \emph{implicit} normalizations.

In implicit normalization we minimize either
\begin{equation}
\sigma(X,\hat D):=\frac{\sum  w_{ij}(\hat d_{ij} -d_{ij}(X))^2}{\sum   w_{ij}^{\ }\hat d_{ij}^2}
\label{eq:implicit1}
\end{equation}
or
\begin{equation}
\sigma(X,\hat D):=\frac{\sum   w_{ij}(\hat d_{ij}-d_{ij}(X))^2}{\sum   w_{ij}^{\ }d_{ij}^2(X) }
\label{eq:implicit2}
\end{equation}
over \(X\) and \(\Delta\in\mathfrak{D}\).

As we have seen, Kruskal (1964a) chooses definition \eqref{eq:implicit2} and
calls the explicitly normalized loss function \emph{normalized stress}. Note that we overload the symbol \(\sigma\) to denote any one of the least squares loss
functions. It will always be clear from the text which \(\sigma\) we are
talking about.

In explicit normalization we minimize the raw stress \(\sigma_R(X,\hat D)\)
from \eqref{eq:rawstressdef}, but we add the explicit constraint
\begin{equation}
\sum   w_{ij}^{\ }d_{ij}^2(X)=1,
\label{eq:explicit1}
\end{equation}
or the constraint
\begin{equation}
\sum   w_{ij}^{\ }\hat d_{ij}^2=1.
\label{eq:explicit2}
\end{equation}
Kruskal and Carroll (1969) and De Leeuw (2019) show that these
four normalizations all lead to essentially the same solution for \(X\)
and \(\hat D\), up to scale factors dictated by the choice of the particular
normalization. It is also possible to normalize both \(X\) and \(\hat D\),
either explicitly or implicitly, and again this will give the same
solutions, suitably normalized. These invariance results assume the
admissible transformations form a closed cone with apex at the origin,
i.e.~if \(\hat D\) is admissible and \(\lambda\geq 0\) then \(\lambda\hat D\)
is admissible as well. The matrices of Euclidean distances \(D(X)\) form a
similar closed cone as well. The non-metric MDS problem is to find an element
of the \(\hat D\) cone \(\mathcal{D}\) and an element of the \(D(X)\) cone where the angle between the two is a small as possible.

In the R version of smacof (De Leeuw and Mair (2009),
Mair, Groenen, and De Leeuw (2022)) we use explicit normalization
\eqref{eq:explicit2}. This is supported by the result, also due to
De Leeuw (1975), that projection on the intersection of the cone of
disparities and the sphere defined by \eqref{eq:explicit2} is equivalent
to first projecting on the cone and then normalizing the projection (see
also Bauschke, Bui, and Wang (2018)).

In the version of non-metric MDS discussed in this manualwe need more flexibility. For algorithmic reasons that may become clear later on, we will go with the original \eqref{eq:nstressdef}, i.e.~with the implicitly normalized
Kruskal's stress. For the final results the choice between normalizations should not make a difference, but the iterative computations will be different for the
diufferent choices.

\subsection{Some thoughts on ALS}\label{some-thoughts-on-als}

I will take this opportunity to clear up some misunderstandings and
confusions that have haunted the early development of non-metric MDS.

\subsubsection{The Single-Phase approach}\label{the-single-phase-approach}

In Kruskal (1964a) defines
\begin{equation}
\sigma(X):=\min_{\hat D\in\mathfrak{D}}\ \sigma(\hat D,X)=\sigma(X,\hat D(X)),
\label{eq:project}
\end{equation}
where \(\sigma(\hat D,X)\) is defined by
\eqref{eq:implicit2}. The minimum in \eqref{eq:project} is over admissible
transformations. In definition \eqref{eq:project}
\begin{equation}
\hat D(X):=\mathop{\text{argmin}}_{\hat D\in\mathfrak{D}}\sigma(X, \hat D).
\label{eq:optscal}
\end{equation}
Normalized stress defined by \eqref{eq:project} is now a
function of \(X\) only. Under some conditions, which are true in Kruskal's
definition of non-metric MDS, there is a simple relation between the
partials of \eqref{eq:implicit2} and those of \eqref{eq:project}.
\begin{equation}
\mathcal{D}\sigma(X)=\mathcal{D}_1\sigma(X,\hat D(X)),
\label{eq:partials}
\end{equation}
where \(\mathcal{D}\sigma(X)\) are the derivatives of
\(\sigma\) from \eqref{eq:project} and \(\mathcal{D}_1\sigma(X,\hat D(X))\)
are the partial derivatives of \(\sigma\) from \eqref{eq:implicit2} with
respect to \(X\). Thus the partials of \(\sigma\) from \eqref{eq:project} can
be computed by evaluating the partials of \(\sigma\) from
\eqref{eq:implicit2} with respect to \(X\) at \((X,\hat D(X))\). This has
created much confusion in the past. The non-metric MDS problem in Kruskal's original formulation is now to
minimize \(\sigma\) from \eqref{eq:project}, which is a function of \(X\)
alone.

Guttman (1968) calls this the \emph{single-phase approach}. A variation of
Kruskal's single-phase approach defines
\begin{equation}
\sigma(X)=\sum w_{ij}(d_{ij}^\#(X)-d_{ij}(X))^2,
\label{eq:rankimage}
\end{equation}
where the \(d_{ij}^\#(X)\) are \emph{Guttman's rank images}, i.e.~the
permutation of the \(d_{ij}(X)\) that makes them monotone with the
\(\delta_{ij}\) (Guttman (1968)). Or, alternatively, define
\begin{equation}
\sigma(X):=\sum   w_{ij}(d_{ij}^\%(X)-d_{ij}(X))^2,
\label{eq:shepard}
\end{equation}
where the \(\hat d_{ij}^\%(X)\) are \emph{Shepard's rank images}, i.e.~the
permutation of the \(\delta_{ij}\) that makes them monotone with the
\(d_{ij}(X)\) (Shepard (1962a), Shepard (1962b), De Leeuw (2017b)).

Minimizing the Shepard or Guttman single-phase loss functions is
computationally more complicated than Kruskal's \emph{monotone regression}
approach, mostly because the rank-image transformations are not
differentiable, and there is no analog of \eqref{eq:partials} and of the
equivalence of the different implicit and explicit normalizations.

\subsubsection{The Two-Phase Approach}\label{the-two-phase-approach}

The \emph{two-phase approach} or \emph{alternating least squares (ALS)} approach
alternates minimization of \(\sigma(\hat D,X)\) over \(X\) for our current
best estimate of \(\hat D\) with minimization of \(\sigma(\hat D,X)\) over
\(\Delta\in\mathfrak{D}\) for our current best value of \(X\). Thus an
update from iteration \(k\) to iteration \(k+1\) looks like
\begin{align}
\hat D^{(k)}&=\mathop{\text{argmin}}_{\hat D\in\mathfrak{D}}\sigma(\hat D,X^{(k)}),\label{eq:step1}\\
X^{(k+1)}&=\mathop{\text{argmin}}_X\sigma(\hat D^{(k)},X).\label{eq:step2}
\end{align}
This ALS approach to MDS was in the air since the early
(unsuccessful) attempts around 1968 of Young and De Leeuw to combine
Torgerson's classic metric MDS method with Kruskal's monotone regression
transformation. All previous implementations of non-metric smacof use
the two-phase approach, and we will do the same in this paper.

As formulated, however, there are some problems with the ALS algorithm.
Step \eqref{eq:step1} is easy to carry out, using monotone regression.
Step \eqref{eq:step2} means solving a metric scaling problem, which is an
iterative proces that requires an infinite number of iterations. Thus,
in the usual implementations, step \eqref{eq:step1} is combined with one
of more iterations of a convergent iterative procedure for metric MDS,
such as smacof. If we take only one of these \emph{inner iterations} the
algorithm becomes indistinguishable from Kruskal's single-phase method.
This has also created much confusion in the past.

In the usual implementations of the ALS approach we solve the first
subproblem \eqref{eq:step1} exactly, while we take only a single step
towards the solution for given \(\hat D\) in the second phase
\eqref{eq:step2}. If we have an infinite iterative procedure to compute
the optimal \(\hat D\in\mathfrak{D}\) for given \(X\), then a more balanced
approach would be to take several inner iterations in the first phase
and several inner iterations in the second phase. How many of each,
nobody knows. In our current implementation of smacof we take several
inner iteration steps in the first phase and a single inner iteration
step in the second phase.

\section{Smacof Notation and Terminology}\label{smacof-notation-and-terminology}

We discuss some standard MDS notation, first introduced in
De Leeuw (1977). This notation is useful for the second phase of the ALS
algorithm, in which solve the metric MDS problem of we minimizing
unnormalized \(\sigma(X,\hat D)\) over \(X\) for fixed \(\hat D\). We will
discuss the first ALS phase later in the paper.

Start with the unit vectors \(e_i\) of length \(n\). They have a non-zero
element equal to one in position \(i\), all other elements are zero. Think
of the \(e_i\) as the columns of the identity matrix.

Using the \(e_i\) we define for all \(i\not= j\) the matrices
\begin{equation}
A_{ij}:=(e_i-e_j)(e_i-e_j)'.
\end{equation}
The \(A_{ij}\) are of order \(n\), symmetric,
doubly-centered, and of rank one. They have four non-zero elements.
Elements \((i,i)\) and \((j,j)\) are equal to \(+1\), elements \((i,j)\) and
\((j,i)\) are \(-1\).

The importance of \(A_{ij}\) in MDS comes from the equation
\begin{equation}
d_{ij}^2(X)=\text{tr}\ X'A_{ij}X.
\label{eq:dfroma}
\end{equation}
In addition we use the fact that the \(A_{ij}\) form a
basis for the \(binom{n}{2}\)-dimensional linear space of all
doubly-centered symmetric matrices.

Expanding the square in the definition of stress gives
\begin{equation}
\sigma(X)=\frac12\{\sum   w_k\delta_k^2-2\ \sum   w_k\delta_kd_k(X)+\sum   w_kd_k^2(X)\}.
\label{eq:expand}
\end{equation}
It is convenient to have notation for the three separate
components of stress from equation \eqref{eq:expand}. Define
\begin{align}
\eta_{\hat D}^2&=\sum   w_{ij}\hat d_{ij}^2,\label{eq:condef}\\
\rho(X)&=\sum   w_{ij}\hat d_{ij}d_{ij}(X),\label{eq:rhodef}\\
\eta^2(X)&=\sum   w_{ij}d_{ij}(X)^2.\label{eq:etadef}
\end{align}
which lead to
\begin{equation}
\sigma(X)=\frac12\left\{\eta_{\hat D}^2-2\rho(X)+\eta^2(X)\right\}.
\label{eq:stressshort}
\end{equation}
We also need
\begin{equation}
\lambda(X)=\frac{\rho(X)}{\eta(X)}.
\label{eq:lambdadef}
\end{equation}

Using the \(A_{ij}\) makes it possible to give matrix expressions for
\(\rho\) and \(\eta^2\). First
\begin{equation}
\eta^2(X)=\text{tr}\ X'VX,
\label{eq:etamat}
\end{equation}
with
\begin{equation}
V:=\sum   w_{ij}A_{ij}.
\label{eq:vdef}
\end{equation}
In the same way
\begin{equation}
\rho(X)=\text{tr}\ X'B(X)X,
\label{eq:rhomat}
\end{equation}
with
\begin{equation}
B(X):=\sum   w_{ij}r_{ij}(X)A_{ij},
\label{eq:bdef}
\end{equation}
with
\begin{equation}
r_{ij}(X):=\begin{cases}\frac{\delta_{ij}}{d_{ij}(X)}&\text{ if }d_{ij}(X)>0,\\
0&\text{ if }d_{ij}(X)=0.
\end{cases}
\end{equation}
Note that \(B\) is a function from the set of \(n\times p\)
configurations into the set of symmetric doubly-dentered matrices of
order \(n\). All matrices of the form \(\sum x_{ij}A_{ij}\), where summation
is over all pairs \((i,j)\) with \(j<i\), are symmetric and doubly-centered.
They have \(-x_{ij}\) as off-diagonal elements while the diagonal elements
\((i,i)\) are \(\sum_{j=1}^nx_{ij}\).

Because \(B(X)\) and \(V\) are non-negative linear combinations of the
\(A_{ij}\) they are both positive semi-definite. Because \(W\) is assumed to
be irreducible the matrix \(V\) has rank \(n-1\), with only vectors
proportional to the vector \(e\) with all elements equal to one in its
null-space (De Leeuw (1977)).

Summarizing the results so far we have
\begin{equation}
\sigma(X)=\frac12\{\eta_{\hat D}^2-\text{tr}\ X'B(X)X+\text{tr}\ X'VX\}.
\label{eq:sigmat}
\end{equation}

Next we define the \emph{Guttman transform} of a configuration \(X\), for given
\(W\) and \(\Delta\), as
\begin{equation}
G(X)=V^+B(X)X,
\label{eq:gudef}
\end{equation}
with \(V^+\) the Moore-Penrose inverse of \(V\). In our
computations we use
\begin{equation}
V^+=(V+\frac{1}{n}ee')^{-1}-\frac{1}{n}ee'
\end{equation}
Also note that in the unweighted case with complete data \(V=nJ\),
where \(J\) is the centering matrix \(I-\frac{1}{n}ee'\), and thus
\(V^+=\frac{1}{n}J\). The Guttman transform is then simply
\(G(X)=n^{-1}B(X)X\).

\section{Intermezzo: Explicit Normalization}\label{intermezzo-explicit-normalization}

\[
\sigma(X,\hat D)=\frac12\frac{\sum w_{ij}(\hat d_{ij}-d_{ij}(X))^2}{\sum w_{ij}d_{ij}^2(X)}
\]
Majorize

\[
\sigma(X,\hat D)\leq\frac12\frac{\eta^2(\hat D)-2\text{tr}\ X'V\overline{Y}+\text{tr}\ X'VX}{\text{tr}\ X'VX}=\frac{\omega(X,Y)}{\eta^2(X)}
\]
Stationary equations
\[
\eta^2(X)(VX-VG(Y))-\omega(X,Y)VX=V\{(\eta^2(X)-\omega(X,Y))X-\eta^2(X)\overline Y\}
\]
So at a minimum \(X\) is proportional to \(\overline{Y}\) or \(X=\alpha\overline{Y}\) for some \(\alpha\). For \ldots{} to be zero we must have
\[
\alpha(\alpha^2\eta^2(\overline Y)-(\eta^2(\hat D)-2\alpha\eta^2(\overline Y)+\alpha^2\eta^2(\overline Y))=\alpha^2\eta^2(\overline Y)
\]
which works out to be
\[
\alpha=\frac{\eta^2(\hat D)}{\eta^2(\overline Y)}
\]
\[
\hat X=\frac{\eta^2(\hat D)}{\eta^2(\overline Y)}\ \overline{Y}
\]
The minimum is equal to

\[
\frac{-\frac{(\eta^2(\overline Y))^2}{\eta^2(\hat D)}+\eta^2(\overline Y)}{\eta^2(\overline Y)}=1-\frac{\eta^2(\overline Y)}{\eta^2(\hat D)}
\]
Use homogeneity of the Guttman transform.

More generally suppose we update with
\[
X=\overline Y+\alpha(Y-\overline Y)
\]
Write
\[
\omega(X,Y)=\eta^2(\hat D)+\text{tr}\ (X-\overline Y)'V(X-\overline Y)-\eta^2(\overline Y)
\]
Thus if \(X(\alpha)=\overline Y+\alpha(Y-\overline Y)\) we have
\[
\omega(\alpha)=\eta^2(\hat D)+\alpha^2\text{tr}\ (Y-\overline Y)'V(Y-\overline Y)-\eta^2(\overline Y)
\]
and
\[
\eta^2(\alpha)=\eta^2(\overline Y)+2\alpha\text{tr}\ (Y-\overline Y)'V\overline Y+\alpha^2\text{tr}\ (Y-\overline Y)'V(Y-\overline Y)
\]
\[
\omega(Y,Y)=\eta^2(\hat D)+\text{tr}\ (Y-\overline Y)'V(Y-\overline Y)-\eta^2(\overline Y)
\]
\[
\frac{\omega(\alpha)}{\eta^2(\alpha)}\leq\sigma(Y)
\]

\section{Properties of Smacof Loss}\label{properties-of-smacof-loss}

\subsection{Derivatives}\label{derivatives}

The Euclidean distance function \(d_{ij}\) from \ldots{} is not differentiable
at configurations \(X\) with \(x_i=x_j\). If \(d_{ij}(X)>0\) then
\[
\mathcal{D}\sigma(X)=\frac{1}{d_{ij}(X)}A_{ij}X
\]
If \(d_{ij}(X)=0\) then \[
D_+d_{ij}(X,Y)=\lim_{\epsilon\downarrow 0}\frac{d_{ij}(X+\epsilon Y)-d_{ij}(X)}{\epsilon}=d_{ij}(Y)
\] which is non-linear in \(Y\), showing non-differentiability.

\[
D_+\sigma(X,Y)=\text{tr}\ Y'(V-B(X))X+\sum\{w_{ij}\delta_{ij}d_{ij}(Y)\mid d_{ij}(X)=0\}
\] This form of the directional derivative is used by De Leeuw (1984) to
show that two independent necessary conditions for a local minimum are
\((V-B(X))X=0\) and \(d_{ij}(X)>0\) for all \((i,j)\) with
\(w_{ij}\delta_{ij}>0\).

\subsubsection{Gradient}\label{gradient}

\[
\mathcal{D}\sigma(X)=(V-B(X))X
\]
At a stationary point \(B(X)X=VX\) or \(V^+B(X)X=X\). Thus a necessary
condition for a local minimum is that \(V^+B(X)\) has at least \(p\)
eigenvalues equal to one. De Leeuw (2014) has shown that if
\(V^+B(X)\lesssim I\) then actually \(X\) is a global minimizer of stress.

\[
\rho(X)=\sum w_{ij}\delta_{ij}(X)
\]

\[
\nabla d_{ij}(X)=\begin{bmatrix}0\\
\frac{x_i-x_j}{d_{ij}(X)}\\0\\-\frac{x_i-x_j}{d_{ij}(X)}\\0\end{bmatrix}
\]

\[
\partial d_{ij}(X)=\left\{\begin{bmatrix}0\\
y\\0\\-y\\0\end{bmatrix}\mid y'y\leq 1\right\}.
\]

\subsubsection{Hessian}\label{hessian}

The results on the Hessian of stress are largely unpublished. So we
summarize them here in this manual, so they'll be even more unpublished.

\[
H_{st}(X):=\sum w_{ij}\frac{\delta_{ij}}{d_{ij}(X)}\left\{\frac{A_{ij}x_sx_t'A_{ij}}{d_{ij}^2(X)}\right\}
\]

\[
H_{st}(X)=\sum   w_{ij}\frac{\delta_{ij}}{d_{ij}^3(X)}(x_{is}-x_{js})(x_{it}-x_{jt})A_{ij}
\]
\[
\mathcal{D}_{st}\sigma(X)=\begin{cases}H_{st}(X)&\text{ if }s\not= t,\\
V-B(X)+H_{st}&\text{ if }s= t.
\end{cases}
\]
If \(I_p\) is the identity matrix of order \(p\), and \(\otimes\) is the
Kronecker product, then \[
\mathcal{D}^2\sigma(X)=I_p\otimes(V - B(X))+ H(X)
\] \[
\sum_{s=1}^p\sum_{t=1}^p y_s'H_{st}y_t=\sum w_{ij}\frac{\delta_{ij}}{d_{ij}(X)}\left\{\frac{(\text{tr} \ Y'A_{ij}X)^2}{d_{ij}^2(X)}\right\}\leq\sum w_{ij}\frac{\delta_{ij}}{d_{ij}(X)}\text{tr}\ Y'A_{ij}Y=\text{tr}\ Y'B(X)Y.
\]
Thus
\[
0\lesssim H\lesssim I_p\otimes B(X),
\]
and
\[
I_p\otimes (V-B(X))\lesssim\mathcal{D}^2\sigma(X)\lesssim I_p\otimes V
\]
At a local minimum of \(\sigma\)
\[
0\lesssim\mathcal{D}^2\sigma(X)\lesssim I_p\otimes V
\]
In comparing the lower bounds on \(\mathcal{D}^2\sigma(X)\) in \ldots{} and
\ldots{} De Leeuw (2014) shows that \(V-B(X)\gtrsim 0\) is sufficient for a
\emph{global} minimum of stress (but far from necessary).

Also \[
\sum_{t=1}^p H_{st}y_t=\sum w_{ij}\frac{\delta_{ij}}{d_{ij}(X)}\left\{\frac{\text{tr}\ Y'A_{ij}X }{d_{ij}^2(X)}\right\}A_{ij}x_s
\] If \(Y=X\) then \(H(X)y=(I_p\otimes B(X))x\) and thus
\[\mathcal{D}^2\sigma(X)x=(I_p\otimes V)x.\] In the unweighted case this
means that \(X\) is an eigenvector of \(\mathcal{D}^2\sigma(X)\) with
eigenvalue \(n\). Inequalities \ldots{} show that this is actually the largest
eigenvalue. Or \((I_p\otimes V)^+\mathcal{D}^2\sigma(X)\lesssim I\).

If \(Y=XT\) with \(T\) anti-symmetric then \(\text{tr}\ Y'A_{ij}X=0\) then
thus \(H(X)y=0\). Thus \[
\sum_{t=1}^p\mathcal{D}_{st}\sigma(X)y_t=(V-B(X))y_t
\] which is zero if \(\mathcal{D}\sigma(X)\) is zero. Thus at a stationary
point of stress \(\mathcal{D}^\sigma(X)\) has \(\frac12p(p-1)\) zero
eigenvalues.

There are several ways to think of the Hessian. The simplest one
(perhaps) is as an \(np\times np\) symmetric matrix (corresponding to

column-major R vector of length \(\frac12 np(np+1)\)). This is what we
would use for a straightforward version of Newton-Raphson.

It is more elegant, however, to think of \(H\) as a symmetric super-matrix
of order \(p\), with as elements \(n\times n\) matrices. And, for some
purposes, such as the pseudo-confidence ellipsoids in De Leeuw (2017a), as
a super-matrix of order \(n\) with as elements \(p\times p\) matrices. Both
the super-matrix interpretations lead to four-dimensional arrays, the
first a \(p\times p\times n\times n\) array, the second an
\(n\times n\times p\times p\) array. The different interpretations lead to
different ways to store the Hessian in memory, and to different ways to
retrieve its elements. Of course we can write routines to transform from
one interpretation to another.

\subsection{Lagrangian}\label{lagrangian}

In our implementation of the smacof algorithm we minimize stress over
configurations with \(\eta(X)=1\), or, equivalently,
\(\sum w_{ij}d_{ij}^2(X)=1\). This means we do not look for \(X\) with
\(\mathcal{D}\sigma(X)=(V-B(X))X=0\), but we look for solutions of \[
(V-B(X))X-\lambda VX=0,\\
\text{tr}\ X'VX= 1.
\] At the solution \[
\lambda=1-\rho(X)
\] and \[
X = \frac{\Gamma(X)}{\eta(\Gamma(X))}
\] Also it is necessary for a local minimum that \[
\Gamma(X)=\rho(X)X
\] Because the Guttman transform is homogeneous of degree zero this
implies \[
\Gamma(\Gamma(X)=\Gamma(X),
\] so although \(X\) is not a fixed point of the Guttman transform,
\(\Gamma(X)\) is.

The second order necessary condition is that \[
H(X)\gtrsim I_p\otimes (\rho(X)V - B(X))
\] is positive

\subsubsection{Kuhn-Tucker Points}\label{kuhn-tucker-points}

bozo

\section*{References}\label{references}
\addcontentsline{toc}{section}{References}

\phantomsection\label{refs}
\begin{CSLReferences}{1}{0}
\bibitem[\citeproctext]{ref-bauschke_bui_wang_18}
Bauschke, H. H., M. N. Bui, and X. Wang. 2018. {``{Projecting onto the Intersection of a Cone and a Sphere}.''} \emph{SIAM Journal on Optimization} 28: 2158--88.

\bibitem[\citeproctext]{ref-deleeuw_U_75a}
De Leeuw, J. 1975. {``{A Normalized Cone Regression Approach to Alternating Least Squares Algorithms}.''} Department of Data Theory FSW/RUL.

\bibitem[\citeproctext]{ref-deleeuw_C_77}
---------. 1977. {``Applications of Convex Analysis to Multidimensional Scaling.''} In \emph{Recent Developments in Statistics}, edited by J. R. Barra, F. Brodeau, G. Romier, and B. Van Cutsem, 133--45. Amsterdam, The Netherlands: North Holland Publishing Company.

\bibitem[\citeproctext]{ref-deleeuw_A_84f}
---------. 1984. {``{Differentiability of Kruskal's Stress at a Local Minimum}.''} \emph{Psychometrika} 49: 111--13.

\bibitem[\citeproctext]{ref-deleeuw_U_14b}
---------. 2014. {``{Bounding, and Sometimes Finding, the Global Minimum in Multidimensional Scaling}.''} UCLA Department of Statistics. \url{https://jansweb.netlify.app/publication/deleeuw-u-14-b/deleeuw-u-14-b.pdf}.

\bibitem[\citeproctext]{ref-deleeuw_E_17q}
---------. 2017a. {``{Pseudo Confidence Regions for MDS}.''} 2017. \url{https://jansweb.netlify.app/publication/deleeuw-e-17-q/deleeuw-e-17-q.pdf}.

\bibitem[\citeproctext]{ref-deleeuw_E_17e}
---------. 2017b. {``{Shepard Non-metric Multidimensional Scaling}.''} 2017. \url{https://jansweb.netlify.app/publication/deleeuw-e-17-e/deleeuw-e-17-e.pdf}.

\bibitem[\citeproctext]{ref-deleeuw_E_19d}
---------. 2019. {``Normalized Cone Regression.''} 2019. \url{https://jansweb.netlify.app/publication/deleeuw-e-19-d/deleeuw-e-19-d.pdf}.

\bibitem[\citeproctext]{ref-deleeuw_mair_A_09c}
De Leeuw, J., and P. Mair. 2009. {``{Multidimensional Scaling Using Majorization: SMACOF in R}.''} \emph{Journal of Statistical Software} 31 (3): 1--30. \url{https://www.jstatsoft.org/article/view/v031i03}.

\bibitem[\citeproctext]{ref-fang_oleary_12}
Fang, H., and D. P. O'Leary. 2012. {``Euclidean Distance Matrix Completion Problems Euclidean Distance Matrix Completion Problems Euclidean Distance Matrix Completion Problems.''} \emph{Oprtimization Methods and Software} 27 (4-5): 695--717.

\bibitem[\citeproctext]{ref-guttman_68}
Guttman, L. 1968. {``{A General Nonmetric Technique for Fitting the Smallest Coordinate Space for a Configuration of Points}.''} \emph{Psychometrika} 33: 469--506.

\bibitem[\citeproctext]{ref-kruskal_64a}
Kruskal, J. B. 1964a. {``{Multidimensional Scaling by Optimizing Goodness of Fit to a Nonmetric Hypothesis}.''} \emph{Psychometrika} 29: 1--27.

\bibitem[\citeproctext]{ref-kruskal_64b}
---------. 1964b. {``{Nonmetric Multidimensional Scaling: a Numerical Method}.''} \emph{Psychometrika} 29: 115--29.

\bibitem[\citeproctext]{ref-kruskal_carroll_69}
Kruskal, J. B., and J. D. Carroll. 1969. {``{Geometrical Models and Badness of Fit Functions}.''} In \emph{Multivariate Analysis, Volume II}, edited by P. R. Krishnaiah, 639--71. North Holland Publishing Company.

\bibitem[\citeproctext]{ref-mair_groenen_deleeuw_A_22}
Mair, P., P. J. F. Groenen, and J. De Leeuw. 2022. {``{More on Multidimensional Scaling in R: smacof Version 2}.''} \emph{Journal of Statistical Software} 102 (10): 1--47. \url{https://www.jstatsoft.org/article/view/v102i10}.

\bibitem[\citeproctext]{ref-shepard_62a}
Shepard, R. N. 1962a. {``{The Analysis of Proximities: Multidimensional Scaling with an Unknown Distance Function. I}.''} \emph{Psychometrika} 27: 125--40.

\bibitem[\citeproctext]{ref-shepard_62b}
---------. 1962b. {``{The Analysis of Proximities: Multidimensional Scaling with an Unknown Distance Function. II}.''} \emph{Psychometrika} 27: 219--46.

\end{CSLReferences}

\end{document}
