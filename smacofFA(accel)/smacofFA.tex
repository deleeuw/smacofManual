% Options for packages loaded elsewhere
\PassOptionsToPackage{unicode}{hyperref}
\PassOptionsToPackage{hyphens}{url}
\PassOptionsToPackage{dvipsnames,svgnames,x11names}{xcolor}
%
\documentclass[
  12pt,
  letterpaper,
  DIV=11,
  numbers=noendperiod]{scrartcl}

\usepackage{amsmath,amssymb}
\usepackage{iftex}
\ifPDFTeX
  \usepackage[T1]{fontenc}
  \usepackage[utf8]{inputenc}
  \usepackage{textcomp} % provide euro and other symbols
\else % if luatex or xetex
  \usepackage{unicode-math}
  \defaultfontfeatures{Scale=MatchLowercase}
  \defaultfontfeatures[\rmfamily]{Ligatures=TeX,Scale=1}
\fi
\usepackage{lmodern}
\ifPDFTeX\else  
    % xetex/luatex font selection
    \setmainfont[]{Times New Roman}
\fi
% Use upquote if available, for straight quotes in verbatim environments
\IfFileExists{upquote.sty}{\usepackage{upquote}}{}
\IfFileExists{microtype.sty}{% use microtype if available
  \usepackage[]{microtype}
  \UseMicrotypeSet[protrusion]{basicmath} % disable protrusion for tt fonts
}{}
\makeatletter
\@ifundefined{KOMAClassName}{% if non-KOMA class
  \IfFileExists{parskip.sty}{%
    \usepackage{parskip}
  }{% else
    \setlength{\parindent}{0pt}
    \setlength{\parskip}{6pt plus 2pt minus 1pt}}
}{% if KOMA class
  \KOMAoptions{parskip=half}}
\makeatother
\usepackage{xcolor}
\setlength{\emergencystretch}{3em} % prevent overfull lines
\setcounter{secnumdepth}{5}
% Make \paragraph and \subparagraph free-standing
\makeatletter
\ifx\paragraph\undefined\else
  \let\oldparagraph\paragraph
  \renewcommand{\paragraph}{
    \@ifstar
      \xxxParagraphStar
      \xxxParagraphNoStar
  }
  \newcommand{\xxxParagraphStar}[1]{\oldparagraph*{#1}\mbox{}}
  \newcommand{\xxxParagraphNoStar}[1]{\oldparagraph{#1}\mbox{}}
\fi
\ifx\subparagraph\undefined\else
  \let\oldsubparagraph\subparagraph
  \renewcommand{\subparagraph}{
    \@ifstar
      \xxxSubParagraphStar
      \xxxSubParagraphNoStar
  }
  \newcommand{\xxxSubParagraphStar}[1]{\oldsubparagraph*{#1}\mbox{}}
  \newcommand{\xxxSubParagraphNoStar}[1]{\oldsubparagraph{#1}\mbox{}}
\fi
\makeatother

\usepackage{color}
\usepackage{fancyvrb}
\newcommand{\VerbBar}{|}
\newcommand{\VERB}{\Verb[commandchars=\\\{\}]}
\DefineVerbatimEnvironment{Highlighting}{Verbatim}{commandchars=\\\{\}}
% Add ',fontsize=\small' for more characters per line
\usepackage{framed}
\definecolor{shadecolor}{RGB}{241,243,245}
\newenvironment{Shaded}{\begin{snugshade}}{\end{snugshade}}
\newcommand{\AlertTok}[1]{\textcolor[rgb]{0.68,0.00,0.00}{#1}}
\newcommand{\AnnotationTok}[1]{\textcolor[rgb]{0.37,0.37,0.37}{#1}}
\newcommand{\AttributeTok}[1]{\textcolor[rgb]{0.40,0.45,0.13}{#1}}
\newcommand{\BaseNTok}[1]{\textcolor[rgb]{0.68,0.00,0.00}{#1}}
\newcommand{\BuiltInTok}[1]{\textcolor[rgb]{0.00,0.23,0.31}{#1}}
\newcommand{\CharTok}[1]{\textcolor[rgb]{0.13,0.47,0.30}{#1}}
\newcommand{\CommentTok}[1]{\textcolor[rgb]{0.37,0.37,0.37}{#1}}
\newcommand{\CommentVarTok}[1]{\textcolor[rgb]{0.37,0.37,0.37}{\textit{#1}}}
\newcommand{\ConstantTok}[1]{\textcolor[rgb]{0.56,0.35,0.01}{#1}}
\newcommand{\ControlFlowTok}[1]{\textcolor[rgb]{0.00,0.23,0.31}{\textbf{#1}}}
\newcommand{\DataTypeTok}[1]{\textcolor[rgb]{0.68,0.00,0.00}{#1}}
\newcommand{\DecValTok}[1]{\textcolor[rgb]{0.68,0.00,0.00}{#1}}
\newcommand{\DocumentationTok}[1]{\textcolor[rgb]{0.37,0.37,0.37}{\textit{#1}}}
\newcommand{\ErrorTok}[1]{\textcolor[rgb]{0.68,0.00,0.00}{#1}}
\newcommand{\ExtensionTok}[1]{\textcolor[rgb]{0.00,0.23,0.31}{#1}}
\newcommand{\FloatTok}[1]{\textcolor[rgb]{0.68,0.00,0.00}{#1}}
\newcommand{\FunctionTok}[1]{\textcolor[rgb]{0.28,0.35,0.67}{#1}}
\newcommand{\ImportTok}[1]{\textcolor[rgb]{0.00,0.46,0.62}{#1}}
\newcommand{\InformationTok}[1]{\textcolor[rgb]{0.37,0.37,0.37}{#1}}
\newcommand{\KeywordTok}[1]{\textcolor[rgb]{0.00,0.23,0.31}{\textbf{#1}}}
\newcommand{\NormalTok}[1]{\textcolor[rgb]{0.00,0.23,0.31}{#1}}
\newcommand{\OperatorTok}[1]{\textcolor[rgb]{0.37,0.37,0.37}{#1}}
\newcommand{\OtherTok}[1]{\textcolor[rgb]{0.00,0.23,0.31}{#1}}
\newcommand{\PreprocessorTok}[1]{\textcolor[rgb]{0.68,0.00,0.00}{#1}}
\newcommand{\RegionMarkerTok}[1]{\textcolor[rgb]{0.00,0.23,0.31}{#1}}
\newcommand{\SpecialCharTok}[1]{\textcolor[rgb]{0.37,0.37,0.37}{#1}}
\newcommand{\SpecialStringTok}[1]{\textcolor[rgb]{0.13,0.47,0.30}{#1}}
\newcommand{\StringTok}[1]{\textcolor[rgb]{0.13,0.47,0.30}{#1}}
\newcommand{\VariableTok}[1]{\textcolor[rgb]{0.07,0.07,0.07}{#1}}
\newcommand{\VerbatimStringTok}[1]{\textcolor[rgb]{0.13,0.47,0.30}{#1}}
\newcommand{\WarningTok}[1]{\textcolor[rgb]{0.37,0.37,0.37}{\textit{#1}}}

\providecommand{\tightlist}{%
  \setlength{\itemsep}{0pt}\setlength{\parskip}{0pt}}\usepackage{longtable,booktabs,array}
\usepackage{calc} % for calculating minipage widths
% Correct order of tables after \paragraph or \subparagraph
\usepackage{etoolbox}
\makeatletter
\patchcmd\longtable{\par}{\if@noskipsec\mbox{}\fi\par}{}{}
\makeatother
% Allow footnotes in longtable head/foot
\IfFileExists{footnotehyper.sty}{\usepackage{footnotehyper}}{\usepackage{footnote}}
\makesavenoteenv{longtable}
\usepackage{graphicx}
\makeatletter
\newsavebox\pandoc@box
\newcommand*\pandocbounded[1]{% scales image to fit in text height/width
  \sbox\pandoc@box{#1}%
  \Gscale@div\@tempa{\textheight}{\dimexpr\ht\pandoc@box+\dp\pandoc@box\relax}%
  \Gscale@div\@tempb{\linewidth}{\wd\pandoc@box}%
  \ifdim\@tempb\p@<\@tempa\p@\let\@tempa\@tempb\fi% select the smaller of both
  \ifdim\@tempa\p@<\p@\scalebox{\@tempa}{\usebox\pandoc@box}%
  \else\usebox{\pandoc@box}%
  \fi%
}
% Set default figure placement to htbp
\def\fps@figure{htbp}
\makeatother
% definitions for citeproc citations
\NewDocumentCommand\citeproctext{}{}
\NewDocumentCommand\citeproc{mm}{%
  \begingroup\def\citeproctext{#2}\cite{#1}\endgroup}
\makeatletter
 % allow citations to break across lines
 \let\@cite@ofmt\@firstofone
 % avoid brackets around text for \cite:
 \def\@biblabel#1{}
 \def\@cite#1#2{{#1\if@tempswa , #2\fi}}
\makeatother
\newlength{\cslhangindent}
\setlength{\cslhangindent}{1.5em}
\newlength{\csllabelwidth}
\setlength{\csllabelwidth}{3em}
\newenvironment{CSLReferences}[2] % #1 hanging-indent, #2 entry-spacing
 {\begin{list}{}{%
  \setlength{\itemindent}{0pt}
  \setlength{\leftmargin}{0pt}
  \setlength{\parsep}{0pt}
  % turn on hanging indent if param 1 is 1
  \ifodd #1
   \setlength{\leftmargin}{\cslhangindent}
   \setlength{\itemindent}{-1\cslhangindent}
  \fi
  % set entry spacing
  \setlength{\itemsep}{#2\baselineskip}}}
 {\end{list}}
\usepackage{calc}
\newcommand{\CSLBlock}[1]{\hfill\break\parbox[t]{\linewidth}{\strut\ignorespaces#1\strut}}
\newcommand{\CSLLeftMargin}[1]{\parbox[t]{\csllabelwidth}{\strut#1\strut}}
\newcommand{\CSLRightInline}[1]{\parbox[t]{\linewidth - \csllabelwidth}{\strut#1\strut}}
\newcommand{\CSLIndent}[1]{\hspace{\cslhangindent}#1}

\usepackage{tcolorbox}
\usepackage{amssymb}
\usepackage{yfonts}
\usepackage{bm}


\newtcolorbox{greybox}{
  colback=white,
  colframe=blue,
  coltext=black,
  boxsep=5pt,
  arc=4pt}
  
\newcommand{\sectionbreak}{\clearpage}

 
\newcommand{\ds}[4]{\sum_{{#1}=1}^{#3}\sum_{{#2}=1}^{#4}}
\newcommand{\us}[3]{\mathop{\sum\sum}_{1\leq{#2}<{#1}\leq{#3}}}

\newcommand{\ol}[1]{\overline{#1}}
\newcommand{\ul}[1]{\underline{#1}}

\newcommand{\amin}[1]{\mathop{\text{argmin}}_{#1}}
\newcommand{\amax}[1]{\mathop{\text{argmax}}_{#1}}

\newcommand{\ci}{\perp\!\!\!\perp}

\newcommand{\mc}[1]{\mathcal{#1}}
\newcommand{\mb}[1]{\mathbb{#1}}
\newcommand{\mf}[1]{\mathfrak{#1}}

\newcommand{\eps}{\epsilon}
\newcommand{\lbd}{\lambda}
\newcommand{\alp}{\alpha}
\newcommand{\df}{=:}
\newcommand{\am}[1]{\mathop{\text{argmin}}_{#1}}
\newcommand{\ls}[2]{\mathop{\sum\sum}_{#1}^{#2}}
\newcommand{\ijs}{\mathop{\sum\sum}_{1\leq i<j\leq n}}
\newcommand{\jis}{\mathop{\sum\sum}_{1\leq j<i\leq n}}
\newcommand{\sij}{\sum_{i=1}^n\sum_{j=1}^n}
	
\KOMAoption{captions}{tableheading}
\makeatletter
\@ifpackageloaded{caption}{}{\usepackage{caption}}
\AtBeginDocument{%
\ifdefined\contentsname
  \renewcommand*\contentsname{Table of contents}
\else
  \newcommand\contentsname{Table of contents}
\fi
\ifdefined\listfigurename
  \renewcommand*\listfigurename{List of Figures}
\else
  \newcommand\listfigurename{List of Figures}
\fi
\ifdefined\listtablename
  \renewcommand*\listtablename{List of Tables}
\else
  \newcommand\listtablename{List of Tables}
\fi
\ifdefined\figurename
  \renewcommand*\figurename{Figure}
\else
  \newcommand\figurename{Figure}
\fi
\ifdefined\tablename
  \renewcommand*\tablename{Table}
\else
  \newcommand\tablename{Table}
\fi
}
\@ifpackageloaded{float}{}{\usepackage{float}}
\floatstyle{ruled}
\@ifundefined{c@chapter}{\newfloat{codelisting}{h}{lop}}{\newfloat{codelisting}{h}{lop}[chapter]}
\floatname{codelisting}{Listing}
\newcommand*\listoflistings{\listof{codelisting}{List of Listings}}
\makeatother
\makeatletter
\makeatother
\makeatletter
\@ifpackageloaded{caption}{}{\usepackage{caption}}
\@ifpackageloaded{subcaption}{}{\usepackage{subcaption}}
\makeatother

\usepackage{bookmark}

\IfFileExists{xurl.sty}{\usepackage{xurl}}{} % add URL line breaks if available
\urlstyle{same} % disable monospaced font for URLs
\hypersetup{
  pdfauthor={Jan de Leeuw},
  colorlinks=true,
  linkcolor={blue},
  filecolor={Maroon},
  citecolor={Blue},
  urlcolor={Blue},
  pdfcreator={LaTeX via pandoc}}


\title{Smacof at 50: A Manual\\
Part x: Acceleration}
\author{Jan de Leeuw}
\date{January 8, 2025}

\begin{document}
\maketitle
\begin{abstract}
TBD
\end{abstract}

\renewcommand*\contentsname{Table of contents}
{
\hypersetup{linkcolor=}
\setcounter{tocdepth}{3}
\tableofcontents
}

\textbf{Note:} This is a working manuscript which will be
expanded/updated frequently. All suggestions for improvement are
welcome. All Rmd, tex, html, pdf, R, and C files are in the public
domain. Attribution will be appreciated, but is not required. The files
can be found at \url{https://github.com/deleeuw} in the repositories
smacofCode, smacofManual, and smacofExamples.

\section{Introduction}\label{introduction}

In this paper we study minimization of the multidimensional scaling
(MDS) loss function \begin{equation}
\sigma(X):=\frac12\mathop{\sum\sum}_{1\leq i<j\leq n} w_{ij}(\delta_{ij}-d_{ij}(X))^2
(\#eq:sdef)
\end{equation} over all \(n\times p\) \emph{configuration} matrices
\(X\). Following Kruskal (\citeproc{ref-kruskal_64a}{1964a}), Kruskal
(\citeproc{ref-kruskal_64b}{1964b}) we call \(\sigma(X)\) the
\emph{stress} of \(X\). The symbol \(:=\) is used for definitions.

In definition @ref(eq:sdef) matrices \(W=\{w_{ij}\}\) and
\(\Delta=\{\delta_{ij}\}\) are known non-negative, symmetric, and
hollow. They contain, respectively, \emph{weights} and
\emph{dissimilarities}. The matrix-valued function \(D\), with
\(D(X)=\{d_{ij}(X)\}\), contains \emph{Euclidean distances} between the
rows of \(X\).

Throughout we assume, without loss of generality, that \(W\) is
irreducible, that \(X\) is column-centered, and that \(\Delta\) is
normalized by \begin{equation}
\frac12\mathop{\sum\sum}_{1\leq i<j\leq n} w_{ij}\delta_{ij}^2=1.
(\#eq:delnorm)
\end{equation}

\subsection{Notation}\label{notation}

It is convenient to have some matrix notation for the MDS problem. We
use the symmetric matrices \(A_{ij}\), of order \(n\), which have \(+1\)
for elements \((i,i)\) and \((j,j)\), \(-1\) for elements \((i,j)\) and
\((j,i)\), and zeroes everywhere else. Using unit vectors \(e_i\) and
\(e_j\) we can write \begin{equation}
A_{ij}:=(e_i-e_j)(e_i-e_j)'.
(\#eq:adef)
\end{equation}

Following De Leeuw (\citeproc{ref-deleeuw_C_77}{1977}) we define
\begin{equation}
\rho(X):=\mathop{\sum\sum}_{1\leq i<j\leq n} w_{ij}\delta_{ij}d_{ij}(X)=\text{tr}\ X'B(X)X,
(\#eq:rhodef)
\end{equation} where \begin{equation}
B(X):=\mathop{\sum\sum}_{1\leq i<j\leq n}w_{ij}\delta_{ij}r_{ij}(X)A_{ij},
(\#eq:bdef)
\end{equation} with \begin{equation}
r_{ij}(X)=\begin{cases}
d_{ij}^{-1}(X)&\text{ if }d_{ij}(X)>0,\\
0&\text{ if }d_{ij}(X)=0.
\end{cases}
(\#eq:rdef)
\end{equation}

Also define \begin{equation}
\eta^2(X):=\mathop{\sum\sum}_{1\leq i<j\leq n}w_{ij}d_{ij}^2(X)=\text{tr}\ X'VX,
(\#eq:etadef)
\end{equation} where \begin{equation}
V:=\mathop{\sum\sum}_{1\leq i<j\leq n}w_{ij}A_{ij}.
(\#eq:vdef)
\end{equation} Thus \begin{equation}
\sigma(X)=1-\rho(X)+\frac12\eta^2(X)=1-\text{tr}\ X'B(X)X+\frac12\text{tr}\ X'VX.
(\#eq:sform)
\end{equation} Both \(B(X)\) and \(V\) are positive semi-definite and
doubly-centered. Because of the irreducibility of \(W\) the matrix \(V\)
has rank \(n-1\), with only the constant vectors in its null space. Both
\(\rho\) and \(\eta\) are positively homogeneous convex functions, with
\(\eta\) being a norm on the space of column-centered configurations.

Note that \(\rho\) is continuous, but it is not differentiable at \(X\)
if \(d_{ij}(X)=0\) for some \((i,j)\) for which \(w_{ij}\delta_{ij}>0\).
Because \begin{equation}
|d_{ij}(X)-d_{ij}(Y)|^2\leq\text{tr}\ (X-Y)'A_{ij}(X-Y)\leq 2\|X-Y\|^2
(\#eq:lipschitz)
\end{equation} we see that \(\rho\), although not differentiable at some
points, is globally Lipschitz. \(\eta^2\) is locally Lipschitz, and
consequently so is \(\sigma\).

\subsection{The Guttman Transform}\label{the-guttman-transform}

The \emph{Guttman transform} of a configuration \(X\), so named by De
Leeuw and Heiser (\citeproc{ref-deleeuw_heiser_C_80}{1980}) to honor the
contribution of Guttman (\citeproc{ref-guttman_68}{1968}), is defined as
the set-valued map \begin{equation}
\Phi(X)=V^+\partial\rho(X),
(\#eq:phidef)
\end{equation} with \(V^+\) the Moore-Penrose inverse of \(V\) and
\(\partial\rho(X)\) the subdifferential of \(\rho\) at \(X\), i.e.~the
set of all \(Z\) such that \(\rho(Y)\geq\rho(X)+\text{tr}\ Z'(Y-X)\) for
all \(Y\). Because \(\rho\) is homogeneous of degree one we have that
\(Z\in\partial\rho(X)\) if and only if \(\text{tr}\ Z'X=\rho(X)\) and
\(\rho(Y)\geq\text{tr}\ Z'Y\) for all \(Y\). For each \(X\) the
subdifferential \(\partial\rho(X)\), and consequently the Guttman
transform, is compact and convex. The map \(\partial\rho\) is also
positively homogeneous of degree zero,
i.e.~\(\partial\rho(\alpha X)=\partial\rho(X)\) for all \(X\) and all
\(\alpha\geq 0\). And consequently so is the Guttman transform.

We start with the subdifferential of the distance function between rows
\(i\) and \(j\) of an \(n\times p\) matrix. Straightforward calculation
gives \begin{equation}
\partial d_{ij}(X)=\begin{cases}
\left\{d_{ij}^{-1}(e_i-e_j)(x_i-x_j)'\right\}&\text{ if }d_{ij}(X)>0,\\
\left\{Z\mid Z=(e_i-e_j)z'\text{ with }z'z\leq1\right\}&\text{ if }d_{ij}(X)=0.
\end{cases}
(\#eq:dsubsef)
\end{equation} Thus if \(d_{ij}(X)>0\), i.e.~if \(d_{ij}\) is
differentiable at \(X\), then \(\partial d_{ij}(X)\) is a singleton,
containing only the gradient at \(X\).

From subdifferential calculus (Rockafellar
(\citeproc{ref-rockafellar_70}{1970}), theorem 23.8 and 23.9) the
subdifferential of \(\rho\) is the linear combination \begin{equation}
\partial\rho(X)=\mathop{\sum\sum}_{1\leq i<j\leq n}w_{ij}\delta_{ij}\partial d_{ij}(X)
(\#eq:subdif)
\end{equation} Summation here is in the Minkovski sense,
i.e.~\(\partial\rho(X)\) is the compact convex set of all linear
combinations
\(\mathop{\sum\sum}_{1\leq i<j\leq n}w_{ij}\delta_{ij}z_{ij}\), with
\(z_{ij}\in\partial d_{ij}(X)\).

It follows that \begin{equation}
\partial\rho(X)=B(X)X+Z
(\#eq:rhosubdef)
\end{equation} with \begin{equation}
Z\in\mathop{\sum\sum}\{w_{ij}\delta_{ij}\partial d_{ij}(X)\mid d_{ij}(X)=0\}.
(\#eq:zsubdef)
\end{equation} It also follows that \begin{equation}
\partial\sigma(X)=VX-\partial\rho(X)
(\#eq:sigsubdef)
\end{equation} Since \(\sigma\) is not convex the subdifferential
\(\partial\sigma(X)\) is the Clarke subdifferential (Clarke
(\citeproc{ref-clarke_75}{1975})).

Now \(X\) is a Clarke stationary point of \(\sigma\) if
\(0\in\partial\sigma(X)\), i.e. if and only if
\(X\in V^+\partial\rho(X)\). This means that stationary points are
generalized fixed points of the Guttman transform. A necessary condition
for \(\sigma\) to have a local minimum at \(X\) is that \(X\) is a
Clarke stationary point. The condition is far from sufficient, however,
since stationary points can also be saddle points or local maxima. De
Leeuw (\citeproc{ref-deleeuw_R_93c}{1993}) shows that stress only has a
single local maximum at the origin \(X=0\), but generally there are many
saddle points.

This little excursion into nonsmooth and convex analysis is rarely
needed in practice. We call a configuration \(X\) \emph{friendly} if
\(d_{ij}(X)>0\) for all \((i,j)\) for which \(w_{ij}\delta_{ij}>0\). In
De Leeuw (\citeproc{ref-deleeuw_A_88b}{1988}) such configurations were
called \emph{usable}, but that seems somewhat misleading, because
configurations which are not ``usable'' in this sense can sometimes even
be optimal. Unfortunately the set of friendly configurations is far from
convex. If \(X\) is friendly, then so is \(-X\), and halfway between the
two is the zero configuration, which is very unfriendly.

The equation \(d_{ij}(X)=0\), or equivalently \(x_i=x_j\), defines a
subspace of configuration space, and a configuration is friendly if it
is not in the union of the subspaces for all \((i,j)\) for which
\(w_{ij}\delta_{ij}=0\) (i.e.~if it is in the intersection of their
complements).

Suppose \(d_{ij}(X)>0\) and \(d_{ij}(Y)>0\). There is an \(\alpha\) sich
that \(d_{ij}(\alpha X+(1-\alpha)Y)=0\) iff
\(x_{is}-x_{js}=\lambda_{ij}(y_{is}-y_{js})\) with \(\lambda_{ij}<0\).

If \(X\) is friendly then \(X+\epsilon Y\) is friendly for \(\epsilon\)
small enough.

If \(X_1,\cdots,X_m\) are friendly then all \(Y\) in their convex hull
are a.s. friendly.

If \(X\) is friendly the rank of \(B(X)\) is \(n-1\). More importantly,
it was shown by De Leeuw (\citeproc{ref-deleeuw_A_84f}{1984}) that if
\(\sigma\) has local minimum at \(X\) then \(X\) is friendly. At
friendly configurations (and thus at local minima) \(\sigma\) is
differentiable, and the subdifferential @ref(eq:sigsubdef) is a
singleton, containing only the gradient. Stationary points then satisfy
\(X=V^+B(X)X\). In the case in which \(w_{ij}\delta_{ij}=0\) for some
\((i,j)\), however, then there can be local minima where \(\sigma\) is
not differentiable. This happens, for example, in multidimensional
unfolding (Mair, De Leeuw, and Wurzer
(\citeproc{ref-mair_deleeuw_wurzer_C_15}{2015})).

By the definition of the subdifferential \(Z\in\partial\rho(X)\) implies
\(\rho(X)\geq\text{tr}\ Z'X\) and \(\rho(Y)\geq\text{tr}\ Z'Y\) for all
\(Y\). If \(d_{ij}(X)>0\) this follows directly from the Cauchy-Schwartz
inequality \begin{equation}
d_{ij}(Y)\geq d_{ij}^{-1}(X)\text{tr}\ X'A_{ij}Y.
(\#eq:csineq)
\end{equation} Multiplying both sides by \(w_{ij}\delta_{ij}\) and
summing gives \begin{equation}
\rho(Y)\geq\text{tr}\ Y'B(X)X
(\#eq:rhoineq)
\end{equation} for all \(Y\), with equality if \(Y=X\). Note that we
also have equality if \(Y=\alpha X\) for some \(\alpha\geq 0\), and more
generally if for all \(i<j\) with \(w_{ij}\delta_{ij}(X)>0\) we have
equality in @ref(eq:csineq).

Using the Guttman transform we can use @ref(eq:rhoineq) to derive the
basic smacof equality \begin{equation}
\sigma(X)=1+\eta^2(X-\Phi(X))-\eta^2(\Phi(X))
(\#eq:smacofequality)
\end{equation} for all \(X\) and the basic smacof inequality
\begin{equation}
\sigma(X)\leq 1+\eta^2(X-\Phi(Y))-\eta^2(\Phi(Y))
(\#eq:smacofinequality)
\end{equation} for all \(X\) and \(Y\).

Taken together @ref(eq:smacofequality) and @ref(eq:smacofinequality)
imply the \emph{sandwich inequality} \begin{equation}
\sigma(\Phi(Y))\leq 1-\eta^2(\Phi(Y))\leq 1+\eta^2(Y-\Phi(Y))-\eta^2(\Phi(Y))=\sigma(Y).
(\#eq:sandwich)
\end{equation} If \(Y\) is not a fixed point of \(\Phi\) then the second
inequality in the chain is strict and thus
\(\sigma(\Phi(Y))<\sigma(Y)\). As we mentioned, the first inequality may
not be strict.

It also follows from @ref(eq:sandwich) that \(\eta^2(\Phi(Y))\leq 1\).
Thus the Guttman transforms are all in a convex and compact set, in fact
an ellipsoid, containing the origin.

\section{Basic Iteration}\label{basic-iteration}

\subsection{Function Values}\label{function-values}

The basic smacof algorithm generates the iterative sequence
\begin{equation}
X^{(k+1)}=\Phi(X^{(k)}),
(\#eq:basic)
\end{equation} where it is understood that we stop iterating if
\(X^{(k)}\) is a fixed point. If \(X^{(k)}\) is not a fixed point it
follows from @ref(eq:sandwich) that
\(\sigma(X^{(k+1)})<\sigma(X^{(k)})\). Thus, without any additional
assumptions, and using basically only the Cauchy-Schwartz inequality,
the algorithm either stops at a fixed point or produces a strictly
decreasing sequence of loss function values. Since stress is bounded
below by zero the sequence \(\sigma(X^{k})\) converges to, say,
\(\sigma_\infty\).

It was clear from the beginning (De Leeuw and Heiser
(\citeproc{ref-deleeuw_heiser_C_77}{1977})) that the case \(p=1\) is
special. The smacof algorithm always stops at a fixed point after a
finite, and usually small, number of iterations. This is not as good as
it sounds, because there are many (up to \(n!\)) local minima. The
unidimensional problem (Mair and De Leeuw
(\citeproc{ref-mair_deleeuw_C_15}{2015})) is essentially combinatorial,
and requires a specialized treatment. In this paper we assume throughout
that \(p\geq 2\).

The original derivation of the smacof algorithm (De Leeuw
(\citeproc{ref-deleeuw_C_77}{1977})) used the theory of maximization a
ratio of norms discussed by Robert (\citeproc{ref-robert_67}{1967}).
Later derivations (De Leeuw and Heiser
(\citeproc{ref-deleeuw_heiser_C_80}{1980}), De Leeuw
(\citeproc{ref-deleeuw_A_88b}{1988})) used the fact that
@ref(eq:smacofinequality) defines a majorization scheme for stress.
Convergence then follows from the general \emph{majorization principle}
(these days mostly known as the \emph{MM principle}), introduced in De
Leeuw (\citeproc{ref-deleeuw_C_94c}{1994}). A recent overview of the MM
approach is Lange (\citeproc{ref-lange_16}{2016}).

It was also realized early on that the smacof algorithm was a special
case of the the difference-of-convex functions algorithm (DCA),
introduced by Pham Dinh Tao around 1980. Pham Dinh also started his work
in the context of ratio's of norms, using Robert's fundamental ideas.
Around 1985 he generalized his approach to minimizing DC functions of
the form \(h=f-g\), with both \(f\) and \(g\) convex. The basic idea is
to use the subgradient inequality \(g(x)\geq g(y)+z'(x-y)\), with
\(z\in\partial g(x)\), to construct the majorization
\(h(x):=f(x)-g(y)-z'(x-y)\). Now \(h\) is obviously convex in \(x\). The
DC algorithm then chooses the successor of \(y\) as the minimizer of
this convex majorizer over \(x\). In smacof the role of \(f\) is played
by \(\eta^2\) and the role of \(g\) by \(\rho\). DCA is applied to MDS
in Le Thi and Tao (\citeproc{ref-lethi_tao_01}{2001}). Extensive recent
surveys of the DC/DCA approach are Le Thi and Tao
(\citeproc{ref-lethi_tao_18}{2018}) and Le Thi and Tao
(\citeproc{ref-lethi_tao_24}{2024}).

Thus the smacof algorithm is both MM and DCA, which means that it
inherits all results that have been established for these more general
classes of algorithms. But additional results can be obtained by using
the special properties of the stress loss function and the smacof
iterations. In the DCA context, for example, the convex subproblem that
must be solved by smacof in each step is quadratic, and has the closed
form solution provided by the Guttman transform.

The loss function values are a bounded decreasing, and thus converging,
sequence. De Leeuw (\citeproc{ref-deleeuw_A_88b}{1988}) derives some
additional smacof-specific results. Using up-arrows and down-arrows for
monotone convergence

\begin{itemize}
\tightlist
\item
  \(\rho(X^{(k)})\uparrow\rho_\infty\),
\item
  \(\eta^2(X^{(k)})\uparrow\eta^2_\infty=\rho_\infty\),
\item
  \(\sigma(X^{(k)})\downarrow\sigma_\infty=1-\rho_\infty\),
\end{itemize}

and, last but not least, the sequence \(\{X^{(k)}\}\) is
\emph{asymptotically regular}, i.e. \begin{equation}
\omega^2(X):=\eta^2(X^{(k+1)}-X^{(k)})\rightarrow 0.
(\#eq:etaconv)
\end{equation} This last, very important, result follows because
\begin{equation}
\eta^2(X^{(k+1)}-X^{(k)})=\eta^2(X^{(k+1)})+\eta^2(X^{(k)})-2\rho(X^{(k)}),
(\#eq:etanull)
\end{equation} which converges to zero because
\(\eta^2_\infty=\rho_\infty\). Note that these results are based
completely on the Cauchy-Schwartz inequality and are consequently true
for the general iteration \(X^{(k+1)}\in V^+\partial\rho(X^{(k)})\),
without assuming differentiability.

If \[
\partial\sigma(X^{(k)})=VX^{(k)}-\partial\rho(X^{(k)}) =V(X^{(k)}-X^{(k+1)})
\] Consequently @ref(eq:etaconv) can equivalently be written as \[
\|\partial\sigma(X^{(k)})\|\rightarrow 0
\]

Strictly spoken, the results so far prove convergence of the scalar
sequences \(\{\rho(X^{(k)})\}, \{\eta^2(X^{(k)})\}\) and
\(\{\eta^2(X^{(k+1)}-X^{(k)})\}\) associated with the iterations, and
they do not prove convergence of the sequence \(X^{(k)}\). But in De
Leeuw (\citeproc{ref-deleeuw_A_88b}{1988}) I argue that these scalar
convergence results are sufficient from a practical point of view. If we
define an \(\epsilon\)-fixed-point as any configuration \(X\) with
\(\eta(X-\Phi(X))<\epsilon\) then smacof produces such an
\(\epsilon\)-fixed-point in a finite number of steps.

Also, we can use the general convergence result in theorem 3.1 of Meyer
(\citeproc{ref-meyer_76}{1976}) to get results about \(\{X^{(k)}\}\).
Because

\begin{itemize}
\tightlist
\item
  the subdifferential is a upper semi-continuous (closed) map,
\item
  all iterates are in the compact set \(\eta^2(X)\leq 1\), and
\item
  \(\Phi\) is strictly monotonic (decreases stress at non-fixed points),
\end{itemize}

it follows that the sequence \(\{X^{(k)}\}\) has accumulation points
(converging subsequences) and that

\begin{itemize}
\tightlist
\item
  all accumulation points are fixed points, and
\item
  all accumulation points have the same function value
  \(\sigma_\infty\).
\end{itemize}

Moreover, from asymptotic regularity and theorem 26.1 of Ostrowski
(\citeproc{ref-ostrowski_73}{1973}),

\begin{itemize}
\tightlist
\item
  either the sequence \(\{X^{(k)}\}\) converges or its accumulation
  points form a continuum (a connected and compact set).
\end{itemize}

In order to prove actual Cauchy convergence, additional conditions are
needed. Meyer (\citeproc{ref-meyer_76}{1976}) proves convergence if the
number of fixed points with function value \(\sigma_\infty\) is finite,
or if the sequence has an accumulation point that is an isolated fixed
point. Both these conditions are not met in MDS, because of rotational
indeterminacy. If \(X_\infty\) is a fixed point, then all elements of
the continuum of rotations of \(X_\infty\) are fixed points.

It should also be mentioned that smacof can converge to stationary
points that are not local minima (and thus saddle points). Suppose all
weights are equal to one, \(\delta_{12}>0\), and
\(\delta_{1j}=\delta_{2j}\) for all \(j>2\). If \(d_{12}(X)=0\) then
also \(d_{12}(\Phi(X))=0\), and \(d_{12}(X)\) will be zero for all
iterates, and thus for all subsequential limits, which consequently
cannot be local minima. Another example uses the result that yet another
necessary consition for a local minimum is that \(X\) has full column
rank. Suppose we start iterations at \((X\mid 0)\), i.e.~\(X\) with some
columns of zeroes added. All updates will also have this form, and
convergence again cannot be to a local minimum. This last example can be
generalized to any \(X\) with rank less than \(p\), because all updates
will then also have rank less than \(p\). As a consequence if \(q>p\)
all local minima for \(p-\)dimensional MDS are saddle points for
\(q-\)dimensional MDS.

In two very recent impressive papers Ram and Sabach
(\citeproc{ref-ram_sabach_24}{2024 (in press)}) and Robini, Wang, and
Zhu (\citeproc{ref-robini_wang_zhu_24}{2024}) use the powerful
Kurdyka-Łojasiewicz (KL) framework (Bolte, Daniilidis, and Lewis
(\citeproc{ref-bolte_daniilidis_lewis_07}{2007}), Bolte, Sabach, and
Teboulle (\citeproc{ref-bolte_sabach_teboulle_14}{2014})) to prove
actual global convergence of the smacof iterates to a fixed point. We
shall use the more classical local convergence analysis, based on the
differentiability of the Guttman transform.

We apply basic iterations to the two-dimensional MDS analysis of the
classical color-circle example from Ekman
(\citeproc{ref-ekman_54}{1954}), which has \(n=14\) points. In our
numerical examples we always use weights equal to one. We always start
with the classical Torgerson-Gower solution and we stop if
\(\sigma(X^{(k)})-\sigma(X^{(k+1)})<1e-15\). We distinguish
\(f-convergence\), which happens if the stress value from one iteration
to the next changes less than a small \(\epsilon\), and
\(x-convergence\), which happens if
\(\eta(\sigma(X^{(k)})-\sigma(X^{(k+1)})\) is less than another small
\(\epsilon\).

The fit for the Ekman example is very good and convergence is rapid. In
iteration 56, the final iteration, stress is 2.1114112739076. The change
CHNG \(\eta(X^{(k)}-X^{(k-1)})\) in the final iteraton is
2.26054683805646e-16. The estimated asymptotic rate of convergence or
\emph{EARC} is the CHNG divided by the CHNG of the previous iteration.
In the final iteration of this analysis it is 0.766978439824377.

\subsection{Asymptotic Rate of
Convergence}\label{asymptotic-rate-of-convergence}

In order to study the asymptotic rate of convergence (ARC) of smacof, we
have to compute the derivative of the Guttman transform and its
eigenvalues (Ortega and Rheinboldt
(\citeproc{ref-ortega_rheinboldt_70}{1970}), chapter 10). Thus we assume
we are in the neighborhood of a configuration where the Guttman
transform is (infinitely many times) differentiable, for example near a
local minimizer.

The derivative of \(\Phi\) at \(X\), first given in De Leeuw
(\citeproc{ref-deleeuw_A_88b}{1988}), is the linear transformation
\(\mathcal{D}\Phi_X\), mapping the space of column-centered
\(n\times p\) matrices into itself. Its value at matrix \(H\) is equal
to \begin{equation}
\mathcal{D}\Phi_X(H)=V^+\sum w_{ij}\frac{\delta_{ij}}{d_{ij}(X)}\left\{A_{ij}H-\frac{\text{tr}\ X'A_{ij}H}{ \text{tr}\ X'A_{ij}X}A_{ij}X\right\}.
(\#eq:jacobian)
\end{equation}

It follows that \(\mathcal{D}\Phi_X(X)=0\) for all \(X\) and the
derivative has at least one zero eigenvalue. If we think of equation
@ref(eq:jacobian) as a linear transformation on the space of all
\(n\times p\) matrices, then there are an additional \(p\) zero
eigenvalues corresponding with translational invariance. If we define
@ref(eq:jacobian) on the space of column-centered matrices, then those
zero eigenvalues disappear.

If \(S\) is anti-symmetric and \(H=XS\) then \(\text{tr}\ X'A_{ij}H=0\)
and thus \(\mathcal{D}\Phi_X(XS)=\Phi(X)S\). If in addition \(X\) is a
fixed point then \(\mathcal{D}\Phi_X(XS)=XS\), which means that at a
fixed point \(\mathcal{D}\Phi_X\) has \(\frac12p(p-1)\) eigenvalues
equal to one. These correspond to the rotational indeterminacy of the
MDS problem and the smacof iterations. It also follows from
\(\mathcal{D}\Phi_X(XS)=\Phi(X)S\) that for all \(X\) and all
anti-symmetric \(S\) the inner product
\(\text{tr}\ \Phi(X)'V\mathcal{D}\Phi_X(XS)\) is zero, i.e.~\(\Phi(X)\)
is orthogonal to the \(\frac12 p(p-1)\) dimensional subspace of all
\(\mathcal{D}\Phi_X(XS)\).

Since \(\Phi(X)=V^+\mathcal{D}\rho(X)\) the derivative of the Guttman
transform has a simple relationship with the second derivatives of
\(\rho\). The second derivative, again from De Leeuw
(\citeproc{ref-deleeuw_A_88b}{1988}), is giuven by the quadratic form
\begin{equation}
\mathcal{D}^2\rho_X(H,H)=\sum w_{ij}\frac{\delta_{ij}}{d_{ij}(X)}\left\{\text{tr}\ H'A_{ij}H-\frac{(\text{tr}\ H'A_{ij}X)^2}{d_{ij}^2(X)}\right\}.
(\#eq:hessian)
\end{equation} Since \(\rho\) is convex, all eigenvalues of
\(\mathcal{D}^2\rho_X\), and thus of \(\mathcal{D}\Phi_X\), are real and
nonnegative. It also follows that if \(G\) and \(H\) are eigenvectors of
\(\mathcal{D}\Phi_X\) with different eigenvalues then
\(\text{tr}\ G'VH=0\). In addition we see from equation @ref(eq:hessian)
that \(0\lesssim\mathcal{D}^2\rho_X\lesssim B(X)\) in the Loewner sense.
Since \(\mathcal{D}^2\sigma_X=V-\mathcal{D}^2\rho_X\), and we have
\(\mathcal{D}^2\sigma_X\gtrsim 0\) at a local minimum, it follows that
\(\mathcal{D}\Phi_X\lesssim I\). Thus all eigenvalues of the derivative
\(\mathcal{D}\Phi_X\) at a local minimum \(X\) are between zero and one.

We compute the Jacobian corresponding to the derivative
\(\mathcal{D}\Phi_X\) in two ways. First with a loop over
\(i=1,\cdots,n\) and \(s=1,\cdots,p\) by\\
setting \(H\) equal to each \(e_i^{\ }e_s'\) in turn in formula
@ref(eq:jacobian). Second, just to be sure, by using the jacobian
function from the numDeriv package (Gilbert and Varadhan
(\citeproc{ref-gilbert_varadhan_19}{2019})). If the two results agree,
we use the one based on @ref(eq:jacobian).

The eigenvalues of the derivative \(\mathcal{D}\Phi_X\) at the solution
are

\begin{verbatim}
 [1]   +1.0000000000   +0.7669964993   +0.7480939418   +0.7185926294
 [5]   +0.7007452309   +0.6920114813   +0.6859492533   +0.6593334529
 [9]   +0.6541779410   +0.6477573343   +0.6237683213   +0.6178713316
[13]   +0.5735285951   +0.5483330653   +0.5260355535   +0.5112510731
[17]   +0.5064703617   +0.5059294792   +0.4919752630   +0.4827646555
[21]   +0.4782034995   +0.4757907675   +0.4682965893   +0.4619226491
[25]   +0.4559704884   +0.0000000000   +0.0000000000   -0.0000000000
\end{verbatim}

Note that the largest non-trivial eigenvalue, which is another and
usually better estimate of the ARC, is equal to the EARC in the final
iteration.

\subsection{Modifications}\label{modifications}

As De Leeuw (\citeproc{ref-deleeuw_A_88b}{1988}) mentions, we cannot
directly apply the basic point-of-attraction theorem 10.1.3 and the
equally basic linear convergence theorem 10.1.4 from Ortega and
Rheinboldt (\citeproc{ref-ortega_rheinboldt_70}{1970}), because at a
fixed point of smacof there are \(\frac12 p(p-1\) eigenvalues equal to
one.

\[\Xi(X)=\Pi(\Phi(X))\] \[
\mathcal{D}\Xi_X(H)=\mathcal{D}\Pi_{\Phi(X)}(D\Phi_X(H))
\] \[
\Xi(X)=\Pi(X,\Phi(X))
\] \[
\mathcal{D}\Xi_X(H)=\Pi(X+H,\Phi(X)+\mathcal{D}\Phi_X(H))=\mathcal{D}_1\Pi_{X,\Phi(X)}(H)+
\mathcal{D}_2\Pi_{X,\Phi(X)}(\mathcal{D}\Phi_X(H)).
\]

\section{Orthogonalization}\label{orthogonalization}

\subsection{Modification}\label{modification}

One way around this problem (De Leeuw
(\citeproc{ref-deleeuw_E_19h}{2019})) is to rotate each update to
orthogonality, i.e.~to principal components. Thus the update formula
becomes \(\Xi(X)=\Pi(\Phi(X))\), with \(\Pi(X)=XL\), where \(L\) are the
right singular vectors of \(X\). The reasoning is simple. If \(\Omega\)
is any differentiable mapping with \(\Omega(XK)=\Omega(X)K\) for all
orthonormal \(K\) then \(\mathcal{D}\Omega_X(XA)=XA\) for all
anti-symmetric \(A\). But if \(\Omega(XK)=\Omega(X)\) for all
orthonormal \(K\) then \(\mathcal{D}\Omega_X(XA)=0\) for all
anti-symmetric \(A\).

With orthogonality restrictions we can expect isolated local minima,
where the largest eigenvalue of the algorithmic map is strictly less
than one. Such local minima are points of attraction, which means
convergence to that point if the iterations get close enough. It also
means that if we assume that there is only a finite number of these
orthogonal stationary points, then the smacof algorithm converges
globally to one of them.

\subsection{Function Values}\label{function-values-1}

This modified algorithm generates the same sequence of \(\rho\),
\(\eta\), and \(\sigma\) values as basic smacof.

But \(\mathfrak{e}\) is a different sequence In fact
\(\mathfrak{e}(XL)=\eta^2(X)+\eta^2(\Phi(X))-2\ \text{tr}\ X'V\Phi(X)\)

Moreover \(\Xi^n(X)=\Pi(\Phi^n(X))\), which means that we can find any
term of the orthogonalized sequence by orthogonalizing the corresponding
term in the basic sequence. Thus, in actual computation, there is no
need to orthogonalize, we may as well compute the basic sequence and
orthogonalize after convergence.

In iteration 55, the final iteration, stress is 2.1114112739076. The
CHNG is 3.864164805803e-16 and the EARC is 0.766992047059491.

\subsection{Asymptotic Rate of
Convergence}\label{asymptotic-rate-of-convergence-1}

Reference: De Leeuw (\citeproc{ref-deleeuw_B_21}{2021}), section
5.4.5.2, De Leeuw (\citeproc{ref-deleeuw_E_19h}{2019})

Suppose the singular value decomposition of \(X\) is \(X=K\Lambda L'\).
We transform \(X\) to \(\Pi(X)=XL=K\Lambda\).

Thus ref \begin{equation}
\mathcal{D}\Pi_X(H)=HL+K\Lambda S.
(\#eq:finalpca)
\end{equation} where \(S\) is anti-symmetric, with off-diagonal elements
\begin{equation}
s_{ij}=-\frac{\lambda_i u_{ij}+\lambda_j u_{ji}}{\lambda_i^2-\lambda_j^2},
(\#eq:spca)
\end{equation} where \(U:= K'HL\).

If \(X\) is orthogonal, then \(L=I\) and \(X'X=\Lambda^2\).

\begin{itemize}
\item
  If \(H=XA\) then \(U=\Lambda L'AL\), \(\Lambda U=\Lambda^2 L'AL\) and
  \(U'\Lambda=L'A'L\Lambda^2\). If \(A\) is anti-symmetric then
  \(\Lambda U+U'\Lambda=\Lambda^2 L'AL-L'AL\Lambda^2\) and
  \(\{\Lambda U+U'\Lambda\}_{ij}=(\lambda_i^2-\lambda_j^2)\{L'AL\}_{ij}\).
  Thus \(S=-L'AL\) and \(\mathcal{D}\Pi_X(XA)=0\).
\item
  If \(H=K\Lambda^{-1}A\) with \(A\) anti-symmetric, then
  \(U=\Lambda^{-1}AL\) \(\Lambda U+U'\Lambda=AL+L'A'\) If \(L=I\) then
  \(S=0\).
\item
  If \(H=K\Lambda^{-1}D\) with \(D\) diagonal, then \(U=\Lambda^{-1}DL\)
  and \(\Lambda U+U'\Lambda=DL+L'D\). If \(L=I\) this is diagonal, and
  thus \(S=0\).
\end{itemize}

Also \(XS=0\) if and only if \(S=0\) if and only if
\(\text{nondiag}(X'H)\) is anti-symmetric. Thus true for \(H=X_\perp B\)
as well as for \(H\) with \(X'H\) us diagonal. If \(H=XU\) then
\(l_i'(H'X+X'H)l_j-\)\{U'\Lambda+\Lambda U\}\emph{\{ij\}=u}\{ji\}\lambda\_j+\lambda\emph{iu}\{ij\}\$
Eigenvalues of \(\mathcal{D}\Pi_X(H)\). \(XL=K\Lambda^\frac12\)
\(H=K\Lambda^\frac12 A+K_\perp B\).

\[
L'(H'X+X'H)L
\] \(X=K\Lambda L'\) \(H=KAL'+K_\perp BL'\) \(L'X'HL=\Lambda A\)
\(\Lambda A+A\Lambda=0\)

We next compute the derivative of \(\Xi\). By the chain rule
\begin{equation}
\mathcal{D}\Xi_X(H)=\mathcal{D}\Pi_{\Phi(X)}(\mathcal{D}\Phi_X(H)).
(\#eq:chain)
\end{equation} Thus, from equations @ref(eq:chain) and @ref(eq:finalpca)
\begin{equation}
\mathcal{D}\Xi_X(H)=\mathcal{D}\Phi_X(H)L+\Phi(X)LS
(\#eq:xideriv)
\end{equation} with \(L\) and \(S\) computed from the singular value
decomposition of \(\Phi(X)\).

At a fixed point of \(\Xi\) we have both \(\Phi(X)=X\) and \(\Pi(X)=X\),
and consequently \(L=I\) and \(X'X=\Lambda\). (Is this true ? We could
have \(\Phi(X)=XK\), because then still \(\Xi(X)=X\))

Equation @ref(eq:xideriv) becomes \begin{equation}
\mathcal{D}\Xi_X(H)=\mathcal{D}\Phi_X(H)+XS,
(\#eq:xiderivfixed)
\end{equation} where now \begin{equation}
s_{ij}=-\frac{(H'X+X'H)_{ij}}{\lambda_i-\lambda_j}.
(\#eq:sdeffixed)
\end{equation}

At a fixed point \(X\) the eigenvectors \(H\) of \(\mathcal{D}\Phi_X\)
with eigenvalue one are of the form \(H=XA\) with \(A\) any
anti-symmetric matrix. From @ref(eq:xiderivfixed) \begin{equation}
\mathcal{D}\Xi_X(XA)=XA+XS,
(\#eq:xiderivasym)
\end{equation} where \begin{equation}
s_{ij}=-\frac{(A'\Lambda+\Lambda A)_{ij}}{\lambda_i-\lambda_j}=-a_{ij}.
(\#eq:sdefasym)
\end{equation} Thus \(\mathcal{D}\Xi_X(XA)=0\) and the unit eigenvalue
has been replaced by a zero eigenvalue.

Moreover, at a fixed point \(X\) of \(\Phi\), if \(H\) is an eigenvector
of \(\mathcal{D}\Phi_X\) with eigenvalue \(\lambda<1\), then
\(H+\lambda^{-1}XS\) is an eigenvector of \(\mathcal{D}\Xi_X\) with
eigenvalue \(\lambda\). This follows from \begin{equation}
\mathcal{D}\Xi_X(H+\lambda^{-1}XS)=\mathcal{D}\Xi_X(H)=\lambda H+XS=\lambda(H+\lambda^{-1}XS).
(\#eq:evaltrans)
\end{equation} Thus, except for the trivial unit eigenvalue which
becomes zero, both sets of eigenvalues are the same, and so is the ARC.

\[
\mathcal{D}\Xi_X(H)=\mathcal{D}\Phi_X(H)+XS
\] \[
\mathcal{D}\Xi_X(X)=\mathcal{D}\Phi_X(X)+XS=XS
\]

If \(\mathcal{D}\Phi_X(H)=\lambda H\) then \[
\text{tr}\ X'V\mathcal{D}\Xi_X(H)=\lambda\ \text{tr}\ X'VH+\text{tr}\ X'VXS=0
\]

which implies \(\lambda = 1\). The eigenvalues of the Jacobian are

\begin{verbatim}
 [1]   +0.7669964950   +0.7480939419   +0.7185926296   +0.7007452320
 [5]   +0.6920114814   +0.6859492533   +0.6593334535   +0.6541779411
 [9]   +0.6477573344   +0.6237683215   +0.6178713316   +0.5735285953
[13]   +0.5483330652   +0.5260355534   +0.5112510730   +0.5064703617
[17]   +0.5059294792   +0.4919752629   +0.4827646555   +0.4782035015
[21]   +0.4757907660   +0.4682965891   +0.4619226497   +0.4559704891
[25]   -0.0000000000   -0.0000000000   +0.0000000000   +0.0000000000
\end{verbatim}

Orthogonalization gives the same EARC as the basic sequence, but the
Jacobian of \(\Xi\) at a local minimum does not have the unit
eigenvalues any more. They are replaced by zeroes, reflecting the fact
that we are iterating on the nonlinear manifold or orthogonal
column-centered matrices.

It is now sufficient for local linear convergence to assume that the
largest eigenvalue of the Jacobian at the solution is strictly less than
one, or alternatively assume that one of the accumulation points is an
isolated local minimum.

\section{Subspace Rotation}\label{subspace-rotation}

\subsection{Modification}\label{modification-1}

Instead of orthogonality we can also require \(X\) to be in the subspace
of all lower triangular column-centered \(n\times p\) matrices (which
means \(x_{ij}=0\) for all \(i<j\)). This also identifies \(X\) in the
manifold of rotated solutions.

In subspace rotation we use a rotation of \(X\) to lower triangular
form. We use the same notation as in the previous section, overloading
some symbols. The transformation of the update is again \(\Pi\) and the
transformed update is \(\Xi\). Thus \(\Xi(X)=\Pi(\Phi(X))\).

Suppose \(X_1\) are the first \(p\) rows of \(X\), and \(X_1'=QR\) is
the QR-decomposition of the transpose. Thus \(Q\) is square orthonormal
and \(R\) is upper triangular. Then \(X_1Q=R'Q'Q=R'\), which is lower
triangular, as desired. Note that the transformation to lower triangular
form only uses the first \(p\) rows of \(X\), and does not depend on the
other \((n - p)\times p\) elements.

\subsection{Function Values}\label{function-values-2}

The results are pretty much the same as for the rotation to principal
components in the previous section.

In iteration 55, the final iteration, stress is 2.1114112739076. CHNG
\(\eta(X^{(k)}-X^{(k+1)})\) is 3.84673519544424e-16 and EARC is
0.766987804354728.

\subsection{Asymptotic Rate of
Convergence}\label{asymptotic-rate-of-convergence-2}

To compute the derivative of \(\Pi\) we first compute the derivative of
the QR decomposition of a square non-singular matrix \(X\), using the
results of De Leeuw (\citeproc{ref-deleeuw_E_23a}{2023}). Perturb
\(X=QR\), with Q square orthonormal and R upper triangular, to
\(X+H=(Q+P)(R+S)\). Collecting the first order terms gives
\begin{equation}
H=QS+PR.
(\#eq:qrfirst)
\end{equation} Because \((Q+P)'(Q+P)=I\) we see that \(Q'P+P'Q=0\), and
thus \(P=QA\) with \(A\) anti-symmetric. \begin{equation}
H=QS+QAR.
(\#eq:qrsecond)
\end{equation} Pre-multiplying by \(Q'\) and post-multiplying by
\(R^{-1}\) gives \begin{equation}
A=Q'HR^{-1}-SR^{-1}
(\#eq:qrthird)
\end{equation} Both \(S\) and \(R^{-1}\) are upper triangular, and so is
their product. Suppose \(\text{lt}\) replaces the upper triangular part
(including the diagonal) of a matrix by zeroes. Then, from
@ref(eq:qrthird), \begin{equation}
\text{lt}(A)=\text{lt}(Q'HR^{-1})
(\#eq:qrfourth)
\end{equation} and by anti-symmetry the upper-triangular part of \(A\)
is minus the transpose of \(\text{lt}(A)\). This gives the derivative
\begin{equation}
\mathcal{D}Q_X(H)=QA.
(\#eq:qrfifth)
\end{equation}

In our rotation procedure we apply QR to \(X_1\), which is the transpose
of the leading \(p\times p\) submatrix of the \(n\times p\) matrix \(X\)
(assumed to be non-singular). Let \(H_1\) be the transpose of the
corresponding submatrix of \(H\). Then \ref(eq:qrfifth) applies with
\(Q\) and \(A\) computed at \(X_1\) and \(H_1\). Thus \[
\mathcal{D}\Pi_X(H)=HQ+XQA
\] (Eigenvalues Jacobian one zero, one negative (equal to trace), 13
minus one, 13 plus one)

and thus \[
\mathcal{D}\Xi_X(H)=\mathcal{D}\Pi_{\Phi(X)}(D\Phi_X(H))=HQ+\Phi(X)QA
\] with \(Q\) and \(A\) now computed at the submatrices
\(\{\Phi(X)\}_1\) and \(\{D\Phi_X(H)\}_1\).

At a fixed point \(\Phi(X)=X\) and both \(Q=I\) and \(R=I\). Thus \[
\mathcal{D}\Xi_X(H)=H+XA.
\]

\section{Subspace Restriction}\label{subspace-restriction}

\subsection{Modification}\label{modification-2}

Method two uses the theory of constrained smacof of De Leeuw and Heiser
(\citeproc{ref-deleeuw_heiser_C_80}{1980}). In this case this means
computing the Guttman update and then projecting it on the subspace of
lower triangular matrices. We create \(p\) column-centered matrices
\(Y_s\), \(s=1,\cdots,p\), of dimension \(n\times(n-s)\), that satisfy
\(Y_s'VY_s=I\) and have their first \(s-1\) rows equal to zero. Now
column \(s\) of \(X\) is restricted to be of the form
\(x_s=Y_s\theta_s\). The transformation \(\Pi\), for dimension \(s\), is
\begin{equation}
\Pi(X)_s=Y_sY_s'Vx_s
(\#eq:subspace)
\end{equation}

Alt: Minimize \[
\text{tr}\ (X_1-Y_1)'V_{11}(X_1-Y_1)+2\text{tr}\ (X_1-Y_1)'V_{12}(X_2-Y_2)+\text{tr}\ (X_2-Y_2)'V_{22}(X_2-Y_2)
\] requiring that \(Y_1\) is upper-triangular. Now \[
X_2-Y_2=V_{22}^{-1}V_{21}(X_1-Y_1)
\] and thus it suffices to minimize \[
\text{tr}\ (X_1-Y_1)'V_{1|2}(X_1-Y_1)
\] with \(V_{1|2}\) the Schur complement
\(V_{11}-V_{12}V_{22}^{-1}V_{21}\) over upper-triangular \(Y\).

Alt: direct

\[
d_{ij}(\theta)=\sqrt{\sum_{s=1}^p\theta_s'Y_s'A_{ij}Y_s\theta_s}.
\] \[
\rho(\theta)=\theta_s'Y_s'B(\theta)Y_s\theta_s\geq\theta_s'Y_s'B(\tilde\theta)Y_s\tilde\theta_s.
\] \[
\theta_s^{(k+1)}=Y_s'B(\theta^{(k)})Y_s\theta_s^{(k)}
\]

\subsection{Function Values}\label{function-values-3}

The subspace restrictions have a devastating effect on the rate of
convergence of the smacof iterations.

Although the final stress is the correct 2.11141127390763, and the final
CHNG is 5.95185745918126e-16, it takes 443 iterations and the EARC is
0.962237154391956.

\subsection{Asymptotic Rate of
Convergence}\label{asymptotic-rate-of-convergence-3}

In this case computing the derivatives of @ref(eq:subspace) is very
simple indeed. We

\begin{verbatim}
 [1]   +0.9622371565   +0.7669637116   +0.7480360811   +0.7105397440
 [5]   +0.7007452013   +0.6920054108   +0.6859241126   +0.6593334167
 [9]   +0.6541666413   +0.6476391845   +0.6234727079   +0.6172949237
[13]   +0.5586355758   +0.5478680315   +0.5260205816   +0.5111279921
[17]   +0.5064290866   +0.4929119272   +0.4843668597   +0.4783221142
[21]   +0.4758041101   +0.4686440280   +0.4665512407   +0.4575763680
[25]   -0.0000000000   -0.0000000000   -0.0000000000   +0.0000000000
\end{verbatim}

\section{Symmetric Iteration}\label{symmetric-iteration}

\[
\Gamma(C):=V^+B(C)CB(C)V^+
\]

\[
B(C)=\sum_{i<j} w_{ij}\frac{\delta_{ij}}{d_{ij}(C)}A_{ij}
\] with \(d_{ij}(C)=\sqrt{\text{tr}\ A_{ij}C}\)

\[
B(C+\Xi)=B(C)-\frac12H(C,\Xi)
\] with \[
H(C,\Xi):=\sum_{i<j}w_{ij}\frac{\delta_{ij}}{d_{ij}(C)}\{\text{tr}\ A_{ij}\Xi\} A_{ij}
\] Note \(H(C,C)=B(C)\). If \(\Xi\) is psd then
\(\text{tr}\ A_{ij}\Xi=d_{ij}^2(\Xi)\) \[
G(C,\Xi):=\lim_{\epsilon\downarrow 0}\frac{\Gamma(C+\epsilon\Xi)-\Gamma(C)}{\epsilon}=-\frac12V^+B(C)CH(C,\Xi)V^+-\frac12V^+H(C,\Xi)CB(C)V^++V^+B(C)\Xi B(C)V^+
\] Note that \(G(C,C)=0\). If \(V^+B(C)C=C\) then \[
G(C,\Xi)=-\frac12CH(C,\Xi)V^+-\frac12V^+H(C,\Xi)C+V^+B(C)\Xi B(C)V^+
\]

Now compare with \(\Gamma(X)\).

\[
H(X,\Theta)=\sum_{i<j} w_{ij}\frac{\delta_{ij}}{d_{ij}^3(X)}\{\text{tr}\ X'A_{ij}\Theta\}A_{ij}
\] Note \(H(X,X)=B(X)\). If \(\Theta=XT\) with \(T\) antisymmetric then
\(H(X,\Theta)=0\). Also note that

\[
G(X,\Theta):=\lim_{\epsilon\downarrow 0}\frac{\Gamma(X+\epsilon\Theta)-\Gamma(C)}{\epsilon}=V^+B(X)\Theta-V^+H(X,\Theta)X
\] If \(\Theta=XT\) with \(T\) antisymmetric then
\(G(X,\Theta)=V^+B(X)XT\) and if also \(V^+B(X)X=X\) then
\(G(X,\Theta)=\Theta\). Thus \(\mathcal{D}\Gamma(X)\) at a stationary
point has \(\frac12p(p-1)\) eigenvalues equal to one. For all \(X\) it a
one zero eigenvalue with eigenvector \(X\).

\section{Relaxed}\label{relaxed}

\subsection{Modification}\label{modification-3}

De Leeuw and Heiser (\citeproc{ref-deleeuw_heiser_C_80}{1980}) first
suggested the ``relaxed'' update \begin{equation}
\Psi(X):=2\Phi(X)-X.
(\#eq:relax)
\end{equation} The reason for recommending @ref(eq:relax) is two-fold.
First, the smacof inequality @ref(eq:smacofinequality) says
\begin{equation}
\sigma(X)\leq 1+\eta^2(X-\Phi(Y))-\eta^2(\Phi(Y)).
(\#eq:smaineq)
\end{equation} If \(X=\alpha\Phi(Y)+(1-\alpha)Y\) then this becomes
\begin{equation}
\sigma(\alpha\Phi(Y)+(1-\alpha)Y)\leq 1+(1-\alpha)^2\eta^2(Y-\Phi(Y))-\eta^2(\Phi(Y))
\end{equation} If \((1-\alpha)^2\leq 1\) then \begin{equation}
1+(1-\alpha)^2\eta^2(Y-\Phi(Y))-\eta^2(\Phi(Y))\leq 1+\eta^2(Y-\Phi(Y))-\eta^2(\Phi(Y))=\sigma(Y)
\end{equation} Thus updating with
\(X^{(k+1)}=\alpha\Phi(X^{(k)})+(1-\alpha)X^{(k)}\) is a stricly
monotone algorithm as long as \(0\leq\alpha\leq 2\).

But if \(\alpha=2\) an \(Y=\lambda X\) !!

The second reason for choosing the relaxed update @ref(eq:relax) given
by De Leeuw and Heiser (\citeproc{ref-deleeuw_heiser_C_80}{1980}) is
that its asymptotic convergence rate is \begin{equation}
\max_s|2\lambda_s-1|=\max(2\lambda_{\text{max}}-1,1-2\lambda_{\text{min}}).
\end{equation} De Leeuw and Heiser
(\citeproc{ref-deleeuw_heiser_C_80}{1980}) then somewhat carelessly
assume that this is equal to \(2\lambda_{\text{max}}-1\) and argue that
if \(\lambda_{\text{max}}=1-\epsilon\) with \(\epsilon\) small, as it
usually is in MDS, then \begin{equation}
2\lambda_{\text{max}}-1=1-2\epsilon\approx(1-\epsilon)^2=\lambda_{\text{max}}^2,
\end{equation} so that the relaxed update requires approximately half
the number of iterations of the basic update. Despite the somewhat
sloppy reasoning, the approximate halving of the number of iterations is
often observed in practice.

\subsection{Function Values}\label{function-values-4}

It turns out (Groenen, Glunt, and Hayden
(\citeproc{ref-groenen_glunt_hayden_96}{1996}), De Leeuw
(\citeproc{ref-deleeuw_R_06b}{2006})), however, that applying the
relaxed update has some unintended consequences, which basically imply
that it should never be used without additional precautions. Let's take
a look at the Ekman results. ::: \{.cell\} ::: \{.cell-output
.cell-output-stdout\}

\begin{verbatim}
itel   25 sold 3.994627066568261 snew 3.994627066568263 chng 3.766431585321326 labd 1.000000000000000 
\end{verbatim}

::: ::: In iteration 25, the final iteration, stress is
3.99462706656826. The ``change'' \(\eta(X^{(k)}-X^{(k+1)})\) is
3.76643158532133 and the estimate of the asymptotic convergence ratio,
the ``change'' divided by the ``change'' of the previous iteration, is
1.

The loss function values converge and the number of iterations is
reduced from 57 to 23. But we see that \(\eta(X^{(k+1)}-X^{(k)})\) does
not converge to zero, and that \(\sigma_k\) converges to a value which
does not correspond to a local minimum of \(\sigma\).

If we check the conditions of theorem 3.1 in Meyer
(\citeproc{ref-meyer_76}{1976}) we see that, although the algorithmic
map is closed and the iterates are in a compact set, \(\Psi\) is not
strictly monotone at some non-fixed points. The problem was first
discussed in Groenen, Glunt, and Hayden
(\citeproc{ref-groenen_glunt_hayden_96}{1996}). Suppose \(X\) is a fixed
point and \(\tau\not= 1\). Then \(\tau\overline{X}\) is not a fixed
point of \(\Psi\), because
\(\Psi(\tau\overline{X})=(2-\tau)\overline{X}\). And \begin{equation}
\sigma(\tau\overline{X})=1-\tau\rho(\overline{X})+\frac12\tau^2\eta^2(\overline{X})=
1-\frac12\tau(2-\tau)\rho(\overline{X})=\sigma((2-\tau)\overline{X})
(\#eq:sigmatau)
\end{equation} Thus the algorithm has convergent subsequences which may
not converge to a fixed point of \(\Psi\) (and thus of \(\Phi\)). And
indeed, the computational results show that the method produces a
sequence \(X^{(k)}\) with two subseqences. If \(\overline{X}\) is a
fixed point of \(\Phi\) then there is a \(\tau>0\) such that the
subsequence with \(k\) even converges to \(\tau\overline{X}\) while the
subsequence with \(k\) odd converges to \((2-\tau)\overline{X}\).

This suggests a simple fix. After convergence of the funcion values we
make a final update using \(\Phi\) instead of \(\Psi\). Computationally
this is simple to do. If the final iteration updates \(X^{(k)}\) to
\(X^{(k+1)}=\Psi(X^{(k)})\) then set the final solution to the average
\(\frac12(X^{(k)}+X^{(k+1)})\). Making this adjustment at the end of the
Ekman sequence gives us a final stress equal to 2.11141127390763.

\subsection{Asymptotic Rate of
Convergence}\label{asymptotic-rate-of-convergence-4}

The eigenvalues of the Jacobian are ::: \{.cell\}

:::

\section{Doubling}\label{doubling}

\subsection{Modification}\label{modification-4}

The analysis in the previous section suggest the update function
\(\Psi^2\), i.e. \[
\Xi(X)=\Psi(\Psi(X)).
\]

\subsection{Function Values}\label{function-values-5}

The algorithm generates the same sequence of function values and
configurations as the relaxed algoroithm \(\Psi\). The only difference
is that we test for convergence, and compute CHNG and EARC, every other
iteration.

With \(\Psi^2\) the algorithm is everywhere strictly monotonic and does
converge to a fixed point. But not all problems associated with \(\Psi\)
have disappeared. If \(X\) is a stationary point of \(\sigma\), and thus
a fixed point of \(\Phi\), \(\Psi\), and \(\Psi^2\), then \(\tau X\) is
a fixed point of \(\Psi^2\) for all \(\tau>0\). Thus we cannot exclude
the possibility that the sequence converges to a fixed point
proportional to \(X\), but not equal to \(X\).

Here are the results for the Ekman data if we use \(\Psi^2\). :::
\{.cell\}

::: In iteration 13, the final iteration, stress is 3.99462706656826.
The CHNG is 3.76583425614514e-16 and the EARC is 0.273802752120992.

Again we need some adjustment. A final update using \(\Phi\) will do the
trick. After this adjustment stress is 3.99462706656826

\subsection{Asymptotic Rate of
Convergence}\label{asymptotic-rate-of-convergence-5}

\[
\mathcal{D}\Psi^2_X(H)=\mathcal{D}\Psi_{\Psi(X)}(\mathcal{D}\Psi_X(H))
\]

The asymptotic convergence rate is \[
\max_s (2\lambda_s-1)^2=\max\{ (2\lambda_\text{max}-1)^2, (2\lambda_\text{min}-1)^2\}
\]

\section{Dilation}\label{dilation}

\subsection{Modification}\label{modification-5}

De Leeuw (\citeproc{ref-deleeuw_R_06b}{2006}) discusses some other ways
to fix the relaxed update problem. The first one, borrowed from Groenen,
Glunt, and Hayden (\citeproc{ref-groenen_glunt_hayden_96}{1996}),
defines Defines \[
\Pi(X):=\frac{\rho(X)}{\eta^2(X)}X
\] and \[
\Xi(X):=\Pi(\Psi(X))
\] \#\# Function Values

Here are the results for the Ekman data if we use dilation. :::
\{.cell\}

::: In iteration 26, the final iteration, stress is 2.1114112739076. The
CHNG is 1.77621337696992e-16 and the EARC is 0.533991473995601.

\subsection{Asymptotic Rate of
Convergence}\label{asymptotic-rate-of-convergence-6}

First, differentiate \(\Pi\) of \ldots{} Using the product and quotient
rules for differentiation we find \[
\mathcal{D}\Pi_X(H)=\frac{\rho(X)}{\eta^2(X)}H+\frac{\eta^2(X)\mathcal{D}\rho_X(H)-\rho(X)\mathcal{D}\eta^2_X(H)}{\eta^4(X)}X
\] Using \[
\mathcal{D}\rho_X(H)=\text{tr}\ H'B(X)X
\] \[
\mathcal{D}\eta^2_X(H)=2\text{tr}\ H'VX
\] this becomes \[
\mathcal{D}\Pi_X(H)=\frac{\rho(X)}{\eta^2(X)}H+\text{tr}\ H'\left\{\frac{\eta^2(X) B(X)X-2\rho(X)VX}{\eta^4(X)}\right\}X
\]

The chain rule says \[
\mathcal{D}\Xi_X(H)=\mathcal{D}\Psi_X(H)-\frac{\text{tr}\ X'V\mathcal{D}\Psi_X(H)}{\text{tr}\ X'VX}X
\] Since \(\mathcal{D}\Psi_X(X)=-X\) we have \[
\mathcal{D}\Xi_X(X)=-X+\frac{\text{tr}\ X'VX}{\text{tr}\ X'VX}X=0
\] Thus the offending eigenvector \(X\) of \(\mathcal{D}\Psi\) is
eliminated.

More generally, if \(\mathcal{D}\Psi_X(H)=\lambda H\) with \(H\not= X\)
then \(\mathcal{D}\Xi_X(H-X)=\lambda (H-X),\) and thus
\(\mathcal{D}\Xi\) has the same eigenvalues as \(\mathcal{D}\Psi\).

\section{Stabilizing}\label{stabilizing}

\subsection{Modification}\label{modification-6}

Another strategy

\begin{equation}
\Xi(X):=\Phi(\Psi(X))
\end{equation}

\[
\mathcal{D}\Xi_X(H)=\mathcal{D}\Phi_{\Psi(X)}(\mathcal{D}\Psi_X(H))=2\mathcal{D}\Phi_{\Psi(X)}(\mathcal{D}\Phi_X(H))-\mathcal{D}\Phi_{\Psi(X)}(H)
\]

\begin{equation}
\max_s|\lambda_s(2\lambda_s-1)|
\end{equation}

\subsection{Function Values}\label{function-values-6}

Here are the results for the Ekman data if we use stabilization. :::
\{.cell\}

::: In iteration 19, the final iteration, stress is 2.1114112739076. The
CHNG is 1.29239746041767e-16 and the EARC is 0.40956832382978.

\subsection{Asymptotic Rate of
Convergence}\label{asymptotic-rate-of-convergence-7}

\section{Averaging}\label{averaging}

\subsection{Modification}\label{modification-7}

\[
\Xi(X):=\frac12\{\Psi(\Psi(X))+\Psi(X))\}
\]

\subsection{Function Values}\label{function-values-7}

\subsection{ARC}\label{arc}

\[
\frac12\{(2\lambda - 1)\{1 + (2\lambda - 1)\}\}=\max_s\lambda_s(2\lambda_s-1)
\]

\section{Benchmarking}\label{benchmarking}

We compare the eight different upgrades using the microbenchmark package
(Mersmann (\citeproc{ref-mersmann_23}{2023})).

\begin{verbatim}
Warning in microbenchmark(smacofAccelerate(delta, xold = xold, ndim = 2, : less
accurate nanosecond times to avoid potential integer overflows
\end{verbatim}

\begin{verbatim}
Unit: milliseconds
                                                                                    expr
 smacofAccelerate(delta, xold = xold, ndim = 2, opt = 1, halt = 2,      verbose = FALSE)
 smacofAccelerate(delta, xold = xold, ndim = 2, opt = 2, halt = 2,      verbose = FALSE)
 smacofAccelerate(delta, xold = xold, ndim = 2, opt = 3, halt = 2,      verbose = FALSE)
 smacofAccelerate(delta, xold = xold, ndim = 2, opt = 4, halt = 2,      verbose = FALSE)
                   smacofAccelerate(delta, ndim = 2, opt = 5, halt = 2, verbose = FALSE)
 smacofAccelerate(delta, xold = xold, ndim = 2, opt = 6, halt = 2,      verbose = FALSE)
 smacofAccelerate(delta, ndim = 2, xold = xold, opt = 7, halt = 2,      verbose = FALSE)
 smacofAccelerate(delta, ndim = 2, xold = xold, opt = 8, halt = 2,      verbose = FALSE)
       min        lq      mean    median        uq       max neval
  3.044455  3.146053  3.448287  3.186110  3.314153  5.294576   100
  4.148954  4.305738  4.739418  4.377754  4.688248  6.815430   100
  3.829769  3.957504  4.165095  4.007606  4.072468  6.222365   100
 26.974228 28.709122 29.319847 29.039111 29.516822 50.105854   100
  1.614129  1.663493  1.838397  1.699819  1.748937  6.934945   100
  1.079079  1.112556  1.199962  1.124487  1.143921  5.839507   100
  1.515032  1.581841  1.704306  1.607754  1.641599  3.645228   100
  1.493630  1.544860  1.655274  1.563576  1.597032  3.725219   100
\end{verbatim}

De Gruijter (\citeproc{ref-degruijter_67}{1967})

\begin{verbatim}
Unit: milliseconds
                                                                                    expr
 smacofAccelerate(delta, xold = xold, ndim = 2, opt = 1, halt = 2,      verbose = FALSE)
 smacofAccelerate(delta, xold = xold, ndim = 2, opt = 2, halt = 2,      verbose = FALSE)
 smacofAccelerate(delta, xold = xold, ndim = 2, opt = 3, halt = 2,      verbose = FALSE)
 smacofAccelerate(delta, xold = xold, ndim = 2, opt = 4, halt = 2,      verbose = FALSE)
                   smacofAccelerate(delta, ndim = 2, opt = 5, halt = 2, verbose = FALSE)
 smacofAccelerate(delta, xold = xold, ndim = 2, opt = 6, halt = 2,      verbose = FALSE)
 smacofAccelerate(delta, ndim = 2, xold = xold, opt = 7, halt = 2,      verbose = FALSE)
 smacofAccelerate(delta, ndim = 2, xold = xold, opt = 8, halt = 2,      verbose = FALSE)
      min       lq     mean   median       uq      max neval
 44.56356 45.42333 47.14609 46.44152 46.87905 68.94703   100
 61.57872 62.66573 64.18375 63.27837 64.45190 83.48600   100
 57.70192 58.66200 60.11207 59.05974 60.60173 78.60196   100
 56.66696 57.81754 58.96594 58.21838 59.09945 77.35921   100
 20.29418 21.77850 21.97197 22.11845 22.37103 24.10640   100
 14.69190 14.94430 15.75168 15.58898 16.44651 18.72437   100
 23.09633 24.79844 25.09315 25.00746 25.33203 27.71325   100
 21.50270 23.05369 23.37570 23.34841 23.64210 26.69973   100
\end{verbatim}

\begin{verbatim}
Unit: milliseconds
                                                                                    expr
 smacofAccelerate(delta, xold = xold, ndim = 2, opt = 1, halt = 2,      verbose = FALSE)
 smacofAccelerate(delta, xold = xold, ndim = 2, opt = 2, halt = 2,      verbose = FALSE)
 smacofAccelerate(delta, xold = xold, ndim = 2, opt = 3, halt = 2,      verbose = FALSE)
 smacofAccelerate(delta, xold = xold, ndim = 2, opt = 4, halt = 2,      verbose = FALSE)
                   smacofAccelerate(delta, ndim = 2, opt = 5, halt = 2, verbose = FALSE)
 smacofAccelerate(delta, xold = xold, ndim = 2, opt = 6, halt = 2,      verbose = FALSE)
 smacofAccelerate(delta, ndim = 2, xold = xold, opt = 7, halt = 2,      verbose = FALSE)
 smacofAccelerate(delta, ndim = 2, xold = xold, opt = 8, halt = 2,      verbose = FALSE)
       min        lq      mean    median        uq       max neval
  42.83639  46.24685  47.50839  46.54912  47.28733  66.76551   100
  55.42913  56.56801  58.14801  57.06767  58.80995  80.01474   100
  52.75601  53.75082  55.24188  54.26299  55.70772  76.80842   100
 107.38769 108.96640 115.01316 110.18508 116.25702 152.92635   100
  18.94048  19.66083  21.17640  20.10187  22.41228  45.11419   100
  14.31864  14.77636  16.61831  14.99965  17.51516  42.79695   100
  24.47405  26.37624  29.26654  27.93303  28.79045  50.26055   100
  23.35577  24.20747  27.54380  26.71755  27.48287  49.37405   100
\end{verbatim}

\section*{(APPENDIX) Appendices}\label{appendix-appendices}
\addcontentsline{toc}{section}{(APPENDIX) Appendices}

\section{Code}\label{code}

\subsection{smacofAccelerate.R}\label{smacofaccelerate.r}

\begin{Shaded}
\begin{Highlighting}[]
\FunctionTok{library}\NormalTok{(MASS)}
\FunctionTok{library}\NormalTok{(microbenchmark)}
\FunctionTok{library}\NormalTok{(numDeriv)}

\NormalTok{smacofAccelerate }\OtherTok{\textless{}{-}} \ControlFlowTok{function}\NormalTok{(delta,}
                             \AttributeTok{wgth =} \DecValTok{1} \SpecialCharTok{{-}} \FunctionTok{diag}\NormalTok{(}\FunctionTok{nrow}\NormalTok{(delta)),}
                             \AttributeTok{ndim =} \DecValTok{2}\NormalTok{,}
                             \AttributeTok{xold =} \FunctionTok{smacofTorgerson}\NormalTok{(delta, ndim),}
                             \AttributeTok{opt =} \DecValTok{1}\NormalTok{,}
                             \AttributeTok{halt =} \DecValTok{0}\NormalTok{,}
                             \AttributeTok{wd =} \DecValTok{4}\NormalTok{,}
                             \AttributeTok{dg =} \DecValTok{15}\NormalTok{,}
                             \AttributeTok{itmax =} \DecValTok{1000}\NormalTok{,}
                             \AttributeTok{epsx =} \FloatTok{1e{-}10}\NormalTok{,}
                             \AttributeTok{epsf =} \FloatTok{1e{-}15}\NormalTok{,}
                             \AttributeTok{verbose =} \DecValTok{2}\NormalTok{) \{}
\NormalTok{  nobj }\OtherTok{\textless{}{-}} \FunctionTok{nrow}\NormalTok{(delta)}
\NormalTok{  vmat }\OtherTok{\textless{}{-}} \SpecialCharTok{{-}}\NormalTok{wgth}
  \FunctionTok{diag}\NormalTok{(vmat) }\OtherTok{\textless{}{-}} \SpecialCharTok{{-}}\FunctionTok{rowSums}\NormalTok{(vmat)}
\NormalTok{  vinv }\OtherTok{\textless{}{-}} \FunctionTok{solve}\NormalTok{(vmat }\SpecialCharTok{+}\NormalTok{ (}\DecValTok{1} \SpecialCharTok{/}\NormalTok{ nobj)) }\SpecialCharTok{{-}}\NormalTok{ (}\DecValTok{1} \SpecialCharTok{/}\NormalTok{ nobj)}
  \ControlFlowTok{if}\NormalTok{ (opt }\SpecialCharTok{==} \DecValTok{4}\NormalTok{) \{}
\NormalTok{    bs }\OtherTok{\textless{}{-}} \FunctionTok{smacofMakeBasis}\NormalTok{(nobj, ndim, vmat)}
\NormalTok{  \}}
\NormalTok{  xold }\OtherTok{\textless{}{-}} \FunctionTok{smacofCenter}\NormalTok{(xold)}
  \ControlFlowTok{if}\NormalTok{ ((opt }\SpecialCharTok{==} \DecValTok{2}\NormalTok{) }\SpecialCharTok{||}\NormalTok{ (opt }\SpecialCharTok{==} \DecValTok{4}\NormalTok{)) \{}
\NormalTok{    xrot }\OtherTok{\textless{}{-}} \FunctionTok{qr.Q}\NormalTok{(}\FunctionTok{qr}\NormalTok{(xold[}\DecValTok{1}\SpecialCharTok{:}\NormalTok{ndim, ]))}
\NormalTok{    xold }\OtherTok{\textless{}{-}}\NormalTok{ xold }\SpecialCharTok{\%*\%}\NormalTok{ xrot}
\NormalTok{  \}}
  \ControlFlowTok{if}\NormalTok{ (opt }\SpecialCharTok{==} \DecValTok{3}\NormalTok{) \{}
\NormalTok{    xrot }\OtherTok{\textless{}{-}} \FunctionTok{svd}\NormalTok{(xold)}\SpecialCharTok{$}\NormalTok{v}
\NormalTok{    xold }\OtherTok{\textless{}{-}}\NormalTok{ xold }\SpecialCharTok{\%*\%}\NormalTok{ xrot}
\NormalTok{  \}}
\NormalTok{  dold }\OtherTok{\textless{}{-}} \FunctionTok{as.matrix}\NormalTok{(}\FunctionTok{dist}\NormalTok{(xold))}
\NormalTok{  sold }\OtherTok{\textless{}{-}} \FunctionTok{sum}\NormalTok{(wgth }\SpecialCharTok{*}\NormalTok{ (delta }\SpecialCharTok{{-}}\NormalTok{ dold) }\SpecialCharTok{\^{}} \DecValTok{2}\NormalTok{)}
\NormalTok{  cold }\OtherTok{\textless{}{-}} \ConstantTok{Inf}
\NormalTok{  itel }\OtherTok{\textless{}{-}} \DecValTok{1}
  \ControlFlowTok{repeat}\NormalTok{ \{}
    \ControlFlowTok{if}\NormalTok{ (opt }\SpecialCharTok{==} \DecValTok{1}\NormalTok{) \{}
\NormalTok{      h }\OtherTok{\textless{}{-}} \FunctionTok{smacofOptionOne}\NormalTok{(xold, delta, wgth, vmat, vinv)}
\NormalTok{    \}}
    \ControlFlowTok{if}\NormalTok{ (opt }\SpecialCharTok{==} \DecValTok{2}\NormalTok{) \{}
\NormalTok{      h }\OtherTok{\textless{}{-}} \FunctionTok{smacofOptionTwo}\NormalTok{(xold, delta, wgth, vmat, vinv)}
\NormalTok{    \}}
    \ControlFlowTok{if}\NormalTok{ (opt }\SpecialCharTok{==} \DecValTok{3}\NormalTok{) \{}
\NormalTok{      h }\OtherTok{\textless{}{-}} \FunctionTok{smacofOptionThree}\NormalTok{(xold, delta, wgth, vmat, vinv)}
\NormalTok{    \}}
    \ControlFlowTok{if}\NormalTok{ (opt }\SpecialCharTok{==} \DecValTok{4}\NormalTok{) \{}
\NormalTok{      h }\OtherTok{\textless{}{-}} \FunctionTok{smacofOptionFour}\NormalTok{(xold, delta, wgth, vmat, vinv, bs)}
\NormalTok{    \}}
    \ControlFlowTok{if}\NormalTok{ (opt }\SpecialCharTok{==} \DecValTok{5}\NormalTok{) \{}
\NormalTok{      h }\OtherTok{\textless{}{-}} \FunctionTok{smacofOptionFive}\NormalTok{(xold, delta, wgth, vmat, vinv)}
\NormalTok{    \}}
    \ControlFlowTok{if}\NormalTok{ (opt }\SpecialCharTok{==} \DecValTok{6}\NormalTok{) \{}
\NormalTok{      h }\OtherTok{\textless{}{-}} \FunctionTok{smacofOptionSix}\NormalTok{(xold, delta, wgth, vmat, vinv)}
\NormalTok{    \}}
    \ControlFlowTok{if}\NormalTok{ (opt }\SpecialCharTok{==} \DecValTok{7}\NormalTok{) \{}
\NormalTok{      h }\OtherTok{\textless{}{-}} \FunctionTok{smacofOptionSeven}\NormalTok{(xold, delta, wgth, vmat, vinv)}
\NormalTok{    \}}
    \ControlFlowTok{if}\NormalTok{ (opt }\SpecialCharTok{==} \DecValTok{8}\NormalTok{) \{}
\NormalTok{      h }\OtherTok{\textless{}{-}} \FunctionTok{smacofOptionEight}\NormalTok{(xold, delta, wgth, vmat, vinv)}
\NormalTok{    \}}
\NormalTok{    labd }\OtherTok{\textless{}{-}} \FunctionTok{sqrt}\NormalTok{((h}\SpecialCharTok{$}\NormalTok{cnew) }\SpecialCharTok{/}\NormalTok{ cold)}
    \ControlFlowTok{if}\NormalTok{ (verbose }\SpecialCharTok{==} \DecValTok{2}\NormalTok{) \{}
      \FunctionTok{smacofLinePrint}\NormalTok{(itel, sold, h}\SpecialCharTok{$}\NormalTok{snew, h}\SpecialCharTok{$}\NormalTok{cnew, labd, }\AttributeTok{wd =}\NormalTok{ wd, }\AttributeTok{dg =}\NormalTok{ dg)}
\NormalTok{    \}}
    \ControlFlowTok{if}\NormalTok{ (halt }\SpecialCharTok{==} \DecValTok{1}\NormalTok{) \{}
\NormalTok{      converge }\OtherTok{\textless{}{-}}\NormalTok{ h}\SpecialCharTok{$}\NormalTok{cnew }\SpecialCharTok{\textless{}}\NormalTok{ epsx}
\NormalTok{    \} }\ControlFlowTok{else}\NormalTok{ \{}
\NormalTok{      converge }\OtherTok{\textless{}{-}}\NormalTok{ (sold }\SpecialCharTok{{-}}\NormalTok{ h}\SpecialCharTok{$}\NormalTok{snew) }\SpecialCharTok{\textless{}}\NormalTok{ epsf}
\NormalTok{    \}}
    \ControlFlowTok{if}\NormalTok{ ((itel }\SpecialCharTok{==}\NormalTok{ itmax) }\SpecialCharTok{||}\NormalTok{ converge) \{}
      \ControlFlowTok{break}
\NormalTok{    \}}
\NormalTok{    itel }\OtherTok{\textless{}{-}}\NormalTok{ itel }\SpecialCharTok{+} \DecValTok{1}
\NormalTok{    sold }\OtherTok{\textless{}{-}}\NormalTok{ h}\SpecialCharTok{$}\NormalTok{snew}
\NormalTok{    xold }\OtherTok{\textless{}{-}}\NormalTok{ h}\SpecialCharTok{$}\NormalTok{xnew}
\NormalTok{    cold }\OtherTok{\textless{}{-}}\NormalTok{ h}\SpecialCharTok{$}\NormalTok{cnew}
\NormalTok{    dold }\OtherTok{\textless{}{-}}\NormalTok{ h}\SpecialCharTok{$}\NormalTok{dnew}
\NormalTok{  \} }\CommentTok{\# end of repeat loop}
  \ControlFlowTok{if}\NormalTok{ (verbose }\SpecialCharTok{==} \DecValTok{1}\NormalTok{) \{}
    \FunctionTok{smacofLinePrint}\NormalTok{(itel, sold, h}\SpecialCharTok{$}\NormalTok{snew, h}\SpecialCharTok{$}\NormalTok{cnew, labd, }\AttributeTok{wd =}\NormalTok{ wd, }\AttributeTok{dg =}\NormalTok{ dg)}
\NormalTok{  \}}
\NormalTok{  adjust }\OtherTok{\textless{}{-}} \FunctionTok{list}\NormalTok{(}\AttributeTok{xnew =} \ConstantTok{NULL}\NormalTok{, }\AttributeTok{dnew =} \ConstantTok{NULL}\NormalTok{, }\AttributeTok{snew =} \ConstantTok{NULL}\NormalTok{)}
  \ControlFlowTok{if}\NormalTok{ (opt }\SpecialCharTok{==} \DecValTok{5}\NormalTok{) \{}
\NormalTok{    adjust}\SpecialCharTok{$}\NormalTok{xnew }\OtherTok{\textless{}{-}}\NormalTok{ (h}\SpecialCharTok{$}\NormalTok{xnew }\SpecialCharTok{+}\NormalTok{ xold) }\SpecialCharTok{/} \DecValTok{2}
\NormalTok{    adjust}\SpecialCharTok{$}\NormalTok{dnew }\OtherTok{\textless{}{-}} \FunctionTok{as.matrix}\NormalTok{(}\FunctionTok{dist}\NormalTok{(adjust}\SpecialCharTok{$}\NormalTok{xnew))}
\NormalTok{    adjust}\SpecialCharTok{$}\NormalTok{snew }\OtherTok{\textless{}{-}} \FunctionTok{sum}\NormalTok{(wgth }\SpecialCharTok{*}\NormalTok{ (delta }\SpecialCharTok{{-}}\NormalTok{ adjust}\SpecialCharTok{$}\NormalTok{dnew) }\SpecialCharTok{\^{}} \DecValTok{2}\NormalTok{)}
\NormalTok{  \}}
  \ControlFlowTok{if}\NormalTok{ (opt }\SpecialCharTok{==} \DecValTok{6}\NormalTok{) \{}
\NormalTok{    bold }\OtherTok{\textless{}{-}} \SpecialCharTok{{-}}\NormalTok{wgth }\SpecialCharTok{*}\NormalTok{ delta }\SpecialCharTok{/}\NormalTok{ (h}\SpecialCharTok{$}\NormalTok{dnew }\SpecialCharTok{+} \FunctionTok{diag}\NormalTok{(nobj))}
    \FunctionTok{diag}\NormalTok{(bold) }\OtherTok{\textless{}{-}} \SpecialCharTok{{-}}\FunctionTok{rowSums}\NormalTok{(bold)}
\NormalTok{    adjust}\SpecialCharTok{$}\NormalTok{xnew }\OtherTok{\textless{}{-}}\NormalTok{ vinv }\SpecialCharTok{\%*\%}\NormalTok{ bold }\SpecialCharTok{\%*\%}\NormalTok{ h}\SpecialCharTok{$}\NormalTok{xnew}
\NormalTok{    adjust}\SpecialCharTok{$}\NormalTok{dnew }\OtherTok{\textless{}{-}} \FunctionTok{as.matrix}\NormalTok{(}\FunctionTok{dist}\NormalTok{(adjust}\SpecialCharTok{$}\NormalTok{xnew))}
\NormalTok{    adjust}\SpecialCharTok{$}\NormalTok{snew }\OtherTok{\textless{}{-}} \FunctionTok{sum}\NormalTok{(wgth }\SpecialCharTok{*}\NormalTok{ (delta }\SpecialCharTok{{-}}\NormalTok{ adjust}\SpecialCharTok{$}\NormalTok{dnew) }\SpecialCharTok{\^{}} \DecValTok{2}\NormalTok{)}
\NormalTok{  \}}
  \FunctionTok{return}\NormalTok{(}
    \FunctionTok{list}\NormalTok{(}
      \AttributeTok{x =}\NormalTok{ h}\SpecialCharTok{$}\NormalTok{xnew,}
      \AttributeTok{s =}\NormalTok{ h}\SpecialCharTok{$}\NormalTok{snew,}
      \AttributeTok{d =}\NormalTok{ h}\SpecialCharTok{$}\NormalTok{dnew,}
      \AttributeTok{itel =}\NormalTok{ itel,}
      \AttributeTok{chng =}\NormalTok{ h}\SpecialCharTok{$}\NormalTok{cnew,}
      \AttributeTok{labd =}\NormalTok{ labd,}
      \AttributeTok{wgth =}\NormalTok{ wgth,}
      \AttributeTok{delta =}\NormalTok{ delta,}
      \AttributeTok{adjust =}\NormalTok{ adjust}
\NormalTok{    )}
\NormalTok{  )}
\NormalTok{\}}

\NormalTok{smacofOptionOne }\OtherTok{\textless{}{-}} \ControlFlowTok{function}\NormalTok{(xold, delta, wgth, vmat, vinv) \{}
\NormalTok{  xnew }\OtherTok{\textless{}{-}} \FunctionTok{smacofCenter}\NormalTok{(}\FunctionTok{smacofGuttman}\NormalTok{(xold, delta, wgth, vinv))}
\NormalTok{  dnew }\OtherTok{\textless{}{-}} \FunctionTok{as.matrix}\NormalTok{(}\FunctionTok{dist}\NormalTok{(xnew))}
\NormalTok{  snew }\OtherTok{\textless{}{-}} \FunctionTok{sum}\NormalTok{(wgth }\SpecialCharTok{*}\NormalTok{ (delta }\SpecialCharTok{{-}}\NormalTok{ dnew) }\SpecialCharTok{\^{}} \DecValTok{2}\NormalTok{)}
\NormalTok{  cnew }\OtherTok{\textless{}{-}} \FunctionTok{sum}\NormalTok{((xold }\SpecialCharTok{{-}}\NormalTok{ xnew) }\SpecialCharTok{*}\NormalTok{ (vmat }\SpecialCharTok{\%*\%}\NormalTok{ (xold }\SpecialCharTok{{-}}\NormalTok{ xnew)))}
  \FunctionTok{return}\NormalTok{(}\FunctionTok{list}\NormalTok{(}
    \AttributeTok{xnew =}\NormalTok{ xnew,}
    \AttributeTok{dnew =}\NormalTok{ dnew,}
    \AttributeTok{snew =}\NormalTok{ snew,}
    \AttributeTok{cnew =}\NormalTok{ cnew}
\NormalTok{  ))}
\NormalTok{\}}

\NormalTok{smacofOptionTwo }\OtherTok{\textless{}{-}} \ControlFlowTok{function}\NormalTok{(xold, delta, wgth, vmat, vinv) \{}
\NormalTok{  ndim }\OtherTok{\textless{}{-}} \FunctionTok{ncol}\NormalTok{(xold)}
\NormalTok{  xbar }\OtherTok{\textless{}{-}} \FunctionTok{smacofCenter}\NormalTok{(}\FunctionTok{smacofGuttman}\NormalTok{(xold, delta, wgth, vinv))}
\NormalTok{  xrot }\OtherTok{\textless{}{-}} \FunctionTok{smacofSignEigenVectors}\NormalTok{(}\FunctionTok{qr.Q}\NormalTok{(}\FunctionTok{qr}\NormalTok{(}\FunctionTok{t}\NormalTok{(xbar[}\DecValTok{1}\SpecialCharTok{:}\NormalTok{ndim, ]))))}
  \CommentTok{\#xrot \textless{}{-} qr.Q(qr(t(xbar[1:ndim, ])))}
\NormalTok{  xnew }\OtherTok{\textless{}{-}}\NormalTok{ xbar }\SpecialCharTok{\%*\%}\NormalTok{ xrot}
\NormalTok{  dnew }\OtherTok{\textless{}{-}} \FunctionTok{as.matrix}\NormalTok{(}\FunctionTok{dist}\NormalTok{(xnew))}
\NormalTok{  snew }\OtherTok{\textless{}{-}} \FunctionTok{sum}\NormalTok{(wgth }\SpecialCharTok{*}\NormalTok{ (delta }\SpecialCharTok{{-}}\NormalTok{ dnew) }\SpecialCharTok{\^{}} \DecValTok{2}\NormalTok{)}
\NormalTok{  cnew }\OtherTok{\textless{}{-}} \FunctionTok{sum}\NormalTok{((xold }\SpecialCharTok{{-}}\NormalTok{ xnew) }\SpecialCharTok{*}\NormalTok{ (vmat }\SpecialCharTok{\%*\%}\NormalTok{ (xold }\SpecialCharTok{{-}}\NormalTok{ xnew)))}
  \FunctionTok{return}\NormalTok{(}\FunctionTok{list}\NormalTok{(}
    \AttributeTok{xnew =}\NormalTok{ xnew,}
    \AttributeTok{dnew =}\NormalTok{ dnew,}
    \AttributeTok{snew =}\NormalTok{ snew,}
    \AttributeTok{cnew =}\NormalTok{ cnew}
\NormalTok{  ))}
\NormalTok{\}}

\NormalTok{smacofOptionThree }\OtherTok{\textless{}{-}} \ControlFlowTok{function}\NormalTok{(xold, delta, wgth, vmat, vinv) \{}
\NormalTok{  xbar }\OtherTok{\textless{}{-}} \FunctionTok{smacofCenter}\NormalTok{(}\FunctionTok{smacofGuttman}\NormalTok{(xold, delta, wgth, vinv))}
\NormalTok{  xrot }\OtherTok{\textless{}{-}} \FunctionTok{smacofSignEigenVectors}\NormalTok{(}\FunctionTok{svd}\NormalTok{(xbar)}\SpecialCharTok{$}\NormalTok{v)}
\NormalTok{  xnew }\OtherTok{\textless{}{-}}\NormalTok{ xbar }\SpecialCharTok{\%*\%}\NormalTok{ xrot}
\NormalTok{  dnew }\OtherTok{\textless{}{-}} \FunctionTok{as.matrix}\NormalTok{(}\FunctionTok{dist}\NormalTok{(xnew))}
\NormalTok{  snew }\OtherTok{\textless{}{-}} \FunctionTok{sum}\NormalTok{(wgth }\SpecialCharTok{*}\NormalTok{ (delta }\SpecialCharTok{{-}}\NormalTok{ dnew) }\SpecialCharTok{\^{}} \DecValTok{2}\NormalTok{)}
\NormalTok{  cnew }\OtherTok{\textless{}{-}} \FunctionTok{sum}\NormalTok{((xold }\SpecialCharTok{{-}}\NormalTok{ xnew) }\SpecialCharTok{*}\NormalTok{ (vmat }\SpecialCharTok{\%*\%}\NormalTok{ (xold }\SpecialCharTok{{-}}\NormalTok{ xnew)))}
  \FunctionTok{return}\NormalTok{(}\FunctionTok{list}\NormalTok{(}
    \AttributeTok{xnew =}\NormalTok{ xnew,}
    \AttributeTok{dnew =}\NormalTok{ dnew,}
    \AttributeTok{snew =}\NormalTok{ snew,}
    \AttributeTok{cnew =}\NormalTok{ cnew}
\NormalTok{  ))}
\NormalTok{\}}

\NormalTok{smacofOptionFour }\OtherTok{\textless{}{-}} \ControlFlowTok{function}\NormalTok{(xold, delta, wgth, vmat, vinv, bs) \{}
\NormalTok{  ndim }\OtherTok{\textless{}{-}} \FunctionTok{ncol}\NormalTok{(xold)}
\NormalTok{  nobj }\OtherTok{\textless{}{-}} \FunctionTok{nrow}\NormalTok{(xold)}
\NormalTok{  xnew }\OtherTok{\textless{}{-}} \FunctionTok{matrix}\NormalTok{(}\DecValTok{0}\NormalTok{, nobj, ndim)}
\NormalTok{  xbar }\OtherTok{\textless{}{-}} \FunctionTok{smacofCenter}\NormalTok{(}\FunctionTok{smacofGuttman}\NormalTok{(xold, delta, wgth, vinv))}
  \ControlFlowTok{for}\NormalTok{ (s }\ControlFlowTok{in} \DecValTok{1}\SpecialCharTok{:}\NormalTok{ndim) \{}
\NormalTok{    aux }\OtherTok{\textless{}{-}} \FunctionTok{crossprod}\NormalTok{(bs[[s]], vmat }\SpecialCharTok{\%*\%}\NormalTok{ xbar[, s])}
\NormalTok{    xnew[, s] }\OtherTok{\textless{}{-}}\NormalTok{ bs[[s]] }\SpecialCharTok{\%*\%}\NormalTok{ aux}
\NormalTok{  \}}
\NormalTok{  dnew }\OtherTok{\textless{}{-}} \FunctionTok{as.matrix}\NormalTok{(}\FunctionTok{dist}\NormalTok{(xnew))}
\NormalTok{  snew }\OtherTok{\textless{}{-}} \FunctionTok{sum}\NormalTok{(wgth }\SpecialCharTok{*}\NormalTok{ (delta }\SpecialCharTok{{-}}\NormalTok{ dnew) }\SpecialCharTok{\^{}} \DecValTok{2}\NormalTok{)}
\NormalTok{  cnew }\OtherTok{\textless{}{-}} \FunctionTok{sum}\NormalTok{((xold }\SpecialCharTok{{-}}\NormalTok{ xnew) }\SpecialCharTok{*}\NormalTok{ (vmat }\SpecialCharTok{\%*\%}\NormalTok{ (xold }\SpecialCharTok{{-}}\NormalTok{ xnew)))}
  \FunctionTok{return}\NormalTok{(}\FunctionTok{list}\NormalTok{(}
    \AttributeTok{xnew =}\NormalTok{ xnew,}
    \AttributeTok{dnew =}\NormalTok{ dnew,}
    \AttributeTok{snew =}\NormalTok{ snew,}
    \AttributeTok{cnew =}\NormalTok{ cnew}
\NormalTok{  ))}
\NormalTok{\}}


\NormalTok{smacofOptionFive }\OtherTok{\textless{}{-}} \ControlFlowTok{function}\NormalTok{(xold, delta, wgth, vmat, vinv) \{}
\NormalTok{  xbar }\OtherTok{\textless{}{-}} \FunctionTok{smacofCenter}\NormalTok{(}\FunctionTok{smacofGuttman}\NormalTok{(xold, delta, wgth, vinv))}
\NormalTok{  xnew }\OtherTok{\textless{}{-}} \DecValTok{2} \SpecialCharTok{*}\NormalTok{ xbar }\SpecialCharTok{{-}}\NormalTok{ xold}
\NormalTok{  dnew }\OtherTok{\textless{}{-}} \FunctionTok{as.matrix}\NormalTok{(}\FunctionTok{dist}\NormalTok{(xnew))}
\NormalTok{  snew }\OtherTok{\textless{}{-}} \FunctionTok{sum}\NormalTok{(wgth }\SpecialCharTok{*}\NormalTok{ (delta }\SpecialCharTok{{-}}\NormalTok{ dnew) }\SpecialCharTok{\^{}} \DecValTok{2}\NormalTok{)}
\NormalTok{  cnew }\OtherTok{\textless{}{-}} \FunctionTok{sum}\NormalTok{((xold }\SpecialCharTok{{-}}\NormalTok{ xnew) }\SpecialCharTok{*}\NormalTok{ (vmat }\SpecialCharTok{\%*\%}\NormalTok{ (xold }\SpecialCharTok{{-}}\NormalTok{ xnew)))}
  \FunctionTok{return}\NormalTok{(}\FunctionTok{list}\NormalTok{(}
    \AttributeTok{xnew =}\NormalTok{ xnew,}
    \AttributeTok{dnew =}\NormalTok{ dnew,}
    \AttributeTok{snew =}\NormalTok{ snew,}
    \AttributeTok{cnew =}\NormalTok{ cnew}
\NormalTok{  ))}
\NormalTok{\}}

\NormalTok{smacofOptionSix }\OtherTok{\textless{}{-}} \ControlFlowTok{function}\NormalTok{(xold, delta, wgth, vmat, vinv) \{}
\NormalTok{  nobj }\OtherTok{\textless{}{-}} \FunctionTok{nrow}\NormalTok{(xold)}
\NormalTok{  xbar }\OtherTok{\textless{}{-}} \FunctionTok{smacofCenter}\NormalTok{(}\FunctionTok{smacofGuttman}\NormalTok{(xold, delta, wgth, vinv))}
\NormalTok{  xaux }\OtherTok{\textless{}{-}} \DecValTok{2} \SpecialCharTok{*}\NormalTok{ xbar }\SpecialCharTok{{-}}\NormalTok{ xold}
\NormalTok{  daux }\OtherTok{\textless{}{-}} \FunctionTok{as.matrix}\NormalTok{(}\FunctionTok{dist}\NormalTok{(xaux))}
\NormalTok{  baux }\OtherTok{\textless{}{-}} \SpecialCharTok{{-}}\NormalTok{wgth }\SpecialCharTok{*}\NormalTok{ delta }\SpecialCharTok{/}\NormalTok{ (daux }\SpecialCharTok{+} \FunctionTok{diag}\NormalTok{(nobj))}
  \FunctionTok{diag}\NormalTok{(baux) }\OtherTok{\textless{}{-}} \SpecialCharTok{{-}}\FunctionTok{rowSums}\NormalTok{(baux)}
\NormalTok{  xbaz }\OtherTok{\textless{}{-}}\NormalTok{ vinv }\SpecialCharTok{\%*\%}\NormalTok{ baux }\SpecialCharTok{\%*\%}\NormalTok{ xaux}
\NormalTok{  xnew }\OtherTok{\textless{}{-}} \DecValTok{2} \SpecialCharTok{*}\NormalTok{ xbaz }\SpecialCharTok{{-}}\NormalTok{ xaux}
\NormalTok{  dnew }\OtherTok{\textless{}{-}} \FunctionTok{as.matrix}\NormalTok{(}\FunctionTok{dist}\NormalTok{(xnew))}
\NormalTok{  snew }\OtherTok{\textless{}{-}} \FunctionTok{sum}\NormalTok{(wgth }\SpecialCharTok{*}\NormalTok{ (delta }\SpecialCharTok{{-}}\NormalTok{ dnew) }\SpecialCharTok{\^{}} \DecValTok{2}\NormalTok{)}
\NormalTok{  cnew }\OtherTok{\textless{}{-}} \FunctionTok{sum}\NormalTok{((xold }\SpecialCharTok{{-}}\NormalTok{ xnew) }\SpecialCharTok{*}\NormalTok{ (vmat }\SpecialCharTok{\%*\%}\NormalTok{ (xold }\SpecialCharTok{{-}}\NormalTok{ xnew)))}
  \FunctionTok{return}\NormalTok{(}\FunctionTok{list}\NormalTok{(}
    \AttributeTok{xnew =}\NormalTok{ xnew,}
    \AttributeTok{dnew =}\NormalTok{ dnew,}
    \AttributeTok{snew =}\NormalTok{ snew,}
    \AttributeTok{cnew =}\NormalTok{ cnew}
\NormalTok{  ))}
\NormalTok{\}}

\NormalTok{smacofOptionSeven }\OtherTok{\textless{}{-}} \ControlFlowTok{function}\NormalTok{(xold, delta, wgth, vmat, vinv) \{}
\NormalTok{  xbar }\OtherTok{\textless{}{-}} \FunctionTok{smacofCenter}\NormalTok{(}\FunctionTok{smacofGuttman}\NormalTok{(xold, delta, wgth, vinv))}
\NormalTok{  xaux }\OtherTok{\textless{}{-}} \DecValTok{2} \SpecialCharTok{*}\NormalTok{ xbar }\SpecialCharTok{{-}}\NormalTok{ xold}
\NormalTok{  daux }\OtherTok{\textless{}{-}} \FunctionTok{as.matrix}\NormalTok{(}\FunctionTok{dist}\NormalTok{(xaux))}
\NormalTok{  alpa }\OtherTok{\textless{}{-}} \FunctionTok{sum}\NormalTok{(wgth }\SpecialCharTok{*}\NormalTok{ daux }\SpecialCharTok{*}\NormalTok{ delta) }\SpecialCharTok{/} \FunctionTok{sum}\NormalTok{(wgth }\SpecialCharTok{*}\NormalTok{ daux }\SpecialCharTok{\^{}} \DecValTok{2}\NormalTok{)}
\NormalTok{  xnew }\OtherTok{\textless{}{-}}\NormalTok{ alpa }\SpecialCharTok{*}\NormalTok{ xaux}
\NormalTok{  dnew }\OtherTok{\textless{}{-}}\NormalTok{ alpa }\SpecialCharTok{*}\NormalTok{ daux}
\NormalTok{  snew }\OtherTok{\textless{}{-}} \FunctionTok{sum}\NormalTok{(wgth }\SpecialCharTok{*}\NormalTok{ (delta }\SpecialCharTok{{-}}\NormalTok{ dnew) }\SpecialCharTok{\^{}} \DecValTok{2}\NormalTok{)}
\NormalTok{  cnew }\OtherTok{\textless{}{-}} \FunctionTok{sum}\NormalTok{((xold }\SpecialCharTok{{-}}\NormalTok{ xnew) }\SpecialCharTok{*}\NormalTok{ (vmat }\SpecialCharTok{\%*\%}\NormalTok{ (xold }\SpecialCharTok{{-}}\NormalTok{ xnew)))}
  \FunctionTok{return}\NormalTok{(}\FunctionTok{list}\NormalTok{(}
    \AttributeTok{xnew =}\NormalTok{ xnew,}
    \AttributeTok{dnew =}\NormalTok{ dnew,}
    \AttributeTok{snew =}\NormalTok{ snew,}
    \AttributeTok{cnew =}\NormalTok{ cnew}
\NormalTok{  ))}
\NormalTok{\}}

\NormalTok{smacofOptionEight }\OtherTok{\textless{}{-}} \ControlFlowTok{function}\NormalTok{(xold, delta, wgth, vmat, vinv) \{}
\NormalTok{  nobj }\OtherTok{\textless{}{-}} \FunctionTok{nrow}\NormalTok{(xold)}
\NormalTok{  xbar }\OtherTok{\textless{}{-}} \FunctionTok{smacofCenter}\NormalTok{(}\FunctionTok{smacofGuttman}\NormalTok{(xold, delta, wgth, vinv))}
\NormalTok{  xaux }\OtherTok{\textless{}{-}} \DecValTok{2} \SpecialCharTok{*}\NormalTok{ xbar }\SpecialCharTok{{-}}\NormalTok{ xold}
\NormalTok{  daux }\OtherTok{\textless{}{-}} \FunctionTok{as.matrix}\NormalTok{(}\FunctionTok{dist}\NormalTok{(xaux))}
\NormalTok{  baux }\OtherTok{\textless{}{-}} \SpecialCharTok{{-}}\NormalTok{wgth }\SpecialCharTok{*}\NormalTok{ delta }\SpecialCharTok{/}\NormalTok{ (daux }\SpecialCharTok{+} \FunctionTok{diag}\NormalTok{(nobj))}
  \FunctionTok{diag}\NormalTok{(baux) }\OtherTok{\textless{}{-}} \SpecialCharTok{{-}}\FunctionTok{rowSums}\NormalTok{(baux)}
\NormalTok{  xnew }\OtherTok{\textless{}{-}}\NormalTok{ vinv }\SpecialCharTok{\%*\%}\NormalTok{ baux }\SpecialCharTok{\%*\%}\NormalTok{ xaux}
\NormalTok{  dnew }\OtherTok{\textless{}{-}} \FunctionTok{as.matrix}\NormalTok{(}\FunctionTok{dist}\NormalTok{(xnew))}
\NormalTok{  snew }\OtherTok{\textless{}{-}} \FunctionTok{sum}\NormalTok{(wgth }\SpecialCharTok{*}\NormalTok{ (delta }\SpecialCharTok{{-}}\NormalTok{ dnew) }\SpecialCharTok{\^{}} \DecValTok{2}\NormalTok{)}
\NormalTok{  cnew }\OtherTok{\textless{}{-}} \FunctionTok{sum}\NormalTok{((xold }\SpecialCharTok{{-}}\NormalTok{ xnew) }\SpecialCharTok{*}\NormalTok{ (vmat }\SpecialCharTok{\%*\%}\NormalTok{ (xold }\SpecialCharTok{{-}}\NormalTok{ xnew)))}
  \FunctionTok{return}\NormalTok{(}\FunctionTok{list}\NormalTok{(}
    \AttributeTok{xnew =}\NormalTok{ xnew,}
    \AttributeTok{dnew =}\NormalTok{ dnew,}
    \AttributeTok{snew =}\NormalTok{ snew,}
    \AttributeTok{cnew =}\NormalTok{ cnew}
\NormalTok{  ))}
\NormalTok{\}}
\end{Highlighting}
\end{Shaded}

\subsection{smacofDerivatives.R}\label{smacofderivatives.r}

\begin{Shaded}
\begin{Highlighting}[]
\FunctionTok{library}\NormalTok{(numDeriv)}

\NormalTok{smacofRhoHessian }\OtherTok{\textless{}{-}} \ControlFlowTok{function}\NormalTok{(x, delta, wgth) \{}
\NormalTok{  nobj }\OtherTok{\textless{}{-}} \FunctionTok{nrow}\NormalTok{(x)}
\NormalTok{  ndim }\OtherTok{\textless{}{-}} \FunctionTok{ncol}\NormalTok{(x)}
\NormalTok{  ntot }\OtherTok{\textless{}{-}}\NormalTok{ nobj }\SpecialCharTok{*}\NormalTok{ ndim}
\NormalTok{  dmat }\OtherTok{\textless{}{-}} \FunctionTok{as.matrix}\NormalTok{(}\FunctionTok{dist}\NormalTok{(x))}
\NormalTok{  fac1 }\OtherTok{\textless{}{-}}\NormalTok{ wgth }\SpecialCharTok{*}\NormalTok{ delta }\SpecialCharTok{/}\NormalTok{ (dmat }\SpecialCharTok{+} \FunctionTok{diag}\NormalTok{(nobj))}
\NormalTok{  fac2 }\OtherTok{\textless{}{-}}\NormalTok{ wgth }\SpecialCharTok{*}\NormalTok{ delta }\SpecialCharTok{/}\NormalTok{ ((dmat }\SpecialCharTok{+} \FunctionTok{diag}\NormalTok{(nobj)) }\SpecialCharTok{\^{}} \DecValTok{3}\NormalTok{)}
\NormalTok{  bmat }\OtherTok{\textless{}{-}} \SpecialCharTok{{-}}\NormalTok{fac1}
  \FunctionTok{diag}\NormalTok{(bmat) }\OtherTok{\textless{}{-}} \SpecialCharTok{{-}}\FunctionTok{rowSums}\NormalTok{(bmat)}
\NormalTok{  hess }\OtherTok{\textless{}{-}} \FunctionTok{matrix}\NormalTok{(}\DecValTok{0}\NormalTok{, ntot, ntot)}
  \ControlFlowTok{for}\NormalTok{ (s }\ControlFlowTok{in} \DecValTok{1}\SpecialCharTok{:}\NormalTok{ndim) \{}
\NormalTok{    ns }\OtherTok{\textless{}{-}}\NormalTok{ (s }\SpecialCharTok{{-}} \DecValTok{1}\NormalTok{) }\SpecialCharTok{*}\NormalTok{ nobj }\SpecialCharTok{+} \DecValTok{1}\SpecialCharTok{:}\NormalTok{nobj}
\NormalTok{    hess[ns, ns] }\OtherTok{\textless{}{-}}\NormalTok{ bmat}
    \ControlFlowTok{for}\NormalTok{ (t }\ControlFlowTok{in} \DecValTok{1}\SpecialCharTok{:}\NormalTok{ndim) \{}
\NormalTok{      nt }\OtherTok{\textless{}{-}}\NormalTok{ (t }\SpecialCharTok{{-}} \DecValTok{1}\NormalTok{) }\SpecialCharTok{*}\NormalTok{ nobj }\SpecialCharTok{+} \DecValTok{1}\SpecialCharTok{:}\NormalTok{nobj}
\NormalTok{      ds }\OtherTok{\textless{}{-}} \FunctionTok{outer}\NormalTok{(x[, s], x[, s], }\StringTok{"{-}"}\NormalTok{)}
\NormalTok{      dt }\OtherTok{\textless{}{-}} \FunctionTok{outer}\NormalTok{(x[, t], x[, t], }\StringTok{"{-}"}\NormalTok{)}
\NormalTok{      aux }\OtherTok{\textless{}{-}} \SpecialCharTok{{-}}\NormalTok{fac2 }\SpecialCharTok{*}\NormalTok{ ds }\SpecialCharTok{*}\NormalTok{ dt}
      \FunctionTok{diag}\NormalTok{(aux) }\OtherTok{\textless{}{-}} \SpecialCharTok{{-}}\FunctionTok{rowSums}\NormalTok{(aux)}
\NormalTok{      hess[ns, nt] }\OtherTok{\textless{}{-}}\NormalTok{ hess[ns, nt] }\SpecialCharTok{{-}}\NormalTok{ aux}
\NormalTok{    \}}
\NormalTok{  \}}
  \FunctionTok{return}\NormalTok{(hess)}
\NormalTok{\}}

\NormalTok{smacofBasicDerivative }\OtherTok{\textless{}{-}} \ControlFlowTok{function}\NormalTok{(x, h, delta, wgth, vinv, dmat) \{}
\NormalTok{  nobj }\OtherTok{\textless{}{-}} \FunctionTok{nrow}\NormalTok{(x)}
\NormalTok{  ndim }\OtherTok{\textless{}{-}} \FunctionTok{ncol}\NormalTok{(x)}
\NormalTok{  bmat }\OtherTok{\textless{}{-}}\NormalTok{ wgth }\SpecialCharTok{*}\NormalTok{ delta }\SpecialCharTok{/}\NormalTok{ (dmat }\SpecialCharTok{+} \FunctionTok{diag}\NormalTok{(nobj))}
\NormalTok{  bmat }\OtherTok{\textless{}{-}} \SpecialCharTok{{-}}\NormalTok{bmat}
  \FunctionTok{diag}\NormalTok{(bmat) }\OtherTok{\textless{}{-}} \SpecialCharTok{{-}}\FunctionTok{rowSums}\NormalTok{(bmat)}
\NormalTok{  hmat }\OtherTok{\textless{}{-}}\NormalTok{ wgth }\SpecialCharTok{*}\NormalTok{ delta }\SpecialCharTok{/}\NormalTok{ ((dmat }\SpecialCharTok{+} \FunctionTok{diag}\NormalTok{(nobj)) }\SpecialCharTok{\^{}} \DecValTok{3}\NormalTok{)}
  \ControlFlowTok{for}\NormalTok{ (i }\ControlFlowTok{in} \DecValTok{1}\SpecialCharTok{:}\NormalTok{nobj) \{}
    \ControlFlowTok{for}\NormalTok{ (j }\ControlFlowTok{in} \DecValTok{1}\SpecialCharTok{:}\NormalTok{nobj) \{}
\NormalTok{      xhij }\OtherTok{\textless{}{-}} \FunctionTok{sum}\NormalTok{((x[i, ] }\SpecialCharTok{{-}}\NormalTok{ x[j, ]) }\SpecialCharTok{*}\NormalTok{ (h[i, ] }\SpecialCharTok{{-}}\NormalTok{ h[j, ]))}
\NormalTok{      hmat[i, j] }\OtherTok{\textless{}{-}}\NormalTok{ hmat[i, j] }\SpecialCharTok{*}\NormalTok{ xhij}
\NormalTok{    \}}
\NormalTok{  \}}
\NormalTok{  hmat }\OtherTok{\textless{}{-}} \SpecialCharTok{{-}}\NormalTok{hmat}
  \FunctionTok{diag}\NormalTok{(hmat) }\OtherTok{\textless{}{-}} \SpecialCharTok{{-}}\FunctionTok{rowSums}\NormalTok{(hmat)}
\NormalTok{  deri }\OtherTok{\textless{}{-}}\NormalTok{ vinv }\SpecialCharTok{\%*\%}\NormalTok{ (bmat }\SpecialCharTok{\%*\%}\NormalTok{ h }\SpecialCharTok{{-}}\NormalTok{ hmat }\SpecialCharTok{\%*\%}\NormalTok{ x)}
  \FunctionTok{return}\NormalTok{(deri)}
\NormalTok{\}}

\NormalTok{smacofBasicJacobianFormula }\OtherTok{\textless{}{-}} \ControlFlowTok{function}\NormalTok{(x, delta, wgth) \{}
\NormalTok{  nobj }\OtherTok{\textless{}{-}} \FunctionTok{nrow}\NormalTok{(x)}
\NormalTok{  ndim }\OtherTok{\textless{}{-}} \FunctionTok{ncol}\NormalTok{(x)}
\NormalTok{  ntot }\OtherTok{\textless{}{-}}\NormalTok{ nobj }\SpecialCharTok{*}\NormalTok{ ndim}
\NormalTok{  dmat }\OtherTok{\textless{}{-}} \FunctionTok{as.matrix}\NormalTok{(}\FunctionTok{dist}\NormalTok{(x))}
\NormalTok{  vmat }\OtherTok{\textless{}{-}} \SpecialCharTok{{-}}\NormalTok{wgth}
  \FunctionTok{diag}\NormalTok{(vmat) }\OtherTok{\textless{}{-}} \SpecialCharTok{{-}}\FunctionTok{rowSums}\NormalTok{(vmat)}
\NormalTok{  vinv }\OtherTok{\textless{}{-}} \FunctionTok{solve}\NormalTok{(vmat }\SpecialCharTok{+}\NormalTok{ (}\DecValTok{1} \SpecialCharTok{/}\NormalTok{ nobj)) }\SpecialCharTok{{-}}\NormalTok{ (}\DecValTok{1} \SpecialCharTok{/}\NormalTok{ nobj)}
\NormalTok{  jacob }\OtherTok{\textless{}{-}} \FunctionTok{matrix}\NormalTok{(}\DecValTok{0}\NormalTok{, ntot, ntot)}
\NormalTok{  e }\OtherTok{\textless{}{-}} \ControlFlowTok{function}\NormalTok{(i, n) \{}
    \FunctionTok{ifelse}\NormalTok{(i }\SpecialCharTok{==} \DecValTok{1}\SpecialCharTok{:}\NormalTok{n, }\DecValTok{1}\NormalTok{, }\DecValTok{0}\NormalTok{)}
\NormalTok{  \}}
\NormalTok{  k }\OtherTok{\textless{}{-}} \DecValTok{1}
  \ControlFlowTok{for}\NormalTok{ (j }\ControlFlowTok{in} \DecValTok{1}\SpecialCharTok{:}\NormalTok{ndim) \{}
    \ControlFlowTok{for}\NormalTok{ (i }\ControlFlowTok{in} \DecValTok{1}\SpecialCharTok{:}\NormalTok{nobj) \{}
\NormalTok{      h }\OtherTok{\textless{}{-}} \FunctionTok{outer}\NormalTok{(}\FunctionTok{e}\NormalTok{(i, nobj), }\FunctionTok{e}\NormalTok{(j, ndim))}
\NormalTok{      r }\OtherTok{\textless{}{-}} \FunctionTok{smacofBasicDerivative}\NormalTok{(x, h, delta, wgth, vinv, dmat)}
\NormalTok{      jacob[, k] }\OtherTok{\textless{}{-}} \FunctionTok{as.vector}\NormalTok{(r)}
\NormalTok{      k }\OtherTok{\textless{}{-}}\NormalTok{ k }\SpecialCharTok{+} \DecValTok{1}
\NormalTok{    \}}
\NormalTok{  \}}
  \FunctionTok{return}\NormalTok{(jacob)}
\NormalTok{\}}

\NormalTok{smacofBasicJacobianNumerical }\OtherTok{\textless{}{-}} \ControlFlowTok{function}\NormalTok{(x, delta, wgth) \{}
\NormalTok{  nobj }\OtherTok{\textless{}{-}} \FunctionTok{nrow}\NormalTok{(x)}
\NormalTok{  ndim }\OtherTok{\textless{}{-}} \FunctionTok{ncol}\NormalTok{(x)}
\NormalTok{  ntot }\OtherTok{\textless{}{-}}\NormalTok{ nobj }\SpecialCharTok{*}\NormalTok{ ndim}
\NormalTok{  dmat }\OtherTok{\textless{}{-}} \FunctionTok{as.matrix}\NormalTok{(}\FunctionTok{dist}\NormalTok{(x))}
\NormalTok{  vmat }\OtherTok{\textless{}{-}} \SpecialCharTok{{-}}\NormalTok{wgth}
  \FunctionTok{diag}\NormalTok{(vmat) }\OtherTok{\textless{}{-}} \SpecialCharTok{{-}}\FunctionTok{rowSums}\NormalTok{(vmat)}
\NormalTok{  vinv }\OtherTok{\textless{}{-}} \FunctionTok{solve}\NormalTok{(vmat }\SpecialCharTok{+}\NormalTok{ (}\DecValTok{1} \SpecialCharTok{/}\NormalTok{ nobj)) }\SpecialCharTok{{-}}\NormalTok{ (}\DecValTok{1} \SpecialCharTok{/}\NormalTok{ nobj)}
\NormalTok{  func }\OtherTok{\textless{}{-}} \ControlFlowTok{function}\NormalTok{(x, nobj, ndim, delta, wgth) \{}
\NormalTok{    x }\OtherTok{\textless{}{-}} \FunctionTok{matrix}\NormalTok{(x, nobj, ndim)}
\NormalTok{    xbar }\OtherTok{\textless{}{-}} \FunctionTok{smacofGuttman}\NormalTok{(x, delta, wgth, vinv)}
    \FunctionTok{return}\NormalTok{(}\FunctionTok{as.vector}\NormalTok{(xbar))}
\NormalTok{  \}}
\NormalTok{  jacob }\OtherTok{\textless{}{-}} \FunctionTok{jacobian}\NormalTok{(}
\NormalTok{    func,}
    \FunctionTok{as.vector}\NormalTok{(x),}
    \AttributeTok{nobj =}\NormalTok{ nobj,}
    \AttributeTok{ndim =}\NormalTok{ ndim,}
    \AttributeTok{delta =}\NormalTok{ delta,}
    \AttributeTok{wgth =}\NormalTok{ wgth}
\NormalTok{  )}
  \FunctionTok{return}\NormalTok{(jacob)}
\NormalTok{\}}

\NormalTok{smacofPCADerivative }\OtherTok{\textless{}{-}} \ControlFlowTok{function}\NormalTok{(x, h, delta, wgth, vinv, dmat) \{}
\NormalTok{  xbar }\OtherTok{\textless{}{-}} \FunctionTok{smacofGuttman}\NormalTok{(x, delta, wgth, vinv)}
\NormalTok{  dexh }\OtherTok{\textless{}{-}} \FunctionTok{smacofBasicDerivative}\NormalTok{(x, h, delta, wgth, vinv, dmat)}
\NormalTok{  deri }\OtherTok{\textless{}{-}} \FunctionTok{PCADerivative}\NormalTok{(xbar, dexh)}
  \FunctionTok{return}\NormalTok{(deri)}
\NormalTok{\}}

\NormalTok{smacofPCAJacobianFormula }\OtherTok{\textless{}{-}} \ControlFlowTok{function}\NormalTok{(x, delta, wgth) \{}
\NormalTok{  nobj }\OtherTok{\textless{}{-}} \FunctionTok{nrow}\NormalTok{(x)}
\NormalTok{  ndim }\OtherTok{\textless{}{-}} \FunctionTok{ncol}\NormalTok{(x)}
\NormalTok{  ntot }\OtherTok{\textless{}{-}}\NormalTok{ nobj }\SpecialCharTok{*}\NormalTok{ ndim}
\NormalTok{  dmat }\OtherTok{\textless{}{-}} \FunctionTok{as.matrix}\NormalTok{(}\FunctionTok{dist}\NormalTok{(x))}
\NormalTok{  vmat }\OtherTok{\textless{}{-}} \SpecialCharTok{{-}}\NormalTok{wgth}
  \FunctionTok{diag}\NormalTok{(vmat) }\OtherTok{\textless{}{-}} \SpecialCharTok{{-}}\FunctionTok{rowSums}\NormalTok{(vmat)}
\NormalTok{  vinv }\OtherTok{\textless{}{-}} \FunctionTok{solve}\NormalTok{(vmat }\SpecialCharTok{+}\NormalTok{ (}\DecValTok{1} \SpecialCharTok{/}\NormalTok{ nobj)) }\SpecialCharTok{{-}}\NormalTok{ (}\DecValTok{1} \SpecialCharTok{/}\NormalTok{ nobj)}
\NormalTok{  jacob }\OtherTok{\textless{}{-}} \FunctionTok{matrix}\NormalTok{(}\DecValTok{0}\NormalTok{, ntot, ntot)}
\NormalTok{  e }\OtherTok{\textless{}{-}} \ControlFlowTok{function}\NormalTok{(i, n) \{}
    \FunctionTok{ifelse}\NormalTok{(i }\SpecialCharTok{==} \DecValTok{1}\SpecialCharTok{:}\NormalTok{n, }\DecValTok{1}\NormalTok{, }\DecValTok{0}\NormalTok{)}
\NormalTok{  \}}
\NormalTok{  k }\OtherTok{\textless{}{-}} \DecValTok{1}
  \ControlFlowTok{for}\NormalTok{ (j }\ControlFlowTok{in} \DecValTok{1}\SpecialCharTok{:}\NormalTok{ndim) \{}
    \ControlFlowTok{for}\NormalTok{ (i }\ControlFlowTok{in} \DecValTok{1}\SpecialCharTok{:}\NormalTok{nobj) \{}
\NormalTok{      h }\OtherTok{\textless{}{-}} \FunctionTok{outer}\NormalTok{(}\FunctionTok{e}\NormalTok{(i, nobj), }\FunctionTok{e}\NormalTok{(j, ndim))}
\NormalTok{      r }\OtherTok{\textless{}{-}} \FunctionTok{smacofPCADerivative}\NormalTok{(x, h, delta, wgth, vinv, dmat)}
\NormalTok{      jacob[, k] }\OtherTok{\textless{}{-}} \FunctionTok{as.vector}\NormalTok{(r)}
\NormalTok{      k }\OtherTok{\textless{}{-}}\NormalTok{ k }\SpecialCharTok{+} \DecValTok{1}
\NormalTok{    \}}
\NormalTok{  \}}
  \FunctionTok{return}\NormalTok{(jacob)}
\NormalTok{\}}

\NormalTok{smacofPCAJacobianNumerical }\OtherTok{\textless{}{-}} \ControlFlowTok{function}\NormalTok{(x, delta, wgth) \{}
\NormalTok{  nobj }\OtherTok{\textless{}{-}} \FunctionTok{nrow}\NormalTok{(x)}
\NormalTok{  ndim }\OtherTok{\textless{}{-}} \FunctionTok{ncol}\NormalTok{(x)}
\NormalTok{  ntot }\OtherTok{\textless{}{-}}\NormalTok{ nobj }\SpecialCharTok{*}\NormalTok{ ndim}
\NormalTok{  dmat }\OtherTok{\textless{}{-}} \FunctionTok{as.matrix}\NormalTok{(}\FunctionTok{dist}\NormalTok{(x))}
\NormalTok{  vmat }\OtherTok{\textless{}{-}} \SpecialCharTok{{-}}\NormalTok{wgth}
  \FunctionTok{diag}\NormalTok{(vmat) }\OtherTok{\textless{}{-}} \SpecialCharTok{{-}}\FunctionTok{rowSums}\NormalTok{(vmat)}
\NormalTok{  vinv }\OtherTok{\textless{}{-}} \FunctionTok{solve}\NormalTok{(vmat }\SpecialCharTok{+}\NormalTok{ (}\DecValTok{1} \SpecialCharTok{/}\NormalTok{ nobj)) }\SpecialCharTok{{-}}\NormalTok{ (}\DecValTok{1} \SpecialCharTok{/}\NormalTok{ nobj)}
\NormalTok{  func }\OtherTok{\textless{}{-}} \ControlFlowTok{function}\NormalTok{(x, nobj, ndim, delta, wgth) \{}
\NormalTok{    x }\OtherTok{\textless{}{-}} \FunctionTok{matrix}\NormalTok{(x, nobj, ndim)}
\NormalTok{    xbar }\OtherTok{\textless{}{-}} \FunctionTok{smacofGuttman}\NormalTok{(x, delta, wgth, vinv)}
\NormalTok{    l }\OtherTok{\textless{}{-}} \FunctionTok{smacofSignEigenVectors}\NormalTok{(}\FunctionTok{svd}\NormalTok{(xbar)}\SpecialCharTok{$}\NormalTok{v)}
    \FunctionTok{return}\NormalTok{(xbar }\SpecialCharTok{\%*\%}\NormalTok{ l)}
\NormalTok{  \}}
\NormalTok{  jacob }\OtherTok{\textless{}{-}} \FunctionTok{jacobian}\NormalTok{(}
\NormalTok{    func,}
    \FunctionTok{as.vector}\NormalTok{(x),}
    \AttributeTok{nobj =}\NormalTok{ nobj,}
    \AttributeTok{ndim =}\NormalTok{ ndim,}
    \AttributeTok{delta =}\NormalTok{ delta,}
    \AttributeTok{wgth =}\NormalTok{ wgth}
\NormalTok{  )}
  \FunctionTok{return}\NormalTok{(jacob)}
\NormalTok{\}}


\NormalTok{smacofQRDerivative }\OtherTok{\textless{}{-}} \ControlFlowTok{function}\NormalTok{(x, h, delta, wgth, vinv, dmat) \{}
\NormalTok{  xbar }\OtherTok{\textless{}{-}} \FunctionTok{smacofGuttman}\NormalTok{(x, delta, wgth, vinv)}
\NormalTok{  dexh }\OtherTok{\textless{}{-}} \FunctionTok{smacofBasicDerivative}\NormalTok{(x, h, delta, wgth, vinv, dmat)}
\NormalTok{  deri }\OtherTok{\textless{}{-}} \FunctionTok{QRDerivative}\NormalTok{(xbar, dexh)}
  \FunctionTok{return}\NormalTok{(deri)}
\NormalTok{\}}

\NormalTok{smacofQRJacobianFormula }\OtherTok{\textless{}{-}} \ControlFlowTok{function}\NormalTok{(x, delta, wgth) \{}
\NormalTok{  nobj }\OtherTok{\textless{}{-}} \FunctionTok{nrow}\NormalTok{(x)}
\NormalTok{  ndim }\OtherTok{\textless{}{-}} \FunctionTok{ncol}\NormalTok{(x)}
\NormalTok{  ntot }\OtherTok{\textless{}{-}}\NormalTok{ nobj }\SpecialCharTok{*}\NormalTok{ ndim}
\NormalTok{  dmat }\OtherTok{\textless{}{-}} \FunctionTok{as.matrix}\NormalTok{(}\FunctionTok{dist}\NormalTok{(x))}
\NormalTok{  vmat }\OtherTok{\textless{}{-}} \SpecialCharTok{{-}}\NormalTok{wgth}
  \FunctionTok{diag}\NormalTok{(vmat) }\OtherTok{\textless{}{-}} \SpecialCharTok{{-}}\FunctionTok{rowSums}\NormalTok{(vmat)}
\NormalTok{  vinv }\OtherTok{\textless{}{-}} \FunctionTok{solve}\NormalTok{(vmat }\SpecialCharTok{+}\NormalTok{ (}\DecValTok{1} \SpecialCharTok{/}\NormalTok{ nobj)) }\SpecialCharTok{{-}}\NormalTok{ (}\DecValTok{1} \SpecialCharTok{/}\NormalTok{ nobj)}
\NormalTok{  jacob }\OtherTok{\textless{}{-}} \FunctionTok{matrix}\NormalTok{(}\DecValTok{0}\NormalTok{, ntot, ntot)}
\NormalTok{  e }\OtherTok{\textless{}{-}} \ControlFlowTok{function}\NormalTok{(i, n) \{}
    \FunctionTok{ifelse}\NormalTok{(i }\SpecialCharTok{==} \DecValTok{1}\SpecialCharTok{:}\NormalTok{n, }\DecValTok{1}\NormalTok{, }\DecValTok{0}\NormalTok{)}
\NormalTok{  \}}
\NormalTok{  k }\OtherTok{\textless{}{-}} \DecValTok{1}
  \ControlFlowTok{for}\NormalTok{ (j }\ControlFlowTok{in} \DecValTok{1}\SpecialCharTok{:}\NormalTok{ndim) \{}
    \ControlFlowTok{for}\NormalTok{ (i }\ControlFlowTok{in} \DecValTok{1}\SpecialCharTok{:}\NormalTok{nobj) \{}
\NormalTok{      h }\OtherTok{\textless{}{-}} \FunctionTok{outer}\NormalTok{(}\FunctionTok{e}\NormalTok{(i, nobj), }\FunctionTok{e}\NormalTok{(j, ndim))}
\NormalTok{      r }\OtherTok{\textless{}{-}} \FunctionTok{smacofQRDerivative}\NormalTok{(x, h, delta, wgth, vinv, dmat)}
\NormalTok{      jacob[, k] }\OtherTok{\textless{}{-}} \FunctionTok{as.vector}\NormalTok{(r)}
\NormalTok{      k }\OtherTok{\textless{}{-}}\NormalTok{ k }\SpecialCharTok{+} \DecValTok{1}
\NormalTok{    \}}
\NormalTok{  \}}
  \FunctionTok{return}\NormalTok{(jacob)}
\NormalTok{\}}

\NormalTok{smacofQRJacobianNumerical }\OtherTok{\textless{}{-}} \ControlFlowTok{function}\NormalTok{(x, delta, wgth) \{}
\NormalTok{  nobj }\OtherTok{\textless{}{-}} \FunctionTok{nrow}\NormalTok{(x)}
\NormalTok{  ndim }\OtherTok{\textless{}{-}} \FunctionTok{ncol}\NormalTok{(x)}
\NormalTok{  ntot }\OtherTok{\textless{}{-}}\NormalTok{ nobj }\SpecialCharTok{*}\NormalTok{ ndim}
\NormalTok{  dmat }\OtherTok{\textless{}{-}} \FunctionTok{as.matrix}\NormalTok{(}\FunctionTok{dist}\NormalTok{(x))}
\NormalTok{  vmat }\OtherTok{\textless{}{-}} \SpecialCharTok{{-}}\NormalTok{wgth}
  \FunctionTok{diag}\NormalTok{(vmat) }\OtherTok{\textless{}{-}} \SpecialCharTok{{-}}\FunctionTok{rowSums}\NormalTok{(vmat)}
\NormalTok{  vinv }\OtherTok{\textless{}{-}} \FunctionTok{solve}\NormalTok{(vmat }\SpecialCharTok{+}\NormalTok{ (}\DecValTok{1} \SpecialCharTok{/}\NormalTok{ nobj)) }\SpecialCharTok{{-}}\NormalTok{ (}\DecValTok{1} \SpecialCharTok{/}\NormalTok{ nobj)}
\NormalTok{  func }\OtherTok{\textless{}{-}} \ControlFlowTok{function}\NormalTok{(x, nobj, ndim, delta, wgth) \{}
\NormalTok{    x }\OtherTok{\textless{}{-}} \FunctionTok{matrix}\NormalTok{(x, nobj, ndim)}
\NormalTok{    xbar }\OtherTok{\textless{}{-}} \FunctionTok{smacofGuttman}\NormalTok{(x, delta, wgth, vinv)}
\NormalTok{    l }\OtherTok{\textless{}{-}} \FunctionTok{qr.Q}\NormalTok{(}\FunctionTok{qr}\NormalTok{(}\FunctionTok{t}\NormalTok{(xbar[}\DecValTok{1}\SpecialCharTok{:}\NormalTok{ndim, ])))}
    \FunctionTok{return}\NormalTok{(xbar }\SpecialCharTok{\%*\%}\NormalTok{ l)}
\NormalTok{  \}}
\NormalTok{  jacob }\OtherTok{\textless{}{-}} \FunctionTok{jacobian}\NormalTok{(}
\NormalTok{    func,}
    \FunctionTok{as.vector}\NormalTok{(x),}
    \AttributeTok{nobj =}\NormalTok{ nobj,}
    \AttributeTok{ndim =}\NormalTok{ ndim,}
    \AttributeTok{delta =}\NormalTok{ delta,}
    \AttributeTok{wgth =}\NormalTok{ wgth}
\NormalTok{  )}
  \FunctionTok{return}\NormalTok{(jacob)}
\NormalTok{\}}

\NormalTok{smacofYbasDerivative }\OtherTok{\textless{}{-}} \ControlFlowTok{function}\NormalTok{(x, h, delta, wgth, vmat, vinv, dmat, bs) \{}
\NormalTok{  ndim }\OtherTok{\textless{}{-}} \FunctionTok{ncol}\NormalTok{(x)}
\NormalTok{  nobj }\OtherTok{\textless{}{-}} \FunctionTok{nrow}\NormalTok{(x)}
\NormalTok{  ntot }\OtherTok{\textless{}{-}}\NormalTok{ nobj }\SpecialCharTok{*}\NormalTok{ ndim}
\NormalTok{  dexh }\OtherTok{\textless{}{-}} \FunctionTok{smacofBasicDerivative}\NormalTok{(x, h, delta, wgth, vinv, dmat)}
\NormalTok{  deri }\OtherTok{\textless{}{-}} \FunctionTok{matrix}\NormalTok{(}\DecValTok{0}\NormalTok{, nobj, ndim)}
  \ControlFlowTok{for}\NormalTok{ (i }\ControlFlowTok{in} \DecValTok{1}\SpecialCharTok{:}\NormalTok{ndim) \{}
\NormalTok{    deri[, i] }\OtherTok{\textless{}{-}}\NormalTok{ bs[[i]] }\SpecialCharTok{\%*\%} \FunctionTok{crossprod}\NormalTok{(bs[[i]], vmat }\SpecialCharTok{\%*\%}\NormalTok{ dexh[, i])}
\NormalTok{  \}}
  \FunctionTok{return}\NormalTok{(deri)}
\NormalTok{\}}

\NormalTok{smacofYbasJacobianFormula }\OtherTok{\textless{}{-}} \ControlFlowTok{function}\NormalTok{(x, delta, wgth) \{}
\NormalTok{  nobj }\OtherTok{\textless{}{-}} \FunctionTok{nrow}\NormalTok{(x)}
\NormalTok{  ndim }\OtherTok{\textless{}{-}} \FunctionTok{ncol}\NormalTok{(x)}
\NormalTok{  ntot }\OtherTok{\textless{}{-}}\NormalTok{ nobj }\SpecialCharTok{*}\NormalTok{ ndim}
\NormalTok{  dmat }\OtherTok{\textless{}{-}} \FunctionTok{as.matrix}\NormalTok{(}\FunctionTok{dist}\NormalTok{(x))}
\NormalTok{  vmat }\OtherTok{\textless{}{-}} \SpecialCharTok{{-}}\NormalTok{wgth}
  \FunctionTok{diag}\NormalTok{(vmat) }\OtherTok{\textless{}{-}} \SpecialCharTok{{-}}\FunctionTok{rowSums}\NormalTok{(vmat)}
\NormalTok{  vinv }\OtherTok{\textless{}{-}} \FunctionTok{solve}\NormalTok{(vmat }\SpecialCharTok{+}\NormalTok{ (}\DecValTok{1} \SpecialCharTok{/}\NormalTok{ nobj)) }\SpecialCharTok{{-}}\NormalTok{ (}\DecValTok{1} \SpecialCharTok{/}\NormalTok{ nobj)}
\NormalTok{  bs }\OtherTok{\textless{}{-}} \FunctionTok{smacofMakeBasis}\NormalTok{(nobj, ndim, vmat)}
\NormalTok{  jacob }\OtherTok{\textless{}{-}} \FunctionTok{matrix}\NormalTok{(}\DecValTok{0}\NormalTok{, ntot, ntot)}
\NormalTok{  e }\OtherTok{\textless{}{-}} \ControlFlowTok{function}\NormalTok{(i, n) \{}
    \FunctionTok{ifelse}\NormalTok{(i }\SpecialCharTok{==} \DecValTok{1}\SpecialCharTok{:}\NormalTok{n, }\DecValTok{1}\NormalTok{, }\DecValTok{0}\NormalTok{)}
\NormalTok{  \}}
\NormalTok{  k }\OtherTok{\textless{}{-}} \DecValTok{1}
  \ControlFlowTok{for}\NormalTok{ (j }\ControlFlowTok{in} \DecValTok{1}\SpecialCharTok{:}\NormalTok{ndim) \{}
    \ControlFlowTok{for}\NormalTok{ (i }\ControlFlowTok{in} \DecValTok{1}\SpecialCharTok{:}\NormalTok{nobj) \{}
\NormalTok{      h }\OtherTok{\textless{}{-}} \FunctionTok{outer}\NormalTok{(}\FunctionTok{e}\NormalTok{(i, nobj), }\FunctionTok{e}\NormalTok{(j, ndim))}
\NormalTok{      r }\OtherTok{\textless{}{-}} \FunctionTok{smacofYbasDerivative}\NormalTok{(x, h, delta, wgth, vmat, vinv, dmat, bs)}
\NormalTok{      jacob[, k] }\OtherTok{\textless{}{-}} \FunctionTok{as.vector}\NormalTok{(r)}
\NormalTok{      k }\OtherTok{\textless{}{-}}\NormalTok{ k }\SpecialCharTok{+} \DecValTok{1}
\NormalTok{    \}}
\NormalTok{  \}}
  \FunctionTok{return}\NormalTok{(jacob)}
\NormalTok{\}}

\NormalTok{smacofYbasJacobianNumerical }\OtherTok{\textless{}{-}} \ControlFlowTok{function}\NormalTok{(x, delta, wgth) \{}
\NormalTok{  nobj }\OtherTok{\textless{}{-}} \FunctionTok{nrow}\NormalTok{(x)}
\NormalTok{  ndim }\OtherTok{\textless{}{-}} \FunctionTok{ncol}\NormalTok{(x)}
\NormalTok{  vmat }\OtherTok{\textless{}{-}} \SpecialCharTok{{-}}\NormalTok{wgth}
  \FunctionTok{diag}\NormalTok{(vmat) }\OtherTok{\textless{}{-}} \SpecialCharTok{{-}}\FunctionTok{rowSums}\NormalTok{(vmat)}
\NormalTok{  bs }\OtherTok{\textless{}{-}} \FunctionTok{smacofMakeBasis}\NormalTok{(nobj, ndim, vmat)}
\NormalTok{  vinv }\OtherTok{\textless{}{-}} \FunctionTok{solve}\NormalTok{(vmat }\SpecialCharTok{+}\NormalTok{ (}\DecValTok{1} \SpecialCharTok{/}\NormalTok{ nobj)) }\SpecialCharTok{{-}}\NormalTok{ (}\DecValTok{1} \SpecialCharTok{/}\NormalTok{ nobj)}
\NormalTok{  func }\OtherTok{\textless{}{-}} \ControlFlowTok{function}\NormalTok{(x, nobj, ndim, delta, wgth, vmat, vinv, bs) \{}
\NormalTok{    x }\OtherTok{\textless{}{-}} \FunctionTok{matrix}\NormalTok{(x, nobj, ndim)}
\NormalTok{    xbar }\OtherTok{\textless{}{-}} \FunctionTok{smacofGuttman}\NormalTok{(x, delta, wgth, vinv)}
    \ControlFlowTok{for}\NormalTok{ (i }\ControlFlowTok{in} \DecValTok{1}\SpecialCharTok{:}\NormalTok{ndim) \{}
\NormalTok{      xbar[, i] }\OtherTok{\textless{}{-}}\NormalTok{ bs[[i]] }\SpecialCharTok{\%*\%} \FunctionTok{crossprod}\NormalTok{(bs[[i]], vmat }\SpecialCharTok{\%*\%}\NormalTok{ xbar[, i])}
\NormalTok{    \}}
    \FunctionTok{return}\NormalTok{(xbar)}
\NormalTok{  \}}
\NormalTok{  jacob }\OtherTok{\textless{}{-}} \FunctionTok{jacobian}\NormalTok{(}
\NormalTok{    func,}
    \FunctionTok{as.vector}\NormalTok{(x),}
    \AttributeTok{nobj =}\NormalTok{ nobj,}
    \AttributeTok{ndim =}\NormalTok{ ndim,}
    \AttributeTok{delta =}\NormalTok{ delta,}
    \AttributeTok{wgth =}\NormalTok{ wgth,}
    \AttributeTok{vmat =}\NormalTok{ vmat,}
    \AttributeTok{vinv =}\NormalTok{ vinv,}
    \AttributeTok{bs =}\NormalTok{ bs}
\NormalTok{  )}
  \FunctionTok{return}\NormalTok{(jacob)}
\NormalTok{\}}


\NormalTok{smacofRelaxDerivative }\OtherTok{\textless{}{-}} \ControlFlowTok{function}\NormalTok{(x, h, delta, wgth, vinv, dmat) \{}
\NormalTok{  ndim }\OtherTok{\textless{}{-}} \FunctionTok{ncol}\NormalTok{(x)}
\NormalTok{  nobj }\OtherTok{\textless{}{-}} \FunctionTok{nrow}\NormalTok{(x)}
\NormalTok{  ntot }\OtherTok{\textless{}{-}}\NormalTok{ nobj }\SpecialCharTok{*}\NormalTok{ ndim}
\NormalTok{  dexh }\OtherTok{\textless{}{-}} \FunctionTok{smacofBasicDerivative}\NormalTok{(x, h, delta, wgth, vinv, dmat)}
\NormalTok{  deri }\OtherTok{\textless{}{-}} \DecValTok{2} \SpecialCharTok{*}\NormalTok{ dexh }\SpecialCharTok{{-}}\NormalTok{ h}
  \FunctionTok{return}\NormalTok{(deri)}
\NormalTok{\}}

\NormalTok{smacofRelaxJacobianFormula }\OtherTok{\textless{}{-}} \ControlFlowTok{function}\NormalTok{(x, delta, wgth) \{}
\NormalTok{  nobj }\OtherTok{\textless{}{-}} \FunctionTok{nrow}\NormalTok{(x)}
\NormalTok{  ndim }\OtherTok{\textless{}{-}} \FunctionTok{ncol}\NormalTok{(x)}
\NormalTok{  ntot }\OtherTok{\textless{}{-}}\NormalTok{ nobj }\SpecialCharTok{*}\NormalTok{ ndim}
\NormalTok{  dmat }\OtherTok{\textless{}{-}} \FunctionTok{as.matrix}\NormalTok{(}\FunctionTok{dist}\NormalTok{(x))}
\NormalTok{  vmat }\OtherTok{\textless{}{-}} \SpecialCharTok{{-}}\NormalTok{wgth}
  \FunctionTok{diag}\NormalTok{(vmat) }\OtherTok{\textless{}{-}} \SpecialCharTok{{-}}\FunctionTok{rowSums}\NormalTok{(vmat)}
\NormalTok{  vinv }\OtherTok{\textless{}{-}} \FunctionTok{solve}\NormalTok{(vmat }\SpecialCharTok{+}\NormalTok{ (}\DecValTok{1} \SpecialCharTok{/}\NormalTok{ nobj)) }\SpecialCharTok{{-}}\NormalTok{ (}\DecValTok{1} \SpecialCharTok{/}\NormalTok{ nobj)}
\NormalTok{  jacob }\OtherTok{\textless{}{-}} \FunctionTok{matrix}\NormalTok{(}\DecValTok{0}\NormalTok{, ntot, ntot)}
\NormalTok{  e }\OtherTok{\textless{}{-}} \ControlFlowTok{function}\NormalTok{(i, n) \{}
    \FunctionTok{ifelse}\NormalTok{(i }\SpecialCharTok{==} \DecValTok{1}\SpecialCharTok{:}\NormalTok{n, }\DecValTok{1}\NormalTok{, }\DecValTok{0}\NormalTok{)}
\NormalTok{  \}}
\NormalTok{  k }\OtherTok{\textless{}{-}} \DecValTok{1}
  \ControlFlowTok{for}\NormalTok{ (j }\ControlFlowTok{in} \DecValTok{1}\SpecialCharTok{:}\NormalTok{ndim) \{}
    \ControlFlowTok{for}\NormalTok{ (i }\ControlFlowTok{in} \DecValTok{1}\SpecialCharTok{:}\NormalTok{nobj) \{}
\NormalTok{      h }\OtherTok{\textless{}{-}} \FunctionTok{outer}\NormalTok{(}\FunctionTok{e}\NormalTok{(i, nobj), }\FunctionTok{e}\NormalTok{(j, ndim))}
\NormalTok{      r }\OtherTok{\textless{}{-}} \FunctionTok{smacofRelaxDerivative}\NormalTok{(x, h, delta, wgth, vinv, dmat)}
\NormalTok{      jacob[, k] }\OtherTok{\textless{}{-}} \FunctionTok{as.vector}\NormalTok{(r)}
\NormalTok{      k }\OtherTok{\textless{}{-}}\NormalTok{ k }\SpecialCharTok{+} \DecValTok{1}
\NormalTok{    \}}
\NormalTok{  \}}
  \FunctionTok{return}\NormalTok{(jacob)}
\NormalTok{\}}

\NormalTok{smacofRelaxJacobianNumerical }\OtherTok{\textless{}{-}} \ControlFlowTok{function}\NormalTok{(x, delta, wgth) \{}
\NormalTok{  nobj }\OtherTok{\textless{}{-}} \FunctionTok{nrow}\NormalTok{(x)}
\NormalTok{  ndim }\OtherTok{\textless{}{-}} \FunctionTok{ncol}\NormalTok{(x)}
\NormalTok{  vmat }\OtherTok{\textless{}{-}} \SpecialCharTok{{-}}\NormalTok{wgth}
  \FunctionTok{diag}\NormalTok{(vmat) }\OtherTok{\textless{}{-}} \SpecialCharTok{{-}}\FunctionTok{rowSums}\NormalTok{(vmat)}
\NormalTok{  bs }\OtherTok{\textless{}{-}} \FunctionTok{smacofMakeBasis}\NormalTok{(nobj, ndim, vmat)}
\NormalTok{  vinv }\OtherTok{\textless{}{-}} \FunctionTok{solve}\NormalTok{(vmat }\SpecialCharTok{+}\NormalTok{ (}\DecValTok{1} \SpecialCharTok{/}\NormalTok{ nobj)) }\SpecialCharTok{{-}}\NormalTok{ (}\DecValTok{1} \SpecialCharTok{/}\NormalTok{ nobj)}
\NormalTok{  func }\OtherTok{\textless{}{-}} \ControlFlowTok{function}\NormalTok{(x, nobj, ndim, delta, wgth, vinv) \{}
\NormalTok{    x }\OtherTok{\textless{}{-}} \FunctionTok{matrix}\NormalTok{(x, nobj, ndim)}
\NormalTok{    xbar }\OtherTok{\textless{}{-}} \FunctionTok{smacofGuttman}\NormalTok{(x, delta, wgth, vinv)}
\NormalTok{    xbaz }\OtherTok{\textless{}{-}} \DecValTok{2} \SpecialCharTok{*}\NormalTok{ xbar }\SpecialCharTok{{-}}\NormalTok{ x}
    \FunctionTok{return}\NormalTok{(xbaz)}
\NormalTok{  \}}
\NormalTok{  jacob }\OtherTok{\textless{}{-}} \FunctionTok{jacobian}\NormalTok{(}
\NormalTok{    func,}
    \FunctionTok{as.vector}\NormalTok{(x),}
    \AttributeTok{nobj =}\NormalTok{ nobj,}
    \AttributeTok{ndim =}\NormalTok{ ndim,}
    \AttributeTok{delta =}\NormalTok{ delta,}
    \AttributeTok{wgth =}\NormalTok{ wgth,}
    \AttributeTok{vinv =}\NormalTok{ vinv}
\NormalTok{  )}
  \FunctionTok{return}\NormalTok{(jacob)}
\NormalTok{\}}

\NormalTok{smacofDoubleDerivative }\OtherTok{\textless{}{-}} \ControlFlowTok{function}\NormalTok{(x, h, delta, wgth, vinv, dmat) \{}
\NormalTok{  ndim }\OtherTok{\textless{}{-}} \FunctionTok{ncol}\NormalTok{(x)}
\NormalTok{  nobj }\OtherTok{\textless{}{-}} \FunctionTok{nrow}\NormalTok{(x)}
\NormalTok{  ntot }\OtherTok{\textless{}{-}}\NormalTok{ nobj }\SpecialCharTok{*}\NormalTok{ ndim}
\NormalTok{  dexh }\OtherTok{\textless{}{-}} \FunctionTok{smacofBasicDerivative}\NormalTok{(x, h, delta, wgth, vinv, dmat)}
\NormalTok{  deri }\OtherTok{\textless{}{-}} \DecValTok{2} \SpecialCharTok{*}\NormalTok{ dexh }\SpecialCharTok{{-}}\NormalTok{ h}
  \FunctionTok{return}\NormalTok{(deri)}
\NormalTok{\}}

\NormalTok{smacofDoubleJacobianFormula }\OtherTok{\textless{}{-}} \ControlFlowTok{function}\NormalTok{(x, delta, wgth) \{}
\NormalTok{  nobj }\OtherTok{\textless{}{-}} \FunctionTok{nrow}\NormalTok{(x)}
\NormalTok{  ndim }\OtherTok{\textless{}{-}} \FunctionTok{ncol}\NormalTok{(x)}
\NormalTok{  ntot }\OtherTok{\textless{}{-}}\NormalTok{ nobj }\SpecialCharTok{*}\NormalTok{ ndim}
\NormalTok{  dmat }\OtherTok{\textless{}{-}} \FunctionTok{as.matrix}\NormalTok{(}\FunctionTok{dist}\NormalTok{(x))}
\NormalTok{  vmat }\OtherTok{\textless{}{-}} \SpecialCharTok{{-}}\NormalTok{wgth}
  \FunctionTok{diag}\NormalTok{(vmat) }\OtherTok{\textless{}{-}} \SpecialCharTok{{-}}\FunctionTok{rowSums}\NormalTok{(vmat)}
\NormalTok{  vinv }\OtherTok{\textless{}{-}} \FunctionTok{solve}\NormalTok{(vmat }\SpecialCharTok{+}\NormalTok{ (}\DecValTok{1} \SpecialCharTok{/}\NormalTok{ nobj)) }\SpecialCharTok{{-}}\NormalTok{ (}\DecValTok{1} \SpecialCharTok{/}\NormalTok{ nobj)}
\NormalTok{  jacob }\OtherTok{\textless{}{-}} \FunctionTok{matrix}\NormalTok{(}\DecValTok{0}\NormalTok{, ntot, ntot)}
\NormalTok{  e }\OtherTok{\textless{}{-}} \ControlFlowTok{function}\NormalTok{(i, n) \{}
    \FunctionTok{ifelse}\NormalTok{(i }\SpecialCharTok{==} \DecValTok{1}\SpecialCharTok{:}\NormalTok{n, }\DecValTok{1}\NormalTok{, }\DecValTok{0}\NormalTok{)}
\NormalTok{  \}}
\NormalTok{  k }\OtherTok{\textless{}{-}} \DecValTok{1}
  \ControlFlowTok{for}\NormalTok{ (j }\ControlFlowTok{in} \DecValTok{1}\SpecialCharTok{:}\NormalTok{ndim) \{}
    \ControlFlowTok{for}\NormalTok{ (i }\ControlFlowTok{in} \DecValTok{1}\SpecialCharTok{:}\NormalTok{nobj) \{}
\NormalTok{      h }\OtherTok{\textless{}{-}} \FunctionTok{outer}\NormalTok{(}\FunctionTok{e}\NormalTok{(i, nobj), }\FunctionTok{e}\NormalTok{(j, ndim))}
\NormalTok{      r }\OtherTok{\textless{}{-}} \FunctionTok{smacofDoubleDerivative}\NormalTok{(x, h, delta, wgth, vinv, dmat)}
\NormalTok{      jacob[, k] }\OtherTok{\textless{}{-}} \FunctionTok{as.vector}\NormalTok{(r)}
\NormalTok{      k }\OtherTok{\textless{}{-}}\NormalTok{ k }\SpecialCharTok{+} \DecValTok{1}
\NormalTok{    \}}
\NormalTok{  \}}
  \FunctionTok{return}\NormalTok{(jacob)}
\NormalTok{\}}

\NormalTok{smacofDoubleJacobianNumerical }\OtherTok{\textless{}{-}} \ControlFlowTok{function}\NormalTok{(x, delta, wgth) \{}
\NormalTok{  nobj }\OtherTok{\textless{}{-}} \FunctionTok{nrow}\NormalTok{(x)}
\NormalTok{  ndim }\OtherTok{\textless{}{-}} \FunctionTok{ncol}\NormalTok{(x)}
\NormalTok{  vmat }\OtherTok{\textless{}{-}} \SpecialCharTok{{-}}\NormalTok{wgth}
  \FunctionTok{diag}\NormalTok{(vmat) }\OtherTok{\textless{}{-}} \SpecialCharTok{{-}}\FunctionTok{rowSums}\NormalTok{(vmat)}
\NormalTok{  bs }\OtherTok{\textless{}{-}} \FunctionTok{smacofMakeBasis}\NormalTok{(nobj, ndim, vmat)}
\NormalTok{  vinv }\OtherTok{\textless{}{-}} \FunctionTok{solve}\NormalTok{(vmat }\SpecialCharTok{+}\NormalTok{ (}\DecValTok{1} \SpecialCharTok{/}\NormalTok{ nobj)) }\SpecialCharTok{{-}}\NormalTok{ (}\DecValTok{1} \SpecialCharTok{/}\NormalTok{ nobj)}
\NormalTok{  func }\OtherTok{\textless{}{-}} \ControlFlowTok{function}\NormalTok{(x, nobj, ndim, delta, wgth, vinv) \{}
\NormalTok{    x }\OtherTok{\textless{}{-}} \FunctionTok{matrix}\NormalTok{(x, nobj, ndim)}
\NormalTok{    xbar }\OtherTok{\textless{}{-}} \DecValTok{2} \SpecialCharTok{*} \FunctionTok{smacofGuttman}\NormalTok{(x, delta, wgth, vinv) }\SpecialCharTok{{-}}\NormalTok{ x}
\NormalTok{    xbaz }\OtherTok{\textless{}{-}} \DecValTok{2} \SpecialCharTok{*} \FunctionTok{smacofGuttman}\NormalTok{(xbar, delta, wgth, vinv) }\SpecialCharTok{{-}}\NormalTok{ xbar}
    \FunctionTok{return}\NormalTok{(xbaz)}
\NormalTok{  \}}
\NormalTok{  jacob }\OtherTok{\textless{}{-}} \FunctionTok{jacobian}\NormalTok{(}
\NormalTok{    func,}
    \FunctionTok{as.vector}\NormalTok{(x),}
    \AttributeTok{nobj =}\NormalTok{ nobj,}
    \AttributeTok{ndim =}\NormalTok{ ndim,}
    \AttributeTok{delta =}\NormalTok{ delta,}
    \AttributeTok{wgth =}\NormalTok{ wgth,}
    \AttributeTok{vinv =}\NormalTok{ vinv}
\NormalTok{  )}
  \FunctionTok{return}\NormalTok{(jacob)}
\NormalTok{\}}

\NormalTok{smacofDilateJacobianFormula }\OtherTok{\textless{}{-}} \ControlFlowTok{function}\NormalTok{() \{\}}

\NormalTok{smacofDilateJacobianNumerical }\OtherTok{\textless{}{-}} \ControlFlowTok{function}\NormalTok{() \{\}}


\NormalTok{smacofStabilizeJacobianFormula }\OtherTok{\textless{}{-}} \ControlFlowTok{function}\NormalTok{() \{\}}

\NormalTok{smacofStabilizeJacobianNumerical }\OtherTok{\textless{}{-}} \ControlFlowTok{function}\NormalTok{() \{\}}
\end{Highlighting}
\end{Shaded}

\subsection{smacofPCADerivative.R}\label{smacofpcaderivative.r}

\begin{Shaded}
\begin{Highlighting}[]
\NormalTok{PCADerivative }\OtherTok{\textless{}{-}} \ControlFlowTok{function}\NormalTok{(x, h) \{}
\NormalTok{  ndim }\OtherTok{\textless{}{-}} \FunctionTok{ncol}\NormalTok{(x)}
\NormalTok{  eixx }\OtherTok{\textless{}{-}} \FunctionTok{eigen}\NormalTok{(}\FunctionTok{crossprod}\NormalTok{(x))}
\NormalTok{  evec }\OtherTok{\textless{}{-}}\NormalTok{ eixx}\SpecialCharTok{$}\NormalTok{vectors}
\NormalTok{  evec }\OtherTok{\textless{}{-}}\NormalTok{ evec }\SpecialCharTok{\%*\%} \FunctionTok{diag}\NormalTok{(}\FunctionTok{sign}\NormalTok{(}\FunctionTok{diag}\NormalTok{(evec)))}
\NormalTok{  eval }\OtherTok{\textless{}{-}}\NormalTok{ eixx}\SpecialCharTok{$}\NormalTok{values}
\NormalTok{  xh }\OtherTok{\textless{}{-}} \FunctionTok{crossprod}\NormalTok{(x, h)}
\NormalTok{  xh }\OtherTok{\textless{}{-}}\NormalTok{ xh }\SpecialCharTok{+} \FunctionTok{t}\NormalTok{(xh)}
\NormalTok{  s }\OtherTok{\textless{}{-}} \FunctionTok{matrix}\NormalTok{(}\DecValTok{0}\NormalTok{, ndim, ndim)}
  \ControlFlowTok{for}\NormalTok{ (i }\ControlFlowTok{in} \DecValTok{1}\SpecialCharTok{:}\NormalTok{ndim) \{}
    \ControlFlowTok{for}\NormalTok{ (j }\ControlFlowTok{in} \DecValTok{1}\SpecialCharTok{:}\NormalTok{ndim) \{}
      \ControlFlowTok{if}\NormalTok{ (i }\SpecialCharTok{==}\NormalTok{ j) \{}
        \ControlFlowTok{next}
\NormalTok{      \}}
\NormalTok{      s[i, j] }\OtherTok{\textless{}{-}} \SpecialCharTok{{-}}\FunctionTok{sum}\NormalTok{(evec[, i] }\SpecialCharTok{*}\NormalTok{ (xh }\SpecialCharTok{\%*\%}\NormalTok{ evec[, j])) }\SpecialCharTok{/}\NormalTok{ (eval[i] }\SpecialCharTok{{-}}\NormalTok{ eval[j])}
\NormalTok{    \}}
\NormalTok{  \}}
  \FunctionTok{return}\NormalTok{(h }\SpecialCharTok{\%*\%}\NormalTok{ evec }\SpecialCharTok{+}\NormalTok{ x }\SpecialCharTok{\%*\%}\NormalTok{ evec }\SpecialCharTok{\%*\%}\NormalTok{ s)}
\NormalTok{\}}

\NormalTok{PCAJacobianFormula }\OtherTok{\textless{}{-}} \ControlFlowTok{function}\NormalTok{(x) \{}
\NormalTok{  nobj }\OtherTok{\textless{}{-}} \FunctionTok{nrow}\NormalTok{(x)}
\NormalTok{  ndim }\OtherTok{\textless{}{-}} \FunctionTok{ncol}\NormalTok{(x)}
\NormalTok{  np }\OtherTok{\textless{}{-}}\NormalTok{ nobj }\SpecialCharTok{*}\NormalTok{ ndim}
\NormalTok{  e }\OtherTok{\textless{}{-}} \ControlFlowTok{function}\NormalTok{(i, n) \{}
    \FunctionTok{ifelse}\NormalTok{(i }\SpecialCharTok{==} \DecValTok{1}\SpecialCharTok{:}\NormalTok{n, }\DecValTok{1}\NormalTok{, }\DecValTok{0}\NormalTok{)}
\NormalTok{  \}}
\NormalTok{  d }\OtherTok{\textless{}{-}} \FunctionTok{matrix}\NormalTok{(}\DecValTok{0}\NormalTok{, np, np)}
\NormalTok{  k }\OtherTok{\textless{}{-}} \DecValTok{1}
  \ControlFlowTok{for}\NormalTok{ (j }\ControlFlowTok{in} \DecValTok{1}\SpecialCharTok{:}\NormalTok{ndim) \{}
    \ControlFlowTok{for}\NormalTok{ (i }\ControlFlowTok{in} \DecValTok{1}\SpecialCharTok{:}\NormalTok{nobj) \{}
\NormalTok{      h }\OtherTok{\textless{}{-}} \FunctionTok{outer}\NormalTok{(}\FunctionTok{e}\NormalTok{(i, nobj), }\FunctionTok{e}\NormalTok{(j, ndim))}
\NormalTok{      r }\OtherTok{\textless{}{-}} \FunctionTok{PCADerivative}\NormalTok{(x, h)}
\NormalTok{      d[, k] }\OtherTok{\textless{}{-}} \FunctionTok{as.vector}\NormalTok{(r)}
\NormalTok{      k }\OtherTok{\textless{}{-}}\NormalTok{ k }\SpecialCharTok{+} \DecValTok{1}
\NormalTok{    \}}
\NormalTok{  \}}
  \FunctionTok{return}\NormalTok{(d)}
\NormalTok{\}}

\NormalTok{PCAJacobianNumerical }\OtherTok{\textless{}{-}} \ControlFlowTok{function}\NormalTok{(x) \{}
\NormalTok{  nobj }\OtherTok{\textless{}{-}} \FunctionTok{nrow}\NormalTok{(x)}
\NormalTok{  ndim }\OtherTok{\textless{}{-}} \FunctionTok{ncol}\NormalTok{(x)}
\NormalTok{  func }\OtherTok{\textless{}{-}} \ControlFlowTok{function}\NormalTok{(x, nobj, ndim) \{}
\NormalTok{    x }\OtherTok{\textless{}{-}} \FunctionTok{matrix}\NormalTok{(x, nobj, ndim)}
\NormalTok{    l }\OtherTok{\textless{}{-}} \FunctionTok{svd}\NormalTok{(x)}\SpecialCharTok{$}\NormalTok{v}
\NormalTok{    l }\OtherTok{\textless{}{-}}\NormalTok{ l }\SpecialCharTok{\%*\%} \FunctionTok{diag}\NormalTok{(}\FunctionTok{sign}\NormalTok{(}\FunctionTok{diag}\NormalTok{(l)))}
    \FunctionTok{return}\NormalTok{(x }\SpecialCharTok{\%*\%}\NormalTok{ l)}
\NormalTok{  \}}
\NormalTok{  jacob }\OtherTok{\textless{}{-}} \FunctionTok{jacobian}\NormalTok{(func, }\FunctionTok{as.vector}\NormalTok{(x), }\AttributeTok{nobj =}\NormalTok{ nobj, }\AttributeTok{ndim =}\NormalTok{ ndim)}
  \FunctionTok{return}\NormalTok{(jacob)}
\NormalTok{\}}

\NormalTok{PCATester }\OtherTok{\textless{}{-}} \ControlFlowTok{function}\NormalTok{(x, h) \{}
\NormalTok{  func }\OtherTok{\textless{}{-}} \ControlFlowTok{function}\NormalTok{(x) \{}
\NormalTok{    l }\OtherTok{\textless{}{-}} \FunctionTok{svd}\NormalTok{(x)}\SpecialCharTok{$}\NormalTok{v}
\NormalTok{    l }\OtherTok{\textless{}{-}}\NormalTok{ l }\SpecialCharTok{\%*\%} \FunctionTok{diag}\NormalTok{(}\FunctionTok{sign}\NormalTok{(}\FunctionTok{diag}\NormalTok{(l)))}
    \FunctionTok{return}\NormalTok{(x }\SpecialCharTok{\%*\%}\NormalTok{ l)}
\NormalTok{  \}}
\NormalTok{  x0 }\OtherTok{\textless{}{-}} \FunctionTok{func}\NormalTok{(x)}
\NormalTok{  xh }\OtherTok{\textless{}{-}} \FunctionTok{func}\NormalTok{(x }\SpecialCharTok{+}\NormalTok{ h)}
\NormalTok{  xd }\OtherTok{\textless{}{-}}\NormalTok{ x0 }\SpecialCharTok{+} \FunctionTok{PCADerivative}\NormalTok{(x, h)}
  \FunctionTok{print}\NormalTok{(}\FunctionTok{cbind}\NormalTok{(x0, xh, xd))}
\NormalTok{\}}
\end{Highlighting}
\end{Shaded}

\subsection{smacofQRDerivative.R}\label{smacofqrderivative.r}

\begin{Shaded}
\begin{Highlighting}[]
\NormalTok{QRDerivative }\OtherTok{\textless{}{-}} \ControlFlowTok{function}\NormalTok{(x, h) \{}
\NormalTok{  ndim }\OtherTok{\textless{}{-}} \FunctionTok{ncol}\NormalTok{(x)}
\NormalTok{  z }\OtherTok{\textless{}{-}} \FunctionTok{t}\NormalTok{(x[}\DecValTok{1}\SpecialCharTok{:}\NormalTok{ndim, ])}
\NormalTok{  g }\OtherTok{\textless{}{-}} \FunctionTok{t}\NormalTok{(h[}\DecValTok{1}\SpecialCharTok{:}\NormalTok{ndim, ])}
\NormalTok{  qq }\OtherTok{\textless{}{-}} \FunctionTok{qr}\NormalTok{(z)}
\NormalTok{  q }\OtherTok{\textless{}{-}} \FunctionTok{qr.Q}\NormalTok{(qq)}
\NormalTok{  r }\OtherTok{\textless{}{-}} \FunctionTok{qr.R}\NormalTok{(qq)}
\NormalTok{  b }\OtherTok{\textless{}{-}} \FunctionTok{crossprod}\NormalTok{(q, g }\SpecialCharTok{\%*\%} \FunctionTok{solve}\NormalTok{(r))}
\NormalTok{  a }\OtherTok{\textless{}{-}} \FunctionTok{matrix}\NormalTok{(}\DecValTok{0}\NormalTok{, ndim, ndim)}
\NormalTok{  i }\OtherTok{\textless{}{-}} \FunctionTok{outer}\NormalTok{(}\DecValTok{1}\SpecialCharTok{:}\NormalTok{ndim, }\DecValTok{1}\SpecialCharTok{:}\NormalTok{ndim, }\StringTok{"\textgreater{}"}\NormalTok{)}
\NormalTok{  a }\OtherTok{\textless{}{-}} \FunctionTok{ifelse}\NormalTok{(i, b, }\DecValTok{0}\NormalTok{)}
\NormalTok{  a }\OtherTok{\textless{}{-}}\NormalTok{ a }\SpecialCharTok{{-}} \FunctionTok{t}\NormalTok{(a)}
\NormalTok{  deri }\OtherTok{\textless{}{-}}\NormalTok{ h }\SpecialCharTok{\%*\%}\NormalTok{ q }\SpecialCharTok{+}\NormalTok{ x }\SpecialCharTok{\%*\%}\NormalTok{ q }\SpecialCharTok{\%*\%}\NormalTok{ a}
  \FunctionTok{return}\NormalTok{(deri)}
\NormalTok{\}}

\NormalTok{QRJacobianFormula }\OtherTok{\textless{}{-}} \ControlFlowTok{function}\NormalTok{(x) \{}
\NormalTok{  nobj }\OtherTok{\textless{}{-}} \FunctionTok{nrow}\NormalTok{(x)}
\NormalTok{  ndim }\OtherTok{\textless{}{-}} \FunctionTok{ncol}\NormalTok{(x)}
\NormalTok{  np }\OtherTok{\textless{}{-}}\NormalTok{ nobj }\SpecialCharTok{*}\NormalTok{ ndim}
\NormalTok{  e }\OtherTok{\textless{}{-}} \ControlFlowTok{function}\NormalTok{(i, n) \{}
    \FunctionTok{ifelse}\NormalTok{(i }\SpecialCharTok{==} \DecValTok{1}\SpecialCharTok{:}\NormalTok{n, }\DecValTok{1}\NormalTok{, }\DecValTok{0}\NormalTok{)}
\NormalTok{  \}}
\NormalTok{  d }\OtherTok{\textless{}{-}} \FunctionTok{matrix}\NormalTok{(}\DecValTok{0}\NormalTok{, np, np)}
\NormalTok{  k }\OtherTok{\textless{}{-}} \DecValTok{1}
  \ControlFlowTok{for}\NormalTok{ (j }\ControlFlowTok{in} \DecValTok{1}\SpecialCharTok{:}\NormalTok{ndim) \{}
    \ControlFlowTok{for}\NormalTok{ (i }\ControlFlowTok{in} \DecValTok{1}\SpecialCharTok{:}\NormalTok{nobj) \{}
\NormalTok{      h }\OtherTok{\textless{}{-}} \FunctionTok{outer}\NormalTok{(}\FunctionTok{e}\NormalTok{(i, nobj), }\FunctionTok{e}\NormalTok{(j, ndim))}
\NormalTok{      r }\OtherTok{\textless{}{-}} \FunctionTok{QRDerivative}\NormalTok{(x, h)}
\NormalTok{      d[, k] }\OtherTok{\textless{}{-}} \FunctionTok{as.vector}\NormalTok{(r)}
\NormalTok{      k }\OtherTok{\textless{}{-}}\NormalTok{ k }\SpecialCharTok{+} \DecValTok{1}
\NormalTok{    \}}
\NormalTok{  \}}
  \FunctionTok{return}\NormalTok{(d)}
\NormalTok{\}}

\NormalTok{QRJacobianNumerical }\OtherTok{\textless{}{-}} \ControlFlowTok{function}\NormalTok{(x) \{}
\NormalTok{  nobj }\OtherTok{\textless{}{-}} \FunctionTok{nrow}\NormalTok{(x)}
\NormalTok{  ndim }\OtherTok{\textless{}{-}} \FunctionTok{ncol}\NormalTok{(x)}
\NormalTok{  func }\OtherTok{\textless{}{-}} \ControlFlowTok{function}\NormalTok{(x, nobj, ndim) \{}
\NormalTok{    x }\OtherTok{\textless{}{-}} \FunctionTok{matrix}\NormalTok{(x, nobj, ndim)}
\NormalTok{    l }\OtherTok{\textless{}{-}} \FunctionTok{qr.Q}\NormalTok{(}\FunctionTok{qr}\NormalTok{(}\FunctionTok{t}\NormalTok{(x[}\DecValTok{1}\SpecialCharTok{:}\NormalTok{ndim, ])))}
    \FunctionTok{return}\NormalTok{(x }\SpecialCharTok{\%*\%}\NormalTok{ l)}
\NormalTok{  \}}
\NormalTok{  jacob }\OtherTok{\textless{}{-}} \FunctionTok{jacobian}\NormalTok{(func, }\FunctionTok{as.vector}\NormalTok{(x), }\AttributeTok{nobj =}\NormalTok{ nobj, }\AttributeTok{ndim =}\NormalTok{ ndim)}
  \FunctionTok{return}\NormalTok{(jacob)}
\NormalTok{\}}

\NormalTok{QRTester }\OtherTok{\textless{}{-}} \ControlFlowTok{function}\NormalTok{(x, h) \{}
\NormalTok{  func }\OtherTok{\textless{}{-}} \ControlFlowTok{function}\NormalTok{(x) \{}
\NormalTok{    ndim }\OtherTok{\textless{}{-}} \FunctionTok{ncol}\NormalTok{(x)}
\NormalTok{    l }\OtherTok{\textless{}{-}} \FunctionTok{qr.Q}\NormalTok{(}\FunctionTok{qr}\NormalTok{(}\FunctionTok{t}\NormalTok{(x[}\DecValTok{1}\SpecialCharTok{:}\NormalTok{ndim, ])))}
\NormalTok{    l }\OtherTok{\textless{}{-}}\NormalTok{ l }\SpecialCharTok{\%*\%} \FunctionTok{diag}\NormalTok{(}\FunctionTok{sign}\NormalTok{(}\FunctionTok{diag}\NormalTok{(l)))}
    \FunctionTok{return}\NormalTok{(x }\SpecialCharTok{\%*\%}\NormalTok{ l)}
\NormalTok{  \}}
\NormalTok{  x0 }\OtherTok{\textless{}{-}} \FunctionTok{func}\NormalTok{(x)}
\NormalTok{  xh }\OtherTok{\textless{}{-}} \FunctionTok{func}\NormalTok{(x }\SpecialCharTok{+}\NormalTok{ h)}
\NormalTok{  xd }\OtherTok{\textless{}{-}}\NormalTok{ x0 }\SpecialCharTok{+} \FunctionTok{QRDerivative}\NormalTok{(x, h)}
  \FunctionTok{print}\NormalTok{(}\FunctionTok{cbind}\NormalTok{(x0, xh, xd))}
\NormalTok{\}}
\end{Highlighting}
\end{Shaded}

\subsection{smacofCompare.R}\label{smacofcompare.r}

\begin{Shaded}
\begin{Highlighting}[]
\NormalTok{smacofCompare }\OtherTok{\textless{}{-}} \ControlFlowTok{function}\NormalTok{(delta, }\AttributeTok{ndim =} \DecValTok{2}\NormalTok{) \{}
\NormalTok{  nobj }\OtherTok{\textless{}{-}} \FunctionTok{nrow}\NormalTok{(delta)}
\NormalTok{  wgth }\OtherTok{\textless{}{-}} \DecValTok{1} \SpecialCharTok{{-}} \FunctionTok{diag}\NormalTok{(nobj)}
\NormalTok{  xold }\OtherTok{\textless{}{-}} \FunctionTok{smacofTorgerson}\NormalTok{(delta, ndim)}
  \FunctionTok{return}\NormalTok{(}
    \FunctionTok{microbenchmark}\NormalTok{(}
      \FunctionTok{smacofAccelerate}\NormalTok{(}
\NormalTok{        delta,}
        \AttributeTok{xold =}\NormalTok{ xold,}
        \AttributeTok{ndim =} \DecValTok{2}\NormalTok{,}
        \AttributeTok{opt =} \DecValTok{1}\NormalTok{,}
        \AttributeTok{halt =} \DecValTok{2}\NormalTok{,}
        \AttributeTok{verbose =} \ConstantTok{FALSE}
\NormalTok{      ),}
      \FunctionTok{smacofAccelerate}\NormalTok{(}
\NormalTok{        delta,}
        \AttributeTok{xold =}\NormalTok{ xold,}
        \AttributeTok{ndim =} \DecValTok{2}\NormalTok{,}
        \AttributeTok{opt =} \DecValTok{2}\NormalTok{,}
        \AttributeTok{halt =} \DecValTok{2}\NormalTok{,}
        \AttributeTok{verbose =} \ConstantTok{FALSE}
\NormalTok{      ),}
      \FunctionTok{smacofAccelerate}\NormalTok{(}
\NormalTok{        delta,}
        \AttributeTok{xold =}\NormalTok{ xold,}
        \AttributeTok{ndim =} \DecValTok{2}\NormalTok{,}
        \AttributeTok{opt =} \DecValTok{3}\NormalTok{,}
        \AttributeTok{halt =} \DecValTok{2}\NormalTok{,}
        \AttributeTok{verbose =} \ConstantTok{FALSE}
\NormalTok{      ),}
      \FunctionTok{smacofAccelerate}\NormalTok{(}
\NormalTok{        delta,}
        \AttributeTok{xold =}\NormalTok{ xold,}
        \AttributeTok{ndim =} \DecValTok{2}\NormalTok{,}
        \AttributeTok{opt =} \DecValTok{4}\NormalTok{,}
        \AttributeTok{halt =} \DecValTok{2}\NormalTok{,}
        \AttributeTok{verbose =} \ConstantTok{FALSE}
\NormalTok{      ),}
      \FunctionTok{smacofAccelerate}\NormalTok{(}
\NormalTok{        delta,}
        \AttributeTok{ndim =} \DecValTok{2}\NormalTok{,}
        \AttributeTok{opt =} \DecValTok{5}\NormalTok{,}
        \AttributeTok{halt =} \DecValTok{2}\NormalTok{,}
        \AttributeTok{verbose =} \ConstantTok{FALSE}
\NormalTok{      ),}
      \FunctionTok{smacofAccelerate}\NormalTok{(}
\NormalTok{        delta,}
        \AttributeTok{xold =}\NormalTok{ xold,}
        \AttributeTok{ndim =} \DecValTok{2}\NormalTok{,}
        \AttributeTok{opt =} \DecValTok{6}\NormalTok{,}
        \AttributeTok{halt =} \DecValTok{2}\NormalTok{,}
        \AttributeTok{verbose =} \ConstantTok{FALSE}
\NormalTok{      ),}
      \FunctionTok{smacofAccelerate}\NormalTok{(}
\NormalTok{        delta,}
        \AttributeTok{ndim =} \DecValTok{2}\NormalTok{,}
        \AttributeTok{xold =}\NormalTok{ xold,}
        \AttributeTok{opt =} \DecValTok{7}\NormalTok{,}
        \AttributeTok{halt =} \DecValTok{2}\NormalTok{,}
        \AttributeTok{verbose =} \ConstantTok{FALSE}
\NormalTok{      ),}
      \FunctionTok{smacofAccelerate}\NormalTok{(}
\NormalTok{        delta,}
        \AttributeTok{ndim =} \DecValTok{2}\NormalTok{,}
        \AttributeTok{xold =}\NormalTok{ xold,}
        \AttributeTok{opt =} \DecValTok{8}\NormalTok{,}
        \AttributeTok{halt =} \DecValTok{2}\NormalTok{,}
        \AttributeTok{verbose =} \ConstantTok{FALSE}
\NormalTok{      )}
\NormalTok{    )}
\NormalTok{  )}
\NormalTok{\}}
\end{Highlighting}
\end{Shaded}

\subsection{smacofUtils.R}\label{smacofutils.r}

\begin{Shaded}
\begin{Highlighting}[]
\NormalTok{smacofMatrixPrint }\OtherTok{\textless{}{-}} \ControlFlowTok{function}\NormalTok{(x,}
                   \AttributeTok{digits =} \DecValTok{10}\NormalTok{,}
                   \AttributeTok{width =} \DecValTok{15}\NormalTok{,}
                   \AttributeTok{format =} \StringTok{"f"}\NormalTok{,}
                   \AttributeTok{flag =} \StringTok{"+"}\NormalTok{) \{}
  \FunctionTok{print}\NormalTok{(}\FunctionTok{noquote}\NormalTok{(}
    \FunctionTok{formatC}\NormalTok{(}
\NormalTok{      x,}
      \AttributeTok{digits =}\NormalTok{ digits,}
      \AttributeTok{width =}\NormalTok{ width,}
      \AttributeTok{format =}\NormalTok{ format,}
      \AttributeTok{flag =}\NormalTok{ flag}
\NormalTok{    )}
\NormalTok{  ))}
\NormalTok{\}}

\NormalTok{smacofLinePrint }\OtherTok{\textless{}{-}} \ControlFlowTok{function}\NormalTok{(itel, sold, snew, cnew, labd, wd, dg) \{}
  \FunctionTok{cat}\NormalTok{(}
    \StringTok{"itel"}\NormalTok{,}
    \FunctionTok{formatC}\NormalTok{(itel, }\AttributeTok{width =}\NormalTok{ wd, }\AttributeTok{format =} \StringTok{"d"}\NormalTok{),}
    \StringTok{"sold"}\NormalTok{,}
    \FunctionTok{formatC}\NormalTok{(sold, }\AttributeTok{digits =}\NormalTok{ dg, }\AttributeTok{format =} \StringTok{"f"}\NormalTok{),}
    \StringTok{"snew"}\NormalTok{,}
    \FunctionTok{formatC}\NormalTok{(snew, }\AttributeTok{digits =}\NormalTok{ dg, }\AttributeTok{format =} \StringTok{"f"}\NormalTok{),}
    \StringTok{"chng"}\NormalTok{,}
    \FunctionTok{formatC}\NormalTok{(cnew, }\AttributeTok{digits =}\NormalTok{  dg, }\AttributeTok{format =} \StringTok{"f"}\NormalTok{),}
    \StringTok{"labd"}\NormalTok{,}
    \FunctionTok{formatC}\NormalTok{(labd, }\AttributeTok{digits =}\NormalTok{  dg, }\AttributeTok{format =} \StringTok{"f"}\NormalTok{),}
    \StringTok{"}\SpecialCharTok{\textbackslash{}n}\StringTok{"}
\NormalTok{  )}
\NormalTok{\}}

\NormalTok{smacofMakeBasis }\OtherTok{\textless{}{-}} \ControlFlowTok{function}\NormalTok{(n, ndim, vmat) \{}
\NormalTok{  y }\OtherTok{\textless{}{-}} \FunctionTok{lapply}\NormalTok{(}\DecValTok{1}\SpecialCharTok{:}\NormalTok{ndim, }\ControlFlowTok{function}\NormalTok{(k)}
    \FunctionTok{matrix}\NormalTok{(}\DecValTok{0}\NormalTok{, n, n }\SpecialCharTok{{-}}\NormalTok{ k))}
  \ControlFlowTok{for}\NormalTok{ (s }\ControlFlowTok{in} \DecValTok{0}\SpecialCharTok{:}\NormalTok{(ndim }\SpecialCharTok{{-}} \DecValTok{1}\NormalTok{)) \{}
\NormalTok{    ns }\OtherTok{\textless{}{-}}\NormalTok{ n }\SpecialCharTok{{-}}\NormalTok{ s}
\NormalTok{    aux }\OtherTok{\textless{}{-}} \FunctionTok{qr.Q}\NormalTok{(}\FunctionTok{qr}\NormalTok{(ns }\SpecialCharTok{*} \FunctionTok{diag}\NormalTok{(ns) }\SpecialCharTok{{-}} \DecValTok{1}\NormalTok{))[, }\SpecialCharTok{{-}}\NormalTok{ns]}
\NormalTok{    aux }\OtherTok{\textless{}{-}} \FunctionTok{rbind}\NormalTok{(}\FunctionTok{matrix}\NormalTok{(}\DecValTok{0}\NormalTok{, s, ns }\SpecialCharTok{{-}} \DecValTok{1}\NormalTok{), aux)}
\NormalTok{    sux }\OtherTok{\textless{}{-}} \FunctionTok{crossprod}\NormalTok{(aux, vmat }\SpecialCharTok{\%*\%}\NormalTok{ aux)}
\NormalTok{    y[[s }\SpecialCharTok{+} \DecValTok{1}\NormalTok{]] }\OtherTok{\textless{}{-}}\NormalTok{ aux }\SpecialCharTok{\%*\%} \FunctionTok{smacofMatrixPower}\NormalTok{(sux, }\SpecialCharTok{{-}}\FloatTok{0.5}\NormalTok{)}
\NormalTok{  \}}
  \FunctionTok{return}\NormalTok{(y)}
\NormalTok{\}}

\NormalTok{smacofMakeBisas }\OtherTok{\textless{}{-}} \ControlFlowTok{function}\NormalTok{(n, ndim, vmat) \{}
\NormalTok{  y }\OtherTok{\textless{}{-}} \FunctionTok{rep}\NormalTok{(}\FunctionTok{list}\NormalTok{(}\FunctionTok{matrix}\NormalTok{(}\DecValTok{0}\NormalTok{, n, n }\SpecialCharTok{{-}} \DecValTok{1}\NormalTok{)), ndim)}
  \ControlFlowTok{for}\NormalTok{ (s }\ControlFlowTok{in} \DecValTok{0}\SpecialCharTok{:}\NormalTok{(ndim }\SpecialCharTok{{-}} \DecValTok{1}\NormalTok{)) \{}
\NormalTok{    ns }\OtherTok{\textless{}{-}}\NormalTok{ n}
\NormalTok{    aux }\OtherTok{\textless{}{-}} \FunctionTok{qr.Q}\NormalTok{(}\FunctionTok{qr}\NormalTok{(n }\SpecialCharTok{*} \FunctionTok{diag}\NormalTok{(n) }\SpecialCharTok{{-}} \DecValTok{1}\NormalTok{))[, }\SpecialCharTok{{-}}\NormalTok{n]}
\NormalTok{    sux }\OtherTok{\textless{}{-}} \FunctionTok{crossprod}\NormalTok{(aux, vmat }\SpecialCharTok{\%*\%}\NormalTok{ aux)}
\NormalTok{    y[[s }\SpecialCharTok{+} \DecValTok{1}\NormalTok{]] }\OtherTok{\textless{}{-}}\NormalTok{ aux }\SpecialCharTok{\%*\%} \FunctionTok{smacofMatrixPower}\NormalTok{(sux, }\SpecialCharTok{{-}}\FloatTok{0.5}\NormalTok{)}
\NormalTok{  \}}
  \FunctionTok{return}\NormalTok{(y)}
\NormalTok{\}}

\NormalTok{smacofTorgerson }\OtherTok{\textless{}{-}} \ControlFlowTok{function}\NormalTok{(delta, ndim) \{}
\NormalTok{  n }\OtherTok{\textless{}{-}} \FunctionTok{nrow}\NormalTok{(delta)}
\NormalTok{  dd }\OtherTok{\textless{}{-}}\NormalTok{ delta }\SpecialCharTok{\^{}} \DecValTok{2}
\NormalTok{  rd }\OtherTok{\textless{}{-}} \FunctionTok{rowSums}\NormalTok{(dd) }\SpecialCharTok{/}\NormalTok{ n}
\NormalTok{  sd }\OtherTok{\textless{}{-}} \FunctionTok{mean}\NormalTok{(dd)}
\NormalTok{  cc }\OtherTok{\textless{}{-}} \SpecialCharTok{{-}}\NormalTok{.}\DecValTok{5} \SpecialCharTok{*}\NormalTok{ (dd }\SpecialCharTok{{-}} \FunctionTok{outer}\NormalTok{(rd, rd, }\StringTok{"+"}\NormalTok{) }\SpecialCharTok{+}\NormalTok{ sd)}
\NormalTok{  ee }\OtherTok{\textless{}{-}} \FunctionTok{eigen}\NormalTok{(cc)}
\NormalTok{  x }\OtherTok{\textless{}{-}}\NormalTok{ ee}\SpecialCharTok{$}\NormalTok{vectors[, }\DecValTok{1}\SpecialCharTok{:}\NormalTok{ndim] }\SpecialCharTok{\%*\%} \FunctionTok{diag}\NormalTok{(}\FunctionTok{sqrt}\NormalTok{(ee}\SpecialCharTok{$}\NormalTok{values[}\DecValTok{1}\SpecialCharTok{:}\NormalTok{ndim]))}
  \FunctionTok{return}\NormalTok{(x)}
\NormalTok{\}}

\NormalTok{smacofGuttman }\OtherTok{\textless{}{-}} \ControlFlowTok{function}\NormalTok{(x, delta, wgth, vinv) \{}
\NormalTok{  nobj }\OtherTok{\textless{}{-}} \FunctionTok{nrow}\NormalTok{(x)}
\NormalTok{  dmat }\OtherTok{\textless{}{-}} \FunctionTok{as.matrix}\NormalTok{(}\FunctionTok{dist}\NormalTok{(x))}
\NormalTok{  bmat }\OtherTok{\textless{}{-}} \SpecialCharTok{{-}}\NormalTok{wgth }\SpecialCharTok{*}\NormalTok{ delta }\SpecialCharTok{/}\NormalTok{ (dmat }\SpecialCharTok{+} \FunctionTok{diag}\NormalTok{(nobj))}
  \FunctionTok{diag}\NormalTok{(bmat) }\OtherTok{\textless{}{-}} \SpecialCharTok{{-}}\FunctionTok{rowSums}\NormalTok{(bmat)}
  \FunctionTok{return}\NormalTok{(vinv }\SpecialCharTok{\%*\%}\NormalTok{ bmat }\SpecialCharTok{\%*\%}\NormalTok{ x)}
\NormalTok{\}}

\NormalTok{smacofCenter }\OtherTok{\textless{}{-}} \ControlFlowTok{function}\NormalTok{(x) \{}
  \FunctionTok{return}\NormalTok{(}\FunctionTok{apply}\NormalTok{(x, }\DecValTok{2}\NormalTok{, }\ControlFlowTok{function}\NormalTok{(x)}
\NormalTok{    x }\SpecialCharTok{{-}} \FunctionTok{mean}\NormalTok{(x)))}
\NormalTok{\}}

\NormalTok{smacofSignEigenVectors }\OtherTok{\textless{}{-}} \ControlFlowTok{function}\NormalTok{(x) \{}
  \FunctionTok{return}\NormalTok{(x }\SpecialCharTok{\%*\%} \FunctionTok{diag}\NormalTok{(}\FunctionTok{sign}\NormalTok{(}\FunctionTok{diag}\NormalTok{(x))))}
\NormalTok{\}}

\NormalTok{smacofMatrixPower }\OtherTok{\textless{}{-}} \ControlFlowTok{function}\NormalTok{(s, power) \{}
\NormalTok{  e }\OtherTok{\textless{}{-}} \FunctionTok{eigen}\NormalTok{(s)}
\NormalTok{  eval }\OtherTok{\textless{}{-}}\NormalTok{ e}\SpecialCharTok{$}\NormalTok{values}
\NormalTok{  evec }\OtherTok{\textless{}{-}}\NormalTok{ e}\SpecialCharTok{$}\NormalTok{vectors}
\NormalTok{  dval }\OtherTok{\textless{}{-}} \FunctionTok{ifelse}\NormalTok{(}\FunctionTok{abs}\NormalTok{(eval) }\SpecialCharTok{\textless{}} \FloatTok{1e{-}10}\NormalTok{, }\DecValTok{0}\NormalTok{, }\FunctionTok{abs}\NormalTok{(eval) }\SpecialCharTok{\^{}}\NormalTok{ power)}
  \FunctionTok{return}\NormalTok{(}\FunctionTok{tcrossprod}\NormalTok{(evec }\SpecialCharTok{\%*\%} \FunctionTok{diag}\NormalTok{(dval), evec))}
\NormalTok{\}}
\end{Highlighting}
\end{Shaded}

\section*{References}\label{references}
\addcontentsline{toc}{section}{References}

\phantomsection\label{refs}
\begin{CSLReferences}{1}{0}
\bibitem[\citeproctext]{ref-bolte_daniilidis_lewis_07}
Bolte, J., A Daniilidis, and A. Lewis. 2007. {``The {Ł}ojasiewicz
Inequality for Nonsmooth Subanalytic Functions with Applications to
Subgradient Dynamical Systems.''} \emph{SIAM Journal of Optimization} 17
(4): 1205--23.

\bibitem[\citeproctext]{ref-bolte_sabach_teboulle_14}
Bolte, J., S. Sabach, and M. Teboulle. 2014. {``Proximal Alternating
Linearized Minimization for Nonconvex and Nonsmooth Problems.''}
\emph{Mathematical Programming} 146: 459--94.

\bibitem[\citeproctext]{ref-clarke_75}
Clarke, F. H. 1975. {``Generalized Gradients and Applications.''}
\emph{Transactions of the American Mathematical Society} 205: 247--62.

\bibitem[\citeproctext]{ref-degruijter_67}
De Gruijter, D. N. M. 1967. {``{The Cognitive Structure of Dutch
Political Parties in 1966}.''} Report E019-67. Psychological Institute,
University of Leiden.

\bibitem[\citeproctext]{ref-deleeuw_C_77}
De Leeuw, J. 1977. {``Applications of Convex Analysis to
Multidimensional Scaling.''} In \emph{Recent Developments in
Statistics}, edited by J. R. Barra, F. Brodeau, G. Romier, and B. Van
Cutsem, 133--45. Amsterdam, The Netherlands: North Holland Publishing
Company.

\bibitem[\citeproctext]{ref-deleeuw_A_84f}
---------. 1984. {``{Differentiability of Kruskal's Stress at a Local
Minimum}.''} \emph{Psychometrika} 49: 111--13.

\bibitem[\citeproctext]{ref-deleeuw_A_88b}
---------. 1988. {``Convergence of the Majorization Method for
Multidimensional Scaling.''} \emph{Journal of Classification} 5:
163--80.

\bibitem[\citeproctext]{ref-deleeuw_R_93c}
---------. 1993. {``Fitting Distances by Least Squares.''} Preprint
Series 130. Los Angeles, CA: UCLA Department of Statistics.
\url{https://jansweb.netlify.app/publication/deleeuw-r-93-c/deleeuw-r-93-c.pdf}.

\bibitem[\citeproctext]{ref-deleeuw_C_94c}
---------. 1994. {``{Block Relaxation Algorithms in Statistics}.''} In
\emph{Information Systems and Data Analysis}, edited by H. H. Bock, W.
Lenski, and M. M. Richter, 308--24. Berlin: Springer Verlag.
\url{https://jansweb.netlify.app/publication/deleeuw-c-94-c/deleeuw-c-94-c.pdf}.

\bibitem[\citeproctext]{ref-deleeuw_R_06b}
---------. 2006. {``{Accelerated Least Squares Multidimensional
Scaling}.''} Preprint Series 493. Los Angeles, CA: UCLA Department of
Statistics.
\url{https://jansweb.netlify.app/publication/deleeuw-r-06-b/deleeuw-r-06-b.pdf}.

\bibitem[\citeproctext]{ref-deleeuw_E_19h}
---------. 2019. {``{Convergence of SMACOF}.''} 2019.
\url{https://jansweb.netlify.app/publication/deleeuw-e-19-h/deleeuw-e-19-h.pdf}.

\bibitem[\citeproctext]{ref-deleeuw_B_21}
---------. 2021. \emph{Least Squares Euclidean Multidimensional
Scaling}.
\url{https://jansweb.netlify.app/publication/deleeuw-b-21-a/deleeuw-b-21-a.pdf}.

\bibitem[\citeproctext]{ref-deleeuw_E_23a}
---------. 2023. {``Differentiating the QR Decomposition.''} 2023.
\url{https://jansweb.netlify.app/publication/deleeuw-e-23-a/deleeuw-e-23-a.pdf}.

\bibitem[\citeproctext]{ref-deleeuw_heiser_C_77}
De Leeuw, J., and W. J. Heiser. 1977. {``Convergence of Correction
Matrix Algorithms for Multidimensional Scaling.''} In \emph{Geometric
Representations of Relational Data}, edited by J. C. Lingoes, 735--53.
Ann Arbor, Michigan: Mathesis Press.

\bibitem[\citeproctext]{ref-deleeuw_heiser_C_80}
---------. 1980. {``Multidimensional Scaling with Restrictions on the
Configuration.''} In \emph{Multivariate Analysis, Volume {V}}, edited by
P. R. Krishnaiah, 501--22. Amsterdam, The Netherlands: North Holland
Publishing Company.

\bibitem[\citeproctext]{ref-ekman_54}
Ekman, G. 1954. {``{Dimensions of Color Vision}.''} \emph{Journal of
Psychology} 38: 467--74.

\bibitem[\citeproctext]{ref-gilbert_varadhan_19}
Gilbert, P., and R. Varadhan. 2019. \emph{{numDeriv: Accurate Numerical
Derivatives}}. \url{https://CRAN.R-project.org/package=numDeriv}.

\bibitem[\citeproctext]{ref-groenen_glunt_hayden_96}
Groenen, P. J. F., W. Glunt, and T. L. Hayden. 1996. {``Fast Algorithms
for Multidimensional Scaling: A Comparison of Majorization and Spectral
Gradient Methods.''} Department of Data Theory, University of Leiden.

\bibitem[\citeproctext]{ref-guttman_68}
Guttman, L. 1968. {``{A General Nonmetric Technique for Fitting the
Smallest Coordinate Space for a Configuration of Points}.''}
\emph{Psychometrika} 33: 469--506.

\bibitem[\citeproctext]{ref-kruskal_64a}
Kruskal, J. B. 1964a. {``{Multidimensional Scaling by Optimizing
Goodness of Fit to a Nonmetric Hypothesis}.''} \emph{Psychometrika} 29:
1--27.

\bibitem[\citeproctext]{ref-kruskal_64b}
---------. 1964b. {``{Nonmetric Multidimensional Scaling: a Numerical
Method}.''} \emph{Psychometrika} 29: 115--29.

\bibitem[\citeproctext]{ref-lange_16}
Lange, K. 2016. \emph{MM Optimization Algorithms}. SIAM.

\bibitem[\citeproctext]{ref-lethi_tao_01}
Le Thi, H. A., and P. D. Tao. 2001. {``D.c. Programming Approach to the
Multidimensional Scaling Problem.''} In \emph{From Local to Global
Optimization}, edited by A. Migdalas, P. M. Pardalos, and P. Värbrand,
231--76. Springer Verlag.

\bibitem[\citeproctext]{ref-lethi_tao_18}
---------. 2018. {``{DC Programming and DCA: Thirty Years of
Developments}.''} \emph{Mathematical Programming, Series B}.

\bibitem[\citeproctext]{ref-lethi_tao_24}
---------. 2024. {``Open Issues and Recent Advances in DC Programming
and DCA.''} \emph{Journal of Global Optimization} 88: 533--90.

\bibitem[\citeproctext]{ref-mair_deleeuw_C_15}
Mair, P., and J. De Leeuw. 2015. {``{Unidimensional Scaling}.''} In
\emph{Wiley {StatsRef}: Statistics Reference Online}, 1--3. Wiley.

\bibitem[\citeproctext]{ref-mair_deleeuw_wurzer_C_15}
Mair, P., J. De Leeuw, and M. Wurzer. 2015. {``{Multidimensional
Unfolding}.''} In \emph{Wiley {StatsRef}: Statistics Reference Online},
1--4. Wiley.

\bibitem[\citeproctext]{ref-mersmann_23}
Mersmann, O. 2023. \emph{{microbenchmark: Accurate Timing Functions}}.
\url{https://CRAN.R-project.org/package=microbenchmark}.

\bibitem[\citeproctext]{ref-meyer_76}
Meyer, R. R. 1976. {``{Sufficient Conditions for the Convergence of
Monotonic Mathematical Programming Algorithms}.''} \emph{Journal of
Computer and System Sciences} 12: 108--21.

\bibitem[\citeproctext]{ref-ortega_rheinboldt_70}
Ortega, J. M., and W. C. Rheinboldt. 1970. \emph{{Iterative Solution of
Nonlinear Equations in Several Variables}}. New York, N.Y.: Academic
Press.

\bibitem[\citeproctext]{ref-ostrowski_73}
Ostrowski, A. M. 1973. \emph{Solution of Equations in Euclidean and
Banach Spaces}. Third Edition of Solution of Equations and Systems of
Equations. Academic Press.

\bibitem[\citeproctext]{ref-ram_sabach_24}
Ram, N., and S. Sabach. 2024 (in press). {``A Globally Convergent
Inertial First-Order Optimization Method for Multidimensional
Scaling.''} \emph{Journal of Optimization Theory and Applications}, 2024
(in press).

\bibitem[\citeproctext]{ref-robert_67}
Robert, F. 1967. {``{Calcul du Rapport Maximal de Deux Normes sur
\(\mathbb{R}^n\)}.''} \emph{Revue Francaise d'Automatique,
d'Informatique Et De Recherche Operationelle} 1: 97--118.

\bibitem[\citeproctext]{ref-robini_wang_zhu_24}
Robini, M., L. Wang, and Y. Zhu. 2024. {``The Appeals of Quadratic
Majorization-Minimization.''} \emph{Journal of Global Optimization} 89:
509--58.

\bibitem[\citeproctext]{ref-rockafellar_70}
Rockafellar, R. T. 1970. \emph{Convex Analysis}. Princeton University
Press.

\end{CSLReferences}




\end{document}
